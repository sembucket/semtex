% $Id$
% CHAPTER ONE
% 1389 words at 26/4/99

\setcounter{endnote}{0}

\chapter{Preface}
\label{ch:intro}
\addtoendnotes{\protect\section*{Preface}}

Irrigation was introduced to Australia by European settlers, probably
no earlier than the 1820s.  The Aboriginal people constructed weirs
and excavated channels but only to trap fish.  Accordingly there was
no long-term experience of irrigation as in many other parts of the
world, but it has achieved great importance in Australia, with impact
in cities as well as the countryside.  Development at first depended
on the skill and enterprise of individual landholders, notably in
Tasmania, who were able to draw on nearby rivers.  Unsuccessful
efforts were made in the 1840s and 1870s to develop complex irrigation
schemes and until the 1880s most irrigation involved pumping rather
than gravitation from perennial streams.

When agricultural settlement extended on the mainland to inland tracts
with low unreliable rainfall the need for water soon led to interest
in irrigation and help was sought from colonial governments.  The
advice they first obtained from hydraulic engineers did not favour
irrigation but its progress in the United States encouraged popular
agitation in favour of government action at a time when the advantages
were being demonstrated in many parts of Australia by pastoralists,
farmers and Chinese market-gardeners.  At last steps were taken in
Victoria to establish irrigation through government loans to locally
based irrigation trusts and by agreement with two Canadians who had
developed irrigation settlements in California.  These projects were
sustained for a time by the prosperity which then prevailed in
Victoria; they suffered during the subsequent depression but provided
valuable though costly experience.

A major outcome in the 20th century was to enshrine irrigation as a
matter primarily dependent on government intervention.  In
south-eastern Australia the governments of the three south-eastern
states on the mainland set up bodies with authority to establish
irrigation settlements catering for local residents and migrants, and
to regulate the use of water for irrigation.  In Tasmania, however,
different arrangements were made.

My interest developed through work as a soil scientist with the
Commonwealth Scientific and Industrial Research Organisation (CSIRO).
Its Division of Soils had been established in 1929, mainly in response
to problems of waterlogging and salinity affecting irrigated lands in
the Murray River valley.  It undertook classification and mapping of
soils, concentrating at first on the numerous irrigation settlements
and extending gradually to other areas.  When I joined that Division
in 1946 it was providing information on the soils of areas intended
for war service land settlement and one of my first assignments was to
take part in a survey of farm land at Loxton, South Australia, before
establishment there of an irrigated horticultural settlement.  Related
work was undertaken elsewhere by colleagues in CSIRO and by others in
state government bodies.  The publication of soil maps and information
on irrigation problems provided me with an introduction to the
complexity of irrigation in Australia.

Extending the domain of irrigation was widely accepted as progress
during the mid-20th century but the 1970s brought doubts, with the
focus on Murray Valley salinity problems and criticial studies of the
economics of irrigation.  These concerns influenced the changes in the
1980s affecting management of the River Murray and control of the
water industry in different States.  However, serious problems continue
to arise with irrigation.

During the 1970s I realised that the history of Australian irrigation
had attracted little attention and began its study.  The main
contributions then included the essay by Alfred Deakin, the biography
of George Chaffey, which provided information on the Chaffey
irrigation colonies of Mildura and Renmark on the River Murray, a
popular account of irrigation in the Murray valley and near the
Murrumbidgee River, and a chapter of a book mainly concerned with
economic aspects of Australian irrigation.\fn{\citet{deakin1892};
\citet{alexander1928}; \citet{hill1937}; \citet{davidson1969}.}

In the last quarter of the 20th century, the history of Australian
irrigation has been referred to in works dealing with institutions
responsible for water supply and irrigation, the celebration of the
Australian bicentenary, and the centenary of the Mildura and Renmark
settlements.  Only one of these publications is devoted entirely to
the development of irrigation in an Australian state: Tasmania.  It
was followed recently by an account of the history of Australian
irrigation, which deals mainly with engineering aspects of events in
the twentieth century.\fn{\citet{masoncox1994}; \citet{hallows1996}.}

My aim has been for a comprehensive treatment of the pioneering period
of irrigation, showing the diffusion of experience among the colonies
or states, the contributions made by particular landholders,
engineers, and public figures, and the involvement of people from
various ethnic origins.  This book is concerned with developments up
to 1920 when the contemporary mix of government and private enterprise
was achieved.  While irrigation was always undertaken primarily for
better production by landholders in country districts, it had begun to
affect urban life through involvement with sewerage and by the
watering of parks, gardens and playing fields.

What is meant by irrigation?  Although definitions vary, they all
agree that the practice involves the application of water to the land.
Some assume that irrigation is concerned only with crop production,
but would exclude certain arrangements involving methods regarded as
primitive.  George Gordon, the hydraulic engineer with wide interest
in Victorian irrigation, advised that it is to be understood as:
%\begin{spacing}{1.0}
\begin{quote}
	the conveyance of water without labour from a point where it
	is collected, or made available, to the lands to which it is
	to be applied, and its distribution over these lands. The term
	is also so used in connection with the utilisation of the
	sewage water of towns \ldots\fn{\citet{gordon1878}.}
\end{quote}
%\end{spacing}
Another view is that irrigation includes `any process, other than
natural precipitation, which supplies water to crops'.  Irrigation
will be regarded in this account in the widest sense as the
application of water to land for a particular purpose.  Generally in
Australia it has been used in support of rural
industries.\fn{\citet{stern1979}.}

This book considers development of irrigation through three
significant periods.  The earliest runs from the establishment of
penal colonies to the impact of gold discoveries.  The unusual feature
of Australian irrigation is its lack of antiquity, and
chapter\,\ref{ch:early} concerns initiation of the practice.  The
remarkable early Tasmanian interest in its use and the relative lack
of it on the mainland are dealt with in chapters\,\ref{ch:tas}
and~\ref{ch:mainland}.  For the next period, distinguished by
\citet{roberts1924} as marking the emergence of
agriculture, chapter\,\ref{ch:emergence} deals with irrigation in
south-eastern Australia by European farmers and Chinese
market-gardeners and with the campaign for an ambitious scheme for
provision of irrigation and transport by an extensive canal in
Victoria.

Irrigation in the last period, starting about 1875, became involved
with closer settlement in forms which included the gradual selection
of small holdings, the planned irrigation areas, and various
cooperative settlements.  Independent irrigators made an impressive
contribution, shown in chapter\,\ref{ch:indep}, the first for this
period, but they were gradually overshadowed by the involvement of
some colonial or state governments in irrigation following official
enquiries and legislation as outlined in chapter\,\ref{ch:inquiries}.
Then follow several chapters roughly in chronological order: four
dealing with irrigation used under different arrangements for closer
settlement (chapters\,\ref{ch:trusts}--\ref{ch:groups}), and others
concerning the association of irrigation with cities
(chapter\,\ref{ch:sewage}), the development of relations between states
concerning irrigation in the Murray--Darling drainage basin
(chapters\,\ref{ch:murray} and~\ref{ch:stateint}), and with the use of
underground water (chapter\,\ref{ch:underground}).

The fact that irrigation is a human accomplishment determined that my
account of its pioneering phase in Australia should give attention to
the interests and attitudes of the main players in the story together
with the engineering and agricultural aspects of the development.  My
study showed that progress in the pioneering phase of Australian
irrigation has involved serious controversy, occasional romanticism,
and technology transfer combined with local innovation; these
considerations should be reflected in the following account.

%\section{References}
%1. A.Deakin, Irrigation in Australia, in the Year-Book of Australia
%   for 1892,pp 81-96.
%2. J.A.Alexander, The Life of George Chaffey; A Story Of Irrigation
%    Beginnings In California And Australia. 1928.
%3. Ernestine Hill, Water Into Gold,1937.
%4. B.R.Davidson, Australia Wet Or Dry?; The Physical And Economic
%    Limits To The Expansion Of Irrigation, 1969.
%5. Margaret Mason-Cox, Lifeblood Of A Colony; A History Of Irrigation
%    In Tasmania, 1994.
%6. P.J.Hallows \& D.G.Thompson, The History Of Irrigation In Australia,
%   (1996?).
%7.  G.Gordon, Lecture on Irrigation . . . and Drainage, 1878-79.
%8. P.Stern, Small-scale Irrigation, 1979, as cited by A.Pacey,
%    Technology In World Civilization, 1990.
%9. S.H.Roberts,  History of Australian Land Settlement (1788-1920), 19 4.
