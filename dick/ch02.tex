% $Id$
% CHAPTER TWO
%3584 wds at 29/4/99

\setcounter{endnote}{0}

\chapter{The Earliest Irrigators}
\label{ch:early}
\addtoendnotes{\protect\section*{Chapter \thechapter}}
\markboth{\textsc{Chapter \thechapter. The Earliest Irrigators}}{}

Deliberate application of water from streams or springs to land in
Australia apparently began between 1820 and 1830.  There is no
indication that this practice followed direction from British
authorities, nor could it have been a response to the advice of
E.\,G.~Wakefield\index{Wakefield, E.\,G.}, the celebrated theorist of
colonisation, whose interest in Australian irrigation was expressed in
1829 and 1834.  Whether entirely fortuitous or owing something to
contemporary usage in Britain, its adoption certainly relied on the
interest and ingenuity of individuals far apart.  The first irrigation
was undertaken either in New South Wales or in Tasmania.  Elsewhere
there is evidence of irrigation before 1835 only from Western
Australia where settlement began in 1829.

\section*{New South Wales}\index{New South Wales}
\label{sec:nsw}

It may appear curious that practical interest in irrigation in New
South Wales was not shown until at least thirty years after the
settlement at Port Jackson though probably much sooner in Tasmania.
The delay in starting the practice in New South Wales arose
principally from environmental constraints: climate and hydrology.

Governor Phillip's\index{Governor Phillip} decision in 1788 to locate
the penal settlement at Sydney\index{Sydney} was more or less forced
by his objections to Botany Bay\index{Bay!Botany} and the urgency of
finding an alternative site to accommodate more than a thousand people
soon arriving in the First Fleet.  Though suitable for mariners,
Sydney Cove gave immediate access only to land which was either too
rocky or too sandy for successful food production, provided only a
small stream draining nearby swamps, and was encumbered by many trees
too hard for the axes brought from England.  A better site for
agriculture was soon found near the head of Port Jackson;\index{Port
Jackson} the settlement there came to be known as Parramatta and its
growth for a few years outstripped that of Sydney\,---\,testifying to
the disadvantages of its site.  But the environs of Parramatta and its
residents, who generally had no previous agricultural experience,
failed to produce the amounts of food required; grain still had to be
imported.

Early exploration near Sydney had revealed the existence of a large
river draining from the distant hills known as the Blue Mountains and
discharging to the sea at Broken Bay, an inlet about 20 miles north of
Port Jackson.  This river, designated the
Hawkesbury,\index{River!Hawkesbury} was found to traverse an extensive
alluvial plain regarded as having great potential for food production
but too isolated at that stage from Sydney or Parramatta.  Its
occupation then by Aborigines was a further initial deterrent to
British intrusion.  Farming began there in 1794 after~22 settlers were
each granted small holdings near the Hawkesbury and a road was made
from Sydney. Transport of goods to and from the new settlement relied
on the river until Governor Macquarie's\index{Governor Phillip} order
for road improvement was achieved with completion in 1813 of the
turnpike road from Sydney to the river at Windsor.\fn{HRA I 1, 470,
483; \textsl{Sydney Gazette}, 27/11/1813, cited by
\citet[p.\,57]{bowd1982}.}

The Hawkesbury River generated many floods; there were ten between
1799 and 1819 which devastated crops nearby but despite this handicap
the district gradually came to be regarded as the granary of New South
Wales. Many settlers there had been convicts who qualified for small
grants of land after completion of their term of punishment. One was
Lawrence May\index{May, L.} who may have been the earliest irrigator
in Australia. At the age of eighteen he had been sentenced to death
for breaking into a shop in Dublin but the penalty was changed to
transportation and he arrived in Australia in 1791 aboard the Queen,
known as a horror ship. He began farming near the Hawkesbury after
being granted a small holding in 1800 and was successful enough to
establish a horse-driven flour mill in 1815; by 1828 he held 96\,ac at
Pitt Town.\fn{\citet[p.\,22]{johnson1991};
\citet[p.\,143]{bowd1982}.}

Floods on the Hawkesbury were a response to high rainfall in nearby
mountains, though the plain which the river traverses is by no means a
dry area.  The average annual rainfall at Sydney exceeds 47\,in, much
more than at any other early settlement on the Australian coast, while
at Windsor on the Hawkesbury the average is near 35\,in. The annual
rainfall is variable but its seasonal distribution is fairly uniform
except for a relatively dry spring.  Neither farmers nor pastoralists
had serious lack of water in most seasons and in the event of drought
there was generally scope for moving livestock to fresh grazing areas
on the Cumberland Plain between the mountains and the coast, or even
beyond the mountains to the Bathurst district following its discovery
after the drought of 1813.  There had been prolonged droughts in the
Sydney region frequently since 1788, but that commencing in 1826 was
the longest and most widespread yet known.\fn{J.\,B.~Henson, Eng.\
Assoc.\ NSW Proc.\ \textbf{6}, 1889; Comm.\ Bur.\ Meteorol.\ Bull.\
\textbf{48}, 1967.}

In describing this difficult period, the explorer Charles Sturt wrote:
\begin{quote}
	The year 1826 was remarkable for the commencement of one of
	those fearful droughts to which we have reason to believe the
	climate of New South Wales is periodically subject.  It
	continued during the following years with unabated severity.
	The surface of the earth became so parched that minor
	vegetation ceased upon it.  Culinary herbs were raised with
	difficulty, crops failed even in the most favourable
	situations.  Settlers drove their flocks and herds to distant
	tracts for pasture and water, neither remaining in the located
	districts.  The interior suffered equally with the
	coast.\fn{\citet{sturt1833}.}
\end{quote}

Conditions at Dunheved,\index{Dunheved} about 12 miles south of Pitt
Town\index{Pitt Town} on the Hawkesbury, were described in some detail
during the drought by Harriet King in letters to her husband, Phillip
Parker King, during 1827 and 1828. There was no complete lack of rain
there but livestock and crops suffered greatly until the spring of
1828 when falls of rain ensured some return from the wheat
crop.\fn{\citet{walsh1967}.}

It was during this serious drought that Lawrence May drew attention to
his efforts at irrigation.  The \textsl{Sydney Gazette} of
September~1, 1828, reported news it gathered a week previously:
\begin{quote}
	A Mr Lawrence May, a respectable settler at Pitt Town, has had
	the courage to erect a pump for the purpose of irrigating his
	land, an experiment, we believe, perfectly novel and
	unprecedented in the Colony. The pump is placed on the margin
	of the river, and conveys the water through lead pipes into a
	ditch, or trench, where it is conducted at pleasure, by means
	of furrows to any part of his farm. It is calculated that it
	will discharge 20 tons an hour, and requires only two men to
	work it. The first trial is to be made on Monday next, and a
	considerable number of respectable gentlemen intend to be
	present at so interesting a scene.
\end{quote}

Whether this trial was actually made is uncertain --- the Sydney
Gazette carried no news of it.  Lawrence May would have abandoned the
demonstration in the event of timely rainfall; it is possible that the
rain in the spring of that year, as mentioned by Harriet King, made
May's trial unnecessary.  There is another consideration which may
have deterred him.  Water then available for irrigation had become
brackish near Pitt Town and other farming settlements due to influx of
tidal water during the diminished flow of the Hawkesbury; normally the
river was fresh there and for 30~miles downstream.\fn{D.\,G.~Bowd,
personal communication.}  Whether the scheme reported in the press was
realistic for coping with a drought-stricken crop is uncertain: it
appears that arrangements for pumping would have provided water to a
depth of less than a quarter of an inch over 8\,ac after eight hours
of pumping.  Nevertheless nothing detracts from Lawrence May's
originality and ingenuity in his scheme for irrigation, apparently for
the first time in the colony.  His use of lead piping indicates a
capacity for procuring material then relatively scarce, though
probably used in construction of the first fountain erected in Sydney
at about that time.

Another indication of early interest in irrigation in the colony is
the report that James Mac\-ar\-th\-ur,\index{Macarthur, J.} son of the
pioneer wool-grower, invented a scheme for mechanical irrigation in
the late 1820s.  The time of this development --- assuming it occurred
in Australia --- would have been before he travelled to England in
1828, where he remained for a few years.  As with Lawrence May,
experience of the protracted drought of 1826--29 probably impelled
Macarthur to devise some means of watering plants of some value to
him.  These were in the gardens, orchards and vineyards at his
residence.\fn{\citet[p.\,492]{ellis1978}.}

During a further dry period during 1834--35 in the Sydney area, Sir
John Jamison\index{Jamison, J.} took some interest in irrigation from
the Nepean River, which is continuous with the Hawkesbury, but it is
uncertain whether his plans came to fruition.  Jamison owned the
imposing Regentville mansion and land on the Nepean near Penrith; he
was a wealthy man and a leading member of the Agricultural and
Horticultural Society of New South Wales, founded in 1822.  Early in
1835 a Sydney newspaper reported the importation of machinery for use
in irrigation at Regentville.  James Backhouse\index{Backhouse, J.}
visited the property in October 1835, noticed the effects of the
prolonged dry season on the state of Jamison's vineyard, but made no
mention of irrigation undertaken by the owner.\fn{\textsl{Sydney
Monitor}, 3 Jan.\ 1835;
\citet[p.\,338]{backhouse1843}.}

\section*{Tasmania}\index{Tasmania}
\label{sec:tas}

In 1830 there was reference in the Hobart Town Almanack to irrigation
of a property near Hobart.\index{Hobart} This may seem quite out of
character with the circumstances of land use on the island at the
time; drought was never the serious handicap it so often proved to be
on the mainland.  Yet apparently irrigation was contemplated in
Tasmania in little more than twenty years after settlement at Hobart
and it achieved some popularity in the next two decades.  The
developments depended on the association of settlement with river
valleys, experience gained locally in the control of water resources,
and the availability of convict labour.

George Arthur\index{Arthur, G.} arrived in Hobart in 1824 as the new
lieutenant-governor, shortly before Tasmania was made a colony
separate from New South Wales.  As a young army officer he had served
his regiment in different parts of the Mediterranean region.  Later he
was put in charge of the Colonial Office's interests in British
Honduras, where one of his activities during an eight-year term as
superintendent was to begin drainage of its swamps.  The latter
experience may have had significance for his private interest in
draining a swamp on the Derwent River\index{River!Derwent} near
Bridgewater,\index{Bridgewater} upstream of Hobart though still in its
tidal reaches. The swamp covered approximately 200\,ac of his farm
which has been referred to as Dutch or Marsh Farm.\index{Marsh Farm}
An embankment was constructed with sluice gates to allow drainage to
the river at low tide, preparatory to the production of meadow hay.
The sluice gates also may have been intended to allow the entry of
water for irrigation during periods of high river levels, when the
stream provided water of good quality.  The levels of the river near
the farm were subject to a tidal range of 3 to 4 feet at the highest
spring tide.\fn{\citet{shaw1967}; \citet{mckay1962};
J.~Moore-Robinson, The Haunted House, \textsl{Hobart Mercury}, 6 July
1935.}

Dr James Ross made reference to irrigation of this swamp in his
Almanack for 1830:
\begin{quote}
	A noble embankment has been completed, damming out the river
        effectually, which can, by sluices, be again let in so that
        about 200 acres may be most successfully irrigated in summer
        time, an advantage unequalled in the island, particularly in a
        dry summer. The quantity of rich meadow thus recovered from
        the river will always afford an adequate supply of
        hay.\fn{\textsl{Hobart Town Almanack} 1830, p.\,186.}
\end{quote}

Arthur's interest in this swamp started in 1826 when he began
acquiring properties which together made up Marsh Farm with an area of
almost 1400\,ac, including the swamp.  Productivity of the drained
swamp brought favourable comment from visitors; it became a show place
during the 1830s as indicated in 1833 by Mrs~Princeps.  Almost
directly across the river, a similar formation was reclaimed by Arthur
Davies\index{Davies, A.} and managed by the distinguished convict
Henry Savery;\index{Savery, H.} its embankment also involved sluice
gates but no indication of their use for irrigation is given in a
comprehensive account of both swamps.  The main problem in using these
areas was apparently drainage: wide internal drains were needed as
well as maintenance of the sluice gates.\fn{\citet{gowlland1980}.}

When George Arthur left Tasmania in 1836 his properties were
administered by his former secretary, William Thomas
Parramore,\index{Parramore, W.\,T.} who in March 1849 wrote to Arthur
about drainage of the swamp.  His letter tells of advice given by
James Blackburn,\index{Blackburn, J.} the engineer then engaged on
bridge construction nearby at Bridgewater, and by Roderic
O'Connor,\index{O'Connor, R.}  well known for his practical knowledge
of farming in Tasmania.  The engineer proposed the installation of
pumps to improve the swamp and in view of the probable expense
Parramore consulted O'Connor.  The latter advised Parramore that:
\begin{quote}
	\ldots he had a strong Lincolnshire navigator who understood
	drainage and banking by practice and that if it was possible
	to recover the Marsh this man would do it. \ldots His opinion
	I regret to say is far from encouraging.  The Fens, he says,
	in Lincolnshire that have been reclaimed rest on a substratum
	of clay four or five feet beneath the surface, upon this the
	banks are made and they are puddled with the clay.  Now at the
	Marsh there is peat to the depth of ten feet at least, a pole
	can be thrust down to that depth and no solid bottom is
	reached.\fn{Sir George Arthur papers, ML Ref.\ FM4/2688,
	Letter 13 March 1849, from W.~Parramore to Sir George Arthur.}
\end{quote}
This information suggests that the swamp could not be drained
successfully because of the depth of peat, so a need for irrigation
was unlikely to arise.  This farm was sold in 1854.

Another use of irrigation, associated with water-powered mills, may
have preceded the drainage of Governor Arthur's farm.  In one case,
relating to the property on the Derwent River once known as
Humphreyville but later as Bushy Park,\index{Bushy Park} `extensive
irrigation works with a flour mill' were reported as installed about
1820 by the owner, Adolarius William Henry Humphrey,\index{Humphrey,
A,\,W.\,H.}  an important public servant.  Humphrey's farm was visited
in December 1826 by the land commissioner Roderic O'Connor, who found
it presented `one of the most gratifying sights in the colony',
commented on the crops and livestock but failed to mention any use of
irrigation.  However, there were facilities for diversion of water
from an adjoining stream when the property was bought by Ebenezer
Shoobridge\index{Shoobridge, E.} in 1864.\fn{\citet{stancombe1968};
\citet[pp.\,56--58]{masoncox1994}; \citet{mckay1962}.}

Irrigation in the Midlands\index{Midlands} as early as 1825 has been
suggested by reference to diaries kept by James Cubbiston
Sutherland.\i<ndex{Sutherland, J,\,C.} He has been credited with
construction in March 1825 of an `irrigation channel from a brook to
his farm'. However, Sutherland's diary for~1 and~2 March 1825 shows
that one of his assigned servants and another man worked with Andrew
Gatenby\index{Gatenby, A.} `in cutting a course for the water through
the lagoon'.  Gatenby and Sutherland had travelled together to
Tasmania in 1823 and soon obtained grants of land on the Isis
River\index{River!Isis} near its confluence with the
Macquarie,\index{River!Macquarie} Gatenby's Barton homestead being two
miles from Sutherland's Rothbury.  Later in March, when water was
flowing after rain, Sutherland's men spent three days with Gatenby
`securing the embankment for his mill dam', and Gatenby's mill was
recorded as `at work' before April.  These entries indicate that
Sutherland had no direct involvement in cutting a channel but was
providing help to a neighbour and friend who sought adequate water for
his mill.\fn{\citet[p.\,138]{morgan1992}; J.\,C. Sutherland Diaries,
TA,
\textbf{1}; \citet[p.\,64]{brown1941};
\citet[pp.\,22--23]{mckay1962}.}

In 1829 there were 29 flour mills\index{flour} operating on the
island, with all but four driven by water.  Practically all had been
installed since 1818.  Whether the water leaving such mills was
returned to its natural course or allowed to spread over the ground
was possibly not a matter deserving comment although irrigation may
have been involved. However, by the late 1830s at least one instance
of the association of water-mills with irrigation had become known to
a Tasmanian writer, David Burn.\fn{\citet[p.\,129]{linge1979};
\citet[p.\,96]{burn1840}.}

\section*{Western Australia}\index{Western Australia}

British settlement on the Swan River\index{River!Swan} began in 1829
and irrigation was undertaken by 1834.  The town of Perth\index{Perth}
was established in 1830 on the banks of the Swan River in its tidal
reach but fresh water was available from local swamps and springs.
Captain Frederick Irwin,\index{Irwin, F.} in charge of the military
detachment to protect the new settlement, reported the use of
irrigation before 1835 on his property in the Swan River valley.  He
came to Western Australia in June 1829, remained there more than four
years, and returned in 1837 for seventeen years.  Irwin soon had a
house on land acquired in Perth and later added to his property by
acquisitions further inland.  Together with his cousin, William
Mackie,\index{Mackie, W.} the advocate-general, he took up 3240\,ac in
the Swan River valley in 1829--30 and 7000\,ac near York on the Avon
River.\index{River!Avon} Their farm known as Henley Park\index{Henley
Park} was on the Helena River,\index{River!Helena} tributary to the
Swan.  It was one of several properties in the district that were
acquired by members of the new colonial administration.  A prime
attraction there was the extent of alluvial soils superior in quality
to the sandy types so prevalent on the Perth coastal plain.  However,
it shared the general lack of adequate rain in summer.  The locality
had the advantages of being just beyond the tidal influence on the
river and accessible by boat from Fremantle and Perth.
Guildford,\index{Guildford} 7~miles northeast of Perth, became its
main township.\fn{\citet{mossenson1967};
\citet{honniball1967};
\citet[appendix]{ogle1839}.}

Irwin and Mackie were extremely fortunate in having Richard
Edwards\index{Edwards, R.} as manager of their rural properties.  He
was one of the hundreds brought out in 1830 by Thomas
Peel,\index{Peel, T.} whose late arrival led to a debacle for his
grandiose scheme to settle up to a million\,ac south of the Swan
River.  His chosen area had then been allotted to others and Peel was
offered inferior land further south; various upsets to his plans made
the hundreds of prospective settlers discontented and his scheme
failed by 1833.  Edwards had skills in brick and tile-making as well
as in farming, gardening, brick-laying, lime-burning, and brewing.
One of his sons was a carpenter; another was a ploughman.  It was
Edwards who carried out irrigation at Henley Park.

Irwin gave handsome testimony to Edwards and his work in his book
written early in 1835 in England.
\begin{quote}
	In the improvement of the gardens he takes peculiar delight,
	and is very successful, having a good knowledge of
	horticulture, acquired by serving an apprenticeship to a
	market gardener.  The spot he fixed upon for his first one was
	a somewhat elevated morass, on sloping ground, separated from
	the house by a ravine, and covered with rank vegetation owing
	to latent springs.  These, after burning off the surface, he
	dug out, and formed into circular wells of close and
	substantial brickwork, rising several layers above the
	surface; from these wells, at different elevations, he is
	enabled to conduct the water in channels to almost every part
	of the garden.  When the last accounts left, he was
	constructing earthen pipes for the purpose of completing his
	plans of irrigation, and also for conveying water across the
	ravine to the height on which the house is situated.  In this
	garden, and in another larger one, hereafter to be noticed,
	almost every kind of vegetable, and as many sorts of
	fruit-trees as have been introduced from tropical and
	extra-tropical countries, are found to flourish.  Among the
	former was the mangel-wurzel, already mentioned as having a
	root six feet in circumference, the tomato grows luxuriantly,
	weighed down with the load of its beautiful
	fruit.\fn{\citet[pp.\,57--60]{irwin1835}.}
\end{quote}

\section*{Conclusion}
\label{sec:conc}

The several references to early irrigation include those with some
telling details concerning use of water in aid of plant growth and
others giving only the bare mention of irrigation. These latter
references are unsatisfactory since in one case the term irrigation
related to channeling to convey water to a mill.  The most compelling
evidence for early irrigation clearly relates to Lawrence May in 1828,
but so far there is nothing to show that his plans for irrigation were
actually carried out.\fn{\citet{morgan1992}.}

%\section*{References}
%1. HRA I 1, 470,483.
%2. Sydney Gazette,27/11/1813, cited by D.G. Bowd, Macquarie Country
%   1982, p.57.
%3. K.Johnson \& M.Flynn, Convicts Of The Queen, p.22, in Exiles From
%     Erin (ed R.Reece), 1991.
%4. Bowd 1982, p.134, \& K.Johnson \& M.Flynn, Convicts Of The Queen, 1991.
%5. J.B.Henson, Eng.Assoc.NSW Proc. vol.6, 1889.
%6. Comm.Bur.Meteorol.Bull. 48, 1967.
%7. C.Sturt, Two Expeditions Into The Interior Of Southern
%   Australia,1833,vol.1,p.1.
%8. Dorothy Walsh(ed), The Admiral's Wife. Mrs Phillip Parker King, A
%    Selection Of Letters 1817-56, 1967.
%9. Sydney Gazette, 1/9/1828.
%10. Pers. comm. D.G.Bowd.
%11. M.H.Ellis, John Macarthur, 1978, p.492.
%12. Sydney Monitor, 3/1/1835.
%13. J.Backhouse, A Narrative Of A Visit To The Australian Colonies,
%    1843, p.338.
%14. A.G.L.Shaw, Sir George Arthur, ADB vol.1,p.32.
%15. J.Moore-Robinson, The Haunted House, Hobart Mercury, 6/7/1935.
%16. Anne McKay(ed), Journals Of The Land Commissioners For Van
%      Diemen's Land, 1826-28, Hobart, 1962, p.114.
%17. Hobart Town Almanack 1830, p.186.
%18. R.W.Gowlland, Some Van Diemens Land Affairs, 1980, p.118.
%19. R.W.Gowlland, 1980.
%20. Sir George Arthur papers, ML Ref. FM4/2688, Letter 13 March
%      1849,from W.Parramore to Sir George Arthur.
%21. G.H.Stancombe, ADB vol.1,1968, p.565, \& Margaret Mason-Cox,
%      Lifeblood Of A Colony, A History Of Irrigation In Tasmania,
%      1994, pp56-58.
%22. Anne McKay(ed), 1962.
%23. Margaret Mason-Cox, 1994 pp.56-58.
%24. Sharon Morgan, Land Settlement In Early Tasmania, 1992, p.138.
%25. P.L.Brown(ed). Clyde Company Papers, vol.1, 1941 p.64.
%26. J.C.Sutherland Diaries, TA, Vol.1, \& Anne McKay(ed),1962  pp22-23.
%27. G.J.R.Linge, Industrial Awakening, 1979, p.129.			
%28. David Burn, A Picture Of Van Diemens Land, 1840/1973,p.96.
%29. D.Mossenson, F.C.Irwin, ADB vol.2, 1967, p.5.
%30. J.H.M.Honniball, W.H.Mackie, ADB vol.2,p.174, \& N.Ogle, The Colony
%      of Western Australia, 1839/1977, Appendix.
%31. F.C.Irwin, The State And Position Of Western Australia, 1835, pp57-60.
%32. Sharon Morgan, 1992.
