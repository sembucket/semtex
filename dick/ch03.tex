% $Id$
% CHAPTER THREE
% at 29/4/99 4898 wds

\setcounter{endnote}{0}
\chapter{Tasmania 1835--1855}
\label{ch:tas}\addtoendnotes{\protect\section*{Chapter \thechapter}}
\markboth{\textsc{Chapter \thechapter. Tasmania}}{}

Although irrigation apparently failed to take hold in Tasmania in the
1820s, it certainly flourished in the late 1830s to a much greater
extent than in any mainland colony.  This development on land rarely
troubled by serious drought may seem remarkable.  Once adopted, the
practice continued in several parts of Tasmania for many years.
Individual landholders displayed initiative, technical advice was
available, and the authorities took some action to develop this
facility.

Although dominated by use as a penal settlement, Tasmania attracted
many settlers, able until 1834 to secure free grants of land, to an
environment somewhat comparable to parts of the British isles.
Attractive valleys with rivers fed from lakes and mountains, a mild
climate, and low frequency of widespread droughts appealed to British
adventurers, farmers, and retired army officers.  A further appeal for
prospective settlers was the availability of convict labour.  Under
the assignment system then in force, a landholder was allotted a
number of servants in proportion to the size of his property on
condition that he clothed, supported and accommodated them at his own
expense.  This supply of labour was additional to other servants
brought from Britain or India. The settlement in Tasmania had a
dual-purpose nature: `both a free colony and a
gaol'.\fn{\citet[p.\,97]{roberts1924}; K. Fitzpatrick~(1967), ADB
\textbf{1}, p.\,413.}

Contemporary accounts point to use of irrigation between 1835 and 1840
but the identity of some operators is unknown and details of its
application are generally unavailable.  James Backhouse, the Quaker
missionary who travelled widely in Australia and recorded much of its
natural and social features, lived in Tasmania during 1832--34 and
made a return visit in February 1836 when he first learnt of
irrigation recently undertaken near the town of New Norfolk on the
Derwent River.  David Burn's well-informed articles \citep{burn1840}
were published in London, with the claim that irrigation in Tasmania
was `a practice that has recently become one of general adoption.'
Arthur Cotton, an engineer who took a special interest in Tasmanian
irrigation from 1839 to 1843, wrote in 1840 that several landowners in
different parts of the island had increased production by irrigation
which was used on improved and natural pasture and in wheat-growing.
He also wrote under a pseudonym in 1849 that for the previous fifteen
years in Tasmania, lucerne under irrigation had provided a cutting
`about fifteen inches high every three weeks, throughout the spring
and summer'.\fn{\citet[p.\,348]{backhouse1843}; \citet{cotton1842};
\citet{cotton1849}.  `Delta' was identified as Authur Cotton by G.~Gordon
\citeyearpar{gordon1900}.}

\section*{The Irrigators}

One of the earliest irrigators was Alexander Reid (1783--1858), who
came to the island in 1822 and obtained grants of land on the River
Clyde, a tributary of the Derwent. On this property of 1400\,ac known
as Ratho, Reid gained success as a wool producer.  Ratho was
advertised for sale in 1837 with advice that the English grass
paddocks were then under irrigation from the Clyde.\fn{\textsl{Hobart Town
Courier}, 20 Oct.\ 1837, as cited by \citet[p.\,17]{masoncox1994}.}

Michael Fenton (1789--1874) was probably the first to combine
irrigation with the use of a water-mill.  He came to the island in
1828 after twenty years with the British army in India and obtained a
grant of 1970\,ac on a tributary of the Derwent River near New
Norfolk.  In 1829 he went to live on the property, later identified as
Fenton Forest, where a few years later a group of 76 men, women, and
children arrived as his indentured servants.  David Burn mentioned his
use of irrigation, apparently before 1840:
\begin{quote}
	Taking advantage of the latter river (Russell Falls), \ldots
	Captain Fenton \ldots has formed a watercourse whereby he
	drives a threshing machine and flour mill, supplies his house,
	and then irrigates an extended flat of the richest
	soil.\fn{\citet{robson1967}; \citet[p.\,96]{burn1840}}
\end{quote}

Another early irrigator may have been Francis Bryant, who leased
Redlands comprising 1550\,ac on the Derwent River near New Norfolk.
The \textsl{Austral-Asiatic Review} claimed in 1843 that Bryant was
the first settler in the colony to practice
irrigation.\fn{\citet[p.\,6]{masoncox1994}.}

From 1840 there was greater interest.  One of the first to become
involved then was William Kermode.  In that year he tried irrigation
on his Mona Vale estate in the Midlands, using water from a tributary
of the Macquarie River.  Neighbours who soon followed suit included
Thomas Parramore of Wetmore (3062\,ac), Samuel Horton of Somercotes
(1045\,ac) and Phillip Smith of Syndal.  At about the same time
irrigation was begun at Lawrenny, a large estate lying between the
Clyde and Ouse Rivers and owned by Edward Lord, one of the richest men
in the colony.  Kingston, a property 12 miles south-east of Launceston
was advertised for sale by Edmund Bryant in 1841 as having 200\,ac
sown down to English grass and clover, all under irrigation from the
Ben Lomond Rivulet.  Hunterston, a property of more than 4000\,ac held
by the family of Myles Patterson on the Shannon River---part of the
Derwent river system---was irrigated from January 1842.  David
Jamieson irrigated wheat before 1843 on his property, Glen Leith, near
New Norfolk.  The property of 1800\,ac known as Blair on the Clyde
River was advertised for sale in 1843 on the death of its owner,
William Allardyce, with information that 60\,ac of English grasses
were under irrigation.\fn{\citet[p.\,40]{masoncox1994}, `Bruni',
Irrigation in Tasmania; \textsl{Australasian}, 3 Nov.\ 1883;
\textsl{Hobart Town Courier}, 12 Nov.\ 1841; K.\,R.~von~Steiglitz, The
History Of Bothwell, 1958, cited by \citet[p.\,26]{masoncox1994};
\citet[p.\,417]{strzelecki1845}; \citet[p.\,20]{masoncox1994}.}

The popularity of irrigation in Tasmania in the early 40s is suggested
by advertisements of properties for sale.  Prospective buyers of the
Rosenvale estate of Anthony Williams on the Back River, New Norfolk,
were advised that its area of 160\,ac (65\,ha) included a dam and
sluice gate with troughing erected for irrigating the meadow land,
through which the Back River flowed with a sufficiency of water at all
times to turn a mill.  The Ashby estate of 2800\,ac (1130\,ha) on the
Macquarie River was advertised with the claim that following the
successful outcome of water conservation at Tombs Lake, `a very large
proportion of the property \ldots may be subjected to irrigation.'
Several notices in March and April 1842 concerning properties to let
or for sale in northern Tasmania made reference to possibilities for
irrigation: several farms to let in the Deloraine district, about 25
miles west of Launceston, were claimed as `being capable of
irrigation'; another four properties were similarly mentioned
later.\fn{\textsl{Hobart Town Courier}, 8 Mar.\ 1840;
\textsl{Launceston Advertiser}, 18 Feb.\ 1841; \textsl{Cornwall
Chronicle}, 5 Mar.\ 1842, 9 Apr.\ 1842.}

Even more telling of the vogue for irrigation was the publication of a
news item under the heading `Want of Water' which reported the plight
of those whose mills were `unable to work at present for the want of
water; the cause of which is ascribed to be the mania for
irrigation'.\fn{\textsl{Cornwall Chronicle}, 9 Apr.\ 1842.}  This may
be the earliest indication that irrigation in Tasmania might lead to
disputes over the riparian rights of landholders with river frontages.
However, irrigation in the early 40s does not appear to have involved
a very large area: half of the dozen or so properties then involved no
more than 2500\,ac under irrigation.

In the period 1845--50 there were apparently few additions to the
irrigated properties.  Part of an area of 6000\,ac near Deloraine in
northern Tasmania, granted to Thomas Archer in the 1820s; was
irrigated from the Meander River by 1848, when his son William Archer
went to live there, designed the house named Ches\-hunt, and later used
his engineering skill to develop irrigation on the property.  Isaac
Sherwin, a Launceston merchant, retired in 1845 to Sherwood, his
father's property near the Clyde.  He began irrigation there about
1847 and later extended its use after constructing a tunnel to provide
more water.  It is probable that irrigation at Dennistoun, also near
the Clyde River, was begun in this period.  Irrigation for hop-growing
may have been started by Ebenezer Shoobridge about 1849 at Glen Ayr,
near Richmond.\fn{\citet[pp.\,28, 24, 50]{masoncox1994}; Tasm.\
J.\,Agric.~\& Hortic.\ August 1860}

During 1850--55 irrigation failed to attract more landholders,
although there is evidence of additional areas provided by William
Archer at Saundridge near Cressy.\fn{\citet[p.\,30]{masoncox1994}.}

\section*{Motivation}

Tasmanian irrigation developed in response to favourable opportunities
for exports.  Sales of wool to Britain became significant by the mid
1820s and in 1830 the island supplied more wool than the mainland.
Rapid extension of sheep-grazing in south-eastern Australia during the
1830s involved many Tasmanians who crossed Bass Strait to become
squatters on new sheep runs south of the River Murray or made
settlements soon designated as Melbourne and Geelong on Port Phillip
Bay.  The initial success of pastoralists in this part of the mainland
created a demand for sheep to stock the extensive natural pastures,
both in the Port Phillip District and, after 1836, in the new colony
of South Australia.  The new settlements also required foodstuffs and
for a period Tasmania became the granary for people everywhere on the
mainland and was also an important source of potatoes.  The area under
wheat in Tasmania rose from 31\,156\,ac in 1830 to 60\,813\,ac in
1840.  These favourable conditions for landholders during the 1830s
enabled some to build mansions, or to make farm improvements including
irrigation which would both increase production and enhance the market
value of properties.  A vast amount of agricultural and construction
work before the early 1840s was done by assigned
convicts.\fn{\citet[p.\,207]{fitzpatrick1941};
\citet[p.\,133]{hartwell1954};
\citet[p.\,445]{coghlan1918}.}

The principal use of irrigation in this period was to improve fodder
production in summer and autumn for sheep and cattle. Summer grazing
was available on river flats, parts of which were swamps of no value
in wetter seasons.  Irrigation of these lowlands in drier seasons gave
better production of fodder; it generally relied on gravitation of
stream water diverted by a weir upstream.  This arrangement, most
appropriate for a property large enough to embrace the stream frontage
suitable for a gravitational system, was also generally required for
water-mills, so an association of milling and irrigation developed in
some places and proved advantageous in summer and autumn.  Utilisation
of low-lying areas by drainage combined with their irrigation in drier
seasons was apparently intended before 1830 for Governor Arthur's
Marsh Farm; it was certainly achieved by William Kermode on his
property in the Midlands.  Irrigation was used for some crops of
wheat, though rainfall in winter and spring was generally sufficient
to produce good crops.

\section*{Technology Transfer}

The early irrigation in Tasmania may be related to British experience
following invention of artificial flooding or `floating meadows' by
Rowland Vaughan in Herefordshire during the 16th century.  By the
1850s irrigated meadows in eleven English counties accounted for more
than 8~per cent of their total area under grass, so the practice may
have been known to Tasmanian landholders trying to follow the best
British farming methods.  `Floating the meadows' has been compared
with the `marcite' system of irrigation used in northern Italy, where
relatively warm river water is applied to meadows in winter and early
spring with the aim of displacing the colder stagnant water in the
soil and stimulating early production of fodder.  This winter
irrigation is referred to in English references as floating the
meadows, but the title of Robert Vaughan's book (1610) shows a wider
interpretation: `The Most Approved and Long Experienced Water Workes.
Containing the Manner of Winter and Summer Drowning of Meadow and
Pastures.'\fn{\citet[pp.\,110--5]{kerridge1973};
\citet[p.\,1042]{mingay1989}.}

Adoption of irrigation on the island may have been influenced also by
the arrival of British people aware of its value from residence in
India.  The attractions of Tasmania for these people probably became
known through movement of British regiments between India and
Australia, The most obvious early example of reference to Indian
experience is given by the activities of Arthur Cotton in Tasmania.
After service with the East India Company's military forces as an
irrigation engineer in southern India, he arrived late in 1838 on the
first of two long periods of sick leave.  He visited different parts
of the settled districts during 1839 and 1840, witnessed the existing
use of irrigation, and personally advised landholders on the technical
measures involved.\fn{\citet{blackburn1985}.}

Arthur Cotton had been involved with the maintenance of
long-established irrigation systems dependent on monsoon rains.  He
had experience of diverting water from a large river to supply an
extensive irrigation system at Tanjore, and he was familiar with the
widespread use of reservoirs (tanks) to conserve water for crop
production in the dry season.  In Tasmania he was thus able to advise
on dams, water conservation, and irrigation systems.  The full extent
of his assistance to individual landholders is unknown but many
details of his activity in the Macquarie River valley have been
recorded.  Before returning to duty in India in 1840, Cotton set out
his views on irrigation in Tasmania in a paper published later by the
Tasmanian Society.  His statement notes the opportunities for
irrigation in a land where streams generally had either a lake or
lagoon at their headwaters, thus providing opportunities for storage
by construction of embankments or dams.  He also gave estimates of
evaporation, detailed advice on construction of dams and races, and
discussed various mechanical ways of raising water from
streams.\fn{\citet{cotton1842}.}

Finding that much Tasmanian land was capable of improvement and that
agricultural production would remain the main source of wealth on the
island for many years, Arthur Cotton emphasised the first-rate
importance of irrigation.  Accordingly he maintained that before
private rights and interests became well entrenched, community
interests in natural resources should be affirmed, and in particular:
\begin{quote}
	Care should at once be taken that every river and natural
	reservoir should be, as far as possible, kept from falling
	into the hands of individuals, in such a manner as to place
	the districts connected with them in a situation of dependence
	upon them for this invaluable
	treasure.
\end{quote}

There was further recourse to British experience in India after the
appointment of Hugh Cotton as Deputy Surveyor General in 1842.  He had
served the East India Company in the same capacity as his brother and
showed his continuing interest in irrigation in a lecture on that
subject given to the Hobart Mechanics' Institute in July 1843.  On
that occasion Hugh Cotton quoted extensively from his brother's paper
after dealing with the `principles of irrigation as a science' and
their application in southern India.  He then proposed in November
1843 that surveys be made with respect to possible irrigation of land
on the extensive plains traversed by the Macquarie, Elizabeth, and
Lake Rivers; Eardley-Wilmot, the governor following Franklin, approved
this move in November 1843, instructing him to commence the work and
providing technical assistance for the survey.\fn{H.\,C.~Cotton
(1843), Lecture on Irrigation, delivered at the Hall of the Mechanics'
Institute, Hobart Town, Jul.\ 14, Papers Relative to the Irrigation
of Lands in Tasmania, Tas.\ Govt Printer 1855.}

\section*{Science and Irrigation}

Hugh Cotton's reference to science would have been well received by
members of his audience.  Five years previously, Governor Franklin, a
distinguished arctic explorer, had begun regular meetings in Hobart of
gentlemen with firm interests in natural science.  Thus began the
Tasmanian Society which soon became the leading scientific society in
Australia with 70 members in Australian colonies, New Zealand, and
Europe.  Franklin welcomed several visitors with scientific interests,
including Arthur Cotton and Paul Strzelecki, two men with marked
interests in Australian irrigation.  During his two years (1840--42)
in Tasmania, Strzelecki visited many parts of the island and
determined altitudes of many mountains, inland lakes, and estates as
data useful for development of irrigation.  He also recorded the use
of irrigation on several estates, and by publication brought
irrigation in Australia to the attention of readers there and
overseas.  Both men were honoured in 1842 by the Midland Agricultural
Association when they were elected as honorary members.\fn{\textsl{Van
Diemen's Land Chronicle} 30 July 1841; \citet{strzelecki1845}.}

During the period of greatest interest in irrigation development, when
Arthur Cotton and Paul Strzelecki were displaying their interest in
this matter, Governor Franklin was receptive to calls from the
Midlands for State control of reservoir sites and he later tried to
give some support to water conservation schemes by interpreting them
as in the public interest and therefore eligible for a supply of
convict labour.

\section*{Water Conservation for Irrigation}

Details of irrigation development before 1845 in different parts of
the island are unrecorded except for part of the Macquarie River
valley where several landholders had unofficial help from Arthur
Cotton from 1840 to 1843 and where Hugh Cotton was active later.  The
Macquarie is a major tributary of the Tamar, whose discharge compares
with those of important streams in southern Australia. According to
David Burn, `no river can be more dangerous and uncertain than the
Macquarie---in winter an impetuous torrent; in summer a mere chain
of occasional stagnant pools'.\fn{\citet{burn1840}.}

All development began from William Kermode's ineffective efforts to
irrigate from a dam on the Blackman River, a tributary of the
Macquarie, and Arthur Cotton's help in searching for a better supply.
They first considered Cotton's scheme for a canal diverting water to
the Blackman River from Lake Sorell at the west; this proved beyond
the capacity of the landowner.\fn{\citet[p.\,93]{masoncox1994}.}

\section*{Tooms Lake and Long Marsh}

The alternative---to conserve water at Tooms Marsh at the headwaters
of the Macquarie River---was most attractive as it required only
construction of a dam.  The promising scheme was put to a meeting of
local landholders in autumn of 1840, when in order to proceed with the
project they sought and won government action to reserve the marsh
from private occupation.  Then they erected a dam, later raised to
14\,ft, which effectively held back water soon known as Tooms Lake and
provided a beneficial release of water in the following
autumn.\fn{\textsl{Hobart Town Courier}, 16 Apr.\ 1841.}

Later in 1840, Adam Jackson made a survey on a northern branch of the
Macquarie River and discovered a long marsh which drained via a gorge
suitable for the erection of a high dam with much greater capacity
than Tooms Lake.  Before the year ended, Arthur Cotton had returned to
duty in India after writing his long paper on irrigation.  The fact
that Tooms Lake was holding a satisfactory volume of water stimulated
local landholders to act jointly for conservation of even more water
for irrigation.

Before Cotton's return from India in the spring of 1841, the local
land-holders, notably Robert Kermode, Andrew Smith, and George Scott,
worked to improve the dam at Tooms Lake and to win government support
for a dam at Long Marsh.  In the autumn of 1841 they asked for that
site and Tooms Lake to be reserved by government for the benefit of
irrigation.  This was soon agreed to but no further steps were taken
then, probably due to uncertainty about the availability of labour
during changes in the control of convicts.\fn{\textsl{Launceston
Examiner}, 18 Dec.\ 1847.}

The long-established system of convict assignment to settlers ended in
1840 on introduction of the probationary system decided on by
authorities in Britain.  This change coincided with the end of
transportation to New South Wales.  Tasmania then received more
convicts at a time when there were many problems in organising the
probation gangs of 200 to 300 men and allocating them to road works
and other tasks of public interest.  These difficulties, in a time of
economic depression, tried the capabilities of three Governors:
Franklin, Eardley-Wilmot, and Denison.

In the autumn of 1842 the local settlers resumed their campaign for a
dam at Long Marsh with a request that a probation gang should be
allocated for its erection.  This was agreed to by Governor Franklin
on the basis that the work would be of public interest.  However, the
settlers were required to provide a survey of the site for barracks,
meet the cost of their erection, pay the wages of an overseer for the
gang, and supply estimates for the dam construction.  After agreeing
to Franklin's decision, the settlers arranged for a further survey of
the Long Marsh site with a view to estimates for a dam to be 70\,ft
high.  Arthur Cotton contributed to the project by providing comments
in July 1842 on the estimates, offered to give further advice, and
declined to undertake supervision of the construction work.  A request
for his supervisory help was repeated later and in October Cotton made
a visit to the Long Marsh site, not long before his return to
India.\fn{\textsl{Launceston Examiner}, 18 Dec.\ 1847, Boyes to
Kermode; \citet[p.\,69]{gowlland1980}; personal
communication, R.\,W.~Gowlland.}

The barracks were completed in 1843 and work then began on the Long
Marsh dam, to be 60\,ft high with the aim of flooding an area greater
than 400\,ac to an average depth of
30\,ft.\fn{\citet{scarborough1975}.}

Late in 1843 Hugh Cotton, the deputy surveyor-general, was made
responsible for an irrigation survey in the valley of the Macquarie
River; he visited the Long Marsh site and reported that the dam was
under construction and might be finished by June 1845.  Governor
Eardley-Wilmot had directed Cotton to undertake the survey and was
apparently in favour of irrigation development, but he was soon
obliged to end government expense on the Long Marsh dam as it was no
longer deemed in the public interest.  Local settlers were given the
option, which they declined, of meeting all costs for use of convict
labour at Long Marsh.  Operations there were suspended early in 1844
but Cotton's irrigation surveys were not completed until mid-1845,
when he was given other duties.

The relative importance of Tooms Lake and Long Marsh for water
conservation and the scope for local irrigation are indicated in
Cotton's report to the Colonial Secretary on 13 July 1844:
\begin{quote}
	Tooms Lake is an extensive shallow reservoir formed by a low
	embankment, reclaiming when full about 14 million cubic yards
	of water.  It is complete, having been formed with the
	assistance of Government by the efforts of a body of settlers
	possessing property on the banks of the river below.  The Long
	Marsh is an extensive flat, receiving the drainage of a far
	greater tract of country than Tooms Lake; and may, by means of
	a short but high embankment, be made to retain 50 or 60
	million cubic yards of water. This work was undertaken, and
	carried out to a certain extent by government labour,
	conjointly with private subscription, but has been
	discontinued.  The first work to be done is the completion of
	the embankment; and I give it in my plan a base sufficient for
	its being raised to a height of 80 feet, when I calculate that
	it will retain all the water flowing into the marsh in one
	season.\fn{J. \& PP Tas.\ 1886, vol.\,IX, Paper~140,
	Irrigation: Report \& Estimates by Major Cotton, Longford 13
	Jul.\ 1844.}
\end{quote}

In that report, Cotton estimated that distribution of water to be held
in Tooms Lake and at Long Marsh by way of two main channels would
allow irrigation of 18\,000\,ac near the Macquarie River and provide
water to Tunbridge, Ross, and Campbell Town, all at a cost of
\pounds40\,000 using free labour.  He later reported that another project
using water to be retained in a reservoir proposed at Kearneys Bogs on
the Elizabeth River would provide for irrigation of a further
20\,000\,ac.\fn{Cotton (1844); Tas.\ H/A 1879, vol.\,37, Paper
no.\,69. Rept by H.\,C.~Cotton, 2 Apr.\ 1845.}

Repeated efforts were made over several years to resume construction
of the Long Marsh dam, all without success.  Despite this failure,
settlers still had the benefit of irrigation from Tooms Lake, the
arrangements being controlled by their own committee of management.
There was a noticeable benefit on Mona Vale, the large estate owned by
William Kermode.  Strzelecki gave high praise to its farm
improvements, especially the drainage of a marshy area and its
irrigation with water stored in Tooms Lake.\fn{\textsl{Launceston
Examiner} 18 Dec.~1947; \citet[pp.\,283--5]{strzelecki1845}.}
Another distinguished visitor who gave a glowing account of the
irrigation at Mona Vale a few years later was the soldier and author,
Godfrey Charles Mundy. He reported that
\begin{quote}
	Mr Kermode has \ldots carried irrigation to a greater
	perfection than any other person perhaps in the Australian
	colonies.  Of this he presently gave us proof by diverging
	from the direct road to the house, and bringing us to a wide
	tract of refreshing verdure lying in a gentle hollow.  Here
	are 500 acres laid down in English grasses, divided by English
	quick hedges into convenient enclosures, along each of which
	are water-ducts with dam-gates whereby he is enabled to throw
	the whole or part under water in the driest season. This
	valuable plot of ground, which will probably feed as many
	sheep as 15\,000 acres of the native pastures, was originally
	a swamp \ldots The swamp was by him thoroughly drained and
	cleared, the brook that supplied it was dammed back so as to
	form a reservoir and the precious element was thus rendered
	available when and where
	wanted.\fn{\citet[p.\,235]{mundy1852}.}
\end{quote}

\section*{Governor Denison and Irrigation}

William Denison, a military engineer, became Governor in January 1847
and left that post in January 1855 to become Governor-General of
Australia in Sydney.  He saw agricultural production as the basis of
the colonial economy and accordingly gave some support to development
of irrigation: he helped Hugh Cotton, once in charge of the irrigation
survey, and contributed a paper in 1851 on dams in relation to
irrigation to the Royal Society of Tasmania, which he had helped to
establish. 

Denison followed previous governors of Tasmania in acquiring land; he
engaged in agriculture near Launceston, where he drained a swamp with
convict labour and grew turnips and potatoes.  He corresponded with
Deas Thomson, the Colonial Secretary in Sydney on several matters,
including personal exchanges of farm produce.  Deas Thomson had three
letters in 1850 from Denison with opinions about legislation he was
considering for irrigation.  In June 1850 he was contemplating the
introduction of legislation `to enable the people on the banks of any
river to meet and constitute themselves into a municipal body for the
purpose of regulating the supply of water
\ldots' However, this matter was put aside because of greater concern with
transportation, the effects of the gold rushes, and constitutional
issues.\fn{\citet[vol.\,1, p.\,203]{denison1870}; Deas Thomson
Papers, ML A1531--2, vol.\,II, pp.\,534--546.}

During Denison's term of office, several prominent landowners were
confirmed as nominated members of the Legislative Council, to which
they were later elected.  The fact that some of these men had taken up
irrigation entirely on their own initiative had significance for
future development of this facility.

\section*{Conclusion}

The progress of irrigation in the early part of the period relied on
favourable economic conditions and the many opportunities to make use
of perennial streams.  Export of livestock to new settlements on the
nearby coast of the mainland, good prices for wool, and sale of grain
to the Sydney settlement all favoured adoption of a practice
recommended by engineers coming to Tasmania from India and possibly
known also from its use in Britain.  Moreover, the penal system
provided many free settlers with convicts as servants under the
assignment system.  Landholders enjoyed relatively prosperous
conditions in the late 1830s, when interest in irrigation was
particularly evident.

From 1840 there was gradual deterioration in the economy with decline
in wool prices and at the same time the assignment system was
abolished in favour of the probation system.  Opportunities for export
of livestock and grain to the mainland declined, as did the Tasmanian
revenue from land sales.  Neither the Governors nor the landholders
could be so optimistic in their support of irrigation development,
though Governor Franklin even in 1842 promised to provide a convict
labour gang for construction of a dam at Long Marsh.  That work was
terminated abruptly next year by Eardley-Wilmot, the next governor,
and the settlers who stood to benefit from the project refused the
opportunity of funding the scheme themselves.  Irrigation continued as
a private responsibility except for the cooperative arrangements
concerning use of water from Tooms Lake.

%\section*{References}
%1. S.H.Roberts, History Of Australian Land Settlement(1788-1920),1924,
%   p.97.
%2. Kathleen Fitzpatrick, ADB vol.1,p.413.
%3. J.Backhouse, A Narrative Of A Visit To The Australian Colonies,
%   1843, p.348.
%4. D.Burn, A Picture Of Van Diemen's Land, 1840, p.141.
%5. A.T.Cotton, 1842, On Irrigation In Tasmania,
%   Tasm. J.Nat.Sci. 1:81-93, 161-187.
%6. Delta, Australia Felix Monthly Magazine, June 1849, p.34. The
%    author of this anonymous article was identified as Arthur Cotton
%    by G.Gordon in his paper Irrigation in Victoria,
%    Min.Proc.Inst.Civil Eng.,vol.CXLII, 1900,pp 326-333.
%7. Hobart Town Courier, 20/10/1837, as cited by M.Mason-Cox, Lifeblood
%    Of A Colony, 1994,p.17.
%8. L.L.Robson, Michael Fenton, ADB vol.1,p.371.
%9. D.Burn, 1840, p.96.
%10. Margaret Mason-Cox, 1994, p.6.
%11. Margaret Mason-Cox, 1994, p.40, \& 'Bruni',Irrigation in Tasmania,
%      Australasian.3/11/1883.
%12. Hobart Town Courier, 12/11/1841.
%13. K.R.von Steiglitz, The History Of Bothwell, 1958, cited by
%    Mason-Cox 1994,p.26.
%14. P.E.de Strzelecki, Physical Description Of New South Wales And Van
%      Diemen's Land 1845, p.417.
%15. Margaret Mason-Cox, 1994, p.20.
%16. Hobart Town Courier, 8/3/1840.
%17. Launceston Advertiser, 18/2/1841.
%18. Cornwall Chronicle, 5/3/1842, 9/4/1842.
%19. Cornwall Chronicle, 9/4/1842.
%20. Margaret Mason-Cox, 1994, p.28.
%21. Tasm.J.Agric.\& Hortic. August 1860.
%22. Margaret Mason-Cox,1994, p.24.
%23. Margaret Mason-Cox, 1994, p.50.
%24. Margaret Mason-Cox, 1994 p.30.
%25. B.Fitzpatrick, The British Empire In Australia, 1941, p.207.
%26. R.M.Hartwell, Economic Development Of Van Diemen's Land 1820-1854,
%       1954, p.133.
%27. T.A.Coghlan, Labour And Industry In Australia, 1918/1969, p.445.
%28. E.Kerridge, The Farmers Of Old England, 1973, pp110-115.
%29. G.E.Mingay(ed), The Agrarian History Of England And Wales, Vol.VI,
%       1989, p.1042.
%30. G.Blackburn, Arthur Cotton And Irrigation In Tasmania 1839-43,
%      1985 Pap.Proc.R.Soc.Tasm. 119: 1-5.
%31. A.T.Cotton, 1842. 
%32. A.T.Cotton, 1842, pp177-78.
%33. H.C.Cotton, Lecture On Irrigation Delivered At The Hall Of The
%      Mechanics' Institute, Hobart Town, July 14th, 1843.
%34. Papers Relative To The Irrigation Of Lands In Tasmania,
%    Tasm.Govt.Printer 1855.
%35. Van Diemen's Land Chronicle 30/7/1841.
%36. P.E.de Strzelecki, 1845.
%37. D.Burn, 1840.
%38. Margaret Mason-Cox, 1994, p.93.
%39. Hobart Town Courier, 16/4/1841.
%40. Launceston Examiner, 18/12/1847.
%41. Launceston Examiner, 18/12/1847, Boyes to Kermode.
%42. R.W.Gowlland, Some Van Diemens Land Affairs, 1980, p.69, \&
%      Pers.Comm.  R.W.Gowlland 4/5/1984.
%43. D.H.Scarborough \& I.M.Brand, Tasm.J.Agric. 1975,pp. 227-230.
%44. J.\& Pap.Parl Tasm. 1886, vol.IX, Paper 140, Irrigation, Report \&
%     Estimates By Major Cotton, Longford 13/7/1844.
%45. Tas. H/A 1879, v.37, Paper No.69. Rept by H.C.Cotton, 2/4/1845.
%46. Launceston Examiner 18/12/1847.
%47. P.E.de Strzelecki 1845, pp. 283-285.
%48. G.Mundy, Our Antipodes, 1852, p.235.
%49. W.Denison, Varieties Of Vice-Regal Life, 1870, vol. 1, p.203.
%50. Deas Thomson Papers, ML A1531-2, vol.II, pp. 534-546.
