% $Id$
% CHAPTER FOUR
% 5977 wds at 29/4/99

\chapter{Australian Mainland 1835--1855}


At the beginning of this period, British settlement on the mainland
was confined to nineteen counties of New South Wales with 39 million
ac(61,000 sq.miles) - not three times the area of Tasmania - and ten
counties of the smaller area occupied in Western Australia. From 1835
the initiatives at several places on the coast, leading to permanent
occupation with land sales, gave scope for agriculture while vast
inland tracts of New South Wales were being taken for pastoral runs by
numerous squatters without secure rights of occupation.

Until 1851 the territory later known as Victoria was identified as the
Port Phillip District of New South Wales, the colony so affected by
wool production that its number of sheep increased to 13 million in
1850. At that time, the South Australian colony, begun in 1836, held
only a million sheep.1 Queensland was not separated from New South
Wales until 1859.

Irrigation was a matter almost ignored by officials and
landholders. For pastoralists the main problems then were to secure
herdsmen and water for livestock; farmers in most districts were more
concerned with market prospects than with lack of rain for crops. The
colony to record most instances of irrigation was South Australia
where many settlers were intent on farming yet found themselves in a
tract of land not well endowed with rainfall.

This chapter considers both the evidence of irrigation undertaken and
the views of prominent observers on the need and scope for irrigation.

\section{New South Wales \& Victoria}

There was movement inland by pastoralists from the nineteen counties
administered from Sydney and from Port Phillip and other places on the
south coast. Sheep and cattle were soon grazing in semi-arid country
with palatable herbage and major streams. Graziers liked to select
sites for their headquarters close to rivers but above flood level and
thus might irrigate small areas by pumping. The only person known to
do this was J.L.Phelps of Canally, a pastoral station on the
Murrumbidgee River near its junction with the Murray. His practice was
reported by Nehemiah Bartley who spent a few weeks at Canally in
November 1853:
\begin{quote}
	Mr John Lecky Phelps of Canally . . . was a man much in
	advance of his time. While other people, for 150 miles around,
	had no vegetables, he cultivated a half-acre on the river
	bank, with potatoes, green peas, French beans, cabbages, and
	kept it irrigated by a very simple process for rain was
	uncertain in that far inland spot. He had a Californian wooden
	pump, about six inches square, with its end fixed in the
	river, and about 150 feet of 'Osnaburg' hose from it to the
	top of the garden which was, perhaps, three feet higher then
	the lower end by the river and, half an hour of hand pumping
	every morning, sent the water flowing zig zag, backwards and
	forwards, and in and out, through all the well-kept furrows
	and beds of the enclosure, and the vegetation was always fresh
	and green at Canally garden.2
\end{quote}

Irrigation was certainly used by the late 1840s at Buninyong, 56 miles
west of Melbourne. This was the base for the large area held by the
Learmonth brothers and at the time it was the only inland township,
though unofficial, south of the River Murray. Among its facilities
were an inn and a boarding school.

Arthur Cotton, the irrigation engineer who made two visits from India
to Tasmania during 1838-43, came to Buninyong in 1848 as guest of the
Learmonths, his brothers-in-law. Wool prices were then low and many
pastoralists had turned to producing tallow by boiling down the
carcases of their sheep. Some cauldrons used in this process could
hold 300 sheep.3 Francis Peter Labilliere, an historian of early
Victoria, described the irrigation carried out by Cotton:
\begin{quote}
	The Messrs Learmonth at Buninyong.. . annually 'boiled down'
	so many of their sheep, that they found that it answered their
	purpose to have a place of their own, instead of sending their
	fat stock, as was generally done, to a public 'boiling down'
	establishment. The author well remembers visiting it a year or
	two before the gold discovery and seeing numbers of sheep
	prepared for the process. No finer mutton could have
	existed. The workingmen came in and helped themselves to
	whatever joints or slices they liked to take out of the sheep,
	as they hung skinned and ready to be cut up, before being
	thrown into the melting vat. This was of large size, and in it
	the meat was left till all the tallow was extracted from
	it. The offal remaining consisted of flesh, and of rich liquid
	which would have made fine soup or jelly. This was applied to
	the strangest purpose for which such a thing was ever
	used. Major, now Sir Arthur Cotton - being on furlough from
	India, on a visit to his brothers-in-law the Messrs Learmonth
	- tried an irrigation experiment at their place on a limited
	scale. He had a dam thrown across a small creek, and with some
	sluices and trenches to convey the water, was able to carry
	the soup or gravy and other offal from the boiling down
	establishment over some of the adjoining land.  Magnificent
	crops were thus produced on some of the fields and splendid
	vegetables and fruit in the garden.'4
\end{quote}

Pastoralists were prohibited from cultivating the land they held under
licence except for their own use, but it is doubtful if this condition
was strictly enforced. It has been claimed that in the Port Phillip
District 'definite records can be cited showing that pastoralists as
long ago as 1842 cultivated and irrigated by the furrow system
considerable areas of land'.5

The probability of irrigation on pastoral stations is suggested also
by experience recorded by William Howitt, the author and traveller who
spent two years in Victoria at the height of the gold rushes. William
Howitt was relatively fortunate during his travels to various
goldfields in having support from Lieutenant-Governor La Trobe in the
form of introductions to several pastoralists, some of whom supplied
vegetables and fruit to his party. Prominent among these landholders
were the Forlonges of Seven Hills (Euroa) in north-eastern
Victoria. Howitt called at their homestead in the late summer of 1853
when water and fodder for horses were exceedingly scarce along the
main road.6 The Forlonges provided plentiful supplies of fruit and
vegetables:
\begin{quote}
	The garden lies on the other side of the creek, in the flat a
	quarter of a mile from the house. It is, I suppose, five or
	six acres in extent and is full of splendid fruit in season -
	loads of grapes and melons of all kinds especially'.7
\end{quote}

This description suggests some supply of water to the garden from the
creek but Howitt made no reference to irrigation there or elsewhere in
his travels. In another episode, Howitt in October 1853 reported on a
pastoral station near Bendigo with 'a fine market for all its
produce':
\begin{quote}
	I have seen a man from this station going through the diggings
	with a horse and cart, the cart piled with cabbages, and a cabbage 
	stuck on a broomstick as a sign that they were for sale. . .8
\end{quote}

The farming carried on in New South Wales, apparently without
irrigation, was undertaken at many places near the coast, from the
Hunter River in the north to Melbourne, Geelong, Portland and Port
Fairy in the south, as well as inland at Bathurst and Goulburn. Most
of the farms involved would have received rainfall generally
sufficient for crops of wheat and hay.

The apparent general lack of attention by landholders to irrigation
occurred despite reference to it in various publications circulating
in Australia. The earliest of these were the tracts on colonisation by
E.G.Wakefield(1829, 1834), who drew on his experience in Italy to
recommend use of irrigation in Australia. Thomas Mitchell was the
first to foresee the opportunity for irrigation in eastern Australia,
where he conducted four expeditions. He saw the scope for irrigation
in the Murray valley during his 1836 expedition from Sydney to
Portland on the Victorian coast.9 His conclusions, probably composed
during a period of leave in England, show enthusiasm for the use of
irrigation:
\begin{quote}
	The Murray, fed by the lofty mountains on the east, carries to
	the sea a body of fresh water, sufficient to irrigate the
	whole country, which is, in general, so level, even to a great
	distance from its banks, that the abundant waters of the
	river, might probably be turned into canals, for the purpose,
	either of supplying deficiencies of natural irrigation at
	particular places, or of affording the means of transport,
	across the wide plains.10
\end{quote}

From the last of his expeditions out of Sydney, Mitchell offered
further advice on water conservation and irrigation in the colony of
New South Wales, including the southern portion of what is now
Queensland. He concluded that rocky ranges he saw 'afforded the means
of forming reservoirs of water' to be filled by the erratic but
abundant falls of rain. 'Irrigation. .  has not been yet attempted;
the natural fertility of the soil has alone been relied on. . . So
generally available is the structure of the country for the
reservation of water by dams. . . the undulations of the land present
everywhere facilities for constructing reservoirs, which heavy showers
would fill, and thus afford means sufficient for the purpose of
irrigation were not labour now so scarce there. . .' He also
considered that lands along the eastern coast, 'under a lofty range
which supplies abundance of water for the purposes of irrigation, are
well adapted for the cultivation of cotton and sugar'.11

It is interesting that Mitchell's earlier view of irrigation includes
reference to use of canals for transport but this aspect is ignored in
his later book with, instead, some consideration of railways. This may
reflect Mitchell's experience of the fervour for canal construction in
England during the 1830s and the growing popularity of railways there
during the 1840s.

Peter Cunningham(1789-1864 ) spent years at sea as a naval surgeon and
also in the period 1825-1830 acquired more than 1200ac in the Hunter
valley. His experience of Australia is given in two books published in
1827 and 1841. The first was issued before the experience of the
serious drought of 1826-29; the second makes several references to
droughts and to irrigation. His practical interest is shown in the
view:
\begin{quote}
	I have little doubt . . that many portions of Australia which seem to
	the eye incapable of irrigation by canals from the rivers will turn out
	to the contrary when levels are taken; these canals only requiring to
	be commenced at points up the rivers where their beds are higher 
	than the land to be irrigated, such canals often running many miles
	(in irrigating countries) before reaching the latter.12
\end{quote}

The Sydney Gazette referred a letter read to a meeting of the
Legislative Council on 31 August 1841 from a Tasmanian gentleman
concerning his irrigation of 2000 ac and offered its view that there
was
\begin{quote}
	very little prospect of this or any other useful project being set
	on foot until such time as that element of agricultural
	prosperity is in full operation; we mean an 'Agricultural
	Association' without which it is in vain to look for a change
	in the present defective agricultural system as practised in
	this colony.13
\end{quote}

No contribution to discussion of irrigation in Australia could have
had wider impact than the book by Strzelecki in 1845. He spent four
years (1839-43) in the country during which he made an exploration
including ascent of Australia's highest peak, studied geology and
mineralogy of eastern Australia and Tasmania, made soil analyses and
took an active interest in land use including irrigation. The
distinction associated with his naming of Mount Kosciusko helped to
make his book well-known in Australia.

On return to New South Wales from Tasmania in September 1842,
Strzelecki renewed friendly association with James Macarthur, his
companion on the exploration in the Australian Alps but then involved
with affairs of the Australian Agricultural Company which had
ambitious plans for land settlement. He visited the district north of
Sydney where the Company had secured a large grant of land embracing
the valley of the Karua, a stream discharging near Port
Stephens. Strzelecki stayed at Port Stephens with Phillip Parker King,
then in charge of the Company's operations, and for a few months
followed his interests in geology and agriculture before returning to
Sydney early in 1843.14

Strzelecki must have been impressed with the scope for irrigation in
this northern part of the colony, for he gave it prominence in his
views on irrigation:
\begin{quote}
	In New South Wales, the river Karua, and the tributaries of
	the Hunter, afford a most extensive range for the introduction
	of irrigation: the whole country of Cumberland may also be
	laid out in irrigated lands by means of the Grose and
	Warragambia, Hawkesbury and Nepean rivers, and with the aid of
	cheap wooden aqueducts. The river Nepean for the county of
	Campden; the Wollondilly, for Argyleshire; the river Cox for
	the Vale of Clywd; and the Campbell and Macquarie, for
	Bathurst; all offer most valuable water-courses for reclaiming
	and for increasing the productiveness of the comparatively
	sterile lands. The lower portion of Gippsland, sheltered as it
	is, to the northwest and westward by the Dividing range, and
	watered by five fine rivers, may be rendered, by irrigation, a
	most flourishing portion of the colony'.15
\end{quote}

In London during 1853 he became managing director of the Peel River
Land and Mineral Company associated with the Australian Agricultural
Company and thus may have attempted the settlement of migrants on its
Peel River block.16

Quite apart from all these contributions by men with direct experience
of Australia, there is one instance of an initiative for irrigation
coming from British authorities. This occurred in 1847, when the
Colonial Secretary, Earl Grey, sent documents to the Australian
colonial governors concerning irrigation in Italy and requested their
circulation, apparently to encourage use of the practice in
Australia. The Governor of New South Wales, Sir Charles Fitzroy,
apparently failed to make a positive response to the documents.17
However, the lieutenant-governors of South Australia and Tasmania
directed publication in government gazettes.18

The initiative in securing the information from Italy was taken by
Palmerston, then Minister for Foreign Affairs. The British Consul in
Milan furnished several documents concerning various types of
irrigation and drainage in northern Italy, together with references to
crops involved and the administrative arrangements. All this material
covered the period 1804 to 1846.19

Another contribution to the discussion appeared in the Australia Felix
Monthly Magazine, a short-lived periodical published in Geelong in
1849. This was the series of articles attributed anonymously to
'Delta' but actually the work of Arthur Cotton during his last visit
to Australia when he was based at Buninyong, the site of his novel
experiment with irrigation.

Cotton's articles took up the interests of pastoralists. The first
one, in June 1849, described the lessons on water conservation as he
knew it in India and Tasmania and considered the main needs of
pastoralists: water for livestock, sheep-washes, power supply, and
irrigation. His only reference to local irrigation concerned its use
in the boiling-down industry - his own personal experience mentioned
above. His next two articles, entitled 'The Supply of Water in the
Basin of the Murray' were concerned mainly with detailed measures
needed to provide water for pastoralists: especially the excavation of
tanks (dams) and diversion of water from streams. His article referred
to his own experience in the pastoral country where a brother-in-law
then held the Avoca and Wycheproof run.20 Irrigation was not regarded
as an immediate issue but the author envisaged that the Murray Basin
would be developed with irrigation of cereals. This aspect of water
use was to be dealt with in a later contribution, but there was no
further publication of the magazine after October 1849.

There is at least one other echo of Tasmanian experience of irrigation
in the development of a satisfactory water supply for Melbourne during
the early 1850s. By 1849, before the discovery of gold, its population
of 15,000 depended on water taken from the Yarra river alongside the
town; this was no longer satisfactory. The new city surveyor appointed
that year had previously been involved in a company pumping river
water; he was now called on to find a way of improving the water
supply. His proposal in January 1851 was to conduct water by an
aqueduct from marshes at the headwaters of the Plenty River, a
tributary to the Yarra, thus ensuring an unpolluted supply and
avoiding the expense of pumping. His modified scheme, submitted later
that year, incorporated a reservoir to be located in a natural
depression near the headwaters. This scheme would provide for a larger
population in the future and envisaged irrigation of several hundred
acres from the aqueduct in the meantime.21

The surveyor, James Blackburn, had arrived in Melbourne in 1849 after
years of experience as an engineer and architect in Tasmania,
including a period of employment assisting Hugh Cotton with the
irrigation survey in the Macquarie River valley in 1843. The adoption
of his proposal led to construction of the Yan Yean
reservoir. Blackburn's scheme of water conservation may have been
inspired by the basic proposition of Arthur Cotton in Tasmania that
the headwaters of many streams there presented swampy areas which
could be turned into reservoirs; his suggestion regarding irrigation
shows an interest also probably connected with his Tasmanian
experience. Water was brought to Melbourne from the Yan Yean reservoir
in December 1857.

\section{South Australia}

After Charles Sturt explored the River Murray to its mouth in 1826,
his reports encouraged the establishment of a new distinctive
settlement inspired by the views of E.G.Wakefield, promoted by an
unofficial Colonisation Society, and supported by the South Australian
Land Company. When plans for settlement were finalised, the
arrangements agreed to by the British government were a departure from
those of the penal settlements elsewhere and were intended to
facilitate production of primary products by free settlers. The first
settlement was made in July 1836 on Kangaroo Island by the South
Australian Land Company to provide a base for whaling, but the lack of
adequate supplies of fresh water forced its abandonment a few years
later by settlers.

Meanwhile the advantages of alternative settlement on the eastern
margin of Gulf St Vincent had been recognised by the official
surveyors and the site for Adelaide had been chosen after the River
Torrens had been discovered. During inspections of the coast in this
region, John Morphett in November 1836 recorded impressions of the
coastal plain and the mountain range further inland. No doubt
influenced to some extent by his four year's residence in Egypt, he
ventured the opinion that most of the plain 'is a rich light soil,
wanting nothing but irrigation, during the four or five hottest summer
months, to make it eminently productive all the year round.' His
interest in irrigation was further shown in remarks on the inland
mountain range:
\begin{quote}
	Mount Lofty bears nearly east, and the whole of this side of
	the range is intersected with gullies, ravines, and water
	courses, of the deepest kind, bearing evident marks of being
	acted on by powerful torrents. . All the hilly country along
	the coast has a similar character, but in no place is it so
	conspicuous as here. The facilities for damming up, and the
	creation of water power, are greater than I have seen in any
	country in an equal area, and as a probability exists that it
	will be advisable to irrigate during the summer, for the
	second and third crops, this is an inestimable advantage.'22
\end{quote}

The River Torrens is very small in comparison with the Murray, but as
the major stream discharging westward from the Mount Lofty Range it
proved invaluable as the source of fresh water for residents of the
township on its banks. It also provided water for the first irrigation
in the young colony. This was undertaken not by the observant John
Morphett or his English colleagues but by about 200 German migrants
who arrived in November 1838 with August Kavel, their Lutheran
pastor. They came to South Australia under the patronage of George
Angas, a major supporter of the new colony who advanced the money for
their transport. On arrival at Port Adelaide many were suffering from
scurvy and all lacked proper shelter and occupation. George Flaxman,
chief clerk to Angas, accompanied the party to South Australia and
took responsibility for the migrants. They could not obtain sufficient
employment but Flaxman arranged their lease of an Angas property on
the bank of the Torrens, about 4 miles(7 km) north-east of Adelaide,
where in December they started building houses and growing much-needed
vegetables and other food. Their settlement came to be known as
Klemzig.

The Germans at Klemzig were already in debt to Angas for their passage
money and now owed him for the rent of the land; they would also have
to pay dearly for imported cereals and potatoes. However, their
settlement close to the river gave access to water for domestic use,
gardening, and livestock. The Torrens flows strongly for only six
months of the year in response to high rainfall over its headwaters;
it then provided a good supply at Klemzig even in summer from 'a
series of lakes connected by rivulets which in some places are very
narrow and in others shallow but which always ran briskly'.23 At that
time Adelaide had a population greater than 5000 but Ferdinand Kavel,
one of the Klemzig settlers, found that the English residents were not
agriculturalists and his compatriot August Fiedler found that 'The
English had not cultivated any gardens because they did not like the
watering'.24

Growing vegetables at Klemzig was a necessity to combat the scurvy
contracted during the voyage to Australia. Whether they carried water
from the river up its steep bank or used a pump is uncertain, but
watering was a time-consuming task after sundown, understandably so
considering that in 1840 the gardens extended over almost 7 ac.25 The
settlers were soon raising enough vegetables to sell in Adelaide;
although there was no abundant rain in 1839 until June, production was
maintained by the regular hand-watering. The adequacy of river water
at Klemzig is indicated by the fact that apart from its use for
gardening, 'the women did almost all the washing for the people in the
town and had earned a good deal of money from it'.26 Some indication
of the variety of vegetables grown by the settlers in 1839 is given in
the Southern Australian which mentioned lettuces, potatoes, cucumbers,
French, broad, and scarlet beans, carrots, turnips, onions, radishes,
spinach, broccoli, cabbage, and green peas.27 Clearly these German
migrants were important initiators of irrigated market gardening, a
practice for which another group - the Chinese - later achieved
distinction throughout Australia.

There was one German, Johann Menge, who made a plea for irrigation in
South Australia before the arrival of Kavel's people and whose helpful
association with them extended to their gardening at Klemzig. Menge
has been described as eccentric, enigmatic and visionary, but his help
to the Klemzig group appears to have been practical and timely. In
1838, on leaving employment at Kangaroo Island with the South
Australian Land Company, of which Angas was director, he sent a
scathing letter to the resident manager in which he recommended
irrigation on the island.28 He was to have been responsible to Angas
for establishing Kavel's group remote from Adelaide near the junction
of the Darling and the Murray rivers, but this plan was
abandoned. However, Menge did accompany Pastor Schurmann, of the
Dresden Mission Society, on a visit in October 1838 to the Murray at
North-west Bend.29 During this or another visit northerly from
Adelaide, Menge was impressed with the quality of country now
identified as the Barossa Valley, and his proposal early in 1839 for
its occupation by Kavel's group was soon realised. He lived at Klemzig
during 1839 with the Fiedler family, and supplied vegetable seed
gathered from his own garden on Kangaroo Island.30 There appears to be
no evidence of Menge's involvement in the initiation of the settlement
at Klemzig, but it is improbable that Pastor Kavel and Charles
Flaxman, both newcomers to South Australia, would have made the choice
without special advice - and who better to give it than Johann Menge?

The first attempt at irrigation from the River Murray was made by
Edward John Eyre, who in 1841 became a settler on 1400 ac at
Moorundie, 70 miles northeast of Adelaide. Previously he had worked on
pastoral holdings in New South Wales, made two overland trips with
livestock from that colony to Adelaide, and gained distinction as an
explorer - notably for his epic journey in 1841 along the coast to
Albany in Western Australia. Eyre's intention to resume life as a
pastoralist had been demonstrated in 1839 when he acquired land by the
Murray; he was able to occupy the holding after appointment by
Governor Grey in September 1841 as Resident Magistrate on the Murray
and Protector of Aborigines, with a salary of �300 a year. Grey's
decision reflected concern at serious clashes earlier that year
between overlanders and Aborigines near the River Murray; it also
recognised Eyre's reputation for good relations with the natives and
may have been intended as a reward for his recent success in
exploration at a time when he had no income.

Eyre went to live on the river bank in October 1841. The nearest
European along this stream was then more than 60 miles south at
Wellington, but he was soon joined by E.B.Scott, his friend and
companion on some explorations, who also settled near the river and
gave important help to Eyre during residence there. Early in 1842 Eyre
set about preparing to grow wheat as well as arranging for the sale of
small allotments to provide a township. Few sales were completed and
the projected town, to be named for Sturt the explorer, was never
realised. However, that year found soldiers, mounted police, and
building workers numbering altogether more then 40 living at the
settlement.31

Eyre's agricultural efforts suffered setbacks. These derived partly
from his travels on duty as Protector of Aborigines and in the spring
of 1842 when he was employed to search for a missing settler on Eyre
Peninsula. More significantly there was marked variation in annual
rainfall at Moorundie and a serious risk of flooding by the river in
springtime. Reporting to Governor Grey in January 1844, Eyre mentioned
major difficulties in his farming: in 1842 rainfall was insufficient
for cropping but in 1843 the wheat grew luxuriantly after good rains
only to be ruined when the river flooded from September to November.32

Towards the end of 1842, Eyre had the satisfaction of raising a good
crop of wheat from half an acre of land moistened by soakage when the
river level rose. He proposed then to erect sluice dams across
channels normally allowing flood water entry to his river flats, so
that this flow could be regulated to provide for irrigation and for
retention of a supply behind the banks, yet guarding against serious
flooding of his land.33 The necessary work was to be undertaken in
1843. Eyre's 1844 report to Grey also stated that:
\begin{quote}
	From previous experience it had been apparent that the river
	rose periodically several feet and usually overflowed many of
	the alluvial flats lying between the river bank and the fossil
	cliffs - considerable labor and expense were bestowed in
	damming up the entrances by which the water escaped from the
	river to lower levels and in digging canals for the purpose of
	irrigation - these were completely successful as long as the
	river did not attain a greater height than it had risen to in
	1841 and 1842 but upon its rising several feet beyond this
	level there were of course no impediments to its progress -
	the dam and ditches were all destroyed and the whole expanse
	of alluvial flats were again laid under water, in some places
	fully six feet deep and of course all cultivation was
	completely annihilated.34
\end{quote}

This reverse did not deter Eyre; he advised that:
\begin{quote}
	the ensuing year will I trust see these difficulties fully
        conquered and good embankments thrown up at all the openings
        through the river bank - so as to effectively block out the
        highest flood - at the same time that a few shallow ditches
        cut across these lands intended for cultivation will afford
        the important and in Australia almost unknown power of
        completely irrigating at pleasure all such lands. Thus what
        has been inconsiderably deemed as an insuperable objection to
        the Valley of the Murray will I believe eventually prove its
        highest recommendation'.34
\end{quote}

Apparently the embankments were renewed after 1843 but the river rose
high again in 1844 and Eyre expected that much of his crop would be
lost and the balance cut for hay.35 This was Eyre's last year at
Moorundie - he resigned his official post and left for England in
December. Further flooding in later years brought more damage to his
dams; in 1856 the Moorundie settlement was abandoned in favour of a
higher site, several kilometres upstream, where the township of
Blanchetown developed.

Eyre left no details of results of his irrigation. Whether or not he
had real success with it, there is no doubt that he made every effort
to pioneer its use on this river.

There is no doubt about the practice of irrigation near Adelaide in
1843, using the discharge of the River Torrens onto lowland near the
coast, where occasional floods affected an area known as the Reed
Beds. By 1843 wheat-growing was well established in South
Australia. Its cultivation then involved more than 23 000 ac, mainly
on the Adelaide plains, and exports of grain had begun. The colony
easily outstripped cereal production in the area later known as
Victoria, which then had less than 5000ac under wheat and depended on
imports of grain. The spring season in Adelaide during 1843 was
relatively dry, with only 2.14 in. in Sept.-Oct. compared with a
long-term mean of 3.72 in .36 One unnamed farmer at the Reed Beds
experimented with irrigation and as a result the crop yield from the
irrigated portion was one-third better than from the rest.37 This
locality was subdivided soon after settlement and by 1843 had many
landholders, some with scores of hectares devoted to wheatgrowing.

Tasmanian irrigation also got attention in South Australia, when the
Southern Australian (29/12/1843) included a lengthy report of the
lecture on irrigation given in Hobart during 1843 by Major Hugh Cotton
who considered its use in India and Tasmania. The decision to reprint
this material from the Hobart press was made in view of topical
interest following news of irrigation at the Reed Beds.

Irrigation at the Reed Beds was repeated in 1844 when it was  reported that
\begin{quote}
	The beneficial effects of irrigation have never been more
	clearly demonstrated than by this season's crops at the Reed
	Beds. Of fruit the produce is abundant, but the wheat crops,
	generally good, are on some sections extraordinary. At
	Wymondlybury (Dr Addison's farm) the wheat crop is unusually
	heavy, that of the 'White Talavera' standing about 6 feet in
	height, the ears exceedingly well filled, and several we have
	seen (plucked indiscriminately) measure seven inches in
	length. The produce is estimated by competent judges at about
	45 bushels per acre.38
\end{quote}

This farm, the property of Dr Joseph Addison, covered 133 ac on the
north side of the River Torrens, the land lies in the modern suburb of
Fulham. Addison then had 42 ac under wheat. He was one of those
medical men who gave up their profession on migration to Australia.39

Rainfall amounting in places to more than 40 in. annually feeds
streams on both sides of the Mount Lofty range, and it was not long
before irrigation was taken up on its eastern side. This use also
involved another who gave up medicine in favour of farming. John
Rankine, regarded as the founder of Strathalbyn, 35 miles south-east
of Adelaide, obtained land in 1839 on the Angas River after arriving
in Adelaide with two brothers and their families together with other
Scottish migrants.40 They settled in the Strathalbyn district in 1841
where John Rankine's property, at the junction of the Angas River and
Gould Creek, was known as Blackwood. Before 1846, Rankine, according
to Robert Davenport diverted water from the river 'to irrigate his
fertile garden and potato crop'.41

During the 1840s vineyards were established near Adelaide and the
first winery was built in 1845.42 In 1850-52, Alexander Boord
established and irrigated a vineyard at Freshford - near Athelstone
seven miles from Adelaide - using water from the River Torrens.43

\section{Conclusion}

Irrigation failed to secure a firm place on the mainland where
resources - human and physical - were concentrated on producing wool
for export and essential foodstuffs: cereals and meat. Only a few
well-established men undertook irrigation and then only on a limited
scale; there is no indication that the practice became firmly
established anywhere as it had done in Tasmania. Most irrigation was
recorded from the southern part of the continent, particularly in
South Australia. However, greater acquaintance with parts of the
mainland previously unknown to Europeans brought an appreciation of
the value of irrigation in the future.

The discoveries of gold in New South Wales in 1851 led to great
dislocation of pastoral and agricultural production for a few years,
thus impeding further development of irrigation in that colony.

\section{References}

1. B.Fitzpatrick, The British Empire In Australia, 1941, p.137.


2. N.Bartley, Opals And Agates, 1892, p.62.

3. S.H.Roberts, The Squatting Age In Australia 1835-1847,1964, p.205.

4. F.P.Labilliere, Early History Of The Colony Of Victoria, 1878, vol.II,
    p.331.

5. R.V.Billis \& A.S.Kenyon, Pastures New, 1974, p.166.

6. W.Howitt, Land, Labour And Gold, 1855/1972, p.146.

7. W.Howitt, 1972, p.147. 

8. W.Howitt, 1972, p.247.

9. T.L.Mitchell, Three Expeditions Into The Interior Of Eastern Australia,
    1839, vol.II.

10. T.L.Mitchell, 1839, vol II, p.332.

11. T.L.Mitchell, Journal Of An Expedition Into The Interior Of Tropical 
      Australia, 1848, pp. 421-24.	

12. P.Cunningham, Hints For Australian Migrants, 1841, p.4.

13. Sydney Gazette, 2/9/1841.

14. M.Kaluski, Sir Paul E. Strzelecki, 1985.

15. P.E.de Strzelecki, Physical Description Of New South Wales. . ., 1845,
      p.446-47.          		

16. L.Paszkowski, Sir Paul Edmund de Strzelecki, 1997, p.263. 

17. C.J.Lloyd, Either Drought Or Plenty, 1988, p.164.

18. SA Government Gazette, 28/10/1847, \&  Hobart Town Gazette,
      28/9/1847, cited by Margaret Mason-Cox, Lifeblood Of A   
      Colony,1993,p.114.

19. HRA I, vol XXV, pp. 457-79.

20. R.V.Billis \& A.S.Kenyon, Pastoral Pioneers Of Port Phillip, 
      1974,p.168.

21. VicRept.Select Comm. Sewerage and Supply of Water, 1853.

22. J.Morphett, South Australia, 1836/1962, p.12.

23. South Australian Register, 30/1/1878.

24. D.Schubert, Kavel's People, 1985, p.96.

25. D.Schubert, 1985, p.113, 131.

26. D.Schubert, 1985, p.97.

27. Southern Australian, 1/5/1839, cited by Schubert 1985, p.88.

28. Letter by J.Menge, 17/10/1838, SRSA. item 986/9.

29. B.O'Neill, The German Experience Of Australia, 1988, p.29.

30. D.Schubert, 1985, p.96.

31. G.Dutton, Edward John Eyre, 1977, p.152.

32. E.J.Eyre, Reports and letters to Governor Grey from E.J.Eyre at
     Moorunde, 1985.

33. J.C.Hawker, Early Experiences In South Australia, 1899/1975, 2nd
      Series, p.14.

34. E.J.Eyre, Reports And Letters. . .1985, pp.67-69.

35. Moorundie and E.J.Eyre, SRSA item 951/9, Letter from Eyre
       to E.B.Scott, 1/11/1844

36. F.S.Dutton, South Australia And Its Mines, 1846, p.104, \& 
      Commonwealth Yearbook 1972..

37. Southern Australian, 26/12/1843.

38. Southern Australian, 31/12/1844. 

39. J.B.Cleland, Med.J.Aust. 29/10/1938, p.732. 

40. H.J.Stowe, They Built Strathalbyn, 1973.

41. F.S.Dutton, 1846, p.159.

42. J.Fornachon, The Wine Industry Of South Australia, in R.J.Best(ed),
      Introducing South Australia, 1958.		   

43. G.C.Bishop, The Vineyards Of Adelaide, 1977, p.22.

