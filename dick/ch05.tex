% $Id$
% CHAPTER FIVE
% 8681 words at 30/4/99

\setcounter{endnote}{0}

\chapter{Irrigation and Water Conservation 1855--1880}
\index{water conservation}
\label{ch:emergence}\addtoendnotes{\protect\subsection*{Chapter \thechapter}}
\markboth{Chapter \thechapter. Irrigation and Water Conservation}%
{Pioneering Irrigation in Australia}

The Australian colonies were greatly affected in this period by the
impact of mining. Discoveries of gold in New South Wales and Victoria
during 1851 lured thousands of people\,---\,Europeans and
Chinese\,---\,from abroad, to such an extent that the Australian
population rose from less than 450\,000 in 1851 to almost two million
in 1875.  Gold-mining relied heavily on the use of water in alluvial
mining, thus the appropriate techniques for control of water became
widely known in Australia.  Agriculture gradually developed in the
colonies most involved with gold-mining.  Irrigation was taken up in
favoured localities and ambitious plans for its extension attracted
much attention in one colony.  This chapter deals with the gradual
emergence of irrigation in this dynamic period following the squatting
age in Australia.\fn{Comm.\ Bur.\ Census \& Statistics, Demography
Bull.\ no.\,67, 1949.}

\section*{Water, Mining, and Miners} \index{mining} \index{gold}

Water was important in gold mining either as a hindrance to miners or
from its scarcity.  Alluvial \index{gold!alluvial} mining was often so
temporary an activity that elaborate measures for de-watering or
providing a supply were not required at many places.  Instead there
was reliance on portable equipment, including buckets for baling and
the relatively simple Californian pump, described by Howitt who
regarded it as of Chinese origin.  There was also recourse in these
situations to temporary dams and races to deliver water as needed.  An
example in water control was given by miners who had gained experience
in California.\fn{\citet[p.\,99]{howitt1885};
\citet[pp.\,397--409]{smyth1869}.}

Water races became important in several mining districts.  By 1869
their total length in Victoria was 2430 miles , with 1011 in the
Beechworth \index{Beechworth, Vic} district, 418 around Maryborough,
\index{Maryborough, Vic.} 396 at Ballarat, \index{Ballarat, Vic.} 266 in
Gippsland, \index{Gippsland, Vic.} and 264 near Castlemaine.
\index{Castlemaine, Vic.} There were also many reservoirs
constructed in Victoria by the government and in New South Wales by
private companies.  Supervision of water use was a responsibility of
officials on mining fields; these were generally on land still held by
the crown so riparian rights \index{riparian rights} conferred by
freehold land title were rarely involved.  Disputes on mining fields
over use of water are thought to be largely responsible for some of
the clashes between European and Chinese \index{Chinese}
miners.\fn{\citet{smyth1869}; Vic.\ Yearbook 1973, pp.\,161, 215;
\citet[p.\,79]{lloyd1988}.}

As the most superficial deposits of gold became exhausted, there was a
search for gold in the valleys buried beneath lava flows and known as
deep leads.  These required more elaborate methods for removal of
sediment and excess water.  A further development was the reef mining,
undertaken notably at Bendigo, \index{Bendigo, Vic.} Victoria, one of the
richest gold fields in the world during 1870--90.  For this type of
mining, often to depths of several hundred feet, special machinery and
engineering skills were required.  Reef mining determined the
persistence of mining in some localities for decades, thus giving a
basis for several large inland towns with needs for water supply.

Victoria was the colony most notable for its gold resources.  Over the
years more than four times as much gold was mined in Victoria as in
New South Wales.  As a consequence the population of Victoria rose
considerably to be almost 800\,000 in 1875 compared with 600\,000 in
New South Wales, 200\,000 in South Australia, 170\,000 in Queensland,
104\,000 in Tasmania, and 27\,000 in Western Australia.\fn{Comm.\
Bur.\ Census and Statistics, Demography Bull. no.\,67, 1949.}

Miners on the alluvial fields generally had a most demanding and
turbulent life.  The basic diet was meat, bread, sugar, and tea.
Vegetables were expensive if available at all, though many imported
luxury foods were sold to those who `struck it rich'.  Chinese miners
were numerous and distinctive on the alluvial goldfields where many of
them operated in groups organised by creditors who supported their
travel to Australia and to whom the miners had various obligations,
including the discharge of communal duties for cooking and growing
vegetables.  Their diet was based on imported rice, local vegetables,
and dried fish either imported or Australian.  Special areas were
established on the major goldfields as encampments for Chinese subject
to the authority of European protectors.  The more successful Chinese
miners were able to return to their homeland; their less successful
compatriots had to take up other occupations when alluvial mining
declined in the 1860s, or move to newly found goldfields in Queensland
or the Northern Territory of South
Australia.\fn{\citet[p.\,75]{gittins1981}.}

\section*{Agricultural Development}

Agriculture extended in Australia to meet the demand of the growing
population.  At first the gold rushes drew men from pastoral and
agricultural activities.  Many shepherds left their flocks and the
area cultivated for wheat at first declined in gold-mining colonies
but recovered by 1855.  Land was surveyed for new towns and sales
began to provide small holdings offering scope for food production.

A restriction on the use of land for farming arose from the entrenched
position of pastoralists, \index{pastoralists} which continued after
the partial failure of the initial legislation by the early 1860s in
New South Wales and Victoria to allow settlement by selectors in the
pastoral areas.  However, while there was little change in New South
Wales beyond the nineteen counties established much earlier, Victoria
achieved many small holdings, chiefly in its western
district.\fn{\citet[fig.\,42]{roberts1924}.}

The difficulty of land transport also restricted agriculture.  Bullock
teams had long been the mainstay of inland transport during the lack
of country roads, as shown by Howitt.  Farming was largely confined to
coastal areas of the colonies until roads were made, horses largely
replaced bullocks, and railways were begun.  In Victoria
\index{Victoria} the number of
horses \index{horses} rose from less than 30\,000 in 1855 to almost
200\,000 in 1875, and rural holdings increased tenfold in the same
period.  The development of horse-drawn traffic also put a demand on
agriculture through the need for oaten hay.\fn{\citet{howitt1885};
Vic.\ Yearbook, 1973, p.\,1090; \citet{peel1974}.}

In New South Wales \index{New South Wales} gold-mining failed to
sustain major populated centres comparable with those in Victoria.
Instead its farming areas remained at first within the nineteen
counties which included inland towns of Bathurst, \index{Bathurst, NSW}
Mudgee, \index{Mudgee, NSW} Goulburn,
\index{Goulburn, NSW} and Yass, \index{Yass, NSW} all served with
railways by 1880.

South Australian \index{South Australia} agriculture developed first
on the Adelaide plains and extended northwards in proximity to ports
on Gulf St Vincent.  This gave the opportunity to export wheat by sea
to other parts of the mainland as well as developing trade to Victoria
and New South Wales via the Murray River. \index{river!Murray} Apart
from cereal production, the Adelaide plains offered opportunities for
orchards and vineyards.

Agriculture in Tasmania \index{Tasmania} at first declined with the
exodus of people to Victoria but was favoured later by the growing
market for grain and potatoes grown near the northern coast and
shipped to mainland ports.

\section*{Irrigation Used}

References to the use of irrigation are made in many contemporary
publications; they vary greatly in the details concerning area, crop,
and methods of watering.  Knowledge of irrigation during the period is
likely to be incomplete through destruction or loss of relevant
diaries or other personal papers, the inability of journalists to
report all instances of irrigation, and a general disregard of market
gardeners as irrigators.  In those colonies where public interest in
irrigation developed there was some incentive for irrigators to come
forward and report their experience.  Thus in Victoria a conference of
irrigators held in 1890 was attended by men who told of their
irrigation undertaken a few decades earlier.

\subsection*{Tasmania}
\index{Tasmania}

In Tasmania irrigation was still used on several large estates as it
had been during the 1830s and 1840s and several hop-growers later
adopted its use at New Norfolk. \index{New Norfolk, Tas.} Alessandro
Martelli, \index{Martelli, A.} an engineer whose business in Melbourne
dealt with irrigation, visited Tasmania in 1860 to report to an
official commission on irrigation and provided information from the
Derwent river \index{river!Derwent} system and the Macquarie river
\index{river!Macquarie} system in the midlands.

Martelli's first report mentions eleven landholders who were then
using irrigation from the Derwent River or its tributaries.  His
information indicates a supply of water either for horticulture or
pasture for sheep and is more detailed concerning the method of
irrigation which mainly depended on gravitation.  The areas under
irrigation are given only for four landholders, with a total of
approximately 400\,acres.  His second report dealt with irrigation
from the Macquarie River and although it gives no details of its users
or use, the Irrigation Commission to which Martelli reported mentioned
Messrs Kermode, Parramore, Horton and Smith as irrigators in that
district.  The area irrigated by those gentlemen was supplied from
Tooms Lake \index{lake!Tooms} since the 1840s and from details
published in 1883 it appears that not less than 400\,acres were
irrigated from that source during the period.  Martelli's inquiries
did not provide details of irrigation in the northern part of the
island; where William Archer, \index{Archer, W.} chairman of the
irrigation commission, had undertaken before 1855 an elaborate scheme
of irrigation on his Cheshunt property of several thousand acres near
Deloraine.\fn{TasPP nos.~42, 43 of 1861;
\textsl{Australasian}, 3 Nov.\ 1883, `Irrigation in Tasmania', by
`Bruni'; \citet[p.\,28]{masoncox1994}.}

A significant use of irrigation by hop-growers \index{hops} at New
Norfolk was reported in 1874.  Excluding those who were mentioned in
the report by Martelli as irrigators, this group of about 15
landholders irrigated a total of almost 180\,acres of hops in
1874.\fn{\textsl{Hobart Mercury}, 2 Mar.\ 1874, cited by
\citet[p.\,74]{masoncox1994}.}

These sources of information indicate that during the period the total
area under irrigation in Tasmania was probably not more than 2000
acres.

\subsection*{South Australia}
\index{South Australia}

Reports of irrigation in South Australia \index{South Australia} refer
principally to its use for horticultural production.  By 1861 there
were more than 14\,000 people living within 10 miles of the capital
city.  Much of the high rainfall area in the nearby Mount Lofty Ranges
is drained by the River Torrens \index{river!Torrens} crossing the
Adelaide \index{Adelaide, SA} plains.  This stream is markedly inferior in
discharge to rivers associated with other Australian capitals but it
gave opportunities for irrigation, either by pumping from the main
stream or gravitation from its five
tributaries.\fn{\citet[p.\,227]{hirst1973}.}

First Creek \index{creek!First} gave a dependable supply of water in
the section known as Waterfall Gully, \index{Waterfall Gully, SA}
sufficient to power a saw mill \index{sawmill} erected in 1839 at the
end of this gully on the plain.  In 1854 Wilhelm and Augustine Mugge
rented \index{Mugge, W.} \index{Mugge, A.}land in the gully and began
to establish their most successful nursery, market-garden and orange
grove which for at least 30 years were sustained by water channelled
from the creek.  Out on the plain, the same creek supplied channels
and races made by the Clark family to water their garden at
Hazelwood.\fn{\citet[p.\,26]{warburton1981};
\citet[p.\,30]{warburton1977}; \citet[p.\,28]{ward1862}.}

Third Creek \index{creek!Third} enabled C.\,G.~Gwynne, \index{Gwynne,
C.\,G.} previously a legislator and then a judge, to irrigate his
eight acres of oranges and a vineyard at Glynde \index{Glynde, SA} by
1866.  Further downstream, near the Torrens, Rev.\ Thomas Stow
\index{Stow, Rev.\,T.} diverted water to supply his one-acre orangery,
\index{oranges} established in 1854, and a vineyard. \index{vineyard}
At Morialta
\index{Morialta, SA} on the upper reaches of Fourth Creek,
\index{creek!Fourth} John Baker
\index{Baker, J.} had vines, pear \index{pears}
trees and oranges under irrigation by 1862.\fn{\textsl{Adelaide
Observer}, 21 July 1866,
\citet[p.\,64]{warburton1977}, \citet[pp.\,12, 77]{ward1862}.}

Brownhill Creek, \index{creek!Brownhill} rising south of Adelaide, was
utilised by two irrigators.  At Torrens Park, \index{Torrens Park, SA}
R.\,R.~Torrens, \index{Torrens, R.\,R.} a distinguished public
servant, pumped water from it in 1862 for his bananas \index{bananas}
and three acres of oranges.  Dr C.\,G.~Everard \index{Everard, C.\,G.}
made a sluice at Ashford \index{Ashford, SA} to carry creek water to his
garden.\fn{\citet[pp.\,50, 75]{ward1862};
\citet{priess1991}.}

Two owners of land bordering the foothills near Glen Osmond
\index{Glen Osmond, SA} used
irrigation, even though they lacked access to streams, by collecting
runoff from their hills and springs to supply vines, oranges and other
fruit trees.  One was Arthur Hardy \index{Hardy, A.} of Birksgate;
\index{Birksgate, SA} the
other William Milne \index{Milne, W.} at Sunnyside, near
Beaumont.\fn{\citet[pp.\,24, 55]{ward1862}.}\index{Beaumont, SA}

To the west of Adelaide, A.\,H.~Davies \index{Davies, A.\,H.} began
irrigation in 1861 by using a double-action horse-driven pump
\index{pump!horse-driven} at his
well on Moore Farm \index{Moore Farm, SA} near Lockleys
\index{Lockleys, SA}
and running the flow through galvanised pipes laid on the ground to
serve his trees and shrubs.\fn{\citet[p\,15]{ward1862}.}

Only a few irrigators beyond the Adelaide district were known in 1862
to Ebenezer Ward.  \index{Ward, E.} Mr~Winckel was irrigating fruit
trees near Gawler and Joseph Gilbert \index{Gilbert, J.} had prepared
a small reservoir for irrigation in the Barossa \index{Barrossa, SA}
district, apparently at Pewsey
Vale.\fn{\citet[p.\,73]{barker1991}.}\index{Pewsey Vale, SA}

Streams discharging east of the Mount Lofty Ranges were also used for
irrigation; in December 1859 a meeting of 40 farmers at Langhorne
Bridge considered diversion of the Bremer River \index{river!Bremer}
into Mosquito Creek \index{creek!Mosquito} and installation of sluice
gates. \index{sluice gates} It was at Langhorne Creek,
\index{creek!Langhorne} east of the
Mount Lofty Ranges, that the Potts family started to irrigate, at
first using a pump driven by horse power and later, in the 1860s, with
a distinctive system for diverting water from the Bremer river to
inundate their Bleasdale \index{Bleasdale, SA} vineyard.\fn{\textsl{SA
Weekly Chronicle} 24 Aug.\ 1876;
\citet{smith1950}.}\index{vineyard}

\subsection*{New Wouth Wales}
\index{New South Wales} 

Irrigation in New South Wales in the period is recorded from several
widely scattered localities.  On the outskirts of
Sydney,\index{Sydney, NSW} J.\,R.~Atkinson \index{Atkinson, J.\,R.} was
reported as installing an irrigation system in 1856, planned by a
civil engineer, on his Sophienberg property, apparently near George's
River.  \index{river!George's} At the inland mining centre of Mudgee,
\index{Mudgee, NSW} Mr~Dickson in 1868 used a centrifugal pump
\index{pump!centrifugal} to raise
water 15\,feet from a creek for irrigation.  At Bodalla
\index{Bodalla, NSW}
on the south coast, the well-known entrepreneur Thomas Mort
\index{Mort, T.} had
drained swampy land for his dairy farm only to find a need for
irrigation during the serious drought of 1863--65.  He ordered a
horse-drawn pump \index{pump!horse-driven} from England with the
intention of using it for irrigation on the Tuross River
\index{river!Tuross} flats, by
means of a long hose fitted in turn to seven outlets from a branching
system of pipes.  In this way he expected to water up to 45\,acres.

The irrigation before 1855 by Phelps at Canally \index{Canally, NSW} in the
far west of New South Wales was followed in the 1860s by the supply of
water to an orchard \index{orchard} and garden on the Murray Downs
\index{Murray Downs, NSW} pastoral station across the Murray from the
Victorian town of Swan Hill. \index{Swan Hill, Vic.} Suetonius Officer
\index{Officer, S.} had
taken over this holding in 1862 and lived there until 1881.  He
`installed pumps worked by horses, windmills and later steam pumps
\index{pump!steam} and
irrigated large paddocks for lucerne \index{lucerne} and maize
\index{maize} as well
as orchards and groves of Jaffa oranges' \index{orchards}
\index{oranges} apparently
established by 1867.  Another early irrigator in that region was the
pastoralist D.\,H.~Cudmore \index{Cudmore, D.\,H.} who told a
Victorian Royal Commission in 1896 of his irrigation of lucerne and
other crops with Darling River \index{river!Darling} water for `the
past 26 years'.\fn{J.\,R.~Aust.\ Hist.\ Soc., vol.\,34, 1948, p.\,387;
\textsl{Australasian}, 23 Sept.\ 1871, `Irrigation', by T.\,Bath,
p.\,409; Personal communication, C.~Mort, 1992; \citet{hone1967};
G.\,K.~Chapman, Vict.\ Hist.\ Mag., vol.\,22, 1947, p.\,1; Vic.\
RC~Veg.\ Products, 1887, 4th Prog.\ Rept, p.\,109; VicPP no.\,19 of
1896, RC~Mildura, MoE, p.\,16.}

\subsection*{Victoria}
\index{Victoria}

Victoria was the scene of activity by a number of irrigators who
watered pastures, \index{pasture} fruit, \index{fruit} and vegetables
\index{vegetables} in places where water was available from str\-eams,
lakes, springs, or underground supplies.  Before 1860 irrigation had
been started near Melbourne \index{Melbourne, Vic.} and in parts of western
Victoria including the bank of the Murray River
\index{river!Murray} close
to South Australia.  Possibly the earliest record of Victorian
irrigation in this period concerns Rev.\ Thomas Goodwin,
\index{Goodwin, Rev.\,T.} an Englishman
who travelled to Victoria in 1852 and undertook an Anglican mission to
Aborigines.\fn{\citet[p.\,13]{massola1970}.}
\begin{quote}
	In 1855, I went to establish a mission to the Aborigines at
	Yelta, on the lower Murray, about 25 miles below where the
	Mildura \index{Mildura, Vic.} irrigation colony now is.  I was
	told when I went there, `You will not grow a cabbage!.'  I
	said, `I will try!'.  One gentleman told me he would eat his
	hat if I would ever have a garden there.  I knew I would not
	without irrigation.  I first of all erected a hand pump.
	\index{pump!hand} In a few years after I found it necessary to
	construct a windmill \index{windmill} for the working of this
	pump, and I gradually brought about three quarters of an acre
	under cultivation, growing vines, figs and other fruit and
	vegetables.\fn{Vic.\ RC~Veg.\ Products, 8th Prog.\ Rept,
	pp.\,1--3.}
\end{quote}

In one of his annual reports, Goodwin stated that water for the
gardens `had to be pumped from 20 to 30 feet, according to the height
of the flood'.  Aborigines \index{aborigines} undertook some of the
cultivation; Goodwin reported in 1863 that `some achieved quite a
degree of success in this pursuit'.\fn{\citet{massola1970}.}

This achievement and the earlier success on the Murrumbidgee
\index{river!Murrumbidgee}at Canally \index{Canally, NSW} with small-scale 
irrigation may have spurred others to follow their example.  This
probably happened in the case of the well-known garden and orchard
started by the Jamieson brothers on the Victorian side of the Murray
River \index{river!Murray} at Mildura, \index{Mildura, Vic.} the
pastoral station they held from 1847 to 1874.  Their garden apparently
became known to travellers on the river boats \index{river boats} by
the 1870s and it is claimed that Captain Barber of the
\textsl{Cumberoona} obtained fruit from Mildura in
1874.\fn{\citet{hill1937};
\citet[p.\,233]{mudie1965}.}

David Milburn \index{Milburn, D.} began by watering an acre or two of
orchard at Keilor, \index{Keilor, Vic.} 10 miles northwest of
Melbourne, in 1857, using a hand-pump \index{pump!hand} to draw a
supply from the Maribyrnong river. \index{river!Maribyrnong} At first
he held only seven acres of land under lease; later he brought more
than 45\,acres, increased his orchard area, and used horse power
\index{pump!horse-driven} for pumping.  Then in
1870 he purchased an additional 70\,acres, planted more fruit trees,
and erected a chain water-lifter worked by horse power.  This was
supplemented by a windmill \index{windmill} to pump water. His last
improvement for delivering water was to make a weir \index{weir}
across the stream and install hydraulic rams.\fn{Procs First Vic.\
Irrig.\ Conf.~1890; Aqua, vol.\,10, 1959, p.\,189, \& vol.\,13,
p.\,152.}

Others at the 1890 conference of irrigators who referred to their
early use of irrigation were G.~Eason
\index{Eason, G.} who in 1858 took water from a spring near Buninyong
\index{Buninyong, Vic.} to irrigate vegetables at first and later
established fruit trees; John Garden
\index{Garden, J.}  who
irrigated vines \index{vineyards} at Taradale \index{Taradale, Vic.} in
1861; Charles Hugh Lyon \index{Lyon, C.\,H.} who irrigated at Ballan
\index{Ballan, Vic.} on the property he held there from 1853 to 1866;
T.\,G.~Pearce \index{Pearce, T.\,G.} who began irrigating at Bacchus
Marsh \index{Bacchus Marsh, Vic.} in 1867; and Daniel Vince
\index{Vince, D.} whose orchard at Bridgewater \index{Bridgewater,
Vic.} was first irrigated about 1865.\fn{\citet[p.\,170]{billis1930};
Procs First Vic.\ Irrig.\ Conf.~1890.}

Some early irrigators failed to attend the 1890 conference though
their activity is known from other records.  One of them is indicated
in the Victorian prize essay by F.~Acheson~(1860), \index{Acheson, F.}
whose only mention of Victorian irrigation referred to Bacchus Marsh.
This was probably undertaken by Mr~James, reported by James Young
\index{Young, J.} in
1866 as an irrigator some six years previously.  Another reference to
early irrigation is given in the report of a resources survey
undertaken by the Philosophical Institute in 1860.  It mentions
irrigation of potatoes \index{potatoes} at Glendaruel,
\index{Glendaruel, Vic.} near the Clunes goldfield.\fn{\citet{acheson1860};
J.~Young, `Irrigation', A lecture delivered in the Mechanics'
Institute, Bacchus Marsh, 28 Sept.\ 1866; Trans.\ Phil.\ Inst.\ Vic.,
vol.\,4, 1860.}

An ambitious scheme of irrigation was begun in 1859 near the Yarra
River \index{river!Yarra} at Heidelberg, \index{Heidelburg, Vic.} now
a Melbourne suburb, by Sydney Ricardo.  \index{Ricardo, S.} He began
with a system adopted in England by a prominent agriculturist
J.\,J.~Mechi of \index{Meci, J.\,J.} Tiptree Hall in Essex.  Ricardo's
farm included 185\,acres on the south bank of the river, with land up
to 120\,feet above the river and alluvial flats to be used with
irrigation.  He relied on a 12\,horsepower steam engine and
double-action pump
\index{pump!steam-driven} to raise
about 200\,gallons per hour through a main pipe to an earthen storage
dam of large capacity on the high ground.  Water then gravitated
through underground pipes to outlets at intervals of 78\,yards, from
which two men could water the land with a hose at the rate of five
acres per day.  Ricardo's irrigation first involved a crop of turnips.
\index{turnips} He did not
complete the installation over all the suitable land: the Mechi system
was abandoned in favour of an Italian one which distributed water
along surface channels and required levelling of the ground.  Late in
1860 this alternative system was tried on the first area to be
prepared\,---\,approximately 10\,acres.  This undertaking had involved
employment of an irrigation engineer and the outlay of several
thousand pounds for equipment and earthworks.  When complete it would
have allowed irrigation of
100\,acres.\fn{\citet[p.\,174]{thomson1972};
\textsl{Victorian Farmers Journal and Gardeners' Friend}, 7 July 1860.}

The irrigation works installed for Sidney Ricardo were referrred to in
an advertisement which announced that a similar scheme was being
installed by Mr~Teague, an engineer, for Messrs Elms and Bladier at
Adelaide Vale, a locality on the Campaspe River \index{river!Campaspe}
near Bendigo. \index{Bendigo, Vic.} James Bladier, \index{Bladier, J.}
a Frenchman, was one of the first vignerons \index{vineyards} in the
Bendigo district.  No record of irrigation by Elms and Bladier at
Adelaide Vale \index{Adelaide Vale, Vic.} has been found, but when
Andrew O'Keefe \index{O'Keefe, A.} purchased the property of
1600\,acres in 1868 there was equipment to irrigate 20\,acres with an
orchard,
\index{orchard} garden, \index{garden} and a
vineyard \index{vineyards} of excellent varieties which had previously
produced thirty hogsheads of wine annually.  O'Keefe used irrigation
at Adelaide Vale for dairy \index{dairy} production in the 1870s.  A
description of his irrigation system mentions its use of hydrants and
hoses, indicating similarity to the first system installed for Sidney
Ricardo at Heidelberg in 1859.  O'Keefe's irrigation at Adelaide Vale
was described by Sankey in 1871.\fn{\textsl{Victorian Farmers Journal
and Gardeners' Friend}, 6 Oct.\ 1860; \citet[p.\,129]{benwell1960};
\citet[p.\,27]{healy1990}; VicPP no.\,48 of 1871, Appendix 28.}

The Learmonth brothers, \index{Learmonth Bros.} whose involvement with
Arthur Cotton's irrigation at Buninyong in 1849 is reported above,
moved to the Burrumbeet \index{Burrumbeet, Vic.} district during the
1850s and established Ercildoun \index{Ercildoun, Vic.}
homestead. Their irrigation of lucerne \index{lucerne} in 1860--61 was
reported by G.~Graham.\fn{\citet{brown1967}; Footnote, A.\,S.~Kenyon,
J.\,Agric.\ Vic., vol.\,10, 1912, pp.\,658--661.}\index{Graham, G.}

Probably the largest single area under irrigation during this period
in Victoria was the 500\,acres of sown pasture watered in summer by
Thomas Bath \index{Bath, T.} from Lake Learmonth
\index{lake!Learmonth} in the Ballarat \index{Ballarat, Vic.}
district from 1867.\fn{\textsl{Australasian}, 23 Sep.\ 1871, p.\,409.}

Information on the irrigation of vegetables in this period is very
limited.  Chinese miners \index{Chinese} grew vegetables for
themselves at many mining fields and made use of traditional methods
including application of matured human excrement. \index{human
excrement} An early reference to Chinese gardening on the Ballarat
goldfields appeared in the 1860 report of the Victorian Central Board
of Health:
\begin{quote}
	The Chinese cultivate gardens at several of the encampments
	where they raise a good quantity of vegetables peculiar to
	their country.  Urine is carefully preserved at such
	encampments and is universally used for watering the
	plants.\fn{VicPP no.\,71, 1859--60. Fifth Ann.\ Rept
	Central Board of Health, Vic.\ V\&P, LA vol.\, 4.}
\end{quote}

Market-gardening \index{market gardens} by Chinese and Europeans began
to flourish as the population increased; the number of Chinese
market-gardeners at Ballarat \index{Ballarat, Vic.} in 1868 was given
as 150.  In the Melbourne suburb of Brighton, \index{Brighton, Vic.}
market-gardening by Europeans developed in the widespread sandy soils
which received manure and night-soil transported from the city and
water obtained from shallow wells.\fn{\citet{cannon1974};
\textsl{Ill.\ Aust.\ News}, 1 Jan.\ 1869.}

Edwin Carton Booth, \index{Booth, E.\,C.} author of two books on life
in Australia, was one of the first to indicate Chinese in Victoria as
irrigators:
\begin{quote}
	The Chinese gardeners of the colony have shown that with a
	sufficient water supply they could grow any and every thing
	and at any time of the year. A Chinese gardener in Victoria
	would be esteemed a very poor hand at his trade if he did not
	take half a dozen crops of his ground in the course of a
	twelvemonth.\fn{\citet[p.\,185]{booth1869}.}
\end{quote}

The variety of garden produce offered by Chinese in Victoria during
the 1870s was record\-ed:
\begin{quote}
	In Melbourne and its suburbs, in Bendigo and Beechworth and
	Castlemaine, wherever men were gathered together, there, by
	the earliest dawn would be seen the Chinese market gardener
	with his wicker baskets of cool fresh lettuce and cabbage
	along on either end of a long bamboo.  The reign of `mutton
	and damper' with, in the cynically jocular language of the
	shepherd and digger `damper and mutton for a change' was over;
	cabbage and lettuce, potatoes, turnips and carrots, then
	bunches of grapes and baskets of tomatoes, fruits and flowers
	in plenty, and at cheap rates. They became teachers of the
	people.\fn{\citet[pp.\,31--32]{booth1873}.}
\end{quote}

Details of irrigation undertaken by Chinese or European market
gardeners in this period have not been obtained, though the locations
then recorded for these producers near streams and other sources of
water suggest that this practice was widely used.

\section*{Irrigation Advised}

Sir William Denison, Governor of New South Wales,
\index{Governor!Denison} gave an address to
the Philosophical Society of New South Wales in November 1856 on
irrigation.  He gave details of irrigation in Italy and India,
referred to the use of sewerage water near Milan and Edinburgh, and
recommended the adoption of a scheme which he was confident `holds out
the prospect of increased prosperity to the great staple interests of
New South Wales'.  A long report of Denison's paper was published in
the \emph{Sydney Morning Herald} of 13 Nov.\ 1856 and a short report
was given in the \emph{Melbourne Argus} of 1 Dec.\ 1856. 

At much the same time there were expressions of interest in irrigation
by scientists in Victoria.  In 1857 the Philosophical Institute of
Victoria published a paper on irrigation by F\,.C.~Christy
\index{Christy, F.\,C.} which considered some oportunities for water
conservation and irrigation in Victoria, and the contribution by
C.~Hodgkinson \index{Hodgkinson, C.}  on development of the productive
capacity of the country adjacent to the upper Murray River, which
estimated the scope for increased production there by use of
irrigation.  Discussion of Christy's paper in May 1856 led to
appointment of a commission of inquiry into irrigation, including
Christy, Hodgkinson, G.~Holmes, \index{Holmes, G.} Prof~Hearn,
\index{Hearn, Prof} R.~Brough Smyth, \index{Smyth, R.\,B.} J.~Brache,
\index{Brache, J.} and Prof~Wilson. \index{Wilson, Prof} No statement
of activity by this group appeared in subsequent publications by the
Institute, but a few years later it published a long report from its
Committee on Resources of the Colony of Victoria.  Agriculture and
horticulture provided one of the seven headings considered;
agricultural resources were reported at length by Professor Mueller
\index{Mueller, Prof} and four colleagues.  They considered three main
geographical divisions of the State, referred to current use of
irrigation, and recommended it as an aid to production in western and
northern Victoria.  This report must have influenced the Institute,
which became the Royal Society of Victoria,
\index{Royal Society!Victoria} to draw
public attention to questions of Victorian
resources.\fn{\textsl{Melbourne Argus}, 1 Dec.\ 1856; Trans.\ Phil.\
Inst.\ Vic., vol.\,1. 1857, Proc.\ p.\,xxvi; Trans.\ Phil.\ Inst.\
Vic.\ vol.\,4, 1860, Rept Resources Colony Vic.}

A competition was announced by the Royal Society of Victoria in 1860
for prize essays on Victorian resources. Four topics were given,
including water, agriculture, gold and manufactures. Competitors were
given six months to complete their entry for one of the four prizes
given by the Victorian government, each worth \pounds125. Frederick
Acheson, \index{Acheson, F.} a civil engineer, won a prize for his
essay on water storage, which made several references to prospects for
irrigation in different parts of the colony and mentioned current
benefits of irrigation at Bacchus Marsh. \index{Bacchus Marsh, Vic.}
He provided particulars for different streams, including their rates
of discharge at specified localities in December 1859.  For irrigation
on the northern plains he proposed a scheme of artificial channels to
distribute water from streams across the adjacent country; these were
shown on a map included in the collection of prize essays published by
the government in 1861.  The potential value of irrigation was
referred to also in the prize essay by Charles Mayes,~CE,
\index{Mayes, C.} on
development of resources.\fn{The Victorian Government Prize Essays
1860, 1861.}

The sources available to newspapers for information and views on
irrigation included the lectures delivered at Mechanics Institutes
\index{Mechanics Institutes} and
papers read at meetings of scientific societies.  So, in 1860, John
Julius Stutzer, \index{Stutzer, J.\,J.} public school inspector in
Tasmania, lectured on irrigation at the Mechanics Institute in Hobart
and was reported in the \textsl{Hobart Mercury}. His address referred to
recent interest in irrigation in England and Italy and mentioned
Tasmanian attention to the use of Blackman River,
\index{river!Blackman} which might be supplied from Lake Sorell
\index{lake!Sorell} by means of a connecting tunnel.  Soon after this
Stutzer transferred to Victoria to become editor firstly of the weekly
\textsl{Yeoman and Australian Acclimatizer} and later of the
\textsl{Australasian}.  In 1857 H.\,M.~Hull,
\index{Hull, H.\,M.} erstwhile colleague of Denison, addressed the
Royal Society of Tasmania \index{Royal Society!Tasmania} on irrigation
in the district of Cumberland, where irrigation had been practised for
15 or 16 years.  At the conclusion of the meeting, the society passed
a resolution recommending the government to help with development of
irrigation in various districts.\fn{\textsl{Hobart Mercury}, 10 Aug.\
1860;
\textsl{Hobart Town Courier}, 7 Aug.\ 1857.}

\section*{Water Conservation and Irrigation\\ Schemes}
\index{irrigation!schemes}

Alluvial mining \index{gold!alluvial} in the Bendigo \index{Bendigo,
Vic.}  district of Victoria had made it one of the richest goldfields
in the world by 1860 despite a scarcity of water.  A private company
formed in 1859 to provide waterworks was unable to cope with the
problem and the Victorian Government accepted responsibility a few
years later.  In 1864, James Brady, \index{Brady, J.} a civil engineer
who had been involved with the supply of water to Melbourne from the
Yan Yean reservoir, won the competition for the best proposal for an
improved water supply for Bendigo and Castlemaine. \index{Castlemaine,
Vic.}  His scheme covered the main requirements\,---\,for alluvial
mining and urban water supply\,---\,and introduced the matter of
irrigation, to be developed as alluvial mining declined.  His
proposal, which involved harnessing the Coliban River,
\index{river!Coliban} draining better watered land at the south, was
undertaken with the agreement of J.\,F.~Sullivan, \index{Sullivan,
J.\,F} the Minister of Mines and a Bendigo
resident.\fn{\citet{quaife1976}.}

Mr Sullivan as the Minister for Mines also had an interest in the
scheme under consideration for irrigation and water supply in the
Bacchus Marsh \index{Bacchus Marsh, Vic.} and Melton \index{Melton,
Vic.}  districts west of Melbourne.  Irrigation had been undertaken
for some years previously in the former district and unsuccessful
proposals involving government support had been made as early as 1859
for its more extended use there.  Interest was revived later, as
reported by James Young, \index{Young, J.} a prominent resident, in
his lecture in Bacchus Marsh in September 1866.  Mr Young explained
that recent legislation on Victorian waterworks allowed the government
to advance funds to provide water supply for irrigation and domestic
purposes.  Accordingly a scheme had emerged which would provide water
for irrigation of thousands of acres at the Marsh and for domestic
supply at Melton.  It would depend on creation of a reservoir five
miles north of Bacchus Marsh, with distribution by channels.  The
scheme had been put to the Minister (Mr~Sullivan) with favourable
response, and now required the potential ratepayers for irrigation to
guarantee repayment of funds advanced.  This scheme was not realised
but irrigation at the Marsh was soon used by several
landholders.\fn{J.~Young, Irrigation, A Lecture Delivered In The
Mechanics' Institute, Bacchus Marsh, 28 Sept.\ 1866.}

By 1868 one of the main reservoirs in the Coliban scheme was
incomplete and another had not been started yet expenditure was
mounting.  After a few years of inconclusive investigations by
Victorian engineers, the government decided to call in
Lieutenant-Colonel R.\,H.~Sankey \index{Sankey, R.\,H.} of the Royal
Engineers in India.  He was asked to answer 16 questions concerning
the Coliban scheme and a reservoir near Geelong, \index{Geelong, Vic.}
and invited to comment on any other related matters he chose. Sankey
spent several months in Victoria during 1871 and in August and
September submitted reports to the Minister of Mines. His main report
with numerous appendices runs to nearly 120 pages.\fn{VicPP no.\,48 of
1871.}

Sankey advised that despite evidence of faulty design and extravagance
in the Coliban project, \index{irrigation!scheme!Coliban} the
government should proceed to complete the works.  In responding to the
Minister's invitation, he considered that inclusion of irrigation as
one of the aims for the Coliban scheme was inappropriate because
provision of water for mining required elevated reservoirs and
channels but irrigation should involve low-level reservoirs and very
large volumes of water.  However, no effort was made by the government
to exclude possible irrigation from the scheme.

Another matter raised by Sankey was the need to establish a government
office concerned with `waterworks as a special branch of the
department charged with execution of public works in the colony.
Further, that it would be of the utmost importance to the country to
obtain the service of a thoroughly trained hydraulic engineer to act
as head of this special branch'.  Sankey also advocated establishing
careful registration of rainfall and discharge of rivers as
`professional knowledge and research, time and patience are in these
matters necessary'.\fn{\citet[quotation, pp.81--82]{powell1989}; VicPP
no.\,48 of 1871, appendix no.\,28, p.\,113.}

Before returning to India, Sankey also reported on a scheme proposed
for water supply in Adelaide, where there was some interest in its
possible use for irrigation.\fn{SAPP no.\,97 of 1871.}

The Coliban project was continued, leading to completion of the
Mal\-msbury reservoir in 1873--74, and steps were taken, in line with
Sankey's advice, to appoint an experienced hydraulic engineer to deal
with water supply problems in Victoria.  George Gordon
\index{Gordon, G.} was chosen for this work; he had been deputy chief
engineer of the Madras Irrigation and Canal Company before its
collapse.  It had been formed in 1863 to construct canals for
irrigation and navigation between the Tungabhadra River \index{India}
and the coast, but it failed to complete more then one section,
between Kurnool and Cuddapah in Andhra Pradesh.  In 1872 Gordon became
chief hydraulic engineer for the Victorian Board of Land and
Works.\fn{George Gordon papers, LaTL.}

Detailed proposals for greater use of irrigation in Tasmania
\index{Tasmania} received
attention by legislators in 1861.  In July 1860, the Governor, Sir
Henry Young, \index{Governor!Young} opened the parliamentary session
and announced the Government's intention to introduce legislation
concerning irrigation.  Isaac Sherwin, \index{Sherwin, I.} a prominent
irrigator and member of the House of Assembly, later moved for the
appointment of a Select Committee on legislation concerning
irrigation. The Committee met four times and its work was criticised
as slovenly by the
\textsl{Hobart Mercury}.  A fresh approach was made in the House of
Assembly during October by William Archer \index{Archer, W.} and
Dr~Officer, with the result that an Irrigation Commission was agreed
on and its members were appointed by the Governor.  These were Philip
Gell, \index{Gell, P.} who had worked on the irrigation survey of the
1840s, and William Langdon, \index{Langdon, W.} both from the
Legislative Council; William Archer, \index{Archer, W.} James
Maclanachan \index{Maclanachan, J.} and Dr~Robert Officer
\index{Officer, R.} from the
House of Assembly, together with Richard Dry, \index{Dry, R.}
erstwhile legislator, and the Surveyor-General, James
Erskine\index{Calder, J.\,E.}  Calder.\fn{\textsl{Hobart Mercury}, 10
Aug., 2 Oct., 31 Oct., 6 Nov.\ 1860.}

The Irrigation Commission's report advocated legislation declaring all
str\-eams and lakes to be public property, and included the
Surveyor-General's comments on proposals by Alessandro Martelli
\index{Martelli, A.} 
concerning Tasmanian irrigation, Major Cotton's \index{Cotton, A.}
reports in the 1840s, and three draft Bills dealing with irrigation.
Martelli's further report was also presented as Paper no.~43 at the
same time.  In it he provided proposals for greater use of irrigation
in addition to the fairly detailed statement, referred to above, of
existing irrigation in the Derwent
valley. \index{irrigation!scheme!Derwent valley} He detailed the cost
of works in the Derwent valley to provide for a Weasel Plains channel,
\index{Weasel Plains, Tas.}  irrigating 4000\,acres and supply water
to the town of Bothwell,
\index{Bothwell, Tas.} a High
Plains channel irrigating 6000\,acres and supplying the town of
Hamilton,
\index{Hamilton, Tas.} and a Derwent channel to irrigate 800\,acres to the
plains opposite New Norfolk.  \index{New Norfolk, Tas.} These works
together with improvements to existing irrigation on three larger
properties would enable almost 15\,000\,acres to be
irrigated.\fn{Tas.\ HA Paper no.\,43, Sep.\ 1861.}

The Irrigation Commission was followed by another select committee of
the House of Assembly, which was appointed in September 1861, with
William Archer \index{Archer, W.} as chairman, and presented its
report next month.  The committee recommended that the Government
should frame and introduce a Bill dealing with irrigation. It proposed
23 provisions to be included in legislation, including streams and
lakes to be declared public property, the power to proclaim any
portion of the colony as an Irrigation District, and the appointment
of elected trustees to manage irrigation in each District.\fn{Tas.\ HA
Paper no.\,112 of 1861.}

Despite the prolonged interest of Tasmanian legislators in irrigation
and the obvious concern of several to introduce enabling legislation,
no statute dealing with this matter was effected until the Irrigation
and Drainage Act of 1868.  This provided only the legal framework to
sanction the installation of drains or irrigation channels by a
landowner across the land of a
neighbour.\fn{\citet[p.\,143]{masoncox1994}.}

The increase in the population of Sydney \index{Sydney, NSW} led to
the substitution of one water supply scheme for another: at first the
Tank Stream, then Busby's Bore, to be followed by reliance on Botany
Swamps.  During the 1860s parliamentary committees considered problems
of the water supply and in 1867 a Royal Commission was appointed to
make a full inquiry into the water supply for Sydney and suburbs.  It
found the existing supply inadequate and recommended that the Upper
Nepean River \index{river!Nepean} should be the city's principal water
source.  The possibility of a dual use for the recommended scheme was
shown in the Commission's finding that it would allow for irrigation
near the city.  \index{irrigation!scheme!Nepean river} It was not till
1880 that construction began on this project; it was extended
later.\fn{\citet[pp.\,133-36]{lloyd1988}.}

\section*{The Victorian Canal Scheme}\index{canal}

Northern Victoria was the scene of the first effort to irrigate a
large tract of inland Australia.  The proposal in 1871 of a canal drew
attention to a way of using local water resources for irrigation and
transport in a relatively dry, unproductive, and sparsely populated
part of the colony.  Its extensive northern plains stretch from a
chain of uplands to the Murray River. \index{river!Murray} They are
crossed by several streams, perennial at the east where they drain
extensive mountainous terrain but intermittent in the west where some
fail to reach the Murray River.

The entry of pastoralists \index{pastoralists} into northern Victoria
completed the occupation by squatters \index{squatters} of all land in
the colony suitable for grazing.  They were soon issued with licences
for temporary leases, uncontested by other migrants until the decline
of alluvial gold-mining in the uplands made many miners look forward
to settling on the land.  To meet this demand, Victorian legislation
on land tenure provided for the selection of 640-acre holdings, but
many pastoralists found ways of using its provisions to acquire
freehold title to large areas.  Most of the arable land with good
rainfall in south-western Victoria thus continued in pastoral
use. Later, legislation with the requirement of residence by selectors
on blocks of 320\,acres facilitated agricultural settlement on crown
land, though at that stage the bulk of arable land still held by the
crown was extensive only in northern Victoria.

By 1870 selectors were moving north into the former pastoral runs.
Transport facilities were generally inadequate; the only railway was
the line from Melbourne to Bendigo \index{Bendigo, Vic.} thence along
the Campaspe River \index{river!Campaspe} to Echuca.  \index{Echuca,
Vic.}  Settlement along this stream gave rise to the towns of
Rochester \index{Rochester, Vic.} and Elmore, \index{Elmore, Vic.}
with Bendigo as the virtual capital of the region.

Canals had previously been suggested for different parts of Australia:
for inland irrigation or transport and for shipping near the coast.
The early proposals were inspired by widespread use of canals in
Europe before the success of railways.  Benjamin Hawkins Dods
\index{Dods, B.\,H.} (1834--1896) initiated the Victorian canal
scheme.  He came to Australia from Scotland in 1849, worked on the
Bendigo goldfields in 1851, supplied hydraulic equipment from his
Melbourne Pump Warehouse in the early 1860s, served the Department of
Mines later in its water supply office, and in 1890 described himself
as an engineer.  He envisaged a canal taking water westward to the
Mallee country
\index{Mallee district} from the Goulburn River. \index{river!Goulburn} 
That perennial stream, wide and deep enough to require ferries at
crossing places, was used by paddle steamers in the 1870s and 1880s to
carry wool and wheat down to Echuca \index{Echuca, Vic.} on the
Murray. \index{river!Murray} Occasionally it generated floods, the
record being in 1870\,---\,probably significant for Dods' proposal a
year later.  The proposed canal would be more than 200 miles long; it
would supply water to irrigate six million acres and provide a means
of transport.  Construction would be undertaken by a company requiring
a free grant from the government of millions of acres as the basis for
its profit.\fn{\citet[p.\,214]{flett1970};
\citet[p.\,50]{buxton1967}.}

Dods and three others applied to the Minister of Lands, J.~McPherson,
\index{McPherson, J.}
in May 1871, before widespread alienation of crown land had occurred,
for a grant or lease for 999 years of three million acres with a
frontage to a canal 60\,feet wide extending from the Goulburn River
\index{river!Goulburn} westward to lakes in the Mallee
district. \index{Mallee district} Their application was made under a
section of the Land Act 1869 which allowed for the granting of leases
of crown land `to any person willing to make and construct canals'.
The three men joined with Dods in the application included William
McCulloch, \index{McCulloch, W.} owner of the largest company carrying
freight by road and river in northern Victoria, and Hugh Parker,
\index{Parker, H.}
partner of the woolbroker Richard Goldsborough.  \index{Goldsborough,
R.} The Minister was promptly advised by his departmental secretary,
Clement Hodgkinson, \index{Hodgkinson, C.} a surveyor and engineer,
that the scheme had merit and warranted financial support but not to
the extent of the very large grant of land being sought.  McPherson
refused the application a few weeks later, just before the collapse of
the Duffy Ministry (Apr.\ 1870 to 19 June~1871).

A month later, the promoters put their case to the new Minister for
Lands and Survey; he received their deputation but failed to make a
favourable response.  The scheme was then publicised in an amended
form with the aim of forming the Grand Victorian North-western Canal
Company.  \index{Grand Victorian North-western Canal Co.} As far as
irrigation was concerned, the prospectus of the company indicated that
the northern plains would receive water for cultivated crops and for
intensified livestock production.  The land was regarded by the
promoter as suitable for producing many commodities including maize,
cotton, sugar, wine, oil, opium, oranges, chicory, wheat, oats,
barley, tobacco, and fruit.\fn{\citet[p.\,23]{martin1955};
J.\,H.~McColl, Vict.\ Hist.\ Mag., vol.\,V, 1917, p.\,145;
\citet[p.\,87]{powell1989}; J.\,N.~Churchyard, Aqua, vol.\,7, 1956,
p.\,268.}

Efforts to win support for the canal scheme suffered a setback when
the visiting hydraulic engineer, Lieutenant-Colonel Sankey,
\index{Sankey, R.\,H.}  offered
his opinion.  While the purpose of his visit was to give advice on the
Coliban and Geelong schemes for water conservation, he had been
invited to comment on other aspects of water supply in the colony.
Sankey found that a canal supplying water to irrigate six million
acres would need to be wider than any existing irrigation canal in the
world.  His scornful criticism caused Dods to sue for defamation,
claiming \pounds2000 but ultimately settling for \pounds40 just before
Sankey's departure from the colony in September 1871. Sankey's opinion
led to some withdrawal of support for the scheme and as the
agricultural use of northern Victoria was favoured by good rains for
several years, interest in irrigation declined and the scheme
languished until later.  Meanwhile, railways \index{railways} were
being extended into parts of northern Victoria.\fn{VicPP no.\,48 of
1871, Appendix~28.}

A meeting of 300 people at Bendigo in September 1874 almost
unanimously endorsed further action on the scheme.  Hugh McColl
\index{McColl, H.} 
(1819--1885) became secretary of the company and campaigned vigorously
for it.  McColl came to Australia in 1853, where he worked on
production of two short-lived Melbourne newspapers before moving to
Bendigo and becoming in turn bookseller, commercial traveller, and
mining manager.  He had also been involved with campaigns for improved
water supply for Bendigo, including the Coliban scheme.  Before the
end of 1874 a deputation of six, including Dods and McColl, was
introduced to the Minister for Lands, J.\,J.~Casey \index{Casey,
J.\,J.} of Bendigo, by another local parliamentarian, Robert
Burrowes. \index{Burrowes, R.}  The Minister agreed to look into the
matter.  By this time the scheme had become more ambitious: the main
canal was to be joined in the Wimmera by another running south to
Portland on the coast, and there was a suggestion of another canal
directed northwesterly far into South Australia.  None of these
variations made any difference to the essential dependence of the
canal on supply of water from the Goulburn River.  Support for the
scheme came from rural centres likely to benefit from the scheme,
notably Rochester on the Campaspe River.

The company made strenuous efforts in 1875 to get government
ag\-reement to a flying survey of the intended route for the canal.
The proposal was put to George Gordon, \index{Gordon, G.} then chief
hydraulic engineer in Victoria, only to receive an adverse response.
He was induced to pay a visit to the route of the canal but abandoned
the visit after seeing only a short portion of it.  Gordon prepared a
memorandum on the practicability, utility, and probable financial
success of the scheme, with the conclusion that `if the scheme were
reduced so as not to exceed probable feasibility, it would still be
too costly to be entertained at present'.\fn{VicPP no.\,20 of
1880--81, vol.\,2, Northwestern Canal.}

Hugh McColl persevered with the canal scheme, supported by many in
Bendigo, including parliamentarians and civic leaders.  His experience
as a bookseller may have helped in acquiring publications dealing with
Californian \index{California} canal irrigation by private companies.
One of the first was the San Joaquin and King's River Canal and
Irrigation Company, incorporated in 1871 at San Francisco and
providing water to 16\,000\,acres two years later.  One of McColl's
numerous statements indicates he was familiar with the US~Congress
report of 1874 on the irrigation of the San Joaquin, Tulare, and
Sacramento valleys.  McColl also was able to call on Gustav Thureau
\index{Thureau, G.} to
collect pertinent information during his visit in 1877 to the USA on
behalf of Bendigo mining interests.\fn{D.~Worster, Agric.\ History,
vol.\,56, 1982, p.\,503--509.; \citet[p.\,78]{blainey1963};
\textsl{Australasian} 2 Mar.\ 1878, p.\,282, letter from G.~Thureau.}

Following the onset of relatively dry conditions in 1876, and revival
of interest in the canal scheme, the company was allowed to go ahead
and the survey was made in 1878.  There was no positive outcome from
this work but Dods and McColl continued to press for the
scheme.\fn{J.\,H.~McColl, Vict.\ Hist.\ Mag., vol.\,V, 1917, p.\,156.}

Meanwhile, George Gordon along with many other public servants was
dismissed in 1878 ostensibly as a measure of economy.  His efforts to
gain reinstatement were fruitless but he had the opportunity in 1879
to make public his own views in statements on water supply and
irrigation.  His Ballarat lecture on irrigation addressed many aspects
relevant to conditions in northern Victoria.\fn{G.~Gordon,
Lecture on Irrigation and Drainage, Ballarat, 1878.}

Government indifference to the problems of water supply and irrigation
was turned to personal advantage for McColl in 1880 when his fourth
attempt to enter Parliament was successful with his election for the
Mandurang electorate in northern Victoria.  This victory gave him the
opportunity of persistently taking up issues relating to water supply
and irrigation, never missing an opportunity to advocate the canal
project.  His major success was to win support for schemes of
irrigation in northern Victoria, with involvement of canals filled
with water from the Goulburn River. \index{river!Goulburn}

The great significance of the campaign for the canal scheme was that
it enlisted support from many people in Victoria and developed a
strong lobbying group which pressed for introduction of irrigation
schemes based on diversion of river water.

\section*{The Legislatures}

Establishment of representative government in most Australian colonies
during the 1850s involved a franchise limited by certain requirements
as to income, education, and land ownership.  Payment of elected
legislators was not undertaken before 1870 in Victoria and later in
other colonies, so many legislators were prominent landowners.  In
Tasmania several of these gentlemen had satisfying results from
irrigation before 1850 and were able to secure legislative attention
to the matter.

South Australian \index{South Australia} legislators gave attention to
irrigation in 1860 when an Irrigation Bill was introduced by the first
Reynolds ministry in Sept\-em\-ber.  At this time the Premier, Thomas
Reynolds, \index{Reynolds, T.} was an Adelaide resident and
manufacturer of jam.  He had a garden in the suburb of Mitcham
\index{Mitcham, SA} which
he later proposed to irrigate.  The year 1860 was abnormally dry in
the Adelaide district; this may have been a reason for the
introduction of a Bill on irrigation.  This Bill was intended to
provide for the creation of irrigation districts following petitions
by landowners and subsequent establishment of an irrigation board in
an irrigation district. Each board would be elected by district
landowners and would regulate irrigation in the district.  No
provision was made for crown ownership of rivers or variation of
existing riparian rights.  The Government found that its proposal was
not favoured by the majority of members of the House of Assembly and
it withdrew the Bill before its second
reading.\fn{\citet[p.\,174]{ward1862}.}

Victorian \index{Victoria} legislators were not confronted with
proposals concerning irrigation in this period.  A related issue of
water rights was important during the period of alluvial mining and
Peter Wright, \index{Wright, P.} MLA for Ovens, moved unsuccessfully
in 1862 to introduce a private Bill for better regulation of water
rights in the Beechworth mining district.  This action was regarded by
L.\,R.~East \index{East, L.\,R.} as a proposal `to make the waters of
Victorian streams public property for the people of Victoria' and thus
significant for development of irrigation in that State
Parliament.\fn{VicPD, 1862--63, Vol.\,IX, p.\,99; L.\,R.~East, Aqua,
Aug.\,1958, July 1965.}

In New South Wales \index{New South Wales} irrigation failed to become
an issue warranting special consideration by Parliament but it was not
completely ignored.  A parliamentary Select Committee in 1858
considered navigation on the Murray River system.  In response to a
proposal for irrigation made in London, that no land in the Murray
Valley should be alienated until some general plan for irrigation was
drafted, the committee found that irrigation was not required for many
years, when it could be undertaken by private proprietors, not by the
Government.  There was also the attention, mentioned above, given by a
Royal Commission to possible use of Sydney's water supply for
irrigation.\fn{\citet[p.\,87, pp.\,135--36]{lloyd1988}.}


\closure
Records of irrigation in three colonies on the mainland show that more
than a score of landholders were applying water to a variety of crops,
though mainly on a small scale.  It is likely that a much larger
number of unrecorded irrigators, principally Chinese, were active in
Victoria and New South Wales.  Water was distributed by gravity from
springs and weirs, or pumped from rivers and lakes by human labour,
horse power, wind power, and\,---\,exceptionally\,---\,by steam
engines.  The total area under irrigation increased in the period from
virtually nothing in South Australia, New South Wales, and Victoria to
at least 1000\,acres, the largest single area being Bath's 500 acres
near Lake Learmonth in Victoria.  By contrast the Tasmanian use of
irrigation was initially quite extensive\,---\,involving hundreds of
acres\,---\,and even at the close of the period it was comparable with
that of the entire mainland. Schemes were launched for government
development of irrigation in Victoria by water conservation in the
central highlands, and for an ambitious private development of
irrigation in association with an extensive canal.

%\section*{References}
%1. Comm.Bur.Census \& Statistics, Demography Bull. No. 67, 1949.
%2. W.Howitt, Land,Labour and Gold, 1855/1972, p.99.
%3. R.B.Smyth, The Goldfields And Mineral Districts Of Victoria,   
%     1869/1979,  p.397-409.   	           
%4. R.B.Smyth, 1869/1979.
%5. Vic Yearbook 1973, p.215,  \& C.J.Lloyd, Either Drought Or Plenty,
%    1988,p.79.
%6. Vic Yearbook 1973, p.161.
%7. Comm.Bur.Census and Statistics, Demography Bull. No. 67, 1949.
%8. Jean Gittins, The Diggers From China,1981, p.75.
%9. S.H.Roberts, History Of Australian Land Settlement 1788-1920, 1924,
%    Fig 42.
%10. W.Howitt, 1855/1972.
%11. VicYearbook, 1973, p.1090.
%12. Lynnette J.Peel, Rural Industry In The Port Phillip Region 1835-1880,
%      1974.		       
%13. TasPP No.43 of 1861.
%14. TasPP No.42 of 1861.
%15. Australasian, 3/11/1883, Irrigation In Tasmania, by 'Bruni'.
%16. Margaret Mason-Cox, Lifeblood of a Colony, 1994, p.28.
%17. Hobart Mercury, 2/3/1874, cited by Margaret Mason-Cox, p.74.
%18. J.B.Hirst, Adelaide And The Country 1870-1917, 1973, p.227.
%19. Elizabeth Warburton, The Paddocks Beneath, 1981, p.26.
%20. J.W.Warburton, Five Creeks Of The River Torrens, 1977, p.30.
%21. E.Ward, The Vineyards And Orchards Of South Australia, 1862, p.28.
%22. Adelaide Observer, 21/7/1866, J.W.Warburton, 1977, p.64.
%23. E.Ward, 1862, p.12.
%24. E.Ward, 1862, p.77.
%25. E.Ward, 1862, p.75, \& K.Preiss \&Pamela Oborn, The Torrens Park
%      Estate, 1991.
%26. E.Ward, 1862, p.50.
%27. E.Ward, 1862, p.55, \& p.24.
%28. E.Ward, 1862, p.15.
%29. Sue Barker(ed), Explore The Barossa, 1991, p.73.
%30. SAWeekly Chronicle 24/8/1876.
%31. W.B.Smith, Centenary History Of Bleasdale, Langhornes Creek, S.A.
%       1950.
%32. J.R.Aust.Hist.Soc., vol.34, 1948, p.387.
%33. Australasian, 23/9/1871, Irrigation, by T.Bath, p.409.
%34. Personal communication, C.Mort, 1992.
%35. J.Ann Hone, Officer, C.M. and S.H.,ADB vol.5, p.357.
%36. G.K.Chapman, Vict.Hist.Mag., vol.22, 1947, p.1 \& Vic
%      R.C.Veg.Products,1887,4th Prog.Rept, p.109.
%37. VicPP No.19 of 1896, R.C.Mildura, MoE, p.16.
%38. A.Massola, Aboriginal Mission Stations In Victoria, 1970, p.13.
%39. Vic R.C.Veg.Products, 8th Prog.Rept, pp1-3.
%40. A.Massola, 1970.
%41. A.Massola, 1970.
%42. Ernestine Hill, Water Into Gold, 1937.
%43. I.Mudie, Riverboats, 1965, p.233.
%44. Procs First Vic Irrig.Conf. 1890.
%45. Aqua, vol.10, 1959, p.189, vol.13, p.152.
%46. R.V.Billis \& A.S.Kenyon, Pastoral Pioneers Of Port Phillip, 1974,
%       p.170, \& Procs First Vic Irrig.Conf. 1890.
%47. F.Acheson, Collection And Storage Of Water In Victoria, in
%      Victorian Government Prize Essays 1861.
%48.  J.Young, Irrigation, A Lecture Delivered In The Mechanics'
%       Institute, Bacchus Marsh, 28 Sept.1866.
%49. Trans.Phil.Inst.Vict., vol.4, 1860.
%50. Kathleen Thomson \& G.Serle, Biographical Register of Victorian
%       Legislature, 1859-1900, 1972,p.174.       		    
%51. Victorian Farmers Journal and Gardeners' Friend, 7/7/1860.
%52. Victorian Farmers Journal and Gardeners' Friend, 6/10/1860.
%53. W.S.Benwell, Journey To Wine In Victoria, 1960, p.129.
%54. Mary Healy, Railways And Pastures: The Australian O'Keefes, 1990,
%      p.27.
%55. Mary Healy, 1990, p.27.
%56. VicPP No.48 of 1871, Appendix 28.
%57. P.L.Brown, ADB,vol.2, p.100.
%58. A.S.Kenyon, J.Agric.Vict.,vol.10, 1912,pp. 658-661. Footnote.
%59. Australasian,  23/9/1871, p.409.
%60. VicPP No. 71, 1859-60. Fifth Ann.Rept Central Board of Health,
%      VicV \& P, LA vol 4.				
%61. M.Cannon, Australia In The Victorian Age, I. Who's Master? Who's
%      Man, 1974, \& and Ill.Aust.News, 1/1/1869.
%62. E.C.Booth, Another England, 1869 p.185.
%63. E.C.Booth, Australia In The 1870s, 1873-76/1975, pp. 31-32.
%64. Melbourne Argus, 1/12/1856.
%65. Trans Phil.Inst.Vict. vol.1.1857, Proc. p.xxvi.
%66. Trans Phil.Inst.Vict. vol.4, 1860, Rept. Resources Colony Vict.
%67. The Victorian Government Prize Essays 1860, 1861.
%68. Hobart Mercury, 10/8/1860.
%69. Hobart Town Courier, 7/8/1857.
%70. G.R.Quaife, ADB,vol.6, p.218.
%71. VicPP No. 48 of 1871.
%72. J.M.Powell, Watering The Garden State, 1989, quotation pp. 81-82.
%73. VicPP No 48 of 1871, appendix No.28, p.113.
%74. SAPP No.97 of 1871.
%75. George Gordon papers, La TL.
%76. Hobart Mercury, 10/8/1860.
%77. Hobart Mercury, 2/10/1860. 31/10/1860, 6/11/1860.
%78. Tas HA Paper no.42, Sept. 1861.
%79. Tas HA Paper No.43.
%80. Tas HA Paper no.112 of 1861.
%81. Margaret Mason-Cox, 1994, p.143.
%82. C.J.Lloyd, Either Drought or Plenty, 1988, p.133.
%83. C.J.Lloyd, 1988, pp135-136.
%84. J.Flett, The History Of Gold Discovery In Victoria, 1970, p.214,
%      and G.L.Buxton, The Riverina, 1861-1891, 1967, p.50.
%85. C.S.Martin, Irrigation And Closer Settlement In The Shepparton 
%       District, 1836-1906, 1955, p.23.	     	 
%86. J.H.McColl, Vict.Hist.Mag.,vol.v,1917,p.145.
%87. J.M.Powell, Watering the Garden State, 1989, p.87.
%88. J.N.Churchyard, Aqua, vol.7, 1956, p.268.
%89. VicPP No. 48 of 1871, Appendix 28.
%90. VicPP No.20 of 1880-81, vol.2, Northwestern Canal.
%91. D.Worster, Agric.History, vol.56, 1982, p.503.
%92. D.Worster, 1982, p.509.
%93. G.Blainey, The Rush That Never Ended,1961, p.78, \& Australasian
%      2/3/1878, p.282, letter from G.Thureau.
%94. J.H.McColl, 1917, p.156.
%95. G.Gordon, Lecture On Irrigation And Drainage, Ballarat, 1878.
%96. E.Ward, 1862, p.174.
%97. VicPD, 1862-63, Vol.IX, p.99.
%98. L.R.East, Aqua, Aug.1958, July 1965.
%99. C.J.Lloyd, 1988, p.87.


