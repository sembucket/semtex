% $Id$
% CHAPTER SIX
% 6631 wds at 30/4/99

\setcounter{endnote}{0}
 
\chapter{Irrigation by Independent Producers from 1880}
\label{ch:indep}\addtoendnotes{\protect\subsection*{Chapter \thechapter}}
%\markboth{Chapter \thechapter. Independent Producers}%
%{Pioneering Irrigation in Australia}

\fancyhead[RE]{\sffamily \small Chapter \thechapter.\ %
               Independent Producers}

At the beginning of this period irrigation in Australia was
characteristically undertaken independently by private individuals or
companies, without involvement in a definite irrigation scheme.  In
Tasmania a few landholders on a tributary of the Macquarie River
continued to use water from Tooms Lake to irrigate their lands by
mutual arrangement, the essence of a group scheme.

The independence of private irrigators was reduced later in those
colonies where state control of water resources was established and
semi-government bodies were formed to manage irrigation in specific
areas.

Private irrigators, who were numerous and culturally diverse, provided
various commodities.  Vegetables, fruit, tobacco, and hops were
produced on small holdings with some use of irrigation by large
numbers of growers, often Chinese.  Cereals, fodder, and sugar cane
were watered almost exclusively by people of European origin,
generally on farms, pastoral stations, or plantations.  Methods of
irrigation were most varied, ranging from laborious hand-watering to
use of mechanical equipment driven by engines in situations requiring
supplies to be lifted, and including flood irrigation where water
could gravitate from a stream, as in Tasmania.

\section*{Cereals}\index{cereals}

Agriculturalists were generally less involved with irrigation. Their
ma\-in crops were cereals adapted to dry-farming, dependent on
relative abundance of rain in winter and spring; they cultivated
comparatively small holdings generally lacking surface water resources
in abundance; and few had time or money to use irrigation.  However,
some farmers in northern Victoria used water from adjacent river
channels for their wh\-eat crops.  Woodford Patchell, \index{Patchell,
W.} a pioneer of irrigation in the region, pumped water from the
Loddon River
\index{river!Loddon} at Kerang \index{Kerang, Vic.} 
to irrigate more than 100\,acres of mixed crops in 1882. John Garden
\index{Garden, J.} 
began pumping from Barr Creek, \index{creek!Barr} one of several
channels near the Murray, to water 300\,acres of cereals near
Cohuna.\index{Cohuna, Vic.} He demonstrated his equipment in November 1882
to 150 people from different parts of the northern plains, including
H.\,R.~Williams \index{Williams, H.\,R.} and \index{McColl, H.} Hugh
McColl\,---\,parliamentarians elected from the region.\fn{NSW VP
1886, RC~Water Conservation, 2nd Rept, MoE Appendix~I, McKinney
pp.\,17--26; \textit{Australasian}, 18 Nov.\ 1882.}

That year marks the beginning of extended use of irrigation in
northern Victoria, which followed the failure of district crops in
1881. Others who took up irrigation about the same time in this part
of the colony included William Webb \index{Webb, W.} of Rochester,
whose pumping plant installed in 1883 was regarded as one of the first
in Victoria; Duncan Leitch, \index{Leitch, D.} whose supply from
Gunbower Creek \index{creek!Gunbower} depended on an inflow from the
Murray River; \index{river!Murray} Mr~Holding who irrigated 200\,acres
of wheat near Gunbower; and David Chrystal, \index{Chrystal, D.} who
irrigated grass and cereals at Torrumbarry \index{Torrumbarry, Vic.}
by pumping from the Murray. The facilities for irrigation with water
from the Murray River led to the watering of nearly 5\,000\,acres of
wheat by scores of irrigators in the Swan Hill \index{Swan Hill, Vic.}
shire in 1883--84.\fn{\textit{Australasian}, 7 Jan.\ 1882;
\textit{J.\,Inst.\ Eng.\ Aust}.\ vol.\,2, (1930), p.\,270; Vic.\ PP
no.\,53 of 1885; MoE p.\,37;
\textit{Leader}, 1 Dec.\ 1883; Vic.\ PP no.\,53 of 1885, MoE p.\,32;
Vic.\ PP vol.\,4, 1884; textit{Vic.\ Yearbook 1884}, p.\,436.}

\section*{Fodder}\index{fodder}

Many pastoralists \index{pastoralists} in south-eastern Australia made
use of irrigation.  Their holdings had been chosen with an eye to a
secure supply of water; in many cases much more than the essential
stock and domestic supply.  The chief interest of pastoralists in
irrigation was to improve their capacity to carry livestock either by
watering natural pasture in dry seasons or fodder crops such as
lucerne or cereal hay.

In the early part of this period there was greater use of irrigation
for fodder crops in places where a start had been made previously.
One of these was Bacchus Marsh \index{Bacchus Marsh, Vic.} in central
Victoria, where in 1883 Michael O'Connell, \index{O'Connell, M.}  one
of several local irrigators, was diverting water by a race from the
Lerderderg River \index{river!Lerderderg} to irrigate 70\,acres of
lucerne. In 1887 there were six steam plants in use for irrigation in
the locality.  Irrigators included Mr~Crook, who began using supplies
from the Lerderderg River in 1878, and W.~Grant \index{Grant, W.} who
irrigated 12\,acres of lucerne in 1885 using a centrifugal pump
\index{pump!centrifugal} and steam \index{pump!steam-driven}
engine.  In the Victorian Gippsland \index{Gippsland, Vic.} district,
the earlier use of water from springs near Sale \index{Sale, Vic.}
apparently prompted wider practice of irrigation at a time of great
demand for supplies to goldfields in the region.  At Charlcote,
\index{Charlcote, Vic.} near Sale, W.\,H.~Palmer pumped water from the Avon
river
\index{river!Avon} to irrigate 300\,acres of grassland.\fn{Vic.\ PP no.\,53
of 1885, p.\,xxxvii--viii, MoE p.\,271;
\textit{Australasian}, 1 Jan.\ 1887;
\cite{williams1936}; \cite[p.\,76]{priestley1984}.}

Western Victorian experience of irrigation was associated with
str\-eams from the Gramp\-ians \index{Grampians mountains} mountains.
In 1886, Samuel Carter \index{Carter, S.} pump\-ed water from the
Wimmera River
\index{river!Wimmera} to irrigate natural pasture and lucerne on
Walmer pastoral \index{station!Walmer} station near
Horsham. \index{Horsham, Vic.} Previously his family held Glenisla
station
\index{station!Glenisla} on the headwaters of the Glenelg River,
\index{river!Glenelg} south of Horsham, where he first gained
experience of irrigation.\fn{\textit{Australasian}, 9 Jan.\ 1886.}

Major developments occurred in the riverine plain stretching from
northern Victoria to the Lachlan River \index{river!Lachlan} in New
South Wales.  Irrigation on the Murray Downs pastoral station
\index{station!Murray Downs} on the
Murray River \index{river!Murray} in New South Wales was extended to a
further 40\,acres, at least partly to produce suitable fodder for
ostrich farming on the station in 1882. Near Echuca, \index{Echuca,
Vic.} a major port for river-boats on the Murray, there were
opportunities for irrigation by pumping from both sides of the river.
On Perricoota station \index{station!Perricoota} on the northern side,
the firm of Robertson~\& Wagner used irrigation to maintain the
homestead garden and to produce cereal hay for horses held there for
use on its well known Cobb~\&~Co \index{Cobb \& Co.}  coach-lines. The
powerful engines and pumps used along this part of the Murray to
irrigate large areas replaced the horse-powered pumps,
\index{pump!horse-powered} windmills, and hydraulic rams used widely
in the past.\fn{\textit{Australasian}, 23 Dec.\ 1882; \textit{Town and
Country Journal}, 6 Aug.\ 1887.}

North of the Murray there was scattered use of irrigation.  The Werai
pastoral station \index{station!Werai} established north-east of
Wakool \index{Wakool, NSW} by the Gwynne family in 1843 was one of the
earliest scenes of irrigation in the region, using water pumped from
Colligen Creek, \index{creek!Colligen} an effluent of the Murrumbidgee
River.  \index{river!Murrumbidgee} Production there of lucerne
\index{lucerne} and
other crops became well known before 1886.  George Mair \index{Mair,
G.} began irrigating fodder crops at Groongal, \index{Groongal, NSW}
on the north side of the Murrumbidgee, about 1882, and in 1887 had a
good fodder crop on 40\,acres, watered by pumping from the river.
James Tyson
\index{Tyson, J.} 
took advantage of floods on the lower Lachlan River
\index{river!Lachlan} by cutting
channels through the river bank and diverting water into adjacent
depressions to give better growth of herbage for his sheep. These
achievements were reported in 1885.  By 1890 there was irrigation by
Phillips~\& Graves at Warbrecan, \index{station!Warbrecan} by David McCaughey
\index{McCaughey, D.}
at Coree, \index{station!Coree} and by Patrick McFarland
\index{McFarland, P.}  at Barooga, \index{station!Barooga} where water
was pumped from the Murray to 100\,acres of lucerne.  After
legislation in 1896 for State control of water rights in New South
Wales, licenses were granted for use of water by private irrigators,
the outstanding use being that of Samuel McCaughey \index{McCaughey,
S.} who in 1900 developed irrigation of fodder and pastures at North
Yanco.\fn{Vic.\ PD 1886, p.\,2763; NSW VP 118 of 1885; NSW RC~Water
Conservation First Rept 1885--86;
\cite[p.\,53]{lloyd1988}; \cite[p.\,246]{buxton1967};
\textit{Australasian}, 24 Jan.\ 1890.} \index{North Yanco estate}

In the valley of the Lachlan River \index{river!Lachlan} pastoralists
began irrigation in the 1880s.  J.\,B.~Donkin, \index{Donkin, J.\,B.}
who held the Lake Cowal station,
\index{station!Lake Cowal} began
irrigating near the Lachlan about 1885.  Messrs Raymond~\& Nicholas
raised water 32\,feet from the river with a centrifugal pump
\index{pump!centrifugal} to irrigate
25\,acres of maize and five acres of potatoes near
Forbes. \index{Forbes, NSW} At Jemmalong, \index{station!Jemmalong}
also near Forbes, N.\,A.~Gatenby
\index{Gatenby, N.\,A.} 
irrigated lucerne from the Lachlan; he gained a government prize
\index{prize!NSW govt} in
1890 for farm irrigation.  Downstream near Lake Cargellico
\index{lake!Cargellico} a syndicate
put up a substantial dam on the Lachlan in 1885 for
irrigation.\fn{\textit{Procs First Vic.\ Irrig.\ Conf}.\ (1890), p.\,
130;
\textit{Leader}, 12 Sep.\ 1885; \textit{NSW Agr.\ Gaz}.\ vol.\,2, (1891),
p.\,163; \cite[p.\,118]{jeffcoat1988}.}

Near the south coast of New South Wales, the irrigation conducted on
the Kameruka \index{Kameruka, NSW} estate of Robert~L. Tooth
\index{Tooth, R.\,L.} was regarded so highly by judges of the
irrigated farms competition that in 1891 it was awarded first prize
\index{prize!NSW govt} for the best system of irrigation in the
colony.  On this farm, just south of Bega,
\index{Bega, NSW} 30\,acres of lucerne were irrigated with water drawn
 from Tantawanglo Creek \index{creek!Tantawanglo} by a pump driven by
a steam engine \index{pump!steam-driven} and raised more than
200\,feet to the top of a hill whence it flowed down through the
lucerne.  This property, where 1300 cows were milked daily in summer,
was then held to be one of the largest dairy farms in the
world.\fn{\textit{NSW Agr.\ Gaz}., vol.\,2, (1891), p.\,163 and
vol.\,3, (1892), p.\,601.}

Another entrant in the irrigated farms competition of 1891 was the
farm of T.\,P.~Wills-Allen \index{Will-Allen, T.\,P.}  in the Gunnedah
\index{Gunnedah, NSW} district of northern New South Wales.  His
introduction to irrigation had come as an involuntary user of waste
water since 1876 when he used a 30\,hp engine to pump water from the
Namoi River \index{river!Namoi} for a sheep-wash.  After its use for
this purpose, the water was allowed to flow over the land, a practice
followed each year until 1884 when Mr Wills-Allen was so converted to
irrigation that he watered 25\,acres of wheaten hay and a stand of
lucerne.\fn{NSW VP 1885, RC~Water Conservation, MoE, pp.\,160--164.}

Irrigation was used on the Darling Downs \index{Darling Downs, Qld} of
Queensland \index{Queensland} by at least two pastoralists in the
1880s.  On Kings Creek, \index{creek!Kings} about 40 miles south of
Toowoomba \index{Toowoomba, Qld} near the railway to Warwick, Atticus
Tooth, \index{Tooth, A.} a cousin of R.\,L.~Tooth involved with
irrigation at Kameruka, New South Wales, watered 22\,acres of black
cracking clay with supplies delivered by a steam-driven
pump. \index{pump!steam-driven} Then in 1888, Francis and Robert Gore
\index{Gore Bros}
began irrigation at Yandilla \index{Yandilla, NSW} on their
90\,000\,acres run on the Condamine River \index{river!Condamine} near
Pittsworth, \index{Pittsworth, Qld} south-west of
Too\-woom\-ba.\fn{\textit{Brisbane Courier}, 28 Mar.\ 1884;
\textit{Queenslander}, 6 Apr.\ 1889.}

In South Australia, J.\,L.~Thompson \index{Thompson, J.L.} at
Beefacres near Adelaide irrigated 12\,acres in 1881, including five
acres of fodder crops as well as fruit and vegetables.\fn{\textit{NSW
Agr.\ Gaz}., vol.\,10, (1899), p.\,802.}

\section*{Fruit}\index{fruit}

Notable production from irrigated orchards was achieved in the more
populous areas and in inland Australia before the establishment of
gr\-oup schemes for irrigated horticulture.

In 1888 a Victorian government prize \index{prize!Victorian govt} for
irrigated crops was awarded to David Milburn \index{Milburn, D.} of
Keilor who used water from the Maribyrnong River
\index{river!Maribyrnong} to irrigate an orchard \index{orchard} of 12
acres and a larger area of other crops.  Milburn was one the earliest
commercial irrigators in Victoria.  The Mason brothers established an
orchard north of Shepparton \index{Shepparton, Vic.} about 1884 and by
1886 they pumped from the Goulburn River \index{river!Goulburn} to
irrigate 70 acres of fruit trees.\fn{\textit{Australasian}, 3 Mar.\
1888, p.\,466;
\cite[p.\,57]{martin1955}.}

Another prize-winning irrigationist in Victoria was Robert Clark,
\index{Clark, R.}  a
miller and politician with fruit trees on his Riversdale property in
the Wimmera at Lower Norton, watered by pumping from Norton Creek.  In
1889 he was awarded a government prize.\fn{\textit{Australasian}, 10
May 1890, p.\,898;
\cite[p.\ xxiv]{powell1973}.}\index{prize!Victorian govt}

Several properties near Sydney were being irrigated in the last
dec\-ade of the century.  Irrigated orchards were entered for the New
South Wales government competition for irrigated farms in 1891.
\index{prize!NSW govt} First prize
for orchards went to Mr~T.~Brien \index{Brien, T.} of Parramatta
\index{Parramatta, NSW} who
irrigated 10\,acres of citrus; he pumped water from a creek with a
six-horsepower engine and used iron pipes to carry supplies all over
his orchard.  G.\,H.~Demp\-sey \index{Dempsey, G.\,H.} followed a
similar arrangement at Emu Plains, \index{Emu Plains, NSW} taking
water from a well with a Tangye pump \index{pump!Tangye} driven by a
two-horsepower Tangye engine and reticulating the water through iron
pipes to his citrus trees.\fn{\textit{NSW Agr.\ Gaz}., vol.\,3,
(1892), p.\,711.}

Near Echuca, \index{Echuca, Vic.} Daniel Matthews \index{Matthews, D.}
pumped water from the Murray Riv\-er \index{river!Murray} in the
mid-1880s to irrigate an orchard \index{orchard} in New South Wales on
the Maloga Mission \index{Maloga Mission, NSW} for Aborigines, which
he established on land selected from the Moira \index{Moira run, NSW}
run leased by John~O'Shanassy. \index{O'Shanassy, J.} On the Victorian
side, A.\,D.~Jeffrey \index{Jeffrey, A.\,D.} began irrigating fruit
trees in 1887 a few miles from Echuca, using a centrifugal pump to
take water from the river.\fn{\textit{Town and Country Journal}, 6
Aug.\ 1887;
\cite[p.\,155]{priestley1965}.}

In Western Australia there had been very little experience of
irrigation before 1890.\index{Western Australia} Discoveries of gold
at Kalgoorlie
\index{Kalgoorlie, WA} a few years
later led to a doubling of the colonial population within 10 years and
encouraged horticultural and agricultural industries.  In 1913 there
were more than 300 instances of small-scale irrigation of orchards and
vegetable gardens in the south-west part of the State, while by 1908
bananas \index{bananas} were being grown under irrigation by Edward
Angelo \index{Angelo, E.} on the Gascoyne River \index{river!Gascoyne}
at Carnarvon.\fn{\cite[p.\,11]{oldham1913};
\cite{bolton1979}.}\index{Carnavon, WA}

In South Australia \index{South Australia} most irrigation of fruit
trees was undertaken near Adelaide in the lower Torrens valley
\index{river!Torrens} where in 1892 there were more than 200 pumps
driven by wind, steam engines, and horse power to draw water for fruit
and vegetables.  After Robert Barr Smith \index{Barr Smith, R.}
acquired the Torrens Park \index{Torrens Park, SA} property near Mitcham,
\index{Mitcham, SA} he built a reservoir
\index{reservoir!Mitcham} in
1875 to water the existing orangery of 600 trees.  This reservoir with
a capacity of two million gallons could provide seven acre-feet of
water\,---\,a great improvement on the arrangement made by the
previous owner, R.\,R.~Torrens. \index{Torrens, R.\,R.} One instance
of irrigation for fruit growing under arid conditions was near Port
Augusta \index{Port Augusta, SA} where mains supply water from the
nearby Flinders Ranges \index{Flinders Ranges, SA} was used at
Stirling North \index{Stirling North, SA} for fruit and vegetables; in
1892 the orchards covered 100\,acres.  One of the reasons for
development of irrigation near Adelaide in the 1890s was to supply
Broken Hill
\index{Broken Hill, NSW} with fresh food; the expansion at Stirling North
was probably in response to the completion of the railway through the
Flinders Ranges to Oodnadatta.\fn{\cite[p.\,21]{green1892};
\cite[p.\,229]{priess1991};
\cite[p.\,62]{anderson1988}.}\index{Oodnadatta, SA}

Irrigation of citrus fruit and vegetables was practised at Bowen,
\index{Bowen, Qld} the earliest settlement in north Queensland, from the
1890s, using underground water on areas which exceeded 600\,acres
before 1920.\fn{H.\,E.\,A.~Eklund, `Irrigation in Queensland',
\textit{Qld Agric.~J}., vols 20~\& 21, (1923, 24).}

Fruit was produced by Chinese \index{Chinese} at several points along
the Darling River.  \index{river!Darling} At Wentworth,
\index{Wentworth, NSW} Chinese
gardens were reported by John Stanley James \index{James, J.\,S.}
(`The Vagabond') in 1885; he found `John's garden on the banks of the
Darling is the only green oasis in the place'. Two years later `the
splendid orangery in the neighbourhood of Wentworth carried on by
Chinamen' was mentioned in the South Australian parliament.  Citrus
sold in Mildura \index{Mildura, Vic.} in 1890 were grown at Wentworth
on trees planted about twenty years earlier\,---\,almost certainly by
Chinese who could not have sustained production in the arid climate
without irrigation from the Darling River. \index{river!Darling} At
Bourke, \index{Bourke, NSW} fruit and vegetables in sufficient
quantity to supply the town were raised by Tim Yang and a score of
Chinese helpers from an irrigated garden of three to four acres in
1885.  Bourke was then a major port on the Darling, where copper ore
from mines at Cobar,
\index{Cobar, NSW} a hundred miles south, was shipped to South
Australia.\fn{\textit{Melbourne Argus}, 10 Jan.\ 1884; SA PD 1887,
p.\,310; NSW VP 1886, RC~Water Conservation, 1st Rept, MoE p.\,223;
\textit{Queenslander}, 25 Oct.\ 1890.}

\section*{Hops} \index{hops}

Hop gardens were irrigated at several places in Tasmania and Victoria.
During the boom in hop production in the 1880s there were reports of
irrigation from different parts of Victoria, but in Tasmania the
expansion was mainly in the Derwent valley, \index{river!Derwent}
where its use became common following trials at New Norfolk \index{New
Norfolk, Tas.} much earlier.  In Victoria hops were grown most
extensively in Gippsland \index{Gippsland, Vic.} at first, but
irrigation was not common there in 1882. In 1883--84 there were five
growers irrigating 84\,acres near Bairnsdale.  \index{Bairnsdale,
Vic.} One instance was the hop-garden of J.\,A.~Taylor \index{Taylor,
J.\,A.}  near Bairnsdale, where in 1885 water was lifted 20 feet from
the Mitchell River \index{river!Mitchell} by steam power
\index{pump!steam-driven}to irrigate 50~acres.\fn{\textit{Australasian}, May
1882, p.\,538; Vic.\ PP no.\,53 of 1885, RC~Water Supply, Further Progr.\
Rept MoE, p.\,255.}

Hop production increased along the Ovens valley \index{river!Ovens} in
the north-east, where about 40 growers irrigated almost 300\,acres in
1883--84. One of the most prominent growers was Hiram Crawford
\index{Crawford, H.} who
used two waterwheels on the Ovens River to irrigate 50\,acres. This
was an enterprise undertaken after his retirement to the district in
1876 after a busy life as gold-digger, coach-line proprietor, and
manager of a shopping arcade in Melbourne.  W.~Lyons \index{Lyons, W.}
of Everton \index{Everton, Vic.} also used a waterwheel for
irrigation.  Another Victorian area where hops were irrigated was in
the Yarra valley \index{river!Yarra} near Healesville.\fn{Vic.\ PP 1884,
Statistical Register;
\cite[p.\,91]{pearce1976};
\cite[p.\,118]{woods1985}; Vic.\ RC~Vegetable Products, 5th Progr.\
Rept 1888, Q8711;
\textit{Australasian}, 18 Mar.\ 1882.}\index{Healesville, Vic.} 

The Chinese \index{Chinese} were involved with hop-gardens in
north-eastern Victoria during the 1880s, though mainly as labourers
for Europeans. The achievement of the Panlook brothers \index{Panlook
Bros.} in the Eurobin
\index{Eurobin, Vic.} 
district as hop-growers dates back to 1890 when four sons of a former
storekeeper on the Buckland goldfields returned to the district and
started growing tobacco and hops.  Within a few years the brothers had
a substantial area under hops which they irrigated. Their farm became
the outstanding Victorian source of hops.\fn{Vic.\ RC~Vegetable
Products, 5th Progr.\ Rept, 1888, Q8699; \cite[p.\,127]{carter1968};
\cite[p.\,106]{robertson1973}.}

Hops were cultivated in some South Australian localities with moderate
rainfall, as in the Adelaide hills and near Mount Gambier. \index{Mount
Gambier, SA} The only record of irrigation refers to David Murray's
\index{Murray, D.} 
Rockford Estate on the Onkaparinga River \index{river!Onkaparinga}
near Mylor, \index{Mylor, SA} where 10\,acres of hops were cultivated
in 1896.\fn{\cite[p.\,70]{pearce1976};
\cite{oneill1974}.}

\section*{Sugar cane} \index{sugar cane}

Irrigation of sugar cane began in north Queensland about 1879.  This
crop had been grown in some coastal districts for many years; its area
had expanded in the 1860s, particularly near Mackay. \index{Mackay,
Qld} The need for adequate soil moisture for this crop during summer
was met at places along the coast where rainfall reached 80 to 100
inches per annum.  Archibald Macmillan \index{Macmillan, A.} saw the
possibility of growing sugar cane under irrigation on the delta of the
Burdekin River, \index{river!Burdekin} where rainfall would be
inadequate for the crop but many freshwater lagoons could provide for
irrigation.  This development occurred after Macmillan floated a sugar
planting company in 1879 and commenced irrigation by pumping from
these ponds.  Then other sugar plantations, also dependent on
irrigation, were established on the delta at a time of boom for the
industry.  High prices for sugar on the world market then stimulated
the establishment on the delta of other plantations with indentured
labour from Pacific islands.\fn{Qld PP, vol.\,iv, 1889, RC~Sugar
Industry;
\cite[p.\,136]{bolton1972}.}

By 1889 the irrigation of cane on the delta involved at least three
plantations: the Pioneer plantation of Drysdale Brothers,
\index{Drysdale Bros} that of the
Young brothers at Kalamia, \index{Kalamai, Qld} and the Seaforth
plantation.  Charles Young \index{Young, C.} of Kalamia gave evidence
to the Royal Commission on the Sugar Industry that irrigation of cane
had been carried out on the estate since 1885. By 1889 the area
commanded by his irrigation channels was 530\,acres.\fn{Qld PP 1889,
vol.\,iv, \&~MoE.}

Irrigation of sugar cane gradually extended to involve almost 8000
acres by 1915 and was used also in the Bundaberg
\index{Bundaberg, Qld} district, where
irrigation began about 1888 at Bingera \index{Bingera, Qld} and about
1900 at Fairymead.\fn{H.\,E.\,A.~Eklund, (1923, 1924), `Irrigation in
Queensland', \textit{Qld Agr.\ J}. vol.\,20, pp.\,105--106, \&
vol.\,21, pp.\,289--308.}\index{Fairymead, Qld}

\section*{Tobacco}\index{tobacco}

Tobacco-growing was taken up by Chinese \index{Chinese} in the 1860s
when alluvial mining in south-eastern Australia was flagging and
imports of tobacco from the United States were curtailed by the Civil
War.  Although production became the virtual monopoly of the Chinese
during the last quarter of the century, there are only occasional
references to their use of irrigation for this crop.  During the 1880s
Ah~Yon \index{Ah Yon} told a government enquiry about his production
of hops and tobacco in north-eastern Victoria; he then had 20 acres
under crop.  Answering a question about watering the tobacco plants,
Ah~Yon said that he only watered twice, both at planting time.  James
Henley, \index{Henley, J.} an American, told the Commission that he
watered his tobacco seedlings in the region only when they were
planted but Chinese watered theirs `a good many times'.  Share-farming
was the method followed by several Chinese tobacco-growers in
north-east Victoria.\fn{Vic.\ RC~Veg.\ Products, 5th Prog.\ Rept,
1888, p.\,30, Q8449;
\cite[pp.\,124--26]{robertson1973}.}

Another report concerns Millicent in south-eastern South Australia
where in 1888 a Royal Commission was told by S.\,J.~Stuckey
\index{Stuckey, S.\,J.} and
R. Slater \index{Slater, R.} that Chinese irrigated tobacco in their
district.\fn{SA PP no.\,28 of 1888, RC~Land Laws of SA, MoE, p.\,107.}

\section*{Vegetables}\index{vegetables}

Irrigation most familiar to Australians was undertaken by Chinese
\index{Chinese} in small market gardens seen in many urban areas.  It
was used also by Chinese gardeners employed on pastoral stations.

Market gardeners of British origin also made use of irrigation in this
period.  Their gardens were more confined to the outskirts of cities.
One case concerns Bacchus Marsh \index{Bacchus Marsh, Vic.} in Victoria,
\index{Victoria} where during the 1880s Mr Pearce pumped from the
Lerderderg River \index{river!Lerderderg} into a race taking water to
an area of chicory and spread it as required through canvas hose
attached to the aqueduct.  The hose was cut in lengths of 20\,feet,
which were joined as required to begin watering the furthest point
then gradually shortened so that a stretch of land was irrigated.
Others cultivated swampy areas at the south east of Melbourne
\index{Melbourne, Vic.} and near
Perth, \index{Perth, WA} thus reducing the need for irrigation.
Market gardens in the lower Torrens valley \index{river!Torrens}
became the main source of vegetables for the Adelaide market,
\index{Adelaide, SA} with subsequent expansion in the 1890s to the
Piccadilly valley
\index{Piccadilly valley, SA} in the Mount Lofty Ranges,
\index{Mount Lofty Ranges, SA} where supplementary irrigation was widely used.
At Piccadilly, \index{Piccadilly, SA} Summertown \index{Summerton, SA}
and Uraidla,
\index{Uraidla, SA} springs were the
main source of water; wells and creeks also made a contribution.
Piping was used to carry water from springs, whose discharge was
improved on some properties by tunnelling into hillsides.  One
installation took advantage of a tunnel made earlier in a search for
gold.  The gardens were usually less than 10 acres in
extent.\fn{\cite[p.\,53]{aasian1885}; \cite{hallack1893}.}

During the last quarter of the century Chinese gardens existed in all
Australian colonies and supplied people in towns and the country.  By
1870 the industry which afforded the Chinese most employment was
`market-gardening of which they had almost a monopoly; 75 per cent of
the whole of the vegetables being grown by Chinese'.  A French visitor
shared this view in 1882, claiming market-gardening and cabinet making
were industries almost completely in the hands of Chinese.  In
Melbourne a visitor in 1890 found that `half of the vegetables sold in
the market are from Chinese gardeners'. The Chinese gardens were noted
for careful cultivation, use of fertilizers including human
excrement,\index{human excrement} and a good supply of water.  During
this period it was recognised that the Chinese presented an example of
what could be achieved with irrigation.\fn{\cite[vol.\,3,
pp.\,1,~331]{coghlan1918};
\cite[p.\,207]{lameslee1883}; \textit{Queenslander}, 6 Sep.\ 1889.}

By 1880 Chinese gardens were common in eastern Australia.  They were
established initially wherever Chinese congregated for alluvial
mining. \index{gold!alluvial} As that type of mining declined in a
district, many Chinese moved off to new minefields, returned to their
homeland, or took up other occupations, notably market-gardening in
south-eastern Austra\-lia, often near centres of reef mining
\index{gold!reef} such as Bendigo \index{Bendigo, Vic.} and Walhalla
\index{Walhalla, Vic.} in Victoria.  In the north the discovery of alluvial
goldfields attracted thousands of Chinese, especially to the Palmer
River in
\index{river!Palmer} north
Queensland \index{Queensland} and Pine Creek \index{creek!Pine} in the
Northern Territory.  \index{Northern Territory} Some of these diggers
came from Victoria and New South Wales but many were new arrivals,
apparently organised in much the same way as the earlier flow of
Chinese gold-seekers to Australia.  Chinese communities began gardens
to cater primarily for themselves but vegetables were also sold to
others.

A good supply of vegetables was important for all in the Australian
colonies.  This produce had an accepted place in the diet of Chinese,
for others it relieved the monotony in many country areas of living
off mutton, bread and tea.  One account of life on a sheep station in
the 80s described the variety of food at a neighbouring station, due
to the presence of a lady and a Chinaman.
\begin{Quote}
	In the train of the squatter's wife come such luxuries and
	delusions as pastry, puddings, and preserves, and the
	beneficent Chinaman employed as a gardener brings in fresh
	`weletables' every day from his continuously irrigated plot of
	land.\fn{\cite[p.\,202]{clark1955}.}
\end{Quote}
But as well as providing relief from a monotonous diet, the vegetables
from Chinese gardens were valuable because `they undoubtedly saved
thousands of Europeans from scurvy in the goldrush
days'. \index{scurvy} This complaint was common among those living in
the inland without access to fruit and vegetables or antidotes such as
lime juice.\fn{\cite[p.\,201]{rolls1981};
\cite[p.\,26]{cannon1973}.}

During the gold rush in the `top end' of the Northern Territory
\index{Northern Territory} in the
1880s, there were numerous Chinese \index{Chinese} gardens between Darwin
\index{Darwin, NT} and Pine Creek. \index{creek!Pine} At Bridge Creek,
\index{creek!Bridge} 
near Adelaide River, \index{river!Adelaide} there were a dozen Chinese
gardeners, who could `ward off the scurvy by plentiful growths of
vegetables'.  Another instance of the contribution made by Chinese
gardeners in combatting scurvy relates to Milparinka
\index{Milparinka, NSW} in the extreme north-west of New South Wales.  It
was the centre of a gold-rush early in the 80s and thousands lived
there.  Some died from typhoid and others from scurvy.  Then a number
of Chinese `began to drift across from the Darling River. They planted
vegetable gardens and almost simultaneously the new diet ended the
disease'~(scurvy).\fn{\cite[p.\,40]{sowden1882};
\cite[p.\,65]{farwell1965}.}

Irrigation by Chinese \index{Chinese} gardeners is recorded from all
colonies though with few accounts for Western
Australia. \index{Western Australia} The involvement of Chinese
gardeners with irrigation in Western Australia in this period was
insignificant until the gold rush in the south-west.  Many Chinese
were brought to the colony as coolies before the goldrush; they worked
main\-ly in the north on pastoral stations.  However, by 1888, after
the first discoveries of gold in the south-west, there were 54 Chinese
engaged in market-gardening in the Perth district and by 1891 that
occupation involved 102 of the barely 900 Chinese in the colony.
Vegetables were grown by Chinese, presumably with irrigation, in the
early days of the Pilbara
\index{Pilbara, WA} goldfield, before its proclamation as
such.\fn{\cite[p.\,54]{oreilly1984}.}  Much of the market-gardening
by Chinese near Perth \index{Perth, WA} was conducted on swamps, where
irrigation may not have been involved but it is likely that the
description of one site on the South Perth foreshore they leased from
1882 was not unique:
\begin{Quote}
	The land was criss-crossed with hand-dug canals, and pitted
	with wells.  Vegetables, varying seasonally, made a changing
	kaleidoscope of colour, their organised rows contrasting
	starkly with the unruly bamboo which flanked the gardens.
	Chinese, struggling under the weight of their yokes carrying
	watering cans, laboured for endless hours hand-watering their
	produce.\fn{\cite[p.\,11]{ryan1995}.}
\end{Quote}

A vivid picture of Chinese irrigation at Ballarat \index{Ballarat,
Vic.} was given by Henry Cornish, \index{Cornish, H.} who travelled
from India in the late 1870s and visited many parts of eastern
Australia. At this renowned area of alluvial gold-min\-ing,
\index{gold!alluvial} where thousands of Chinese were settled, he was
assured that they had taught the colonists the art of market
gardening.
\begin{Quote}
	The way in which the Chinese convert barren wastes of land
	into flourishing gardens is a sight full of instruction for
	English agriculturists. Their method of cultivation is very
	similar to that of the Hindus, the irrigation channels and
	small reservoirs introduced in the Chinaman's garden being
	much the same as we see in India.\fn{\cite[p.\,157]{cornish1880}.}
\end{Quote}

In 1884 an appreciation of Chinese gardeners was given by an
agricultural journalist, apparently T.\,K.~Dow \index{Dow, T.\,K.} who
had previously reported on irrigation in America: at Elmore
\index{Elmore, Vic.} in 
northern Victoria,
\begin{Quote}
	I passed a large Chinese garden on the banks of the Campaspe,
	and the Celestials were pumping water by horse power to
	irrigate the plants.  The Chinese were the first to
	demonstrate the benefits of irrigation in northern Victoria,
	and the sooner they are extensively imitated in this
	particular the better it will be for the condition of our
	settlers in the northern areas.\fn{\textit{Australasian}, 9
	Feb.\ 1884.}
\end{Quote}

Another acknowledgment to Chinese irrigators came in 1892 from Alfred
Deakin, \index{Deakin, A.} a prominent Victorian politician deeply
involved in the development of irrigation: `For the supply of fresh
vegetables from farms within sufficient distance of city markets
irrigation is an effective agency, familiar enough in most towns in
the shape of Chinese gardens'.\fn{\cite{deakin1892}.}

New South Wales, \index{New South Wales} with more than three times
the area of Victoria, lagged behind the latter in population until the
1890s; its wheat-growing was neglected until the 1890s.  Vast inland
areas were held then in large pastoral stations, many of which
employed Chinese \index{Chinese} to raise vegetables and ornamental
plants by means of irrigation.  So common was this practice that one
exception\,---\,at the Murray Downs station \index{station!Murray
Downs} on the Murray\,---\,surprised
\index{river!Murray} the
widely-travelled journalist G.\,A. Brown (`Bruni') \index{Brown,
G.\,A.} sufficiently to remark: `One is so used to find nothing but
Chinese gardeners everywhere north of Victoria that to meet with a
European gardener is a matter of surprise'.  That these Chinese
gardeners relied on irrigation is indicated by the report that about
1890 the work available at some pastoral stations for itinerant
workers or `swaggies' in the Riverina \index{Riverina, NSW} included
working the `Chinese treadwheel' to provide water for irrigated
gardens.\fn{\cite[p.\,131]{wadham1957}; \cite{parker1982};
\textit{Australasian}, 23 Dec.\ 1882.;
\cite[p.\,261]{buxton1967}.}

Chinese gardeners sold food in Sydney \index{Sydney, NSW} and many
country towns. Hugh McKinney, \index{McKinney, H.\,G.} the irrigation
engineer, found `the Chinaman has nev\-er been at a loss to find
suitable places for his garden near towns in the western plains'.
Their activities along the Macquarie River \index{river!Macquarie}
were noticed in 1880 by Marin La~Meslee.  \index{La Meslee, M.}
Chinese gardens along the Darling River \index{river!Darling} supplied
river towns and the mining centres of Cobar \index{Cobar, NSW} and
Broken Hill. \index{Broken Hill, NSW} Up to 1884, according to
McKinney, `with few exceptions, irrigation in New South Wales was
practised only by Chinamen, who in this respect may and very possibly
do claim to have been the pioneers of civilisation'.\fn{H.~McKinney,
\textit{J.\,R.~Soc.\ NSW} 1893, vol.\,27, p.\,384; \cite{lameslee1883}.}

By 1881 Queensland \index{Queensland} held more Chinese
\index{Chinese} than New South
Wales and almost as many as Victoria.  They were scattered from south
to north and were known both as industrious miners and skilful
gardeners.  Al\-exander Boyd \index{Boyd, A.} knew of these people
from his experiences in many districts as a schoolmaster and
agricultural journalist.\fn{\cite{logan1979}.}  In 1882 he described
them:
\begin{Quote}
	As market-gardeners they are matchless.  No soil is so poor
	that a Chinese gardener cannot raise vegetables of every
	description on it.  Manuring and irrigation are the secrets of
	their success.  Every night the gardeners may be seen swinging
	the two man bucket and transferring the necessary moisture
	from the waterhole to the heads of the rows of vegetables,
	whence it permeates through the soil and running along
	innumerable trenches, prepared for the reception of water,
	carries the fertilising medium all over the
	ground.\fn{\cite[p.\,236]{boyd1882}.}
\end{Quote}

Alluvial gold mining was never important in South Australia
\index{South Australia} and its
Chinese \index{Chinese} population remained the least of all the
Australian colonies at the end of the century.  A few Chinese thought
to be descendants of shepherds employed before 1850 became
market-gardeners with a good supply of water in the 80s from First
Creek \index{creek!First} near Waterfall Gully \index{Waterfall Gully,
SA} in Adelaide. \index{Adelaide, SA} Other Chinese raised vegetables
at Innamincka \index{Innamincka, SA} with water drawn from Coopers
Creek
\index{creek!Coopers} in the 1890s.\fn{\cite[pp.\,31,
192]{warburton1981}; \textit{Adelaide Advertiser} 4 Apr.\ 1989;
\cite{tolcher1986}.}

During the period of tin-mining in north-eastern Tasmania,
\index{Tasmania}  from the
70s, there were several Chinese \index{Chinese} gardens in the area.
With the decline of this mining, other gardens were developed to
supply Launceston
\index{Launceston, Tas.} and
Hobart.\index{Hobart, Tas.}

Chinese market gardeners were not invariably irrigators.  In southern
Australia some of these people gave up production of vegetables at the
height of summer.  They did not need irrigation in parts of north
Queensland \index{Queensland} which remained moist throughout the
year.  In regions with two distinct seasons a lack of water in the dry
season would have limited vegetable production.  This may have been
the case in the Northern Territory, \index{Northern Territory} where
the Chinese practice of market gardening is not well documented
despite indications that many people were involved.  During the
gold-rush period in the Territory during 1875--85, Chinese gardens
supplied a variety of produce to miners and to residents of Darwin.
One account refers to many such gardens between Darwin \index{Darwin, NT}
and Pine Creek \index{creek!Pine} during the wet season when sweet
potatoes were an important product.  Julian Tenison-Woods,
\index{Tenison-Woods, J.} a geologist
and cleric well-known in South Australia, visited the Northern
Territory in the dry season of 1886 and reported the production by
Chinese gardeners of maize, sugar-cane, sweet potato and culinary
vegetables alongside the Margaret River. \index{river!Margaret} The
existence of permanent water also at several well-known springs, as at
Rum Jungle, \index{Rum Jungle, NT} apparently made it possible for
vegetable production to be continued by Chinese and other gardeners
using irrigation throughout the prolonged dry season in the
Territory.\fn{\cite{sowden1882}; SA PP no.\,122 of 1886.}

Chinese \index{Chinese} irrigators occupied land more by lease than by
freehold ten\-ure.  Accounts of market-gardening in south-eastern
suburbs of Melbourne \index{Melbourne, Vic.} indicate that Chinese
gardeners leased land from European gardeners.  In the 1880s,
Chen~Ah~Teak
\index{Chen Ah Teak} was the owner or lessee of more than six market
gardens around Sydney.  At Bourke, Tim~Yung \index{Yung, T.} had by
1886 purchased an existing garden of six acres.  In many parts of
northern Australia it is likely that the Chinese irrigators were
squatters.\fn{NSW PP RC~Chinese Gambling, 1891; NSW VP LA vol.\,6,
1885--86, no.\,118a, RC~Water Conservation, First rept, MoE, p.\,223.}

\section*{Means of irrigation}

Irrigation in the period generally required some means of lifting
water to the designated area.  Delivery by gravitation \index{gravity
feed} often required the installation of a weir or dam \index{weir}
\index{dam} to divert part
of the stream flow to a race or canal discharging at the required
area; this system probably was a feature of Tasmanian irrigation but
was not widely used on the mainland.

Pumping of water from perennial rivers and creeks was undertaken by
many irrigators; many of the streams were so deeply incised in the
landscape that pumps were required to lift water to a significant
height.  In order to obtain a supply sufficient to water a large area
without delay, pumps and engines of adequate capacity and power were
needed; generally the essential equipment included a horizontal steam
engine and a centrifugal pump. \index{pump!steam-driven}
\index{pump!centrifugal} This type of pump is adapted to lifting water
to a height of 25 to 30 feet; it was useful to those irrigators on the
riverine plain in Victoria who needed to lift water less than 10 feet.
Some of them depended on adequate supplies in creeks filled by
overflow from the Murray River
\index{river!Murray} between
Echuca \index{Echuca, Vic.} and Swan Hill, \index{Swan Hill, Vic.} and
as Duncan Leitch \index{Leitch, D.} at Gunbower soon found there were
seasons when water was not available for pumping.  Such experience led
to an interest in seeking more dependable supplies of water.\fn{NSW
VP, 1886, RC~Water Conservation, 2nd Rept MoE Appendix I, McKinney
pp.\,17--26; Vic.\ PP no.\,53 of 1885, RC~Water Supply MoE, p.\,37.}

The use of steam engines and centrifugal pumps was adopted at a time
when there were several Australian makers of this equipment.
J.~Robison \index{Robison, J.} of Melbourne was apparently the first
to manufacture centrifugal pumps in Australia; application of his
products to irrigation was advertised frequently in the
press.\fn{\cite[p.\,192]{cannon1975}; J.\,G.~Burnell, (1934) \textit{Inst.\
Eng.\ Aust.\ J.}}

John O'Shanassy, \index{O'Shanassay, J.} a former Premier of Victoria,
acquired the Moira \index{Moira run, NSW} run on the New South Wales
side of the river in 1862 and subsequently took up Madowla Park
\index{Madowla Park, Vic.}  on the Lower Moira
run\,---\,on the opposite si\-de of the river.  His son Matthew
\index{O'Shanassay, M.} was irrigating at Madowla Park in 1887, using
a steam engine and centrifugal pump \index{pump!centrifugal} to
discharge river water into an elevated flume running to the homestead
area three-quarters of a mile away. An unusual feature was the
mounting of engine and pump on a barge to allow irrigation of the
O'Shanassy lands on both sides of the river.\fn{\cite{ingham1974a};
\textit{Town and Country Journal}, 6 Aug.\ 1887, p.\,270;
\cite[p.\,108]{coulson1979}.}

So popular was the more powerful equipment, available from different
engineering firms in Victoria, that in the 1890s there were said to be
90 pumping plants along the river between Swan Hill and
Echuca.\fn{\cite[p.\,127]{feldtmann1973}.}

How did the Chinese \index{Chinese} water their gardens?
Illustrations in the 19th century generally show a man carrying two
buckets of water at the ends of a bamboo stick or yoke carried over
the shoulder.  But there was considerable variety in their arrangement
of water supply.  In New South Wales during the 1880s, Chinese
gardeners were using steam engines to pump water from rivers or wells.
Tim~Yung at Bourke drew water from a well by means of a six-horsepower
steam-driven pump. \index{pump!steam-driven} On the Hawkesbury River,
\index{river!Hawkesbury} Chinese gardeners also used steam power to
pump their water from the river.  Other Chinese used horse-powered
pumps, \index{pump!horse-powered} as at Elmore and Katamatite in the
Goulburn Valley \index{river!Goulburn} of Victoria.  Innamincka
\index{Innamincka, SA} in
South Australia was one of few places known to have a water-wheel
\index{water-wheel} 
raising supplies to a Chinese garden of about one acre.  It was
described as `a magnificent water-wheel, the workmanship of which was
admired by all who saw it'.\fn{\textit{Australasian}, Feb.\ 1884,
p.\,186;
\cite[p.\,51]{dunlop1978};
\cite[p.\,107]{tolcher1986}.}  One of the simplest machines for
raising water was the type of pump \index{pump!belt} first used in
alluvial mining and described in 1853:
\begin{Quote}	
	Others use a Chinese pump, called a belt-pump, which the
	Chinese took to California, and which Californian diggers are
	using here.  The belt-pump consists simply of a long wooden
	pipe or tunnel, about six inches square, at the upper end of
	which is a wheel turning a long band of canvas, the two ends
	of which are sewed together so that it forms a circle.  On
	this band are fixed upright square pieces of board at regular
	distances; and as the wheel is turned, these pieces of board
	move onward with the band, enter the lower end of the tunnel,
	and carrying the water with them, discharge it at the
	mouth.\fn{\cite[p.\,97]{howitt1885}.}
\end{Quote}

In urban areas, some market gardeners depended on mains supply water.
Thus in Haw\-th\-orn, \index{Hawthorn, Vic.} Victoria, Chinese gardens
became established at a distance from the Yarra River
\index{river!Yarra} only after the reticulated supply of Yan Yean
\index{reservoir!Yan Yean} water became available from 1870.  Mains
supply water was also the basis for irrigation in South Australia at
Stirling \index{Stirling North, SA}
North.\fn{\cite[p.\,81]{mcwilliam1978};
\cite{green1892}.}

%\section*{References}
%1. NSW VP 1886, R.C.Water Conservation, 2nd Rept, MoE Appendix I,
%    McKinney pp.17-26.	        
%2. Australasian, 18/11/1882.
%3. Australasian, 7/1/1882.
%4. J.Inst.Eng.Aust. 1930,vol. 2, p.270, VicPP No.53 of 1885, MoE p.37,\&
%    Leader, 1/12/1883, VicPP No. 53 of 1885, MoE p.32.
%5. VicPP Vol.4,1884, VicYearbook 1884, p.436. 
%6. VicPP No 53 of 1885,p.xxxviii.
%7. Australasian, 1/1/1887.
%8. W.Williams, A History Of Bacchus Marsh. . . 1836-1936, 1936, \&
%    VicPP No 53 of 1885, MoE p.271.
%9. Susan Priestley, The Victorians: Making Their Mark, 1984, p.76.
%10. VicPP No.53 of 1885, p.xxxvii.
%11. Australasian, 9/1/1886.
%12. Australasian, 23/12/1882.
%13. Town and country journal, 6/8/1887.
%14. VicPD 1886,p.2763.
%15. NSW VP 118 of 1885.
%16. NSW R.C.Water Conservation First Rept 1885-86; C.J.Lloyd, Either
%      Drought Or Plenty, 1988,p.53.  
%17. G.L.Buxton, The Riverina 1861-1891, 1967. p.246, \&  Australasian,
%      24/1/1890.
%18.  Procs First Vict.Irrig.Conf. 1890, p.130.
%19. Leader, 12/9/1885.
%20. NSW Agr.Gaz. Vol.2, 1891, p.163.
%21. K.Jeffcoat, More Precious Than Gold, 1988, p.118.
%22. NSW Agr.Gaz.vol.2, 1891, p.163.
%23. NSW Agr.Gaz., vol.3, 1892, p.601.
%24. NSW VP 1885, R.C.Water Conservation, MoE, pp160-164.
%25. Brisbane Courier, 28/3/1884.
%26. Queenslander, 6/4/1889.
%27. NSW Agr.Gaz. vol.10, 1899, p.802.
%28. Australasian, 3/3/1888, p.466.
%29. C.S.Martin, Irrigation And Closer Settlement In The Shepparton District,
%      1836-1906, 1955, p.57.
%30. Australasian, 10/5/1890, p.898, J.M.Powell(ed), Yeomen And Bureaucrats,
%      1973, p.xxiv.
%31. NSW Agr.Gaz., vol.3 1892, p.711.
%32. Town and Country Journal, 6/8/1887.
%33. Susan Priestley, Echuca, A Centenary History, 1965, p.155.
%34. H.Oldham, Irrigation And Water Conservation In Western Australia,
%      1913,p.11,  \& ADB vol 7, p.70.
%35. J.J.Green and A.Molineux, Irrigation, 1892, p.21.
%36. K.Preiss \& Pamela Oborn, The Torrens Park Estate, 1991, p.229.
%37. R.J.Anderson, Solid Town, History Of Port Augusta, 1988, p.62.
%38. H.E.A.Eklund, Irrigation In Queensland, Qld Agric.J. vols 20 \& 21, 
%     1923-24.   
%39. Melbourne Argus, 10/1/1884.
%40. SAPD 1887, p.310.
%41. Queenslander, 25/10/1890.
%42. NSW VP1886, R.C.Water Conservation, 1st Rept,MoE p.223.
%43. Australasian, May 1882, p.538.
%44. VicPP No.53 of 1885, R.C.Water Supply, Further Progr.Rept
%      MoE, p.255.
%45. VicPP 1884, Statistical Register.
%46. Helen Pearce, The Hop Industry In Australia, 1976, p.91.
%47. Carole Woods, Beechworth, A Titan's Fields, 1985, p.118.
%48. Vic R.C.Vegetable Products, 5th Progr.Rept 1888, Q8711.
%49. Australasian, 18/3/1882.
%50. Vic R.C.Vegetable Proucts, 5th Progr.Rept,1888, Q8699.
%51. J.Carter, Stout Hearts And Leathery Hands, 1968, p.127; Kay
%      Robertson, Myrtleford, Gateway To The Alps, 1973, p.106.
%52. Helen Pearce, 1976, p.70 \& ADB Vol.5, David Murray, p.319.
%53. QldPP, Vol.iv, 1889, R.C.Sugar Industry.
%54. G.C.Bolton, A Thousand Miles Away, A History Of North Queensland
%       to 1920, 1972, p.136. 		
%55. QldPP 1889, vol.iv, R.C.Sugar Industry.
%56. QldPP 1889, R.C.Sugar Industry, MoE.
%57. H.E.A.Eklund, Irrigation In Queensland, Qld Agr.J. vol.20, 1923,
%      p.105-06, vol.21, 1924, pp. 289-308.
%58. Vic R.C.Veg.Products, 5th Prog.Rept,1888, p.30.
%59. Vic R.C.Veg.Products, 5th Prog.Rept, 1888, Q8449.
%60. Kay Robertson, 1973,pp.124-126.
%61. SAPP No.28 of 1888, R.C.Land Laws of S.A., MoE, p.107.
%62. The Australasian Farmer, 1885, p.53.
%63. E.H.Hallack, Toilers Of The Hills, 1893/1987.
%64. T.A.Coghlan, Labour And Industry In Australia, 1918, vol.3, p.1331.
%65. E.M.La Meslee, The New Australia,1883,  trans.R.Ward 1973, p.207.
%66. Queenslander, 6/9/1889.
%67. C.M.H.Clark, Select Documents In Australian History 1851-1900,
%       1955, p.202.
%68. E.Rolls, A Million Wild Acres, 1981, p.201.
%69. M.Cannon, Australia In The Victorian Age, Life In The Country, 1978,
%       p.26.
%70. W.J.Sowden, The Northern Territory As It Is, 1882, p.40.
%71. G.Farwell, Ghost Towns Of Australia, 1969, p.65.
%72. Jan Ryan, Ancestors: Chinese In Colonial Australia, 1995.
%73. M.J.O'Reilly, Bowyangs And Boomerangs, 1984, p.54.
%74. Jan Ryan, 1995, p.11.
%75. H.Cornish, Under The Southern Cross, 1880, p.157.
%76. Australasian, 9/2/1884.
%77. A.Deakin, Irrigation In Australia, Year-book of Australia,1892.
%78. S.Wadham, R.K.Wilson \& Joyce Wood, Land Utilization In
%      Australia, 1957,p. 131.
%79. K.L.Parker, My Bush Book, 1982.
%80. Australasian, 23/12/1882.
%81. G.L.Buxton, The Riverina 1861-1891, 1967, p.261.
%82. H.McKinney, J.R.Soc.NSW 1893, v.27, p.384.
%83. E.M.La Meslee, 1883, trans. R.Ward, 1973.
%84. H.McKinney, J.R.Soc.NSW 1893, v.27.
%85. G.N.Logan, W.A.J.Boyd, ADB Vol.7, 1979, p.374.
%86. A.J.Boyd, Old Colonials, 1882/1974, p.236.
%87. Elizabeth Warburton, The Paddocks Beneath, A History Of Burnside from
%       Its Beginning, 1981, pp 31 \& 192, \& Adelaide Advertiser 4/4/1989.
%88. Helen M.Tolcher, Drought Or Deluge, Man In The Coopers Creek
%    Region, 1986.
%89. W.J.Sowden, 1882.
%90. SAPP No. 122 of 1886.
%91. NSWPP R.C.Chinese Gambling, 1891.
%92. NSW VP LA Vol.6,1885-86, No 118a , R.C.Water Conservation, First 
%      rept, MoE, p.223. 
%93. NSW VP, 1886, R.C.Water Conservation, 2nd Rept 
%      MoE Appendix I, McKinney p.17-26.		  
%94. VicPP No.53 of 1885, R.C.Water Supply MoE, p.37.
%95. M.Cannon, Australia In The Victorian Age, Life In The Cities, 1988,
%       p.192,\&  J.G.Burnell, 1934 Inst.Eng.Aust.J.
%96. S.M.Ingham, Sir John O'Shanassy, ADB vol.5,1974.
%97. Town and Country Journal, 6/8/1887, p.270.
%98. Helen Coulson, Echuca-Moama, Murray River Neighbours, 1979, p.108.
%99. A.Feldtmann, Swan Hill, 1973, p.127.
%100. Australasian, Feb.1884, p.186, \& A.J.Dunlop, Wide Horizons, 1978,
%        p.51.
%101. Helen M.Tolcher, 1986, p.107.
%102. W.Howitt, Land,Labour And Gold, 1855/1972, p.97.
%103. Gwen McWilliam, Hawthorn Peppercorns, 1978, p.81.
%104. J.J.Green \& A.Molineux, Irrigation, 1892.
