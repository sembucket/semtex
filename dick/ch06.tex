% $Id$
% CHAPTER SIX
% 6631 wds at 30/4/99

\setcounter{endnote}{0}
 
\chapter{Irrigation by Independent Producers from 1880}
\label{ch:indep}\addtoendnotes{\protect\section*{Chapter \thechapter}}
\markboth{
\textsc{Chapter \thechapter. Independent Producers}
}{
}

At the beginning of this period irrigation in Australia was
characteristically undertaken independently by private individuals or
companies, without involvement in a definite irrigation scheme.  In
Tasmania a few landholders on a tributary of the Macquarie River
continued to use water from Tooms Lake to irrigate their lands by
mutual arrangement, the essence of a group scheme.

The independence of private irrigators was reduced later in those
colonies where state control of water resources was established and
semi-government bodies were formed to manage irrigation in specific
areas.

Private irrigators, who were numerous and culturally diverse, provided
various commodities.  Vegetables, fruit, tobacco, and hops were
produced on small holdings with some use of irrigation by large
numbers of growers, often Chinese.  Cereals, fodder, and sugar cane
were watered almost exclusively by people of European origin,
generally on farms, pastoral stations, or plantations.  Methods of
irrigation were most varied, ranging from laborious hand-watering to
use of mechanical equipment driven by engines in situations requiring
supplies to be lifted, and including flood irrigation where water
could gravitate from a stream, as in Tasmania.

\section*{Cereals}

Agriculturalists were generally less involved with irrigation. Their
main crops were cereals adapted to dry-farming dependent on relative
abundance of rain in winter and spring; they cultivated comparatively
small holdings generally lacking surface water resources in abundance;
and few had time or money to use irrigation.  However, some farmers in
northern Victoria used water from adjacent river channels for their
wheat crops.  Woodford Patchell, a pioneer of irrigation in the
region, pumped water from the Loddon River at Kerang to irrigate more
than 100\,ac of mixed crops in 1882.1\fn{NSW V\&P 1886,
RC~Water Conservation, 2nd Rept, MoE Appendix~I, McKinney
pp.\,17--26.}  John Garden began pumping from Barr Creek, one of
several channels near the Murray, to water 300\,ac of cereals near
Cohuna.  He demonstrated his equipment in November 1882 to 150 people
from different parts of the northern plains, including H.\,R.~Williams
and Hugh McColl---parliamentarians elected from the
region.\fn{Australasian, 18 Nov. 1882.}

That year marks the beginning of extended use of irrigation in
northern Victoria, which followed the failure of district crops in
1881.\fn{Australasian, 7 Jan.\ 1882.}  Others who took up
irrigation about the same time in this part of the colony included
William Webb of Rochester, whose pumping plant installed in 1883 was
regarded as one of the first in Victoria; Duncan Leitch, whose supply
from Gunbower Creek depended on an inflow from the Murray River;
Mr~Holding who irrigated 200\,ac of wheat near Gunbower; and David
Chrystal, who irrigated grass and cereals at Torrumbarry by pumping
from the Murray.\fn{J.\,Inst.\ Eng.\ Aust.\ 1930, vol.\,2,
p.\,270, VicPP no.\,53 of 1885, MoE p.\,37, \& Leader, 1 Dec.\ 1883,
VicPP no.\,53 of 1885, MoE p.\,32.}  The facilities for irrigation
with water from the Murray River led to the watering of nearly
5\,000\,ac of wheat by scores of irrigators in the Swan Hill shire in
1883--84.\fn{VicPP vol.\,4, 1884, VicYearbook 1884, p.\,436.}

\section*{Fodder}

Many pastoralists in south-eastern Australia made use of irrigation.
Their holdings had been chosen with an eye to a secure supply of
water; in many cases much more than the essential stock and domestic
supply.  The chief interest of pastoralists in irrigation was to
improve their capacity to carry livestock either by watering natural
pasture in dry seasons or fodder crops such as lucerne or cereal hay.

In the early part of this period there was greater use of irrigation
for fodder crops in places where a start had been made previously.
One of these was Bacchus Marsh in central Victoria, where in 1883
Michael O'Connell, one of several local irrigators, was diverting
water by a race from the Lerderderg River to irrigate 70\,ac of
lucerne.\fn{VicPP no.\,53 of 1885, p.\,xxxviii.}  In 1887 there
were six steam plants in use for irrigation in the
locality.\fn{Australasian, 1 Jan.\ 1887.}  Irrigators included
Mr~Crook, who began using supplies from the Lerderderg River in 1878,
and W.~Grant who irrigated 12\,ac of lucerne in 1885 using a
centrifugal pump and steam engine.\fn{W.~Williams, \textsl{A
History of Bacchus Marsh 1836--1936}, 1936, \& VicPP no.\,53 of 1885,
MoE p.\,271.}  In the Victorian Gippsland district, the earlier use of
water from springs near Sale apparently prompted wider practice of
irrigation at a time of great demand for supplies to goldfields in the
region.\fn{Susan Priestley, \textsl{The Victorians: Making Their
Mark}, 1984, p.\,76.}  At Charlcote, near Sale, W.H.Palmer pumped
water from the Avon river to irrigate 300\,ac of
grassland.\fn{VicPP no.\,53 of 1885, p.\,xxxvii.}

Western Victorian experience of irrigation was associated with streams
from the Grampians mountains.  In 1886 Samuel Carter pumped water from
the Wimmera River to irrigate natural pasture and lucerne on Walmer
pastoral station near Horsham.  Previously his family held Glenisla
station on the headwaters of the Glenelg River, south of Horsham,
where he first gained experience of irrigation.\fn{Australasian,
9 Jan.\ 1886.}

Major developments occurred in the riverine plain stretching from
northern Victoria to the Lachlan River in New South Wales.  Irrigation
on the Murray Downs pastoral station on the Murray River in New South
Wales was extended to a further 40\,ac, at least partly to produce
suitable fodder for ostrich farming on the station in
1882.\fn{Australasian, 23 Dec.\ 1882.}  Near Echuca, a major
port for river-boats on the Murray, there were opportunities for
irrigation by pumping from both sides of the river.  On Perricoota
station on the northern side, the firm of Robertson~\& Wagner used
irrigation to maintain the homestead garden and to produce cereal hay
for horses held there for use on its well known Cobb~\&~Co
coach-lines.\fn{Town and Country Journal, 6 Aug.\ 1887.}  The
powerful engines and pumps used along this part of the Murray to
irrigate large areas replaced the horse-powered pumps, windmills, and
hydraulic rams used widely in the past.

North of the Murray there was scattered use of irrigation.  The Werai
pastoral station established north-east of Wakool by the Gwynne family
in 1843 was one of the earliest scenes of irrigation in the region,
using water pumped from Colligen Creek, an effluent of the
Murrumbidgee River.  Production there of lucerne and other crops
became well known before 1886.\fn{VicPD 1886, p.\,2763.}  George
Mair began irrigating fodder crops at Groongal, on the north side of
the Murrumbidgee, about 1882, and in 1887 had a good fodder crop on
40\,ac, watered by pumping from the river.\fn{NSW V\&P 118 of
1885.}  James Tyson took advantage of floods on the lower Lachlan
River by cutting channels through the river bank and diverting water
into adjacent depressions to give better growth of herbage for his
sheep. These achievements were reported in 1885.\fn{NSW
RC~Water Conservation First Rept 1885--86 \& C.\,J.~Lloyd,
\textsl{Either Drought or Plenty}, 1988, p.\,53.}  By 1890 there was
irrigation by Phillips~\& Graves at Warbrecan, by David McCaughey at
Coree, and by Patrick McFarland at Barooga, where water was pumped
from the Murray to 100\,ac of lucerne.\fn{G.\,L.~Buxton,
\textsl{The Riverina 1861--1891}, 1967. p.\,246, \& Australasian,
24 Jan.\ 1890.}  After legislation in 1896 for State control of water
rights in New South Wales, licenses were granted for use of water by
private irrigators, the outstanding use being that of Samuel McCaughey
who in 1900 developed irrigation of fodder and pastures at North
Yanco.

In the valley of the Lachlan River pastoralists began irrigation in
the 80s.  J.\,B.~Donkin, who held the Lake Cowal station, began
irrigating near the Lachlan about 1885.\fn{Procs First Vic.\
Irrig.\ Conf.\ 1890, p.\, 130.}  Messrs Raymond~\& Nicholas raised
water 32\,ft from the river with a centrifugal pump to irrigate 25\,ac
of maize and 5\,ac of potatoes near Forbes.\fn{Leader, 12 Sep.\
1885.}  At Jemmalong, also near Forbes, N.\,A.~Gatenby irrigated
lucerne from the Lachlan; he gained a government prize in 1890 for
farm irrigation.\fn{NSW Agr.\ Gaz.\ vol.\,2, 1891, p.\,163.}
Downstream near Lake Cargellico a syndicate put up a substantial dam
on the Lachlan in 1885 for irrigation.\fn{K.~Jeffcoat,
\textsl{More Precious Than Gold}, 1988, p.\,118.}

Near the south coast of New South Wales, the irrigation conducted on
the Kameruka estate of Robert~L. Tooth was regarded so highly by
judges of the irrigated farms competition that in 1891 it was awarded
first prize for the best system of irrigation in the
colony.\fn{NSW Agr.\ Gaz.\ vol.\,2, 1891, p.\,163.}  On this
farm, just south of Bega, 30\,ac of lucerne were irrigated with water
drawn from Tantawanglo Creek by a pump driven by a steam engine and
raised more than 200\,ft to the top of a hill whence it flowed down
through the lucerne.  This property, where 1\,300 cows were milked
daily in summer, was then held to be one of the largest dairy farms in
the world.\fn{NSW Agr.\ Gaz.\ vol.\,3, 1892, p.\,601.}

Another entrant in the irrigated farms competition of 1891 was the
farm of T.\,P.~Wills-Allen in the Gunnedah district of northern New
South Wales.  His introduction to irrigation had come as an
involuntary user of waste water since 1876 when he used a 30\,hp
engine to pump water from the Namoi River for a sheep-wash.  After its
use for this purpose, the water was allowed to flow over the land, a
practice followed each year until 1884 when Mr Wills-Allen was so
converted to irrigation that he watered 25\,ac of wheaten hay and a
stand of lucerne.\fn{NSW V\&P 1885, RC~Water Conservation,
MoE, pp.\,160--164.}

Irrigation was used on the Darling Downs of Queensland by at least two
pastoralists in the 1880s.  On Kings Creek, about 40 miles south of
Toowoomba near the railway to Warwick, Atticus Tooth, a cousin of
R.\,L.~Tooth involved with irrigation at Kameruka, New South Wales,
watered 22\,ac of black cracking clay with supplies delivered by a
steam-driven pump.\fn{Brisbane Courier, 28 Mar.\ 1884.}  Then in 1888,
Francis and Robert Gore began irrigation at Yandilla on their
90\,000\,ac run on the Condamine River near Pittsworth, south-west of
Too\-woom\-ba.\fn{Queenslander, 6 Apr.\ 1889.}

In South Australia, J.\,L.~Thompson at Beefacres near Adelaide
irrigated 12\,ac in 1881 including 5\,ac of fodder crops as well as
fruit and vegetables.\fn{NSW Agr.\ Gaz.\ vol.\,10, 1899,
p.\,802.}

\section*{Fruit}

Notable production from irrigated orchards was undertaken in the more
populous areas and in inland Australia before the establishment of
group schemes for irrigated horticulture.

In 1888 a Victorian government prize for irrigated crops was awarded
to David Milburn of Keilor who used water from the Maribyrnong River
to irrigate an orchard of 12 acres and a larger area of other
crops.\fn{Australasian, 3 Mar.\ 1888, p.\,466.}  Milburn was one
the earliest commercial irrigators in Victoria.  The Mason brothers
established an orchard north of Shepparton about 1884 and by 1886 they
pumped from the Goulburn River to irrigate 70 acres of fruit
trees.\fn{C.\,S.~Martin, \textsl{Irrigation and Closer
Settlement in the Shepparton District, 1836--1906}, 1955, p.\,57.}

Another prize-winning irrigationist in Victoria was Robert Clark, a
miller and politician with fruit trees on his Riversdale property in
the Wimmera at Lower Norton, watered by pumping from Norton Creek.  In
1889 he was awarded a government prize.\fn{Australasian, 10 May
1890, p.\,898, \& J.\,M.~Powell~(ed), \textsl{Yeomen and Bureaucrats},
1973, p.\,xxiv.}

Several properties near Sydney were being irrigated in the last decade
of the century.  Irrigated orchards were entered for the New South
Wales government competition for irrigated farms in 1891.  First prize
for orchards went to Mr~T.~Brien of Parramatta who irrigated 10\,ac of
citrus; he pumped water from a creek with a 6\,hp engine and used
iron pipes to carry supplies all over his orchard.  G.\,H.~Dempsey
followed a similar arrangement at Emu Plains, taking water from a well
with a Tangye pump driven by a 2\,hp Tangye engine and reticulating
the water through iron pipes to his citrus trees.\fn{NSW Agr.\
Gaz., vol\,.3 1892, p.\,711.}

Near Echuca, Daniel Matthews pumped water from the Murray River in the
mid-80s to irrigate an orchard in New South Wales on the Maloga
Mission for Aborigines, which he established on land selected from the
Moira run leased by John~O'Shanassy.\fn{Town and Country
Journal, 6 Aug.\ 1887.}  On the Victorian side, A.\,D.~Jeffrey began
irrigating fruit trees in 1887 a few miles from Echuca, using a
centrifugal pump to take water from the river.\fn{Susan
Priestley, \textsl{Echuca, A Centenary History}, 1965, p.\,155.}

In Western Australia there had been very little experience of
irrigation before 1890.  Discoveries of gold at Kalgoorlie a few years
later led to a doubling of the colonial population within 10 years and
encouraged horticultural and agricultural industries.  In 1913 there
were more than 300 instances of small-scale irrigation of orchards and
vegetable gardens in the south-west part of the State, while by 1908
bananas were being grown under irrigation by Edward Angelo on the
Gascoyne River at Carnarvon.\fn{H.~Oldham, \textsl{Irrigation and
Water Conservation in Western Australia}, 1913, p.\,11, \& ADB vol.\,7,
p.\,70.}

In South Australia most irrigation of fruit trees was undertaken near
Adelaide in the lower Torrens valley where in 1892 there were more
than 200 pumps driven by wind, steam engines, and horse power to draw
water for fruit and vegetables.\fn{J.\,J.~Green and A.~Molineux,
\textsl{Irrigation}, 1892, p.\,21.}  After Robert Barr Smith acquired
the Torrens Park property near Mitcham, he built a reservoir in 1875
to water the existing orangery of 600 trees.\fn{K.~Preiss
\& Pamela Oborn, \textsl{The Torrens Park Estate}, 1991, p.\,229.}
This reservoir with a capacity of 2 million gallons (9\,000\,kL) could
provide 7\,ac-ft of water---a great improvement on the arrangement
made by the previous owner, R.\,R.~Torrens.  One instance of
irrigation for fruit growing under arid conditions was near Port
Augusta where mains supply water from the nearby Flinders Range was
used at Stirling North for fruit and vegetables; in 1892 the orchards
covered 40\,ha.\fn{R.\,J.~Anderson, \textsl{Solid Town, History
of Port Augusta}, 1988, p.\,62.}  One of the reasons for development
of irrigation near Adelaide in the 90s was to supply Broken Hill with
fresh food; the expansion at Stirling North was probably in response
to the completion of the railway through the Flinders Ranges to
Oodnadatta.

Irrigation of citrus fruit and vegetables was practised at Bowen, the
earliest settlement in north Queensland, from the 1890s, using
underground water on areas which exceeded 600\,ac before
1920.\fn{H.\,E.\,A.~Eklund, Irrigation in Queensland, Qld
Agric.~J. vols.\,20~\& 21, 1923--24.}

Fruit was produced by Chinese at several points along the Darling
River.  At Wentworth, Chinese gardens were reported by John Stanley
James (`The Vagabond') in 1885; he found `John's garden on the banks
of the Darling is the only green oasis in the
place'.\fn{Melbourne Argus, 10 Jan.\ 1884.}  Two years later
`the splendid orangery in the neighbourhood of Wentworth carried on by
Chinamen' was mentioned in the South Australian
parliament.\fn{SAPD 1887, p.\,310.}  Citrus sold in Mildura in
1890 were grown at Wentworth on trees planted about twenty years
earlier---almost certainly by Chinese who could not have sustained
production in the arid climate without irrigation from the Darling
River.\fn{Queenslander, 25 Oct.\ 1890.}  At Bourke, fruit and
vegetables in sufficient quantity to supply the town were raised by
Tim Yang and a score of Chinese helpers from an irrigated garden of 3
to 4 acres in 1885.  Bourke was then a major port on the Darling,
where copper ore from mines at Cobar, a hundred miles south, was
shipped to South Australia.\fn{NSW V\&P 1886, RC~Water
Conservation, 1st Rept, MoE p.\,223.}

\section*{Hops}

Hop gardens were irrigated at several places in Tasmania and Victoria.
During the boom in hop production in the 80s there were reports of
irrigation from different parts of Victoria, but in Tasmania the
expansion was mainly in the Derwent valley, where its use became
common following trials at New Norfolk much earlier.  In Victoria hops
were grown most extensively in Gippsland at first, but irrigation was
not common there in 1882.\fn{Australasian, May 1882, p.\,538.}
In 1883--84 there were five growers irrigating 84\,ac near Bairnsdale.
One instance was the hop-garden of J.\,A.~Taylor near Bairnsdale,
where in 1885 water was lifted 20 feet from the Mitchell River by
steam power to irrigate 50~acres.\fn{VicPP no.\,53 of 1885,
RC~Water Supply, Further Progr.\ Rept MoE, p.\,255.}

Hop production increased along the Ovens valley in the north-east,
where about 40 growers irrigated almost 300\,ac in
1883--84.\fn{VicPP 1884, Statistical Register.}  One of the most
prominent growers was Hiram Crawford who used two waterwheels on the
Ovens River to irrigate 50\,ac.\fn{Helen Pearce, \textsl{The Hop
Industry in Australia}, 1976, p.\,91.} This was an enterprise
undertaken after his retirement to the district in 1876 after a busy
life as gold-digger, coach-line proprietor, and manager of a shopping
arcade in Melbourne.\fn{Carole Woods, \textsl{Beechworth, A
Titan's Fields}, 1985, p.\,118.}  W.~Lyons of Everton also used a
waterwheel for irrigation.\fn{Vic.\ RC~Vegetable Products,
5th Progr.\ Rept 1888, Q8711.}  Another Victorian area where hops were
irrigated was in the Yarra valley near
Healesville.\fn{Australasian, 18 Mar.\ 1882.}

The Chinese were involved with hop-gardens in north-eastern Victoria
during the 80s, though mainly as labourers for
Europeans.\fn{Vic.\ RC~Vegetable Proucts, 5th Progr.\ Rept,
1888, Q8699.}  The achievement of the Panlook brothers in the Eurobin
district as hop-growers dates back to 1890 when four sons of a former
storekeeper on the Buckland goldfields returned to the district and
started growing tobacco and hops.  Within a few years the brothers had
a substantial area under hops which they irrigated. Their farm became
the outstanding Victorian source of hops.\fn{J.~Carter,
\textsl{Stout Hearts and Leathery Hands}, 1968, p.\,127, \&
Kay Robertson, \textsl{Myrtleford, Gateway to the Alps}, 1973,
p.\,106.}

Hops were cultivated in some South Australian localities with moderate
rainfall, as in the Adelaide hills and near Mount Gambier.  The only
record of irrigation refers to David Murray's Rockford Estate on the
Onkaparinga River near Mylor, where 10\,ac of hops were cultivated in
1896.\fn{Helen Pearce, 1976, p.\,70, \& ADB vol.\,5, David
Murray, p.\,319.}

\section*{Sugar Cane}

Irrigation of sugar cane began in north Queensland about
1879.\fn{QldPP, vol.\,iv, 1889, RC~Sugar Industry.}  This
crop had been grown in some coastal districts for many years; its area
had expanded in the 1860s, particularly near Mackay.  The need for
adequate soil moisture for this crop during summer was met at places
along the coast where rainfall reached 80 to 100 inches per annum.
Archibald Macmillan saw the possibility of growing sugar cane under
irrigation on the delta of the Burdekin River, where rainfall would be
inadequate for the crop but many freshwater lagoons could provide for
irrigation.  This development occurred after Macmillan floated a sugar
planting company in 1879 and commenced irrigation by pumping from
these ponds.\fn{G.\,C.~Bolton, \textsl{A Thousand Miles Away, A
History of North Queensland to 1920}, 1972, p.\,136.}  Then other
sugar plantations, also dependent on irrigation, were established on
the delta at a time of boom for the industry.  High prices for sugar
on the world market then stimulated the establishment on the delta of
other plantations with indentured labour from Pacific islands.

By 1889 the irrigation of cane on the delta involved at least three
plantations: the Pioneer plantation of Drysdale Brothers, that of the
Young brothers at Kalamia, and the Seaforth plantation.\fn{QldPP
1889, vol.\,iv, RC~Sugar Industry.}  Charles Young of Kalamia gave
evidence to the Royal Commission on the Sugar Industry that irrigation
of cane had been carried out on the estate since 1885. By 1889 the
area commanded by his irrigation channels was 530\,ac.\fn{QldPP
1889, RC~Sugar Industry, MoE.}

Irrigation of sugar cane gradually extended to involve almost
8\,000\,ac by 1915 and was used also in the Bundaberg district, where
irrigation began about 1888 at Bingera and about 1900 at
Fairymead.\fn{H.\,E.\,A.~Eklund, Irrigation in Queensland, Qld
Agr.~J. vol.\,20, 1923, pp.\,105--106, \& vol.\,21, 1924,
pp.\,289--308.}

\section*{Tobacco}

Tobacco-growing was taken up by Chinese in the 1860s when alluvial
mining in south-eastern Australia was flagging and imports of tobacco
from the United States were curtailed by the Civil War.  Although
production became the virtual monopoly of the Chinese during the last
quarter of the century, there are only occasional references to their
use of irrigation for this crop.  During the 1880s Ah~Yon told a
government enquiry about his production of hops and tobacco in
north-eastern Victoria; he then had 20 acres under crop.  Answering a
question about watering the tobacco plants, Ah~Yon said that he only
watered twice, both at planting time.\fn{Vic.\ RC~Veg.\ Products,
5th Prog.\ Rept, 1888, p.\,30.}  James Henley, an American, told the
Commission that he watered his tobacco seedlings in the region only
when they were planted but Chinese watered theirs `a good many
times'.\fn{Vic.\ RC~Veg.\ Products, 5th Prog.\ Rept, 1888, Q8449.}
Share-farming was the method followed by several Chinese
tobacco-growers in north-east Victoria.\fn{Kay Robertson, 1973,
pp.\,124--126.}

Another report concerns Millicent in south-eastern South Australia
where in 1888 a Royal Commission was told by S.\,J.~Stuckey and
R.~Slater that Chinese irrigated tobacco in their
district.\fn{SAPP no.\,28 of 1888, RC~Land Laws of SA, MoE,
p.\,107.}

\section*{Vegetables}

Irrigation most familiar to Australians was undertaken by Chinese in
small market gardens seen in many urban areas.  It was used also by
Chinese gardeners employed on pastoral stations.

Market gardeners of British origin also made use of irrigation in this
period.  Their gardens were more confined to the outskirts of cities.
One case concerns Bacchus Marsh in Victoria, where during the 1880s Mr
Pearce pumped from the Lerderderg River into a race taking water to an
area of chicory and spread it as required through canvas hose attached
to the aqueduct.  The hose was cut in lengths of 20\,ft, which were
joined as required to begin watering the furthest point then gradually
shortened so that a stretch of land was irrigated.\fn{The Australasian
Farmer, 1885, p.\,53.}  Others cultivated swampy areas at the south
east of Melbourne and near Perth, thus reducing the need for
irrigation.  Market gardens in the lower Torrens valley became the
main source of vegetables for the Adelaide market, with subsequent
expansion in the 1890s to the Piccadilly valley in the Mount Lofty
Range, where supplementary irrigation was widely used.  At Piccadilly,
Summertown and Uraidla, springs were the main source of water; wells
and creeks also made a contribution.  Piping was used to carry water
from springs, whose discharge was improved on some properties by
tunnelling into hillsides.  One installation took advantage of a
tunnel made earlier in a search for gold.  The gardens were usually
less than 10 acres in extent.\fn{E.\,H.~Hallack, \textsl{Toilers of
the Hills}, 1893/1987.}

During the last quarter of the century Chinese gardens existed in all
Australian colonies and supplied people in towns and the country.  By
1870 the industry which afforded the Chinese most employment was
`market-gardening of which they had almost a monopoly; 75 per cent of
the whole of the vegetables being grown by
Chinese'.\fn{T.\,A.~Coghlan, \textsl{Labour and Industry in
Australia}, 1918, vol.\,3, p.\,1\,331.}  A French visitor shared this
view in 1882, claiming market-gardening and cabinet making were
industries almost completely in the hands of Chinese.\fn{E.\,M.~La
Meslee, \textsl{The New Australia} ,1883, trans.\ R.~Ward 1973,
p.\,207.}  In Melbourne a visitor in 1890 found that `half of the
vegetables sold in the market are from Chinese
gardeners'.\fn{Queenslander, 6 Sep.\ 1889.}  The Chinese gardens were
noted for careful cultivation, use of fertilizers including human
excrement, and a good supply of water.  During this period it was
recognised that the Chinese presented an example of what could be
achieved with irrigation.

By 1880 Chinese gardens were common in eastern Australia.  They were
established initially wherever Chinese congregated for alluvial
mining.  As that type of mining declined in a district, many Chinese
moved off to new minefields, returned to their homeland, or took up
other occupations, notably market-gardening in south-eastern Australia
often near centres of reef mining such as Bendigo and Walhalla in
Victoria.  In the north the discovery of alluvial goldfields attracted
thousands of Chinese, especially to the Palmer River in north
Queensland and Pine Creek in the Northern Territory.  Some of these
diggers came from Victoria and New South Wales but many were new
arrivals, apparently organised in much the same way as the earlier
flow of Chinese gold-seekers to Australia.  Chinese communities began
gardens to cater primarily for themselves but vegetables were also
sold to others.

A good supply of vegetables was important for all in the Australian
colonies.  This produce had an accepted place in the diet of Chinese,
for others it relieved the monotony in many country areas of living
off mutton, bread and tea.  One account of life on a sheep station in
the 80s described the variety of food at a neighbouring station, due
to the presence of a lady and a Chinaman.
\begin{quote}
	In the train of the squatter's wife come such luxuries and
	delusions as pastry, puddings, and preserves, and the
	beneficent Chinaman employed as a gardener brings in fresh
	`weletables' every day from his continuously irrigated plot of
	land.\fn{C.\,M.\,H.~Clark, \textsl{Select Documents in
	Australian History 1851--1900}, 1955, p.\,202.}
\end{quote}
But as well as providing relief from a monotonous diet, the vegetables
from Chinese gardens were valuable because `they undoubtedly saved
thousands of Europeans from scurvy in the goldrush days'.\fn{E.~Rolls,
\textsl{A Million Wild Acres}, 1981, p.\,201.}  This complaint was
common among those living in the inland without access to fruit and
vegetables or antidotes such as lime juice.\fn{M.~Cannon,
\textsl{Australia in the Victorian Age, Life in the Country}, 1978, p.\,26.}

During the gold rush in the `top end' of the Northern Territory in the
1880s, there were numerous Chinese gardens between Darwin and Pine
Creek.  At Bridge Creek, near Adelaide River, there were a dozen
Chinese gardeners, who could `ward off the scurvy by plentiful growths
of vegetables'.\fn{W.\,J.~Sowden, \textsl{The Northern Territory As It
Is}, 1882, p.\,40.}  Another instance of the contribution made by
Chinese gardeners in combatting scurvy relates to Milparinka in the
extreme north-west of New South Wales.  It was the centre of a
gold-rush early in the 80s and thousands lived there.  Some died from
typhoid and others from scurvy.  Then a number of Chinese `began to
drift across from the Darling River. They planted vegetable gardens
and almost simultaneously the new diet ended the
disease'~(scurvy).\fn{G.~Farwell, \textsl{Ghost Towns of Australia},
1969, p.\,65.}

Irrigation by Chinese gardeners is recorded from all colonies though
with few accounts for Western Australia.  The involvement of Chinese
gardeners with irrigation in Western Australia in this period was
insignificant until the gold rush in the south-west.  Many Chinese
were brought to the colony as coolies before the goldrush; they worked
mainly in the north on pastoral stations.  However, by 1888, after the
first discoveries of gold in the south-west, there were 54 Chinese
engaged in market-gardening in the Perth district and by 1891 that
occupation involved 102 of the barely 900 Chinese in the
colony.\fn{Jan Ryan, \textsl{Ancestors: Chinese in Colonial
Australia}, 1995.}  Vegetables were grown by Chinese, presumably with
irrigation, in the early days of the Pilbara goldfield, before its
proclamation as such.\fn{M.\,J.~O'Reilly, \textsl{Bowyangs and
Boomerangs}, 1984, p.\,54.}  Much of the market-gardening by Chinese
near Perth was conducted on swamps, where irrigation may not have been
involved but it is likely that the description of one site on the
South Perth foreshore they leased from 1882 was not unique:
\begin{quote}
	The land was criss-crossed with hand-dug canals, and pitted
	with wells.  Vegetables, varying seasonally, made a changing
	kaleidoscope of colour, their organised rows contrasting
	starkly with the unruly bamboo which flanked the gardens.
	Chinese, struggling under the weight of their yokes carrying
	watering cans, laboured for endless hours hand-watering their
	produce.\fn{Jan Ryan, 1995, p.\,11.}
\end{quote}

A vivid picture of Chinese irrigation at Ballarat was given by Henry
Cornish, who travelled from India in the late 1870s and visited many
parts of eastern Australia. At this renowned area of alluvial
gold-mining, where thousands of Chinese were settled, he was assured
that they had taught the colonists the art of market gardening.
\begin{quote}
	The way in which the Chinese convert barren wastes of land
	into flourishing gardens is a sight full of instruction for
	English agriculturists. Their method of cultivation is very
	similar to that of the Hindus, the irrigation channels and
	small reservoirs introduced in the Chinaman's garden being
	much the same as we see in India.\fn{H.~Cornish, \textsl{Under
	the Southern Cross}, 1880, p.\,157.}
\end{quote}

In 1884 an appreciation of Chinese gardeners was given by an
agricultural journalist, apparently T.\,K.~Dow who had previously
reported on irrigation in America: at Elmore in northern Victoria,
\begin{quote}
	I passed a large Chinese garden on the banks of the Campaspe,
	and the Celestials were pumping water by horse power to
	irrigate the plants.  The Chinese were the first to
	demonstrate the benefits of irrigation in northern Victoria,
	and the sooner they are extensively imitated in this
	particular the better it will be for the condition of our
	settlers in the northern areas.\fn{Australasian, 9 Feb.\
	1884.}
\end{quote}

Another acknowledgment to Chinese irrigators came in 1892 from Alfred
Deakin, a prominent Victorian politician deeply involved in the
development of irrigation: `For the supply of fresh vegetables from
farms within sufficient distance of city markets irrigation is an
effective agency, familiar enough in most towns in the shape of
Chinese gardens'.\fn{A.~Deakin, Irrigation in Australia, Year-book of
Australia, 1892.}

New South Wales, with more than three times the area of Victoria,
lagged behind the latter in population until the 1890s; its
wheat-growing was neglected until the 90s.\fn{S.~Wadham, R.\,K.~Wilson
\& Joyce Wood, \textsl{Land Utilization in Australia}, 1957, p.\,131.}
Vast inland areas were held then in large pastoral stations, many of
which employed Chinese to raise vegetables and ornamental plants by
means of irrigation.\fn{K.\,L.~Parker, \textsl{My Bush Book}, 1982.}
So common was this practice that one exception---at the Murray Downs
station on the Murray---surprised the widely-travelled journalist
G.\,A.~Brown~(`Bruni') sufficiently to remark: `One is so used to find
nothing but Chinese gardeners everywhere north of Victoria that to
meet with a European gardener is a matter of
surprise'.\fn{Australasian, 23 Dec.\ 1882.}  That these Chinese
gardeners relied on irrigation is indicated by the report that about
1890 the work available at some pastoral stations for itinerant
workers or `swaggies' in the Riverina included working the `Chinese
treadwheel' to provide water for irrigated gardens.\fn{G.\,L.~Buxton,
\textsl{The Riverina 1861--1891}, 1967, p.\,261.}

Chinese gardeners sold food in Sydney and many country towns. Hugh
McKinney, the irrigation engineer, found `the Chinaman has never been
at a loss to find suitable places for his garden near towns in the
western plains'.\fn{H.~McKinney, J.\,R.~Soc.\ NSW 1893, vol.\,27,
p.\,384.}  Their activities along the Macquarie River were noticed in
1880 by Marin La~Meslee.\fn{E.\,M.~La Meslee, 1883, trans.\ R.~Ward,
1973.}  Chinese gardens along the Darling River supplied river towns
and the mining centres of Cobar and Broken Hill.  Up to 1884,
according to McKinney, `with few exceptions, irrigation in New South
Wales was practised only by Chinamen, who in this respect may and very
possibly do claim to have been the pioneers of
civilisation'.\fn{H.~McKinney, J.\,R.~Soc.\ NSW 1893, vol.\,27.}

By 1881 Queensland held more Chinese than New South Wales and almost
as many as Victoria.  They were scattered from south to north and were
known both as industrious miners and skilful gardeners.  Alexander
Boyd knew of these people from his experiences in many districts as a
schoolmaster and agricultural journalist.\fn{G.\,N.~Logan,
W.\,A.\,J.~Boyd, ADB vol.\,7, 1979, p.\,374.}  In 1882 he described them:
\begin{quote}
	As market-gardeners they are matchless.  No soil is so poor
	that a Chinese gardener cannot raise vegetables of every
	description on it.  Manuring and irrigation are the secrets of
	their success.  Every night the gardeners may be seen swinging
	the two man bucket and transferring the necessary moisture
	from the waterhole to the heads of the rows of vegetables,
	whence it permeates through the soil and running along
	innumerable trenches, prepared for the reception of water,
	carries the fertilising medium all over the
	ground.\fn{A.\,J.~Boyd, \textsl{Old Colonials}, 1882/1974,
	p.\,236.}
\end{quote}

Alluvial gold mining was never important in South Australia and its
Chinese population remained the least of all the Australian colonies
at the end of the century.  A few Chinese thought to be descendants of
shepherds employed before 1850 became market-gardeners with a good
supply of water in the 80s from First Creek near Waterfall Gully in
Adelaide.\fn{Elizabeth Warburton, \textsl{The Paddocks Beneath, A
History of Burnside from its Beginning}, 1981, pp.\, 31 \& 192, \&
Adelaide Advertiser 4 Apr.\ 1989.}  Other Chinese raised vegetables at
Innamincka with water drawn from Coopers Creek in the 1890s.\fn{Helen
M.~Tolcher, \textsl{Drought or Deluge, Man in the Coopers Creek
Region}, 1986.}

During the period of tin-mining in north-eastern Tasmania, from the
70s, there were several Chinese gardens in the area.  With the decline
of this mining, other gardens were developed to supply Launceston and
Hobart.

Chinese market gardeners were not invariably irrigators.  In southern
Australia some of these people gave up production of vegetables at the
height of summer.  They did not need irrigation in parts of north
Queensland which remained moist throughout the year.  In regions with
two distinct seasons a lack of water in the dry season would have
limited vegetable production.  This may have been the case in the
Northern Territory, where the Chinese practice of market gardening is
not well documented despite indications that many people were
involved.  During the gold-rush period in the Territory during
1875--85, Chinese gardens supplied a variety of produce to miners and
to residents of Darwin.  One account refers to many such gardens
between Darwin and Pine Creek during the wet season when sweet
potatoes were an important product.\fn{W.\,J.~Sowden, 1882.}  Julian
Tenison-Woods, a geologist and cleric well-known in South Australia,
visited the Northern Territory in the dry season of 1886 and reported
the production by Chinese gardeners of maize, sugar- cane, sweet
potato and culinary vegetables alongside the Margaret River.\fn{SAPP
no.\,122 of 1886.}  The existence of permanent water also at several
well-known springs, as at Rum Jungle, apparently made it possible for
vegetable production to be continued by Chinese and other gardeners
using irrigation throughout the prolonged dry season in the Territory.

Chinese irrigators occupied land more by lease than by freehold
tenure.  Accounts of market-gardening in south-eastern suburbs of
Melbourne indicate that Chinese gardeners leased land from European
market-gardeners.  In the 1880s, Chen~Ah~Teak was the owner or lessee
of more than 6 market gardens around Sydney.\fn{NSWPP RC~Chinese
Gambling, 1891.}  At Bourke, Tim~Yung had by 1886 purchased an
existing garden of 6\,ac.\fn{NSW V\&P LA vol.\,6,1885--86, no.\,118a ,
RC~Water Conservation, First rept, MoE, p.\,223.}  In many parts of
northern Australia it is likely that the Chinese irrigators were
squatters.

\section*{Means of irrigation}

Irrigation in the period generally required some means of lifting
water to the designated area.  Delivery by gravitation often required
the installation of a weir or dam to divert part of the stream flow to
a race or canal discharging at the required area; this system probably
was a feature of Tasmanian irrigation but was not widely used on the
mainland.  Pumping of water from perennial rivers and creeks was
undertaken by many irrigators; many of the streams were so deeply
incised in the landscape that pumps were required to lift water to a
significant height.  In order to obtain a supply sufficient to water a
large area without delay, pumps and engines of adequate capacity and
power were needed; generally the essential equipment included a
horizontal steam engine and a centrifugal pump.  This type of pump is
adapted to lifting water to a height of 25 to 30 feet; it was useful
to those irrigators on the riverine plain in Victoria who needed to
lift water less than 10 feet.\fn{NSW V\&P, 1886, RC~Water
Conservation, 2nd Rept MoE Appendix I, McKinney pp.\,17--26.}  Some of
them depended on adequate supplies in creeks filled by overflow from
the Murray River between Echuca and Swan Hill, and as Duncan Leitch at
Gunbower soon found there were seasons when water was not available
for pumping.\fn{VicPP no.\,53 of 1885, RC~Water Supply MoE,
p.\,37.}  Such experience led to an interest in seeking more
dependable supplies of water.

The use of steam engines and centrifugal pumps was taken up at a time
when there were several Australian makers of this equipment.
J.~Robison of Melbourne was apparently the first to manufacture
centrifugal pumps in Australia; application of his products to
irrigation was advertised frequently in the press.\fn{M.~Cannon,
\textsl{Australia in the Victorian Age, Life in the Cities}, 1988,
p.\,192, \& J.\,G.~Burnell, 1934 Inst.\ Eng.\ Aust.\ J.}

John O'Shanassy, a former Premier of Victoria, acquired the Moira run
on the New South Wales side of the river in 1862 and subsequently took
up Madowla Park on the Lower Moira run---on the opposite side of the
river.\fn{S.\,M.~Ingham, Sir John O'Shanassy, ADB vol.\,5, 1974.}  His
son Matthew was irrigating at Madowla Park in 1887, using a steam
engine and centrifugal pump to discharge river water into an elevated
flume running to the homestead area 3/4\,mile away.\fn{Town and
Country Journal, 6 Aug.\ 1887, p.\,270.}  An unusual feature was the
mounting of engine and pump on a barge to allow irrigation of the
O'Shanassy lands on both sides of the river.\fn{Helen Coulson,
\textsl{Echuca--Moama, Murray River Neighbours}, 1979, p.\,108.}

So popular was the more powerful equipment, available from different
engineering firms in Victoria, that in the 1890s there were said to be
90 pumping plants along the river between Swan Hill and
Echuca.\fn{A.~Feldtmann, \textsl{Swan Hill}, 1973, p.\,127.}

How did the Chinese water their gardens?  Illustrations in the 19th
century generally show a man carrying two buckets of water at the ends
of a bamboo stick or yoke carried over the shoulder.  But there was
considerable variety in their arrangement of water supply.  In New
South Wales during the 1880s Chinese gardeners were using steam
engines to pump water from rivers or wells.  Tim~Yung at Bourke drew
water from a well by means of a 6\,hp steam-driven pump.  On the
Hawkesbury River, Chinese gardeners also used steam power to pump
their water from the river.  Other Chinese used horse-powered pumps,
as at Elmore and Katamatite in the Goulburn Valley of
Victoria.\fn{Australasian, Feb.\ 1884, p.\,186, \& A.\,J.~Dunlop,
\textsl{Wide Horizons}, 1978, p.\,51.}  Innamincka in South Australia
was one of few places known to have a water-wheel raising supplies to
a Chinese garden of about one acre.  It was described as `a
magnificent water-wheel, the workmanship of which was admired by all
who saw it'.\fn{Helen M.~Tolcher, 1986, p.\,107.}  One of the simplest
machines for raising water was the type of pump first used in alluvial
mining and described in 1853:
\begin{quote}	
	Others use a Chinese pump, called a belt-pump, which the
	Chinese took to California, and which Californian diggers are
	using here.  The belt- pump consists simply of a long wooden
	pipe or tunnel, about six inches square, at the upper end of
	which is a wheel turning a long band of canvas, the two ends
	of which are sewed together so that it forms a circle.  On
	this band are fixed upright square pieces of board at regular
	distances; and as the wheel is turned, these pieces of board
	move onward with the band, enter the lower end of the tunnel,
	and carrying the water with them, discharge it at the
	mouth.\fn{W.~Howitt, \textsl{Land, Labour and Gold},
	1855/1972, p.\,97.}
\end{quote}

In urban areas, some market gardeners depended on mains supply water.
Thus in Hawthorn, Victoria, Chinese gardens became established at a
distance from the Yarra River only after the reticulated supply of Yan
Yean water became available from 1870.\fn{Gwen McWilliam,
\textsl{Hawthorn Peppercorns}, 1978, p.\,81.}  Mains supply water was
also the basis for irrigation in South Australia at Stirling
North.\fn{J.\,J.~Green \& A.~Molineux, \textsl{Irrigation}, 1892.}

%\section*{References}
%1. NSW V\&P 1886, R.C.Water Conservation, 2nd Rept, MoE Appendix I,
%    McKinney pp.17-26.	        
%2. Australasian, 18/11/1882.
%3. Australasian, 7/1/1882.
%4. J.Inst.Eng.Aust. 1930,vol. 2, p.270, VicPP No.53 of 1885, MoE p.37,\&
%    Leader, 1/12/1883, VicPP No. 53 of 1885, MoE p.32.
%5. VicPP Vol.4,1884, VicYearbook 1884, p.436. 
%6. VicPP No 53 of 1885,p.xxxviii.
%7. Australasian, 1/1/1887.
%8. W.Williams, A History Of Bacchus Marsh. . . 1836-1936, 1936, \&
%    VicPP No 53 of 1885, MoE p.271.
%9. Susan Priestley, The Victorians: Making Their Mark, 1984, p.76.
%10. VicPP No.53 of 1885, p.xxxvii.
%11. Australasian, 9/1/1886.
%12. Australasian, 23/12/1882.
%13. Town and country journal, 6/8/1887.
%14. VicPD 1886,p.2763.
%15. NSW V\&P 118 of 1885.
%16. NSW R.C.Water Conservation First Rept 1885-86; C.J.Lloyd, Either
%      Drought Or Plenty, 1988,p.53.  
%17. G.L.Buxton, The Riverina 1861-1891, 1967. p.246, \&  Australasian,
%      24/1/1890.
%18.  Procs First Vict.Irrig.Conf. 1890, p.130.
%19. Leader, 12/9/1885.
%20. NSW Agr.Gaz. Vol.2, 1891, p.163.
%21. K.Jeffcoat, More Precious Than Gold, 1988, p.118.
%22. NSW Agr.Gaz.vol.2, 1891, p.163.
%23. NSW Agr.Gaz., vol.3, 1892, p.601.
%24. NSW V\&P 1885, R.C.Water Conservation, MoE, pp160-164.
%25. Brisbane Courier, 28/3/1884.
%26. Queenslander, 6/4/1889.
%27. NSW Agr.Gaz. vol.10, 1899, p.802.
%28. Australasian, 3/3/1888, p.466.
%29. C.S.Martin, Irrigation And Closer Settlement In The Shepparton District,
%      1836-1906, 1955, p.57.
%30. Australasian, 10/5/1890, p.898, J.M.Powell(ed), Yeomen And Bureaucrats,
%      1973, p.xxiv.
%31. NSW Agr.Gaz., vol.3 1892, p.711.
%32. Town and Country Journal, 6/8/1887.
%33. Susan Priestley, Echuca, A Centenary History, 1965, p.155.
%34. H.Oldham, Irrigation And Water Conservation In Western Australia,
%      1913,p.11,  \& ADB vol 7, p.70.
%35. J.J.Green and A.Molineux, Irrigation, 1892, p.21.
%36. K.Preiss \& Pamela Oborn, The Torrens Park Estate, 1991, p.229.
%37. R.J.Anderson, Solid Town, History Of Port Augusta, 1988, p.62.
%38. H.E.A.Eklund, Irrigation In Queensland, Qld Agric.J. vols 20 \& 21, 
%     1923-24.   
%39. Melbourne Argus, 10/1/1884.
%40. SAPD 1887, p.310.
%41. Queenslander, 25/10/1890.
%42. NSW V\&P1886, R.C.Water Conservation, 1st Rept,MoE p.223.
%43. Australasian, May 1882, p.538.
%44. VicPP No.53 of 1885, R.C.Water Supply, Further Progr.Rept
%      MoE, p.255.
%45. VicPP 1884, Statistical Register.
%46. Helen Pearce, The Hop Industry In Australia, 1976, p.91.
%47. Carole Woods, Beechworth, A Titan's Fields, 1985, p.118.
%48. Vic R.C.Vegetable Products, 5th Progr.Rept 1888, Q8711.
%49. Australasian, 18/3/1882.
%50. Vic R.C.Vegetable Proucts, 5th Progr.Rept,1888, Q8699.
%51. J.Carter, Stout Hearts And Leathery Hands, 1968, p.127; Kay
%      Robertson, Myrtleford, Gateway To The Alps, 1973, p.106.
%52. Helen Pearce, 1976, p.70 \& ADB Vol.5, David Murray, p.319.
%53. QldPP, Vol.iv, 1889, R.C.Sugar Industry.
%54. G.C.Bolton, A Thousand Miles Away, A History Of North Queensland
%       to 1920, 1972, p.136. 		
%55. QldPP 1889, vol.iv, R.C.Sugar Industry.
%56. QldPP 1889, R.C.Sugar Industry, MoE.
%57. H.E.A.Eklund, Irrigation In Queensland, Qld Agr.J. vol.20, 1923,
%      p.105-06, vol.21, 1924, pp. 289-308.
%58. Vic R.C.Veg.Products, 5th Prog.Rept,1888, p.30.
%59. Vic R.C.Veg.Products, 5th Prog.Rept, 1888, Q8449.
%60. Kay Robertson, 1973,pp.124-126.
%61. SAPP No.28 of 1888, R.C.Land Laws of S.A., MoE, p.107.
%62. The Australasian Farmer, 1885, p.53.
%63. E.H.Hallack, Toilers Of The Hills, 1893/1987.
%64. T.A.Coghlan, Labour And Industry In Australia, 1918, vol.3, p.1331.
%65. E.M.La Meslee, The New Australia,1883,  trans.R.Ward 1973, p.207.
%66. Queenslander, 6/9/1889.
%67. C.M.H.Clark, Select Documents In Australian History 1851-1900,
%       1955, p.202.
%68. E.Rolls, A Million Wild Acres, 1981, p.201.
%69. M.Cannon, Australia In The Victorian Age, Life In The Country, 1978,
%       p.26.
%70. W.J.Sowden, The Northern Territory As It Is, 1882, p.40.
%71. G.Farwell, Ghost Towns Of Australia, 1969, p.65.
%72. Jan Ryan, Ancestors: Chinese In Colonial Australia, 1995.
%73. M.J.O'Reilly, Bowyangs And Boomerangs, 1984, p.54.
%74. Jan Ryan, 1995, p.11.
%75. H.Cornish, Under The Southern Cross, 1880, p.157.
%76. Australasian, 9/2/1884.
%77. A.Deakin, Irrigation In Australia, Year-book of Australia,1892.
%78. S.Wadham, R.K.Wilson \& Joyce Wood, Land Utilization In
%      Australia, 1957,p. 131.
%79. K.L.Parker, My Bush Book, 1982.
%80. Australasian, 23/12/1882.
%81. G.L.Buxton, The Riverina 1861-1891, 1967, p.261.
%82. H.McKinney, J.R.Soc.NSW 1893, v.27, p.384.
%83. E.M.La Meslee, 1883, trans. R.Ward, 1973.
%84. H.McKinney, J.R.Soc.NSW 1893, v.27.
%85. G.N.Logan, W.A.J.Boyd, ADB Vol.7, 1979, p.374.
%86. A.J.Boyd, Old Colonials, 1882/1974, p.236.
%87. Elizabeth Warburton, The Paddocks Beneath, A History Of Burnside from
%       Its Beginning, 1981, pp 31 \& 192, \& Adelaide Advertiser 4/4/1989.
%88. Helen M.Tolcher, Drought Or Deluge, Man In The Coopers Creek
%    Region, 1986.
%89. W.J.Sowden, 1882.
%90. SAPP No. 122 of 1886.
%91. NSWPP R.C.Chinese Gambling, 1891.
%92. NSW V\&P LA Vol.6,1885-86, No 118a , R.C.Water Conservation, First 
%      rept, MoE, p.223. 
%93. NSW V\&P, 1886, R.C.Water Conservation, 2nd Rept 
%      MoE Appendix I, McKinney p.17-26.		  
%94. VicPP No.53 of 1885, R.C.Water Supply MoE, p.37.
%95. M.Cannon, Australia In The Victorian Age, Life In The Cities, 1988,
%       p.192,\&  J.G.Burnell, 1934 Inst.Eng.Aust.J.
%96. S.M.Ingham, Sir John O'Shanassy, ADB vol.5,1974.
%97. Town and Country Journal, 6/8/1887, p.270.
%98. Helen Coulson, Echuca-Moama, Murray River Neighbours, 1979, p.108.
%99. A.Feldtmann, Swan Hill, 1973, p.127.
%100. Australasian, Feb.1884, p.186, \& A.J.Dunlop, Wide Horizons, 1978,
%        p.51.
%101. Helen M.Tolcher, 1986, p.107.
%102. W.Howitt, Land,Labour And Gold, 1855/1972, p.97.
%103. Gwen McWilliam, Hawthorn Peppercorns, 1978, p.81.
%104. J.J.Green \& A.Molineux, Irrigation, 1892.
