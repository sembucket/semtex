% $Id$
% CHAPTER SEVEN
% 6387 words at 30/4/99

\chapter{Official Inquiries and Outcomes 1880--1890}

\index{inquiries}
\label{ch:inquiries}\addtoendnotes{\protect\subsection*{Chapter \thechapter}}
\fancyhead[RE]{\sffamily \small Chapter \thechapter.\ %
               Official Inquiries}

\setcounter{endnote}{0}

Official interest in irrigation on the Australian mainland was first
sh\-own in the 1880s, generally linked to the provision of water
supplies to country areas.  In earlier times the needs of pastoralists
had been met by their own efforts including provision of wells,
excavation of tanks or dams, and some diversion of streams by means of
weirs.  When extensive areas with low rainfall were taken up later for
\index{wheat}wheat-growing on comparatively small holdings, water
supply became crucial.  Its provision was more difficult in those
parts far from perennial streams, as in much of South Australia and
north-western Victoria, than in northern Victoria where desperate
residents might cart water from rivers.

On the other hand, irrigation was not generally regarded in Australia
as a matter of urgency, though its advantages had been demonstrated in
many localities, advocated enthusiastically by promoters of the
Victorian \index{technology!canal}canal scheme, and
endorsed by some engineers with experience of Indian irrigation.  The
following account of official inquiries during the 1880s concerning
rural water supply traces the involvement with irrigation of
authorities in Victoria, New South Wales, South Australia, and
Tasmania.  Reference to official inquiries following government
involvement with irrigation is made in other chapters.

\section*{Victorian Water Conservancy Board}
\index{Vic.\ Water Conservancy Board}

Agricultural settlement of northern Victoria at first proceeded
without great difficulty for several years after 1870.  Then periods
of low rainfall led to poor yields of \index{cereals}cereal crops as
well as shortages of water for homes and livestock.  Some waterworks
already existed in local government areas but they were insufficient.
The Victorian government responded by appointing a Water Conservancy
Board in 1880.  The decision was made in autumn by Robert Clark,
\index{Clark, R.} then MP for Bendigo and Minister for Mines and Water
Supply in the Service Government.  The Board had two members: George
Gordon, \index{Gordon, G.} formerly chief hydraulic engineer, and
Alexander Black, \index{Black, A.} deputy surveyor-general.  They were
directed `to inquire and report, first, as to the feasibility of
providing, at a reasonable cost, a supply of water to the northern
plains for domestic purposes, and for the use of stock; and second, as
to irrigation'. This directive indicates the urgent need for stock and
domestic water supplies and the public interest in irrigation probably
associated with the controversial grand canal scheme
\index{canal!scheme!Vic.} for northern Victoria.\fn{A.\,S.~Kenyon,
\textit{Vic.\ Hist.\ Mag.}, vol.\,10, (1925), p.\,114; Vic.\ PP
no.\,18 of 1881, vol.\,II, LA; G.~Gordon, \textit{Min.\ Proc.\ Inst.\
Civ.\ Eng.}, vol.\,142, (1900), p.\,326.}

Gordon and Black began their work by conferring with shire authorities
about existing works before making their own major proposals for water
supply within each of several areas associated with particular streams
and for the necessary administrative and \index{finance}financial
arrangements.  The results of their work were twelve reports made at
intervals and brought together in the Board's submission of September
1881 entitled Supply of Water to the Northern Plains.\fn{Vic.\ PP
no.\,18 of 1881, LA vol.\,II.}

The schemes proposed by the Board generally involved weirs on major
streams and diversion of supplies to fill earthen tanks at no more
than three miles from any holding. Additional supplies to these tanks
could be provided by provision of
\index{technology!channel}channels and plough lines to
intercept local runoff.  What the Board proposed was really a more
wide\-spread use of stream diversion first used by
pastoralists\,---\,notably from the Wimmera River
\index{river!Wimmera} more than 30 years
previously.\fn{\textit{Australasian}, 8 Apr.\ 1882; L.\,R.~East,
\textit{Vic.\ Hist.\ Mag.}, vol.\,38, (1967), p.\,197.}

The proposals for trusts \index{trust!irrigation!Vic.} to administer
waterworks, given in the fifth of the Board's twelve reports, showed
an intention to decentralise management of water supply in northern
Victoria.  Each trust would be derived from local government bodies in
the same way as for existing urban waterworks trusts; they would have
authority `to carry out the schemes with money lent by the government
at a low rate of interest, secured by mortgages over the works and the
rates to be levied'.\fn{G.~Gordon, \textit{Min.\ Proc.\ Inst.\ Civ.\
Eng.}, vol.\,142, (1900) p.\,326.}

Irrigation was mentioned in the Board's report in 1881 but only in
brief references to the opportunities in the areas covered by its
Goulburn and Gunbower schemes.  Major attention was given in the
Board's two reports on irrigation.  The first, a general statement in
September 1882 relied on experience abroad to support the claim that
extensive irrigation was of no immediate importance for the sparsely
populated areas of northern Victoria.  A scheme to irrigate land
between the Goulburn and Campaspe Rivers was provided in the second
report of March 1884. It involved construction of a weir on the
Goulburn River, \index{river!Goulburn} \index{river!Campaspe} high
enough to allow diversion for storage in a nearby extensive depression
and distribution by \index{technology!channel}channels over
the irrigable area.  The Board's initiation of Goulburn River gaugings
in June 1881 had enabled it to estimate the yield of water.  This was
calculated as allowing irrigation in winter of approximately
200\,000\,acres or 60\,000\,acres in summer.  Though technical
problems of the project were regarded as manageable, the general lack
of population in the area restrained the Board from advocating any
immediate action.\fn{Vic.\ PP no.\,74 of 1882; Supply of water to the
northern plains, Irrigation, Second Rept, 20 Mar.\ 1884, by G.~Gordon
and A.~Black, Vic.\ Dept.\ Mines and Water Supply.}

\subsection*{The Response}

The response varied from commendation in the metropolitan press to the
opposition expressed persistently by Hugh McColl, \index{McColl, H.}
the well-known advocate of the grand canal scheme who early in 1880
was elected to parliament and in 1881 criticised the Board's proposals
as inadequate.  The Government upheld the Board's administrative
proposals and acted promptly by introducing legislation to provide for
waterworks trusts in the northern areas, thus obtaining the Water
Conservation Act 1881. \index{legislation!Vic.!Water Conserv.\ Act
1881} This measure came at the end of a year with abnormally low
rainfall and crop failures in parts of the northern
districts.\fn{J.\,H.~McColl, \textit{Vic.\ Hist.\ Mag.}, vol.\,5,
(1917), p.\,157.}

The first quarter of 1882 was so dry in northern Victoria that
\index{technology!railways}railways were used to bring water to some
districts and several politicians lobbied the government for speedy
action on the recent legislation.  In the Loddon valley
\index{river!Loddon} moves were made to set up a Waterworks Trust.
During February churchmen of several denominations called for a day of
prayers for rain but the Anglican Bishop Moorhouse \index{Moorhouse,
Bishop} advised his followers to look instead to the government for
their water supply.  He journeyed to the
\index{drought}drought-stricken town of Kerang \index{Kerang, Vic.} in
March to lecture on irrigation.  Even though good rains came before
winter, concern about water supply was shown by establishment in 1882
of 12 waterworks trusts under the recent \index{legislation!Vic.!Water
Conserv.\ Act 1883} legislation.\fn{\textit{Australasian}, 4 Mar.\
1882; Vic. Water Conserv.\ Act 1883, no.\,778, 3rd Schedule.}

The Echuca Shire \index{Echuca, Vic.} council had in autumn proposed a
scheme for irrigation of several hundred thousand acres from the
Goulburn River. In July 1882 its representatives and those of the
adjoining Waranga \index{Waranga, Vic.} Shire met in Melbourne to
discuss irrigation of land between the Goulburn and Campaspe
Rivers\,---\,the same area considered in the second report on
irrigation by the Water Conservancy Board in 1884. Being unable to
pursue the matter of irrigation further at that stage, the two bodies
undertook the formation in October 1882 of the United Echuca and
Waranga Waterworks Trust, hoping that the supply
\index{technology!channel}channels to be constructed for
water supply would also serve for delivery of irrigation water.  The
Minister for Water Supply, Charles Young,
\index{Young, C.} was reported as being indifferent about the use of
the water.\fn{\textit{Australasian}, 1 Apr.\ 1882; A.\,S.~Kenyon,
\textit{Vic.\ Hist.\ Mag.}, vol.\,10, (1925), p.\,115;
\cite[p.\,45]{martin1955};
\textit{Australasian}, 16 Dec.\ 1882.}

During 1883 several organisations in the northern districts were
displaying discontent about water supply or irrigation.  A conference
of six shire councils in the northwestern districts sought government
construction of all \index{technology!reservoir}reservoirs and weirs
required for water supply; these were described as national works, a
term soon in wide usage.  The councils also agreed that if the
government would meet their request there should be no need for rural
waterworks trusts.\fn{\textit{Australasian}, 7 July 1883.}

A new development was the formation of irrigation leag\-ues.  Their
origin may have been inspired by the earlier existence of
\index{technology!railways}railway leag\-ues in Victoria.  Probably
the earliest was the North-western Water Conservation and Irrigation
League, \index{North-western Water Conserv.\ and Irrig.\ League} with
members at Swan Hill, \index{Swan Hill, Vic.} Kerang \index{Kerang,
Vic.} and in the Loddon Valley.  \index{river!Loddon} By July 1883 it
had Hugh McColl \index{McColl, H.} as president and Elisha De Garis
\index{De Garis, E.\,C.} as vice-president. De Garis was then a
Methodist preacher at Durham Ox \index{Durham Ox, Vic.} in the Loddon
Valley, a staunch advocate of irrigation who had worked for his church
in different parts of the northern plains, including the Goulburn and
Campaspe valleys. \index{river!Goulburn} \index{river!Campaspe} Later
in the year several irrigation leagues meeting at Echuca
\index{Echuca, Vic.}  agreed to form the Central Irrigation League,
with De Garis as president.\fn{\textit{Australasian}, 3 July, 1 Sep.\
1883.}

Another important development was the decision of two Melbourne
newspapers, the \textit{Age} and \textit{Argus}, to send journalists
to the USA \index{USA} in 1883 to report on agricultural developments.
Their numerous despatches, with many references to irrigation,
appeared in the widely circulated \textit{Australasian} and
\textit{Leader}, weekly papers which from that time gave increasing
attention to irrigation.

These developments came soon after Alfred Deakin's \index{Deakin, A.}
appointment as Minister for Public Works and Water Supply following
the change of government in March 1883.  Taking heed of the strong
irrigation lobby represented by the new irrigation leagues and the
parliamentarian Hugh McColl, \index{McColl, H.} Deakin arranged
amendment of the Water Conservation Act \index{legislation!Vic.!Water
Conserv.\ Act 1883} to provide for irrigation trusts.  This change,
late in 1883, pleased McColl but did nothing to encourage the
formation of such trusts because no funds could be provided by the
government to support irrigation and the method of formation of
irrigation trusts was restrictive and
cumbersome.\fn{\cite[p.\,80]{lanauze1979}.}

Finally, the Water Conservation Board was abolished in 1884 after
completing its second report on irrigation but before completing a
third one dealing with the Gunbower district\,---\,a section of the
riverine plain near the Murray River \index{river!Murray} which
includes the towns of Cohuna \index{Cohuna, Vic.} and \index{Kerang,
Vic.}  Ke\-rang.\fn{G.~Gordon, \textit{Trans.\ Proc.\ Vic.\ Eng.\
Ass.}, vol.\,1, (1883--85), p.\,144.}

\section*{New South Wales Royal Commission on Water Conservation 1884}
\index{Royal Commission!NSW!Water Conserv.\ 1884}

Irrigation was one of several matters considered by the Royal
Commission on Water Conservation appointed in May 1884 in response to
concern about water supply following recent serious
\index{drought}droughts in the colony.  From about 1880 there had been
successful interception of artesian water in parts of the colony and
this led to interest in the extent of this resource.  H.\,C.~Russell
\index{Russell, H.\,C.} had previously estimated from rainfall records
and data on flow of the Darling River \index{river!Darling} that it
failed to discharge much of the flow entering the higher reaches of
the stream from tributaries; he concluded that much of the water went
to underground reserves.

William Lyne, \index{Lyne, W.} representing an electorate near Albury,
was the prime mover for a royal commission to consider the problem of
water conservation.  He had pressed unsuccessfully for an inquiry in
February 1883, when he referred to interest in artesian water and the
recent Victorian report on irrigation by Gordon and Black.  He
returned to this issue in parliament a year later when his request
again mentioned the uncertainty about underground water resources.  He
was not so much concerned at that time to obtain a complete system of
irrigation but wished to see the foundations laid for its development
in the remote future.  The Secretary for Mines, J.\,P.~Abbott,
\index{Abbott, J.\,P.} responded favourably to the request, saying
that a decision depended on finding men to serve on the Commission
`whose services will be of value to the country'.\fn{H.\,C.~Russell,
\textit{J.\,Proc.\ R. Soc.\ NSW}, vol.\,13, (1879), p.\,169; NSW PD
Sess.\ 1883--84, vol.\,11, p.1605--8.}

The Royal Commission was appointed
\begin{Quote}
	to make a diligent and full inquiry into the best method of
	conserving the rainfall and of searching for and developing
	the underground resources supposed to exist in the interior of
	this colo\-ny, and also into the practicability, by a general
	system of water conservation and distribution, of averting the
	disastrous consequ\-en\-ces of the periodical
	\index{drought}droughts to which the colony is from time to
	time subject.\fn{NSW PP, 1885--86, vol.\,6. RC~Water Conserv.,
	First Rept.}
\end{Quote}
The members of the commission were legislators, pastoralists, and
engineers. Its president was William Lyne, a landholder with
experience in Tasmania and Queensland before settling in New South
Wales. Hugh McKinney \index{McKinney, H.\,G.} (1847--1930) was the
engineer assisting the enquiry; he had been involved with irrigation
in India before coming to the colony in 1879 for employment with water
supply bodies.\fn{\cite{cunneen1986}; Lorna M.~Darbishire,
\textit{H.\,G.~McKinney, His Life And Work}, ML MSS 706.}

In its first report in December 1885 the Commission showed that the
scope of its work included enquiries overseas and in various parts of
the colony.  One of its members, the engineer F.\,A.~Franklin,
\index{Franklin, F.A} had previously been directed to make enquiries
about water conservation in India while there on other business.  He
visited several parts of northern India to obtain information on
irrigation \index{technology!canal}canals and also learnt details of
irrigation tanks in southern India and the use of underground water.
His report in July 1884, dealing with canals and irrigation in India,
\index{India} was given in an appendix to the first report.  The
Commission applied to the Colonial Office for relevant information
from \index{Europe}Europe and \index{America}America, leading
eventually to the acquisition of relevant publications from several
countries.\fn{NSW PP 1885--86, vol.\,6. RC~Water Conserv., First
Rept.}

Members of the Commission visited parts of the Riverina and towns in
the north and north-west, and collected evidence from more than one
hundred people, including the Victorian engineer George
Gordon. \index{Gordon, G.}  Charles Robinson, \index{Robinson, C.}
secretary of the Commission, provided a long report in May 1885 on a
recent visit to northern Victoria where he attended meetings of the
Victorian Royal Commission on Water Supply at different townships for
collection of evidence. He provided details of the Waterworks Trusts
and statistics of Victorian irrigation in 1884.\fn{NSW PP, 1885--86,
RC~Water Conserv., First Rept, Appendices.}

The Commission gave attention to possible diversion of water from the
Snowy River \index{river!Snowy} to the Murrumbidgee,
\index{river!Murrumbidgee} as well as making
enquiries about other inland rivers, coastal streams, and underground
water supplies.  The scope of the first report included meteorology,
water storage, irrigation, navigation, \index{riparian rights}riparian
rights, and proposed legislation; apparently the Commission took a
wide view of its responsibilities.

A second report was made in June 1886, dealing extensively with
matters of common interest to Victoria\,---\,including the diversion
and use of waters of the Murray River.  \index{river!Murray} Meetings
had been held with the Victorian Royal Commission on Water Supply in
Melbourne in January 1886 and in Sydney in May 1886.  Although New
South Wales included an extensive stretch of the Murray River, the two
bodies agreed on 10 resolutions concerning use by the two colonies of
\index{flood}flood waters in the main stream and its tributaries under the
management of a Trust with equal representation of the two colonies.
The New South Wales Commission had made visits to Victorian storages
(Yan Yean and Coliban) \index{reservoir!Yan Yean}
\index{reservoir!Coliban} and its engineer, Hugh McKinney,
\index{McKinney, H.\,G.} had paid special attention to the Gunbower area 
of Victoria where he interviewed several irrigators.\fn{NSW PP
1885--86, vol.\,6, RC~Water Conserv., Second Rept \& Appendix.}

The third and final report issued in May 1887 provided general
conclusions and recommendations, together with minutes of evidence
tak\-en since the previous reports.  It also showed in an appendix the
correspondence and publications received following enquiries made
overseas.  The emphasis of the conclusions was on irrigation rather
than supplies for livestock; government works to provide water for
stock were needed only on a limited scale, but the `great object of
water conservation in this colony, and particularly for the country
west of the Dividing Range, is for irrigation'.  The purposes of
irrigation were found to be: the provision of \index{fodder}fodder for
livestock, to provide for reserve stocks of fodder for use in bad
seasons, production of fruit, vegetables and miscellaneous crops, and
an increase in the productive powers of the land.  Legislation giving
state ownership of all rivers and watercourses should be given early
consideration.  Recommendations were made about river gauging records
and schemes for \index{technology!canal}canals for irrigation of
almost one million acres from the Murray \index{river!Murray} and
Murrumbidgee \index{river!Murrumbidgee} Rivers.  The concern of this
report with irrigation is shown also by its attention to the interest
of New South Wales, Victoria, and South Australia in the use of Murray
River waters and its hope that these colonies would confer on the
matter, possibly later in 1887.  By its references to lack of
information and an unsatisfied application for funds to conduct
surveys, this report indicates that the Commissioners were unable to
complete their work.  The Commission was abolished by Henry Parkes,
\index{Parkes, H.} who had become Premier early in 1887.\fn{NSW PP,
1887 vol.\,5, 2nd Sess. RC~Water Conserv., Final Rept.;
\cite[p.\,157]{lloyd1988}}

The Royal Commission produced a vast amount of detailed material on
water supply in country districts of New South Wales and made several
major proposals for stream diversion in the interests of irrigation.
Its recommendation for state ownership was a challenge to the
prevailing system of \index{riparian rights}riparian rights.

\subsection*{The Response}

The recommendations had no significant impact at the time in New South
Wales.  Support was apparently given to the uncontentious mov\-es to
collect more information on stream flow, rainfall, and under\-gr\-ound
water resources, but state ownership of water resources was not
introduced then.  Water conservation had receded from immediate
interest during a run of several years free from
\index{drought}drought except in 1888.\fn{W.\,J.~Gibbs \&
J.\,V.~Maher, \textit{Comm.\ Bur.\ Meterol.\ Bull}.\ no.\,48, (1967).}

There was no group of people comparable with those in northern
Victoria who campaigned strongly for irrigation.  This difference was
related to the comparatively minor role then occupied by agriculture
in New South Wales and the fact that it was confined to parts of the
colony with adequate rainfall for \index{cereals}cereal production.
The country districts were still used mainly for the pastoral
industry; legislation on land tenure continued to hinder the
proliferation of agricultural holdings with residential requirements.
The general administration of water supply remained under the control
of the Minister for Mines, as it had since the goldrush era.

All this was as William Lyne \index{Lyne, W.}  might have expected
when in calling for the Royal Commission in February 1884, he
indicated that he had no immediate expectation for development of a
comprehensive system of irrigation.

\section*{Victorian Royal Commission on Water\\ Supply}
\index{Royal Commission!Vic.!Water Supply 1884} 

Hugh McColl \index{McColl, H.} was the first Victorian legislator to
seek a royal commission on water supply when he raised the matter with
Deakin \index{Deakin, A.} in 1883, not long after Lyne's initiative in
New South Wales. In November 1884, after appointment of the Lyne Royal
Commission on Water Conservation in New South Wales, McColl asked the
Victorian Premier to consider appointing a Royal Commission on Water
Supply. Mr~Service indicated that his government was aware of the
Commission already appointed in New South Wales and would soon attend
to the matter.

Two weeks later the appointment of a Royal Commission on Water Supply
was announced. It was directed `to inquire into the question of Water
Supply, and into matters relating thereto' and `to make diligent and
full inquiry into the operation and effect of the various schemes for
water supply, and into the extent to which the present sources of
water supply are utilised, with a view to ascertain whether further
and better provision can be made for the conservation and distribution
of water for the use of the people'. This decision acknowledged the
continuing dissatisfaction in northern Victoria over water supply,
especially the lack of an irrigation system.\fn{Vic.\ VP LA 2nd
Sess.\ 1883, vol.\,1, Surface Irrigation Canals Memorial, p.\,653;
Vic.\ PD 1884, vol.\,47, p.\,2298; Vic.\ PP no.\,53 of 1885, RC~Water
Supply, Further Progr.\ Rept.}

Members of the Commission were predominantly legislators, including
Hugh McColl among those representing electorates in northern Victoria,
together with three civil engineers, a meteorologist and a surveyor.
Alfred Deakin MP (1856--1919) was President and Stuart Murray,
\index{Murray, S.} an
engineer, was secretary.  Deakin represented a central Victorian
electorate with more than 400\,ac under irrigation near Bacchus Marsh;
he was also a lawyer, journalist, and protege of David Syme,
\index{Syme, D.} the
influential proprietor of \textit{The Age} newspaper.  On his
appointment to the Commission, Alfred Deakin prepared to visit the
United States
\index{USA} to
gather information on irrigation and on Christmas Eve began his
journey.\fn{Deakin Papers, NL MS1540/2, 1884 Diary.}

The Commission produced three progress reports in 1885 and a four\-th
in 1887.  The first report briefly announced the early work: meetings
in Melbourne, visits to waterworks, and collection of evidence; it
includes Deakin's \index{Deakin, A.} long memorandum `Irrigation in
Western \index{America}America. So far as it has relation to the
circumstances of Victoria' resulting from his tour of North America
from 25 January to 12 April.  His visit received much attention in the
Melbourne press as a result of despatches from the two journalists
accompanying him\,---\,J.\,L.~Dow, \index{Dow, J.\,L.} MP and a member
of the Royal Commission, and E.\,S.~Cunningham. \index{Cunningham,
E.\,S.}  Inspections were made principally in central and southern
\index{California}California, New Mexico, Nev\-ada, Colorado, and in
the Mexican Republic. \index{Mexico}

Deakin's report refers to many aspects of irrigation in western
\index{USA}USA: environmental, legal, \index{finance}financial,
technical, agricultural, and social.  The author drew several
conclusions and made recommendations concerning Victorian irrigation.
Seven of these concerned administration: State ownership of all
sources of surface water supply except springs on private lands; State
disposal of water to irrigators, a standard scale for water
measurement, appointment of water-masters to supervise distribution,
provision for irrigators to obtain easement over private lands subject
to compensation, full information to be provided by the State on the
scope for irrigation, and local control of irrigation.  He also added
proposals for agricultural colleges to train irrigators, conferences
of practical irrigators, prizes for successful farming including
irrigation, and experimental irrigation.\fn{Vic.\ PP no.\,19 of 1885,
RC~Water Supply, First Progr.\ Rept.}

The second or Further Progress report is entitled `Report on Some of
the Engineering Features of \index{America}American Irrigation'; it
was made by J.\,D.~Derry, \index{Derry, J.\,D.} the engineer who
accompanied Deakin on his tour in North America.  His report dealt
with \index{technology!dam}dams, weirs, flumes, artesian wells, canal
works, pipes and conduits, the Ontario \index{Canada} colony,
excavating plant and machines, and the distribution of water.\fn{Vic.\
PP no.\,19a of 1885, RC~Water Supply, Further Progr.\ Rept.}

The third or Further Progress report, in August 1885, contains fifteen
pages of findings and recommendations on sources of water supply,
existing waterworks, and purposes of water supply.  It was accompanied
by hundreds of pages of evidence taken in Victoria during 1885.
Interviews in several parts of the colony were generally conducted by
only a few members of the Commission, notably W.\,W.~Culcheth,
\index{Culcheth, W.\,W.} an
engineer.  Districts visited for the purpose included northern
Victoria except for the Goulburn Valley, \index{river!Goulburn} the
Wimmera, \index{river!Wimmera} Gippsland, \index{Gippsland, Vic.} and
central Victoria.  After his return from abroad, Alfred Deakin was
involved in the collection of evidence but only at Bacchus Marsh
\index{Bacchus Marsh, Vic.} and \index{Melton, Vic.} 
Melton\,---\,within his electorate\,---\,and at Melbourne where his
time was spent mainly with E.\,C.~De Garis, \index{De Garis, E.\,C.}
the Methodist preacher who had become the spokesman of the northern
districts movement for irrigation after the death of Hugh McColl
\index{McColl, H.}
earlier in 1885.\fn{Vic.\ PP no.\,53 of 1885, RC~Water Supply, Further
Progr.\ Rept.}

Much of the third progress report concerns urban and rural water
supply rather than irrigation.  Its recommendations or proposals
relevant to irrigation included: to accept the proposal of the New
South Wales Royal Commission on Water Conservation \index{Royal
Commission!NSW!Water Conserv.\ 1884} for a conference on questions of
water supply affecting the two colonies, that the State should
exercise supreme control of ownership over all rivers, lakes, streams,
and sources of water supply, that the current requirements for
formation of irrigation trusts should be made less restrictive, that
loan funds should be made available by the State for construction of
irrigation works by the trusts, and that special provision of loan
funds should be made where great storage
\index{technology!reservoir}reservoirs are needed for water supply.
This third report shows the Commission was satisfied that rural
waterworks trusts \index{trust!waterworks} established a few years
previously were meeting their responsibilities by making supplies
available at no more than three miles from a farm holding for domestic
use and livestock but they were generally not meeting their
\index{finance}financial commitments.

In order to meet the demand for irrigation in the colony, government
had legislated in 1883 for irrigation trusts to assume responsibility,
but their formation had been impeded, owing partly to the complicated
rules involved.  Meanwhile, the development of irrigation in several
areas had been dependent on supplies provided by waterworks trusts
after meeting their principal obligations.  This problem required
solution by simplifying the method of forming the irrigation trusts
\index{trust!irrigation} or as the Commission suggested by creating
watershed trusts \index{trust!watershed} having representatives of
local governments and members elected by irrigators and charged with
responsibility for both kinds of water supply.  Future development of
irrigation required a fund of detailed information on stream flow and
rainfall, and better administration of water supply.

The earlier proposal that the government should construct certain
headworks to be regarded as national works was considered by the
Commission.  The idea was approved though on the condition that those
benefitting from the headworks would eventually meet their cost.

The fourth report, another personal memorandum by Alfred Deakin,
\index{Deakin, A.} gives
information on irrigation in Egypt \index{Egypt} and Italy
\index{Italy} gathered
during brief visits on his way to a conference in England in
1887.\fn{Vic.\ PP no.\,111 of 1887, RC~Water Supply, Further Progr.\
Rept.}

Public interest in the work of the Commission was sustained in 1885 by
the frequent newspaper reports from the two journalists accompanying
Deakin on his North \index{America}American tour, and by government
publication of his memorandum as a book intended for wide circulation
at a low price.  His tour occurred before storage of water for
irrigation had become necessary in the \index{USA}USA and before
salinity \index{salinity} problems received much attention there.  His
visit during the northern winter and spring would have given little
opportunity to see irrigation in progress or the labour involved in
crop harvests.  Deakin's comprehensive report includes his own
enthusiastic view of the value of irrigation for Victoria.

There are two unusual features of this Royal Commission.  Its third
progress report is the only one providing minutes of evidence
presented by individuals to the Commission, but evidence given to a
meeting of the Commission in December 1885 is available only in press
reports.  On that occasion Stephen Cureton and P.\,J.~Van der Byl
spoke of irrigation in America \index{America} and South Africa
\index{South Africa} and gave their views on development of irrigation
in Victoria.  The other peculiarity concerning this Commission is the
lack of a final report.  Clearly the Commission was most productive in
1885 and might have been terminated then but for an expectation of
further important matters for consideration.\fn{\textit{Argus}
(Melbourne), 3 Dec.\ 1885.}

\subsection*{The Response}

Deakin's \index{Deakin, A.} conclusions from his visit to North
\index{America}America were viewed fav\-our\-ably in many quarters.
Engineers were divided in their response.  George Gordon's
\index{Gordon, G.} views on the Deakin report had special interest
because of his work on the Water Conservancy Board and his address on
irrigation and \index{drainage}drainage in 1878.  He provided a long
statement to a meeting of Victorian engineers, showing concern at
Deakin's apparent adoption of \index{America}American irrigation as
the model for Australia and failure to advocate drainage of irrigated
land.  The ensuing discussion showed a mixed response.\fn{G.~Gordon,
Lecture on Irrigation and Drainage, Ballarat, 1878; G.~Gordon,
`American and Australian Irrigation', \textit{Trans.\ Proc.\ Vic.\
Inst.\ Eng.}, volume~1, (1883--85), pp.\,107--53.}

Some account of Deakin's visit to America stirred a response from the
distinguished irrigation engineer, Sir Arthur Cotton, \index{Cotton,
A.}  whose interest in Australia had continued long after his visits
to Tasmania and Victoria before 1850.  His undated comments entitled
`Water in Australia' were published posthumously by his daughter who
thought they were probably written in 1850.  However, they were
apparently composed after Deakin's visit to \index{America}America,
since they refer to `the Committee sent to America'.  Cotton regarded
it as a mistake to direct inquiries to American irrigation, which so
much concerned the distribution of `water in abundance'.  He believed
the great need in Australia was for storage of water which should have
been studied in \index{India}India and \index{France}France, `where
large reservoirs have been constructed.'  His view of American
irrigation gains support from Derry's report which mentions only the
Bear Valley \index{technology!reservoir}reservoir in San Bernardino
county of \index{California}California as a storage
\index{technology!dam}dam supplying water for irrigation.  That dam
was built in 1884 with a capacity of 30\,000\,acre-feet and had an
unusual design.\fn{\cite[p.\,44]{hope1900}; Vic.\ PP no.\,19a of 1885;
\cite[p.\,61]{white1976}.}

The detailed recommendations of the Commission led to new legislation
in 1885 and 1886 together with administrative changes.  These
developments were initiated by Alfred Deakin, who set out on a policy
of governmental support for irrigation, though without direct
responsibility for it.  The first action by government was to ease the
restrictive rules for formation of irrigation trusts with the amending
Act of 1885, which also gave access to government loan funds.  Deakin,
appointed Minister for Water Supply early in 1886, negotiated changes
in administration and also introduced three significant pieces of
legislation affecting irrigation development.  

His major accomplishment was the Irrigation Act
\index{legislation!Vic.!Irrig.\ Act 1886} 1886. It gave effect to
several major recommendations of the Royal Commission: State control
of all rivers, lakes and streams (Rec.\,IV), funding for national
works (Rec.\,XVII), though ignoring the recommendation (XV) of
watershed trusts and it confirmed the importance of irrigation trusts.
Deakin now favoured generous handling of the national works issue,
whereas earlier he had agreed that the loans for such works should be
recouped eventually (Rec.\,XVII).  During the debate on this
legislation, Deakin's view on funding was criticised by Charles Young,
\index{Young, C.} member of the Commission and former Minister for
Water Supply.  Other legislation gave approval for the River Goulburn
Weir, and development of irrigation at Mildura \index{Mildura, Vic.}
by private enterprise.\fn{Vic.\ Statutes no.\,859 Vic.\ Water
Conserv. Act 1885; Vic.\ Statutes no.\,898, 16 Dec.\ 1886;
\cite[p.\,38]{martin1955}.}

The recommendation for changes in administration of water supply was
effected following Deakin's appointment as Minister for Water Supply;
he secured Stuart Murray's \index{Murray, S.} services as Chief
Engineer for this department in September 1886.\fn{A.\,S.~Kenyon,
\textit{Vic.\ Hist.\ Mag.}, volume~10, (1925), p.\,112.}

The appointment of the Royal Commission may have been intended to calm
the agitation by the Central Irrigation League
\index{irrigation!league!Central} but that body was
encouraged to make further demands while the Commission was still
active.  The most significant action of the League was to mount a
`monster' deputation in Melbourne involving 800 people to the Premier
in August 1885 concerning the decisions of its April conference at
Kerang, \index{Kerang, Vic.} when government expenditure of
\pounds3\,000\,000 was sought for irrigation
development.\fn{\textit{Australasian}, Aug.\ 1885, p.\,253.}

Hugh McColl would have been delighted with the results of his earlier
proddings, but he died before Deakin returned from
\index{America}America.  The stage was now set for government support
for irrigation in different parts of Victoria.

\section*{Victorian Royal Commission on Vegetable Products}
\index{Royal Commission!Vic.!Vegetable Products 1885}

The Commission was appointed in September 1885 `to explore the
possibilities and examine the problems of growing ``vegetable
products'' of all descriptions profitably in Victoria'.  It was
terminated in 1892 and a final report was given in June 1894.  The
Commission gathered information from many witnesses in Victoria and
South Australia on the production of vegetable products, including the
value of irrigation.  Eight progress reports with evidence from the
numerous interviews were issued in the period 1886 to 1890.  The first
report includes evidence from Stephen Cureton, \index{Cureton, S.}
then associated with Chaffey Brothers, \index{Chaffey Bros} the fourth
report has evidence from primary producers near Kerang, the fifth
report includes testimony on irrigation of hops, and the eighth report
deals mainly with experience of irrigation at Mildura \index{Mildura,
Vic.}  before March 1890, including evidence by George
Chaffey. \index{Chaffey, G.}  The Commission in its fifth report
advised that irrigation offered the greatest possibilities of
supplying the colony with an integrated primary production, which was
found to be necessary owing to the marked fall in the world price of
\index{wheat}wheat.\fn{Vic.\ PP, RC~Vegetable Products, 1894. Final
Rept; \noibidem\cite[p.\,55]{martin1955}.}

\section*{South Australian proposals involving irrigation}

Except for the Murray River, South Australia lacks streams offering
dependable supplies for irrigation in country areas.  On the other
hand many of its residents were aware of the current interest about
irrigation in Victoria and the government responded to Victorian
initiatives.  Thus in 1883 the South Australian government sought
legislation on water conservation to facilitate schemes for water
supply and irrigation under local control, but the Water Conservation
Act, \index{legislation!SA!Water Conserv.\ Act 1886} related to the
Victorian statute of 1883 with the same title, was not realised until
1886.  It allowed local management of water and irrigation works but
no local community used it.\fn{\cite[p.\,56]{hammerton1986};
Irrigation papers read by J.\,J.~Green and A.~Molineux, May and June
1892, with discussion thereon.  Adelaide, (1893), p.\,25.}

Two government schemes for water supply appeared to offer some
possibility of irrigation in farming areas.  The Beetaloo project,
intended to supply Port Pirie, \index{Port Pirie, SA} the
copper-mining towns of Moonta, \index{Moonta, SA} Wallaroo,
\index{Wallaroo, SA} and Kadina, \index{Kadina, SA} as well as farming
areas as far south as northern Yorke Peninsula, \index{Yorke
Peninsula, SA} was designed by R.\,L.~Mestayer, \index{Mestayer,
R.\,L.} Hydraulic Engineer for the Minister of Public Works.  It was
to be based on a \index{technology!reservoir}reservoir in the Beetaloo
\index{reservoir!Beetaloo} district of the southern Flinders
Ranges. \index{Flinders Ranges, SA} Mestayer believed there would be
insufficient water for irrigation.  The Downer government sought
advice from W.\,W.~Culcheth, \index{Culcheth, W.\,W.} a member of the
Victorian royal commission on water supply who had experience of water
supply and irrigation schemes in India and Victoria.

Culcheth regarded the Mestayer scheme as very costly and proposed
instead a storage reservoir for urban supply and diversion of water
from the Broughton River \index{river!Broughton} by open
\index{technology!channel}channels to the rural areas, with some
allowance for irrigation from the channels.  However, the original
design was adopted and by 1890 the Beetaloo
\index{technology!reservoir}reservoir was completed with a capacity of
838 million gallons, and more than 600 miles of
\index{technology!pipe!cast-iron}cast-iron pipes were laid to supply
the urban centres and dry-farming areas.  Irrigation was not a feature
of the scheme, though the piped supp\-ly probably allowed some
watering of ornamental or vegetable gardens.\fn{SA PP no.\,100 of
1885; SA PP no.\,49 of 1886.}

The Barossa scheme, also involving a special inquiry, was developed by
Mestayer in 1885.  A \index{technology!reservoir}reservoir on the
South Para River, \index{river!South Para} northeast of Adelaide, was
intended to augment the metropolitan supply and provide for irrigation
west of Gawler \index{Gawler, SA} and further south towards
Adelaide. \index{Adelaide, SA} This scheme was referred to Mr
Culcheth, who considered it a `very great mistake' to combine
irrigation with city supply and offered suggestions for meeting the
interest in irrigation.  Despite the Hydraulic Engineer's defence of
the scheme, the government abandoned the intended diversion of water
to Adelaide and for irrigation, and proceeded with construction of the
Barossa \index{technology!reservoir}reservoir
\index{reservoir!Barossa} with a capacity of 993 million gallons to
supply urban and rural supplies.\fn{SA PP no.\,100A of 1885; SA PP
no.\,52 of 1886; SA PP nos.\,50 \& 51 of 1886; J.\,R.~Dridan,
\textit{ANZAAS Hdbk of SA}, (1946), p.\,88.}

These two schemes were generated at a time of economic depression and
severe \index{drought}drought, when development of irrigation appeared
to be timely. No comprehensive inquiry on water supply was made in the
State; the government merely consulted non-government specialists
about individual schemes and published the opinions.

As with the Mildura irrigation project in Victoria, the comparable
Renmark \index{Renmark, SA} scheme was the subject of a special agreement,
between the government and Chaffey Brothers \index{Chaffey Bros} as
promoters, reached following a parliamentary debate.

\section*{Tasmania}
\index{Tasmania}

In 1883 a parliamentary select committee was established following a
proposal by Mr Shoobridge MHA, that it should inquire into and report
upon the probable cost and results of providing a scheme of irrigation
for Tasmania or such portions as the committee may recommend.  The
committee circulated a questionnaire to fourteen landholders whose
replies generally failed to endorse major extension of irrigation.
The committee agreed that physical conditions were against any large
general system of water supply from any one centre, and that
irrigation in Tasmania should be confined to private enterprise, with
some supportive legislative enactment.\fn{Tas PP HA, vol.\,XLV, paper
no.\,143 of 1883.  Irrigation., Rept from Select Committee.}

A revival of interest in irrigating the Midlands led to a scheme
considered in 1887 for a diversion of water from Lake Crescent
\index{lake!Crescent} by means of a
\index{technology!channel}channel and
\index{technology!tunnel}tunnel eastwards to
the Blackman River.  \index{river!Blackman} This led firstly to the
appointment of a Select Committee in June 1888 to examine the effects
on the supply of water in the Clyde River. \index{river!Clyde} The
committee took evidence from residents in the Clyde valley who hoped
to increase their use of irrigation and from residents in the Midlands
who expected the proposed scheme would allow irrigation there of many
thousands of acres.  Its report concluded that with the provision of
certain works, the Lakes Crescent and Sorell \index{lake!Sorell} could
provide an ample supply of water to the Clyde valley and the Tunbridge
district \index{Tunbridge, Tas.} in the Midlands.  The government
introduced the Midland Irrigation Bill \index{legislation!Tas.!Midland
Irrig.\ Bill 1889} in 1889 to provide for the works involved but this
failed to gain the necessary support.\fn{Tas.\ JP vol.\,XV 1888--89,
paper no.\,124, Lake Crescent.  Water Supply, Rept from Select
Committee with MoE. 1888; Tas.\ JP, volume\,XVI 1889;
\cite[pp.\,149--150]{masoncox1994}.}

\closure
The official inquiries referred to above were initiated either by
government or by parliamentarians and included different forms of
inquiry: a semi-govern\-ment advisory board, royal commissions,
parliamentary select committees, and investigations for government
departments by private consultants.

While the Water Conservancy Board in Victoria coped effectively with
the general problem of water supply for the relatively dry
agricultural areas of northern Victoria and its recommendations were
effected by legislation, its approach to the question of irrigation
did not satisfy the government.  The Board was then abolished and the
Royal Commission on water supply was established.  It became the
means of securing wide support for development of irrigation with a
minimum of government involvement apart from state control of water
resources.  In New South Wales, a royal commission also with general
direction to water supply problems likewise paid considerable
attention to measures essential for development of irrigation but the
inquiry failed to get the same degree of popular support as in
Victoria and its work had no immediate impact apart from collection of
data on water resources.

The enquiries in South Australia and Tasmania failed to advance the
use of irrigation; indeed the 1883 select committee in the latter
colony made the important recommendation that a national scheme of
irrigation was unwarranted and that irrigation should be left to
private enterprise.

%\section*{References}
%1. A.S.Kenyon, Vict.Hist.Mag., 1925, vol.10,p.114.
%2. VicPP No.18 of 1881, vol.II, LA, \& G.Gordon, Min.Proc.Inst.Civ.Eng.
%     vol.142, 1900, p.326.
%3. VicPP No. 18 of 1881, LA vol II.
%4. \textit{Australasian}, 8/4/1882.
%5. L.R.East, Vict.Hist.Mag. 1967 vol.38,p.197.
%6. G.Gordon, Min.Proc.Inst.Civ.Eng. 1900, vol.142,p.326.
%7. VicPP No.74 of 1882.
%8. Supply of water to the northern plains, Irrigation, Second Rept, 20/3/1884,
%    by G.Gordon and A.Black, Vic Dept.Mines and Water Supply.
%9. J.H.McColl, Vict.Hist.Mag., 1917, Vol.5, p.157.
%10. \textit{Australasian}, 4/3/1882.
%11. VicWater Conservation Act 1883, No.778, 3rd Schedule.
%12. \textit{Australasian}, 1/4/1882.
%13. A.S.Kenyon, Vict.Hist.Mag. 1925, vol.10, p.115.
%14. C.S.Martin, Irrigation And Closer Settlement In The Shepparton District
%      1836-1906, 1955, p.45.
%15. \textit{Australasian}, 16/12/1882.
%16. \textit{Australasian}, 7/7/1883.
%17. \textit{Australasian}, 3/7/1883.
%18. \textit{Australasian}, 1/9/1883.
%19. J.A.LaNauze, Alfred Deakin, A Biography, 1979, p.80.
%20. G. Gordon, Trans.Proc.Vict.Eng.Ass. vol.1, 1883-85, p.144.
%21. H.C.Russell, J.Proc.R.Soc. NSW, vol.13, 1879, p.169.
%22. NSWPD, Sess.1883-84, vol.11, p.1605-07.
%23. NSWPD, Sess. 1883-84, vol.11, p.1608.
%24. NSWPP, 1885-86, vol.6. R.C.Water Conservation, First Rept.
%25. C.Cunneen, ADB vol.10, p.179.
%26. Lorna M.Darbishire, H.G.McKinney, His Life And Work, ML MSS 706.       
%27. NSWPP 1885-86, vol.6. R.C.Water Conservation, First Rept.
%28. NSWPP, 1885-86, R.C.Water Conservation, First Rept, Appendices.
%29. NSWPP 1885-86, vol.6, R.C.Water Conservation, Second Rept.
%      \& Appendix.
%30. NSWPP, 1887 vol.5, 2nd Sess. R.C.Water Conservation, Final Rept.
%31. C.J.Lloyd, Either Drought Or Plenty, 1988, p.157.
%32. W.J.Gibbs \& J.V.Maher, Comm.Bur.Meterol. Bull No.48, 1967.
%33. Vic VP LA 2nd Sess. 1883, vol.1, Surface Irrigation Canals Memorial,
%      p.653.
%34. VicPD 1884, vol.47, p.2298.
%35. VicPP No.53 of 1885, R.C.Water Supply, Further Progr.Rept.
%36. Deakin Papers, NL MS1540/2, 1884 Diary.
%37. VicPP No.19 of 1885, R.C.Water Supply, First Progr.Rept.
%38. VicPP No.19a of 1885, R.C.Water Supply, Further Progr.Rept.
%39. VicPP No.53 of 1885, R.C.Water Supply, Further Progr.Rept.
%40. VicPP No.111 of 1887, R.C.Water Supply, Further Progr.Rept.
%41. Argus(Melbourne), 3/12/1885.
%42. G.Gordon, Lecture On Irrigation And Drainage, Ballarat, 1878.
%43. G.Gordon, American And Australian Irrigation, Trans.Proc Vict. Inst.Eng.
%      vol.1, 1883-85, pp.107-53. 
%44. Lady E.R.Hope, General Sir Arthur Cotton, R.E.,K.C.S.I., His Life And 
%       Work. 1900, p.44.
%45. VicPP No.19a of 1885.
%46. N.White, Man And Water, A History Of Hydro-technology, 1975, p.61.
%47. Vic Statutes No.859 Victorian Water Conservation Act 1885.
%48. Vic Statutes No.898, 16/12/1886.
%49. C.S.Martin, 1955, p.38.
%50. A.S.Kenyon, Vict.Hist.Mag. 1925, vol.10, p.112.
%51. \textit{Australasian}, August 1885, p.253.
%52. VicPP, R.C.Vegetable Products, 1894. Final Rept.
%53. C.S.Martin, 1955, p.55.
%54. Marianne Hammerton, Water South Australia, 1986, p.56; Irrigation 
%      Papers read by J.J.Green and A.Molineux May and June 1892, with 
%     discussion thereon.  Adelaide, 1893, p.25.
%55. SAPP No.100 of 1885.
%56. SAPP No.49 of 1886.
%57. SAPP No.100A of 1885.
%58. SAPP No.50\&51 of 1886.
%59. SAPP No.52 of 1886, \& J.R.Dridan, ANZAAS Hdbk 
%      of South Australia, 1946, p.88.
%60. TasPP HA, Vol.XLV, Paper No.143 of 1883. Irrigation. Rept. from
%      Select Committee.
%61. Tas JP Vol.XV 1888-89, Paper No.124, Lake Crescent.
%      Water Supply. Rept. from Select Committee with MoE. 1888.
%62. Tas JP, Vol.XVI 1889, \& Margaret Mason-Cox , Lifeblood Of A 
%      Colony, 1994,pp.149-150. 
