% $Id$
% CHAPTER EIGHT
% 4902 words at 1/5/99

\chapter{Experience of Irrigation Trusts}

\index{trust!irrigation|(}
\label{ch:trusts}\addtoendnotes{\protect\subsection*{Chapter \thechapter}}
\fancyhead[RE]{\sffamily \small Chapter \thechapter.\ %
               Irrigation Trusts}

\setcounter{endnote}{0}

Irrigation trusts were established under legislation in different
colonies to manage irrigation systems in specified areas. Most of
these trusts were formed after 1885 in Victoria, where they were
regarded by government as important for its policy of extending the
use of irrigation by a major delegation of responsibility to a form of
local government.  A few trusts were established in New South Wales,
each in response to special legislation.  Two trusts were formed in
response to the \index{finance}financial collapse of the promoters of
existing irrigation settlements.

The trusts were important for development of integrated irrigation
schemes during at least twenty years but only two remained in 1920.
This chapter considers their experience and how well they helped to
extend the use of irrigation.

Incorporated trusts with responsibility for water supply in specified
localities had existed in Australia before legislation provided
specifically for irrigation trusts.  In Tasmania the River Clyde Act
\index{legislation!Tas.!Clyde River Act 1857} of 1857 led to the
appointment of trustees to provide a water supply to the townships of
Bothwell and Hamilton on that river.  The legislation was altered in
1869 to empower the trust to levy rates and to widen its
responsibilities which then came to include control of irrigation with
Clyde River \index{river!Clyde} water.  The numerous waterworks trusts
of northern Victoria, referred to in the previous chapter, depended on
legislation in 1881.\fn{\noibidem\cite[p.\,136]{masoncox1994}.}

\section*{Victorian irrigation trusts}\index{trust!irrigation!Vic.}

Provision for trusts to undertake irrigation was made by legislation
of the Victorian government in 1883, but with no immediate effect.
Applications to form them were made after amending legislation, which
liberalised the conditions and provided access to funds, and the
indications that the government was about to offer
\index{finance}financial advances to the trusts and to provide those
headworks known as national works, which would supply two or more
trusts.  By the time of the major enabling legislation, known as the
Irrigation Act 1886, \index{legislation!Vic.!Irrig.\ Act 1886} several
irrigation trusts had been formed\,---\,all in the lower part of the
Loddon River \index{river!Loddon} valley or adjacent
\index{technology!channel}channel country of the Murray
valley. \index{river!Murray} More trusts were formed under that
legislation in the next few years but only two after 1890.  One more
trust\,---\,the First Mildura Irrigation Trust\,---\,was
\index{trust!irrigation!First Mildura} constituted later under
different legislation referred to in another chapter.\fn{Vic.\ Water
Conservation Act no.\,859 of 18 Dec.\ 1885; Vic.\ PP no.\,53 of 1885,
RC~Water Supply, Further Progress Rept July 1885.}

When the Minister for Water Supply, Alfred Deakin, \index{Deakin, A.}
opened parliamentary debate in June 1886 on comprehensive measures for
extending the use of irrigation, his estimate of the area available
for irrigation was approximately 3.2~million acres with about one
million acres as the total to be watered each year.  A great expansion
of irrigation was expected in northern Victoria, where the total
irrigable area in nine districts was given as almost three million
acres, with more than two-thirds lying in the Goulburn, Loddon, Upper
Murray, and Western Wimmera \index{river!Wimmera} districts.  The
limited data then available on water resources indicated that for all
nine districts no more than 750\,000\,acres could be irrigated in
winter or 370\,000\,acres in summer, with more than a third dependent
on the Goulburn \index{river!Goulburn} River.\fn{Vic.\ PD vol.\,51,
1886, pp.\,415--447; \textit{Leader}, 3 July 1886.}

The first irrigation trust to be formed in Victoria was the Leaghur
and Meering Trust \index{trust!irrigation!Leaghur \& Meering} in 1885.
It embraced a comparatively small area of 10\,300\,acres in the
western part of the lower Loddon valley, where landholders had
previously been operating a cooperative system of irrigating
\index{wheat}wheat crops.  More significant was the formation in 1886
of the Tragowel Plains Trust \index{trust!irrigation!Tragowel Plains}
for a large area east of the Loddon river. \index{river!Loddon} The
procedure for establishing an irrigation trust involved an application
to the water supply department by a majority of landholders in an
agreed area, together with plans and estimates of expenditure.  If the
formation of the trust was approved by the government, which might
seek independent advice, responsibility for the necessary works was
taken by elected members of the trust, who were known as
commissioners, assisted by the trust engineer.  The trust was also
charged with collection of revenue and it controlled expenditure of
loan money, for which no interest was to be charged by government for
the first five years.  As a result of amendments to the original
legislation, the distinction between waterworks trusts serving country
areas and the irrigation trusts became blurred, with the irrigation
trusts supplying stock and domestic supplies in addition to irrigation
and the waterworks trusts were permitted to undertake
irrigation.\fn{Vic.\ PP no.\,20 of 1896, RC~Water Supply, Rept.}

Before the end of 1886 five irrigation trusts with a total area of
130\,000 acres were established under the new Act in the settled parts
of northwestern Victoria: in the Loddon Valley and the nearby area
traversed by the Murray and its effluents.  In 1887 the Swan Hill
Irrigation Trust \index{trust!irrigation!Swan Hill} on the Murray
River \index{river!Murray} was the only one formed, but in 1888
another five were created, mainly in the Loddon Valley.
\index{river!Loddon} The peak year was 1889 when eight more were
formed, including the Rodney Irrigation Trust,
\index{trust!irrigation!Rodney} the first in the Goulburn Valley,
\index{river!Goulburn} three dependent on the Campaspe River
\index{river!Campaspe} and one in southern Victoria at Bacchus
Marsh. \index{Bacchus Marsh, Vic.} Several more trusts, including one
at Bairnsdale \index{Bairnsdale, Vic.} in Gippsland, \index{Gippsland,
Vic.} were formed before the last in 1893.  Altogether, 30 trusts were
constituted.\fn{Vic.\ PP no.\,20 of 1896, RC~Water Supply, Rept.}

\subsection*{National headworks} \index{national headworks|(}

The first of the national headworks was to be a weir on the
\index{weir!Goulburn|(}Goulburn River \index{river!Goulburn} to divert
supplies for irrigation in the valley; it was authorised in 1886 by
the River Goulburn Weir Act \index{legislation!Vic.!R. Goulburn Weir
Act 1886} with an appropriation of \pounds20\,000.  Such a weir had
been advocated in the Grand Northwestern Canal scheme
\index{canal!scheme!Vic.} of the 1870s, and by the Water
Conservancy Board in its proposals for water supply and irrigation
between the Goulburn and Campaspe Rivers.  The initiative in securing
the weir had been taken earlier by the Echuca and Waranga Waterworks
Trust \index{trust!waterworks!Echuca \& Waranga} which wished to use
Goulburn River water for irrigation in addition to stock and domestic
supply.  Delays in settling on a design for the weir led the trust to
install a pumping plant on the river in 1885 to supply its channels,
only to experience serious losses by \index{seepage}seepage,
\index{evaporation}evaporation and channel breakages.  An earthen
embankment across the nearby \index{swamp!Waranga}Waranga Swamp had
been constructed to allow storage of water there and to give the
necessary elevation required for the channel gradient; leakage from
this section of the channel involved serious expense to the trust and
reduced the area to be supplied.\fn{River Goulburn Weir Act, Vic.\
Statute no.\,598 of 16 Dec.\ 1886; \cite[pp.\,35--50]{martin1955}.}

The weir \index{weir!Goulburn} to be constructed by the government on
the Goulburn River was expected to divert water to both sides of the
valley.  Work started in 1887 with expectation of completion in 1889
but \index{flood}floods delayed its opening until March 1890 and its
first supply of water occurred next year.  On its completion, which
provided a storage capacity of 20\,000\,ac\-re-feet, a new
\index{technology!channel}channel on the west side of the
river was made from the weir to connect with those from the pumping
station downstream and then parallel to the trust's Cussen channel
except for a deviation around the north end of Waranga Swamp.  This
national channel later became known as the \index{channel!Stuart
Murray}Stuart Murray channel.  No use of the weir to provide supplies
for the eastern side of the valley was made for many
years.\fn{\cite{murray1892};
\cite{mead1914};
\cite{mccoy1988}.}

Conversion of Waranga Swamp into a
\index{technology!reservoir}reservoir \index{reservoir!Waranga} was
designed in 1892 but work was delayed by the depression of the 90s and
did not start until the \index{drought}drought year of 1902.  The
reservoir first received water from the Goulburn River
\index{river!Goulburn} in October 1905, when its capacity was
197\,000\,acre-feet.  The lack of adequate storage for Goulburn River
water until 1905 seriously curtailed the extension of irrigation
dependent on the river.\fn{\cite[pp.\,106--8]{bossence1965};
\cite{mead1914}.}

Other national works commenced during Deakin's term as Minister for
Water Supply were intended to improve supplies in parts of the Loddon
Valley \index{river!Loddon} and the adjacent Gunbower district where
\index{wheat}wheatgrowers were already making the greatest Victorian
use of irrigation.  The formation of trusts in these districts
represented an enthusiastic response to the new legislation and
emphasised the need for an adequate supply of irrigation water for
landholders in parts of Victoria with low annual rainfall.  One of the
national works was the \index{technology!weir}Laanecoorie weir
\index{weir!Laanecoorie} and \index{technology!dam}dam in the hills on
the upper reaches of the Loddon River.  Work began in January 1889 and
finished by 1892 when it became useful.  Its purpose was `to regulate
the river by storing water during \index{flood}flood so as to maintain
the flow during dry seasons'.  It had an initial storage capacity of
14\,042\,acre-feet, less than the Goulburn Weir \index{weir!Goulburn}
storage and reduced later to 6650\,acre-feet by siltation.  Its
purpose was to serve irrigation trusts of the mid-Loddon valley,
including those near Boort \index{Boort, Vic.} and on the Tragowel
plains.\fn{Vic.\ PP 1884, Statistical Register, Production: Irrigation
to 31 Mar.\ 1884; \cite[p.\,14]{murray1892}; \cite[p.\,17]{east1940}.}

Supplies to areas further downstream were to be augmented by diversion
of water from the Murray, taking advantage of natural flooding into
its Gunbower Creek \index{creek!Gunbower} anabranch, and by improving
flow to the large Kow Swamp, \index{swamp!Kow} which would be made
into a \index{technology!reservoir}reservoir for a
\index{technology!canal}canal to deliver supplies to the
\index{river!Loddon}Loddon River near Kerang.  \index{Kerang, Vic.}
This westward trend of overflow from the Murray River
\index{river!Murray} via Gunbower Creek had been used in the past for
supplies to landholders, as indicated by Gordon and Black in their
1882 proposals for water supply works in the Gunbower district.  The
national works in the district began in 1889 with improvements to the
outlet from the Murray to Gunbower Creek, and construction of a
\index{technology!channel}channel several miles long from the outlet
to Gunbower in order to improve the flow westward to Kow Swamp.
Subsequently there was embankment of this swamp to increase its
capacity by 1894 to 40\,000\,acre-feet and construction of the Macorna
Channel \index{channel!Macorna} conveying water from this storage
basin across the Tragowel Plains to the Loddon River.  Stuart Murray
\index{Murray, S.}  expected beneficial effects of some of these
\index{national headworks|)}modifications to begin in the summer of
1892--93.\fn{\cite[p.\,14]{murray1892}; \cite[pp.\,21--27]{mccoy1988};
Vic.\ PP no.\,35 of 1902--03, Interstate RC~on the River Murray. Rept
and MoE, evidence of A.\,S.~Kenyon.}

\subsection*{Irrigated agriculture}

The parts of northern Victoria \index{Victoria} intended for
irrigation by the trusts were used mainly for \index{cereals}cereal
growing and grazing, both of which would benefit from water supply in
times of low rainfall.  However, as rainfall in these parts fluctuates
from year to year, irrigation would be no advantage to some
landholders in wet years.  The greatest use of irrigation in the years
1891 to 1895 was in 1895 when rainfall in Victoria was generally well
below average following relatively wet years which must have provided
the stored water needed in 1895.  In the wetter years the demand for
irrigation water declined and in a sequence of dry years there was
insufficient water to meet the demand.  In these circumstances the
trusts had difficulty in obtaining income to discharge debts to the
government.\fn{Vic.\ PP no.\,20 of 1896, RC~Water Supply; Comm.\ Bur.\
Meteorol., Results of Rainfall Observations made in Victoria, 1937;
J.\,C.~Foley, \textit{Comm.\ Bur.\ Meteorol.\ Bull.}, no.\,43, (1957),
`Drought in Australia'.}

While the areas of irrigable land and the available water resources
mentioned for the nine northern districts in the government proposals
of 1886 may indicate official hopes at the time, the proposals
represented by the various trusts showed a marked preference for
irrigation in the drier areas west of the Goulburn River together with
a minor response in southern Victoria\,---\,at Bacchus Marsh,
\index{Bacchus Marsh, Vic.}  Werribee, \index{Werribee, Vic.} and
Bairnsdale.  \index{Bairnsdale, Vic.} The lack of interest then in
irrigation to the east of the Goulburn River is indicated by the
failure of efforts by the East Goulburn Irrigation League to form a
trust despite active encouragement in 1890 by the Water Supply
department.\fn{\cite[p.\,70]{martin1955}.}

The problems of the irrigation trusts were referred to by Deakin
\index{Deakin, A.} in
addressing the irrigators' conference of 1890, before completion of
any national works,
\begin{Quote}
	There are at present constituted, or practically constituted,
	25 Irrigation Trusts in Victoria.  Of these 25, one fifth (or
	five Trusts) had some water to sell to their constituents last
	season, and only five.  All of those five drew their supplies
	from uncontrolled streams which, like the majority of our
	Australian streams, are not to be relied upon to contribute
	any considerable flow of water. Not one single Trust in
	Victoria could undertake to sell water to its constituents
	beforehand, because in not one case as yet have the necessary
	works been brought to such a condition as would enable them to
	be certain of being able to deliver what they had sold. Only
	five Trusts could sell any water, and they had only an
	uncertain quantity, and, for all they knew, no quantity at all
	to dispose of.  Of those five Trusts, four are among the
	smallest Trusts in the colony.\fn{A.~Deakin, Speech in
	\textit{Procs.\ First Conf.\ Irrigationists Victoria}, (1890),
	Govt Print\-er, Melb.}
\end{Quote}

\subsection*{Five irrigation trusts}

Water to most of the irrigation trusts in Victoria was provided by
five rivers: the \index{river!Murray}Murray,
\index{river!Goulburn}Goulburn, \index{river!Campaspe}Campaspe,
\index{river!Loddon}Loddon, and \index{river!Wimmera}Wimmera.  The
Murray supplied six irrigation trusts, either directly by pumping or
by gravitation.  The Swan Hill Irrigation Trust
\index{trust!irrigation!Swan Hill} was constituted in 1887 with an
area of 15\,000\,acres, much of it previously under the control of a
waterworks trust providing stock and domestic supplies.  Before 1896
no more than 2669\,acres were irrigated, main\-ly for
\index{cereals}cereal crops, by diversion of water when the river was
at high level. There were efforts to foster production from
\index{fruit}fruit trees and \index{vineyards}vines on holdings of
10\,acres created by subdivision but in 1895 less than 20\,acres were
irrigated for horticulture by pumping from the Murray River.\fn{Vic.\
PP no.\,20 of 1896, RC~Water Supply, Rept.}

The Goulburn River, \index{river!Goulburn} the Victorian tributary of
the Murray with annual flow ten times greater than others, was
intended by the Government to supply by far the largest irrigated area
in the colony.  Irrigation had been sought in 1882 by two local
governments with areas west of the river; it began by using
\index{technology!channel}channels of a waterworks trust supplied at
first by \index{technology!pump}pumping from the river but irrigators
lacked supplies in some years.  In 1889 the Rodney Irrigation Trust
\index{trust!irrigation!Rodney} was formed by excision of territory
from the older waterworks trust and irrigation of small blocks with
\index{fruit}fruit trees and \index{vineyards}vines was carried on
notably at \index{Ardmona, Vic.}Ardmona.  Even when more dependable
supplies of water were provided after completion of the Goulburn Weir,
\index{weir!Goulburn} irrigation was used consistently for
horticultural and \index{fodder}fodder crops, with little involvement
of \index{cereals}cereals and \index{pasture}pastures except when
rainfall was deficient, as in 1895.  The Rodney Trust incurred heavy
debts because of general failure by landholders to contribute to
revenue while supplies of water were not assured.\fn{Vic.\ PP no.\,20
of 1896, RC~Water Supply, Rept Appendix~C, p.\,205.}

The major undertaking for irrigation using water from the Campaspe
River involved the Campaspe Irrigation Trust,
\index{trust!irrigation!Campaspe} established in 1889 with an area of
18\,797\,acres some of which would be irrigated from a weir on the
river.  But although the works cost \pounds52\,959, the sale of water
in 1895 realised less than \pounds3. This failure was linked with
diversions involved in the Coliban scheme from headwaters of the river
to the gold-mining towns of Bendigo and Castlemaine.

Further west, in the Loddon Valley, \index{river!Loddon} several
trusts were constituted, one of the most extensive being the Tragowel
Plains Irrigation Trust. \index{trust!irrigation!Tragowel Plains}
Early in 1884 an irrigation scheme for the district was put forward by
residents, leading to a deputation in March to the Minister for Water
Supply involving a local parliamentarian, Rev.~E.\,C.~De Garis
\index{De Garis, E.\,C.} and another local resident, W.\,W.~Culcheth,
\index{Culcheth, W.\,W.} an irrigation engineer and member of the
royal commission on water supply, was instructed to report on the
scheme and responded with statements in August 1884, and March and
April of 1885.  He found that almost all the area under consideration
for the proposed trust was irrigable.  In March 1886 the Trust was
approved and 15 Commissioners were appointed for three years with De
Garis as chairman.  Irrigation in the trust area of 192\,800\,acres
was undertaken principally for \index{cereals}cereals and
\index{pasture}pastures; in 1891 the area so treated was the largest
of all the trusts in Victoria and it remained so in the subsequent
four years, reaching 25\,403\,acres in 1895.\fn{\textit{Leader}, 3
July 1886.}

The Wimmera River \index{river!Wimmera} was the only stream of any
consequence for irrigation within the area of the West Wimmera
Irrigation and Water Supply Trust \index{trust!irrigation!West
Wimmera} formed in 1888 to embrace 1\,643\,132\,acres. This trust took
over water supply \index{technology!channel}channels from a
Waterworks Trust as well as two important weirs and the Wartook
Reservoir \index{reservoir!Wartook} in the Grampians mountains; it
provided stock and domestic supplies in an extensive
\index{wheat}wheatgrowing district and water for irrigated areas near
Horsham. \index{Horsham, Vic.}

\subsection*{Trusts investigated}

The opening of the Goulburn Weir \index{weir!Goulburn|)} in 1890
marked the high point of Deakin's \index{Deakin, A.} association with
irrigation in Victoria, for he ceased to be Minister when his
government was defeated later in the year.  The next Minister, George
Graham, \index{Graham, G.} was aware of troubles associated with water
supply in country areas but failed to remedy them before his
replacement during the \index{finance|(}financial debacle of 1893 by
J.\,H.~McColl, \index{McColl, J.\,H.} son of the notable Hugh McColl.
This Minister of Water Supply instituted a Waterworks Enquiry Board in
1893 to report on the \index{finance}financial problems of waterworks
trusts.  Next year a new government appointed a Royal Commission on
Water Supply \index{Royal Commission!Vic.!Water Supply 1894} to
enquire into `the \index{finance}financial position and prospects of
the various local bodies that have obtained loans from the state for
the construction of works of water supply'. The six men appointed to
the commission were all legislators and they gave their only report in
1896 together with a long record of evidence.  Considerable attention
was given to the irrigation trusts and their administration by the
Water Supply Department.  Many of the 410 witnesses examined were
irrigators.\fn{\noibidem\cite[p.\,45]{martin1955}; Vic.\ PP no.\,20 of
1896, RC~Water Supply, Rept; Vic.\ PP no.\,21 of 1896, RC~Water
Supply, MoE.}

The Royal Commission found that the cost of irrigation then exceed\-ed
\pounds1.8 million, made up mainly of loans to 30 trusts and the cost
of national headworks, together with liabilities transferred from
waterworks trusts and arrears of interest due to the State.  Trusts
had been constituted without attention to justification for proposed
expenditure or to adequate supply of water.  One of the many trusts in
the Loddon Valley \index{river!Loddon} had used a loan of
\pounds165\,000 to construct \index{technology!channel}channels
capable of watering 200\,000\,acres but the river could not supply
water for even one-fifth of that area.\fn{Vic.\ PP no.\,20 of 1896,
RC~Water Supply, Rept, p.\,185.}

The expenditure of money lent by the government to the trusts was
found to have been beyond the control of the Water Supply Department
and there was evidence of poor internal management of funds by trusts.
Advances to them were made upon recommendations of Alfred Deakin,
\index{Deakin, A.}  then Minister for Water Supply.  The most glaring
examples of uncontrolled advances to trusts occurred with the private
irrigation trusts, which initially were not entitled to borrow money
from the State.  The arrangements concerning one such trust, the
Werribee Irrigation Trust, \index{trust!irrigation!Werribee} proved to
be a source of embarrassment later to Deakin when he attempted to
represent his departmental head as responsible for irregularity but
was finally obliged to admit his own fault.  Decisions to form trusts
in some cases involved persons who were not bona-fide owners of land
in the area involved.  This criticism applied particularly to private
irrigation trusts, which each involved one landholder\,---\,either an
individual or a syndicate.\fn{Vic.\ PD 18 Dec.\ 1896, p.\,4738.}

Irregularities concerning formation of trusts were found by the
Commission to include pressure from the government water supply
department to secure their formation.  Residents in different parts of
the Loddon Valley \index{river!Loddon} had complained of being forced
to form trusts in response to threats from the department that
otherwise their existing supply of irrigation water would be stopped.
In the relevant case of the Leaghur and Meering Trust,
\index{trust!irrigation!Leaghur \& Meering} its area had included
cooperative irrigation of \index{wheat}wheat by means of works
constructed for the purpose prior to the formation of the trust in
1885.

There had been unfortunate clashes of interest between irrigation
trusts dependent on one source of water and also those between
irrigation trusts and water supply trusts constituted earlier.  In the
latter case there was the example of friction between the Loddon
United Waterworks Trust \index{trust!waterworks!Loddon United} (1882)
and the Tragowel Plains Irrigation Trust
\index{trust!irrigation!Tragowel Plains} whose area was transferred
from the Loddon United Waterworks Trust.\fn{\cite{sharland1971}.}

Inquiries about the character of soil and its suitability for
irrigation in trust areas were generally lacking.  At Swan Hill,
\index{Swan Hill, Vic.} the trust area included marshy ground; its
liability to flooding from the Murray led to the erection of a levee
bank at additional expense.  In the Loddon Valley unsuitable ground
classed as irrigable had been alluded to by Stuart Murray:
\index{Murray, S.}
\begin{Quote}
	rough crab-holey nature of much of its surface\,---\,a feature
	that greatly detracts from its adaptability for irrigation.
	Much of the land that has been irrigated is so uneven that the
	crops are patchy, being over-watered in some spots and
	insufficiently watered in others.  These unevennesses can, and
	it may be presumed will, be reduced by surface levelling; but
	this will cost money and require time.\fn{\cite[p.\,13]{murray1892}.}
\end{Quote}

Two aspects of the construction of works by the trusts attracted
criticism in the Commission's report.  Advisers to the trusts included
engineers attracted to Victoria by the prospect of employment but
without good experience of Australian conditions.  Some had been
appointed whose competence was in question and it emerged that the
Water Supply Department had not always made sufficient inquiries about
the qualifications of advisers.  There was also the serious
discrepancy between the estimated and actual cost of many works, with
some projects costing several times the estimated figure.  As well
there were the cases where irrigation
\index{technology!channel}channels had been provided far in excess of
the actual scope for irrigation.

Stuart \index{Murray, S.}Murray, head of the Water Supply Department,
told the Commission of his reservations about many procedures
involving the irrigation trusts, which occurred despite his advice.
He claimed that the problems of the trusts were affected by political
pressure.  His evidence touched on the promotion of irrigation trusts
in the interests of land speculators, as part of the Commission's
report shows:
\begin{Quote}
	Under the present law preliminary expenses in connexion with
	the formation of Irrigation Trusts are allowed to be paid out
	of loan.  This practice we think should be abandoned.  If
	Trust promoters had had in the past to pay preliminary
	expenses out of their own pockets many of the Trusts which are
	at present in existence would, we are convinced, not have been
	constituted.  The evidence of the Chief Engineer of Water
	Supply is to the effect that a great many of the irrigation
	schemes have been initiated in the interests, in the first
	place, of syndicates and dealers in land.  If this contention
	be correct, and from our inquiries we are inclined to believe
	that it is, it can be readily understood what an incentive the
	fact of being able to pay preliminary expenses out of State
	money was to speculators.\fn{Vic.\ PP no.\,20 of 1896,
	RC~Water Supply, Rept, p.\,191.}
\end{Quote}

\subsection*{Termination of Victorian Irrigation Trusts}

Despite the findings of the 1894 Royal Commission, irrigation trusts
continued to operate and were given some relief from their burden of
debts.  Unfortunately these continued to mount though some progress
was made in extending irrigation.  In the dry years following 1895 the
deficiencies of Victorian water supply for irrigation and for stock
and domestic use became more evident.  Agricultural settlement of the
northwestern Mallee \index{Mallee district} district had begun in the
1880s, requiring considerable diversion of water for stock and
domestic supply from the \index{river!Wimmera}Wimmera river system
that might otherwise have been used for local irrigation schemes.  In
the Loddon valley there was insufficient water to satisfy the
competing demands.  As the \index{drought}drought reached its worst in
1902--03, the government faced a time of reckoning.  George Swinburne,
\index{Swinburne, G.} an engineer turned company director, was elected
to parliament in 1902.  He accompanied the Premier, William Irvine,
\index{Irvine, W.} on a tour of northern districts in 1903, became
Minister for Water Supply in 1904 and quickly submitted his Water Bill
aimed at major reforms.  The Water Act \index{legislation!Vic.!Water
Act 1905} of 1905 provided for a Commission independent of control and
with three members to take charge of all assets and liabilities of the
various rural trusts except that managing irrigation at Mildura.  At
that time storage capacity for country water supplies was
approximately 175\,000\,acre-feet, with 75\,000\,acres under
irrigation.\fn{\cite[p.\,34]{boorman1942}; \cite[p.\,90]{martin1955}.}

\section*{Trusts in New South Wales\\ and South Australia}
\index{trust!irrigation!NSW} \index{trust!irrigation!SA}

Irrigation trusts were established in New South Wales from 1890.  The
Wentworth municipal government was constituted as an irrigation trust
by the Wentworth Irrigation Act \index{legislation!NSW!Wentworth
Irrig.\ Act 1890} of 1890, with control of more than 10\,000\,acres of
low-level alluvial country east of Wentworth to be irrigated by
pumping from the Murray River.  \index{river!Murray} The trust
accepted the recommendation of H.\,G.~McKinney that initially
development should be confined to 1500\,acres near Wentworth, with
water to be lifted 25 feet from the river by pumping.  The
\index{finance}financial collapse in 1893 made progress impossible and
members of the trust sought its dissolution.  This was agreed to by
the government which by 1896 had called tenders for the pumping plant
and provision of distribution
\index{technology!channel}channels. The scheme attracted
few settlers at first\,---\,there were only about 400\,acres leased by
1906 when the government undertook a publicity campaign to attract
settlers\,---\,with some success.  The Hay Irrigation Trust,
\index{trust!irrigation!Hay} also under municipal control, was
established in 1892 with an area of 17\,147 acres but the area was
reduced to 4000\,acres in 1893 and only 600\,acres was under
irrigation in 1902, using water drawn from the Murrumbid\-gee River.
\index{river!Murrumbidgee} A similar scheme for Balranald
\index{Balranald, NSW} involving 2000\,acres was started in 1893 but
only 60\,acres were leased by 1906.\fn{H.\,G.~McKinney, `The Wentworth
Irrigation Scheme', \textit{Agr.\ Gaz.\ NSW}, vol.\,7, (1896),
pp.\,469--70; \cite[p.\ 113]{jeffcoat1988}; \cite{williamson1968}.}

The only other irrigation trust in New South Wales before 1920 was the
shortlived Murrumbidgee Irrigation Trust
\index{trust!irrigation!Murrumbidgee} set up under legislation in 1910
to control development of irrigation north of the Murrumbidgee River
\index{river!Murrumbidgee} using water gravitating from the Burrinjuk
Dam \index{technology!dam}\index{dam!Burrinjuk} then under
construction to supply 250\,000\,acres.  This trust had three members:
the government ministers for public works, lands, and agriculture; its
secretary and executive officer was the chief irrigation engineer.
The enabling legislation was repealed in 1912 with passage of the
Irrigation Act \index{legislation!NSW!Irrig.\ Act 1912} which gave
control of developments in the Murrumbidgee Irrigation Area to the
Commissioner for Water Conservation and Irrigation, Mr
L.\,A.\,B.~Wade.\fn{\cite{lloyd1990}.} \index{Wade, L.\,A.\,B.}

Although the Irrigation Act of 1912 made provision for establishment
of trusts to administer water supply districts and irrigation, there
was no use of it before 1920 to create irrigation trusts.

In South Australia the only irrigation trust was that established for
the Renmark \index{Renmark, SA} irrigation scheme under the authority
of the Renmark Irrigation Trusts Act \index{legislation!SA!Renmark
Irrig.\ Trusts Act 1892} 1892.  The Renmark Irrigation Trust no.~1
\index{trust!irrigation!Renmark no.~1} was
controlled by a board of seven irrigators elected by those owning 10
or more acres, its inaugural meeting was in January 1894 after the
collapse of the Chaffey Brothers
companies.\fn{\cite[p.\,149]{wells1986}.}

\closure
Formation of many irrigation trusts in Victoria led to provision of
various works necessary for extension of irrigation.  Five to ten
years later, the annual mean irrigated area under their control
reached more than 50\,000\,acres for the period 1891--95.  This mainly
concerned supplementary irrigation in northern Victoria of cereal and
fodder crops and pastures, all of which had been grown there widely
and successfully before the extension of irrigation except in the
occasional years of low rainfall.  Considerable debts incurred by the
irrigation trusts led to serious criticism by a royal commission of
the government administration involved.

Three irrigation trusts formed in drier parts of New South Wales led
to limited use of irrigation for horticultural crops near Wentworth
and for pasture or fodder crops near Hay and Balranald.  These trusts
were not involved with major \index{finance|)}financial problems as in
Victoria.

Two trusts were created to provide satisfactory management of the
irrigation settlements, at Mildura in Victoria and Renmark in South
Australia, after the financial collapse of the promoting company.
Both trusts continue to function
\index{trust!irrigation|)}satisfactorily.

%\section*{References}
%1. Margaret Mason-Cox, Lifeblood Of A Colony, A History of Irrigation
%    In Tasmania, 1994, p, 136.
%2. VicWater Conservation Act No.859 of 18/12/1885, \& VicPP No.53
%    of 1885, R.C.Water Supply, Further Progress Rept. July 1885.
%3. VicPD vol.51, 1886, pp. 415-447.
%4. Leader, 3 July 1886.
%5. VicPP No.20 of 1896, R.C.Water Supply, Rept.
%6. River Goulburn Weir Act, VicStatute No.598 of 16/12/1886 \& 
%    C.S.Martin, Irrigation And Closer Settlement In The Shepparton District
%    1836-1906, 1955, p.35.
%7. C.S.Martin, 1955, pp49-50.
%8. S.Murray, Irrigation In Victoria: Its Position And Prospects, 1892,
%     Govt.Printer, Melbourne.
%9. E.Mead, Irrigation In Victoria, pp 255-268 in BAAS Handbook to
%    Victoria, 1914, \& C.G.McCoy, The Supply Of Water For Irrigation In
%    Victoria From 1881 To 1981, 1988.
%10. W.H.Bossence, Murchison, 1965,pp 106-108, \& E.Mead, 1914.
%11. VicPP 1884, Statistical Register, Production: irrigation to 31/3/1884.
%12. S.Murray, 1892, p.14.
%13. L.R.East, Erosion And Water Supply, in Soil Erosion In Victoria, 1940,
%      p.17.
%14. S.Murray, 1892, p.14, \& C.G.McCoy, 1988, pp. 21-27. 
%15. S.Murray 1892, \& VicPP No.35 of 1902-03, Interstate R.C. on the River
%      Murray. Rept and MoE, evidence of A.S.Kenyon.
%16. S.Murray, 1892, p.14.
%17. VicPP No.20 of 1896, R.C.Water Supply, \& Comm.Bur.Meteorol.
%      Results Of Rainfall Observations Made In Victoria, 1937, \& J.C.Foley,
%     Comm.Bur.Meteorol.Bull No.43,1957, Drought in Australia. 
%18, C.S.Martin, 1955, p.70.
%19. A.Deakin, Speech in Procs First Conf.Irrigationists Victoria,1890, 
%      Govt.Printer, Melb.
%20. VicPP No.20 of 1896, R.C.Water Supply, Rept.
%21. VicPP No.20 of 1896, R.C.Water Supply, Rept. Appendix c, p.205.
%22. Leader, 3/7/1886.
%23. C.S.Martin, 1955, p.45.
%24. VicPP No.20 of 1896, R.C.Water Supply. Rept.
%25. VicPP No 20 of 1896, R.C.Water Supply,Rept..\& VicPP No.21 of
%       1896, R.C.Water Supply,MoE.
%26. VicPP No. 20 of 1896, R.C.Water Supply Rept. p.185.
%27. VicPD 18/12/1896, p.4738.
%28. M.Sharland, These Verdant Plains, History Of East Loddon Shire, 1971.
%29. S.Murray, 1892, p.13.
%30. VicPP No.20 of 1896, R.C.Water Supply, Rept, p.191.
%31. H.L.Boorman, Irrigation And Water Supply Development In Victoria, 
%       1942, Govt.Printer, Melb., p.34, \& C.S.Martin 1955, p.90.
%32. H.G.McKinney, The Wentworth Irrigation Scheme, Agr.Gaz.NSW. 1896, 
%      vol.7, pp 469-470.
%33. K.Jeffcoat, More Precious Than Gold, An Illustrated History Of Water In   
%      New South Wales, 1988, p.113.
%34. W.H.Williamson, Water - From Tank Stream To Snowy Scheme, Ch. 2 in
%      A Century of Scientific Progress, Sydney 1968.
%35. C.J.Lloyd, L.A.G.Wade, 1990, ADB vol.12, p.342.
%36. S.Wells, Paddle Steamers To Cornucopia, The Renmark-Mildura 
%      Experiment of 1887, 1986, p.149.

