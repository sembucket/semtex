% $Id$
% CHAPTER NINE 
% At 1/5/99 effective length was 5948 words

\setcounter{endnote}{0}

\chapter{Irrigation Colonies at Mildura and Renmark}
\label{ch:colonies}\addtoendnotes{\protect\section*{Chapter \thechapter}}
\markboth{
\textsc{Chapter \thechapter. Mildura \& Renmark}
}{
}

Many settlements for irrigation of relatively unoccupied land were
begun abroad late in the 19th century and became known as irrigation
colonies.  Those in southern California were promoted commercially,
with sales of land and supply of water for orchards and grape
vines.\fn{O.\,O.~Winther, The Colony System of Southern California,
Agric.\ History, 1953, vol.\,27, pp.\,94--102.}  In the Punjab
province of India, the British government built major irrigation
canals and made settlements, also known as canal colonies, for
production of wheat and cotton by colonists drawn from other
areas.\fn{D.\,G.~Harris, \textsl{Irrigation in India}, 1923, London.}

The spread of irrigation in America encouraged one Australian in 1884
to propose an irrigation settlement on the Murray River in South
Australia but it was left to George and W.\,B.~Chaffey, Canadian
brothers with experience of Californian irrigation, to launch the
first irrigation colonies in 1887 at Mildura and Renmark on that
river.  These ventures provided models for other irrigation schemes in
Australia and therefore deserve detailed consideration.

\section*{Chaffey enterprises in California}

George Chaffey (1848--1932) and his brother William Benjamin
(1856--1926) had successfully launched two colonies in California by
1884.  William and a younger brother left Ontario, Canada, in 1878
with parents who were persuaded by a fellow Canadian to follow his
example by acquiring land in the irrigation colony known as Riverside,
then embracing about 9000\,ac of land 60 miles east of Los
Angeles. Founded in 1870, its orchards and vines were then watered
from canals supplied by gravity from the Santa Ana river.  In 1876 it
gained a railway connection across the continent and by 1880 Riverside
was distinguished by its successful introduction of Washington navel
oranges to California.\fn{T.~Patterson, \textsl{A Colony for
California, Riverside's First Hundred Years}, 1971.}  As the sales of
land were made by promoters also engaged in selling water from their
canal, settlers faced difficulties with water supply as land sales
improved.  Elwood Mead, the American who later took charge of
irrigation in Victoria, described the situation in the region:
\begin{quote}
	the spirit of speculation in which Californian civilisation
	was born soon fastened upon irrigation and ran a mad race
	through Southern California\ldots Men began to dream of a new
	race of millionaires, created by making merchandise of the
	melting snows, by selling `rights' to the `renting' of water,
	and collecting annual toll from a new class of society, to be
	known as `water tenants'.\fn{E.~Mead, Rise and future of
	irrigation in the United States, p.\,591 in USDA Yearbook of
	Agriculture 1899, Washington USA 1900.}
\end{quote}

George Chaffey, a mechanical engineer with experience of steamboats on
inland waterways in the United States and Canada, visited Riverside in
1880, and decided to stay after finding opportunities there and
elsewhere in southern California for his talents and initiative.  In
1882 the brothers began promotion of their first irrigation colony,
known as Etiwanda.  They purchased 1000\,ac on the slopes at the
foot of the lofty San Gabriel mountains with rights to a limited water
supply from mountain canyons.  Land was sold in blocks of 10\,ac on
the basis that a buyer gained shares in a mutual water company in
proportion to acreage held and was thus assured of water supply.  Its
supply for irrigation was by gravitation in concrete pipes, thus
eliminating losses suffered with earthen channels.

Encouraged by successful promotion of irrigation at Etiwanda, the
Chaffeys in 1883 launched a colony named Ontario on their purchase of
6216\,ac.  Its situation and water supply were similar to those of
Etiwanda and it was adjacent to a railway.  The appeal of growing
oranges was then important in promoting irrigation colonies in the
region, and within a few years more than 100 sales, mainly for
holdings of 10 or 20\,ac at Ontario, had been made to people from
various parts of the United States and Canada.\fn{J.\,A.~Alexander,
\textsl{The Life of George Chaffey}, 1928, \& R.~L.~Gentilcore, Agric.\ 
History, 1960, vol.\,34, pp.\,77--87.}

The use of concrete water pipes at Ontario was referred to in an
official Victorian report.\fn{VicPP no.\,19A of 1885. RC~Water
Supply, Further Progress Rept. \& J.\,E.\,M.~Vincent, \textsl{The
Australian Irrigation Colonies on the River Murray in Victoria and
South Australia}, 1889, p.\,101.}  An additional feature was the
establishment of a College of Agriculture.  Ontario followed the
example of Riverside by concentration on fruit production from small
holdings.

The success of Ontario became known to Australians visiting USA in
1885 to gain information for the Victorian Royal Commission on Water
Supply.\fn{VicPP no.\,19 of 1885, RC~Water Supply.}  Alfred Deakin,
chairman of the Commission, had been advised while in San Francisco to
look at irrigation colonies in southern California.  At Los Angeles
Deakin together with two Melbourne journalists met George Chaffey, at
Ontario 35 miles distant there was William Chaffey to show them around
the settlement, and at Riverside Alfred Deakin dined with Luther Holt,
whose idea of a mutual water company had been used with advantage for
the Chaffey colonies.

\section*{From California to Australia}

An off-hand suggestion at the Ontario colony by one of the Melbourne
journalists to William Chaffey that the brothers might do well in
Australia is the basis for claims that they were invited to develop
irrigation there.\fn{J.\,A.~Alexander, 1928, p.\,91.}  George and
W.\,B.~Chaffey independently were tempted by the prospects in
Australia and there was less reason for remaining in California after
the death of their father in 1884.  Even before Deakin's visit to
California, George had been influenced by advice from a much-travelled
Englishman, Stephen Cureton, who had spent several months in Australia
and claimed that land for irrigation could be acquired there cheaply.
The brothers decided to send him to Australia as their agent.  He
reached Melbourne in the spring of 1885.

A diary kept by Deakin indicates that Cureton called on him early in
October 1885 and had several meetings with him over three
months.\fn{Diary 1885, Deakin papers 1540/2/4, NL.}  Stephen Cureton,
born June 1845 in Shropshire, England, gained prominence in Victoria
from his close association with the Chaffey brothers.  Initially he
was accepted in Victoria as an authority on irrigation and
horticulture, giving evidence to two royal commissions and a public
lecture later at Shepparton.  He soon advised George Chaffey to travel
to Melbourne.

George Chaffey reached Victoria on 13 February 1886 and began a series
of meetings with Alfred Deakin, then Minister for Water Supply, whose
diary shows that they stretched over three
months.\fn{J.\,A.~Alexander, 1928, p.\,101, \& Diary 1886, Deakin
papers 1540/2/5 NL.}  Chaffey sought a substantial grant of land on
which to develop irrigation as in California; Deakin knew the
obstacles to land grants for such a purpose but was optimistic that a
deal could be arranged.  George Chaffey was sufficiently certain of
the outcome to cable his brother, advising him to come to Australia
and authorising him to sell their Californian assets.  The Chaffeys
sold their Californian assets in March 1886, possibly for the assessed
value of \$310\,000 (\pounds62\,000) reported for Ontario in 1886;
their expenditure to mid 1884 was given as \$238\,500
(\pounds47\,700).\fn{J.\,A.~Alexander, 1928, p.\,98, R.\,L.~Gentilcore
1960, \& J.\,E.\,M.~Vincent,1889, p.\,98, re equivalence of US\$5 to
\pounds1.}

\section*{Mildura chosen for irrigation}

Deakin advised George Chaffey to become familiar with the Murray
valley.  He travelled to Echuca and downstream as far as the Kulkyne
pastoral station in the mallee country.\fn{J.\,A.~Alexander, p.\,108.}
There he learnt that the Mildura station, not far distant to the
northwest, might be suitable for irrigation.  It is unlikely that this
was his first information on Mildura as Cureton had apparently called
there on a trip by steamer down the Murray during his earlier visit to
Australia.  The extensive pastoral station or run known as Mildura
embraced approximately 64\,000\,ac and had been occupied for many
years by the Jamieson brothers, who until 1878 grazed thousands of
sheep on its grassy country near the river and in the adjacent mallee
scrub.\fn{J.\,A.~Alexander, p.\,109, \& A.\,S.~Kenyon, The story of
the Mallee, Vic.\ Hist.\ Mag.\ 1914--1915.}  Their homestead by the
river was known to many travellers along that stream who were
attracted by its garden and orchard irrigated from the river.  Rabbits
began infesting the district by 1878 and so reduced the capacity of
the current landholder, the Tapalin Pastoral Company, to graze
livestock that it became insolvent and the Commercial Bank of
Adelaide, its main creditor, crashed in February 1886.\fn{S.~Wells,
\textsl{Paddle Steamers to Cornucopia, The Renmark--Mildura Experiment
of 1887}, 1986, p.\,40.}

After returning to Melbourne and consulting Deakin about Mildura,
George Chaffey apparently travelled to Adelaide and then proceeded by
train and paddle steamer to Wentworth, the busy port at the junction
of the Darling and Murray Rivers and only 15 miles from the Mildura
homestead.\fn{J.\,A.~Alexander, 1928, p.\,108.}  George Chaffey met
William Paterson, manager of the Mildura station, at Wentworth and
asked to be shown that pastoral run.  Paterson agreed only after
Chaffey produced a letter from a bank manager seeking payment from the
station lessee and identifying Chaffey as a probable buyer of the
station.  George Chaffey spent at least three autumn days at Mildura
examining pine-covered sandhills, plains covered with mallee eucalypts
and bluebush, and flats near the river covered with box eucalypts.  He
looked at the soils and subsoils, enquired about the rise and fall of
the river and the features of a particular billabong, and even got the
use of a theodolite to find the height of land above the river.  The
visitor was convinced of the suitability of this country for
irrigation and told the bewildered manager of his intention to buy the
place.  All this was told later by Paterson to Price Fletcher (`Jethro
Tull'), the agricultural journalist of a Queensland
newspaper.\fn{Queenslander, 1 Nov.\ 1890.}

The timing and circumstances of these inspections by George Chaffey of
land near the River Murray in Victoria and South Australia are
uncertain in view of Deakin's relevant diary entries for 1886 which do
not indicate the appropriate long intervals between appointments.
These show:
\begin{tabbing}
        12345679012345\=123456789012345\kill
	14 February \> George Chaffey\\
	17 February \> Chaffey\\
	10 March    \> Chaffey back\\
	19 March    \> Chaffey \\
	22 March    \> Chaffey's proposal \\
	29 March    \> Chaffey \\
	 1 April    \> Chaffey \pounds500 \\
	15 April    \> Chaffey \\
	22 April    \> Chaffey \\
	23 December \> Chaffeys \\
	24 December \> Chaffeys.\fn{Diary 1886, Deakin Papers 1540/2/5, NL.}\\
\end{tabbing}
The interval of almost three weeks before 10 March should have given
time for Chaffey's inspection along the river from Echuca as far as
Kulkyne and his return to Melbourne, in which case a return visit to
Mildura via Adelaide during the first fortnight in April is
problematic considering that the Melbourne to Adelaide railway was
then incomplete.\fn{L.\,J.~Blake, \textsl{The Land of the Lowan, 100
Years in Nhill and West Wimmera}, 1976, p.\,100.}

\section*{Negotiations for the Mildura project}

After his visit to Mildura, Chaffey started discussions with Deakin
about transferring the Mildura land to the firm of Chaffey Brothers.
There were many complications arising from the current arrangements
for use of the mallee country in Victoria.  That vast tract of public
or crown land with low rainfall and a dense cover of eucalypt scrub
was divided into many large leasehold blocks held by pastoralists,
several of these made up the Mildura station.  The leases were now
held by the liquidators of the failed Adelaide bank and by some
coincidence the principal liquidator was a prominent South Australian
politician, Thomas Playford.  The only pieces of freehold land in the
region were those which could be bought to include a station homestead
and could not exceed 640\,ac.  There was such a title for the land
including the Mildura homestead on the river.  Any arrangement to
accommodate the Chaffey request would need to provide compensation for
cancellation of unexpired leases and make new boundaries.

The availability of water for irrigation in this region of low
rainfall and hot summers was another important consideration.  Records
of the river level were available for Mildura; and monthly changes in
the volume of river water for the past 20 years were known for the
upstream port of Echuca.  The Chaffeys expected to follow their
Californian procedure of irrigating fruit trees and vines, which
needed abundant water in the hot dry summer, a time of reduced river
flow.  As a result of conferences in 1885 between the appropriate
royal commissions in Victoria and New South Wales, it was expected
that although the River Murray does not lie in Victoria its waters
would be divided equally between the two colonies.  That arrangement
would have been prejudicial to South Australian interests in river
traffic, so the fact that a prominent South Australian politician was
now in control of the Mildura leases meant that the Victorian
government had to be careful in determining how much water the
Chaffeys should be allowed to pump from the river for irrigation.

Another aspect of development at Mildura was transport.  The Victorian
railhead nearest to Mildura was then at Kerang, more than 150 miles
distant by coach.  River transport upstream from Mildura was subject
to interruption when levels dropped in summer and autumn but was then
more reliable downstream thanks to seasonal inflow from the Darling
River at Wentworth.  A railway connection to Melbourne would have
involved laying an additional 160 miles of track across unproductive
mallee country.  Unless this extension was made, the Mildura
irrigation scheme would have to rely mainly on river transport.

Negotiations between Deakin and Chaffey were completed in October
1886, when an announcement was made in the Victorian Parliament during
debate on the Waterworks Construction Encouragement Bill.  Their
agreement provided that under certain conditions the Chaffey Brothers
would receive a grant of 50\,000\,ac at Mildura, and the right to
acquire adjoining land, to the extent of 200\,000\,ac at a cost of
\pounds2/ac.  The main conditions were to establish an irrigation
settlement on the area it was proposed to grant, and to make permanent
improvements on it such as irrigation channels, installation of
pumping equipment, roads and bridges, all involving expenditure to the
extent of \pounds300\,000 within 20 years, with specified outlays in
each five-year term.  As expenditure mounted, the Chaffeys would
gradually acquire freehold title to an increasing portion of the
50\,000\,ac; this was specified as 1\,ac for each \pounds5
expenditure, with the limitation that no grant was to be for less than
640\,ac.\fn{J.\,A.~Alexander, 1928, p.\,118.}  They were also to
provide an agricultural college and establish fruit packing and
canning, and the allotments offered for sale were to be no larger than
80\,ac for horticulture and 160\,ac for
agriculture.\fn{J.\,A.~Alexander, 1928, p.\,119.} The Chaffeys were to
take water for irrigation from the Murray but only as authorised by
the government.

News of this agreement provoked lively debate on the Bill in
parliament, continuing long enough to threaten its safe conduct and
thus jeopardise negotiations with the Chaffeys.  Finally, to avoid
defeat, Deakin and the government were obliged to allow tenders to be
called for proposed irrigation at Mildura.  There was promise at this
stage of a rival to the Chaffeys with the announcement that a
Victorian politician was travelling to Britain to seek the necessary
finance.  Prospects for the Chaffeys looked bleak when William Chaffey
reached Melbourne late in December 1886.

\section*{Starting two settlements}

At this point, the South Australian government took the opportunity of
offering the Chaffeys land by the Murray for irrigation development as
proposed for Mildura.  Land held under pastoral lease as the Bookmark
station near the Victorian border was to be made available to the
Chaffeys, who in January 1887 spent several days together with the
Conservator of Water (J.\,W.~Jones) inspecting land along the Murray
before accepting the offer in February 1887 for an area later known as
Renmark.\fn{Kapunda Herald, 18 Jan.\ 1887 \& S.~Wells, 1986,
pp.\,48--49, 65.}  They had chosen land much less elevated above the
river than at Mildura.  It ranged from clays partly subject to
flooding to sandy ground up to 60\,ft above the river and covered with
native pines.  After returning from this part of the colony they soon
found that tenders for the Mildura project had closed without any
submissions.  The Chaffeys were still interested in the Mildura
scheme, as was the Victorian government, so a new arrangement was made
between the parties to ensure the development.  Meanwhile the South
Australian proposal awaited endorsement in a new parliament to be
elected in the coming winter.

Even before the new Victorian agreement known as the Indenture
received vice-regal consent at the end of May, the Chaffeys began work
at Mildura with survey parties.  In the coming winter the work of
clearing and ploughing began with huge steam tractors from England.
They were under the control of a young engineer, Peter McLaren, who
remained in Mildura to become resident engineer for Chaffey Brothers.
It was also time to give serious attention to Renmark as the assent to
the agreement concerning that area had been given.  George Chaffey
took up residence then at Renmark; William Chaffey was occupied mainly
with business affairs in Melbourne leading to establishment of the
firm Chaffey Brothers Ltd in October 1887, its directors being the two
Chaffeys, Stephen Cureton, and William Paterson, the Mildura station
manager.\fn{S.\,Wells, 1986, p.\,118.}  A site six miles downstream
from the Mildura homestead had first been selected for the township
but it later became known as `Old Mildura', and was devoted to
irrigation of Lord Ranfurley's orchards and vineyards.\fn{S.~Wells,
1986, p.\,69.}  This change of plans followed Chaffeys' purchase in
August 1887 of the freehold land associated with the
homestead.\fn{S.~Wells, 1986, p.\,56.}  As clearing and surveys
proceeded, the land was divided into blocks for sale, with access by a
gridwork of roads as in their Ontario settlement in California.  Plans
were made for the system of pumps required to supply water from the
river to channels at heights from 50 to 90\,ft above the river in
order to supply the higher land south of the township and the shallow
valley or trough to the southwest.\fn{B.\,E.~Butler, Aeolian landscape
materials and forms: examples from the Mildura region, p.\,40, in
\textsl{Aeolian Landscapes in the Semi-arid Zone of South-eastern
Australia}. Procs Mildura Conf.\ 1979,\,1980, Wagga Wagga.}  At first
only two streets in the township were distinguished by personal names:
the spacious main avenue was named for Deakin. and another along the
river frontage was named for Cureton, apparently in honour of his
efforts to establish the settlement.

It might have been expected that the Chaffeys would deliver irrigation
water by piping as they had done at Ontario but the slopes of the land
selected for irrigation at Mildura and Renmark were regarded as
insufficient for gravitation of water through pipes of the style used
by the Chaffeys in California.  However, South Australian patents were
taken out later by the Chaffeys for their composite pipes and pipe
moulds.\fn{D.\,A.~Cumming and G.~Moxham, \textsl{They Built South
Australia}, p.\,35, 1986, Adelaide.}  During 1887 plans were completed
for the location at Mildura of pumping stations and the positions of
the main irrigation channels.

Meanwhile at Renmark progress had been made since August 1887 with
surveying and clearing about 4000\,ac of low-lying land near an
effluent of the river.  George Chaffey's move there put him in a good
position to make adequate arrangements for the river transport vital
to the development of both settlements.  His enterprise soon brought
about the formation of the River Murray Navigation Co.\ Ltd with
possession of steam-boats and the services of Captain Hugh King, an
experienced river-man.\fn{S.~Wells, 1986, pp.\,84--86.}  Subsequently
Charles Chaffey, the third brother, took up residence at Renmark as
manager of the project and remained there for several years.

Irrigation at both sites required centrifugal pumps to lift water from
the river in stages.  At Mildura the supply of water to channels up to
85\,ft above the river would involve four stages; at Renmark three
stages were needed.  The initial lift at both places involved pumping
from the river into abandoned channels or effluents which could serve
also as reservoirs.

During 1888 contractors at both settlements made earthen channels
cemented in places to reduce seepage; settlers began planting their
allotments with fruit trees and vines; and some equipment was
installed to pump water from the river.  Temporary arrangements
allowed watering of some 80\,ac at Mildura using a centrifugal pump
and a portable engine and limited irrigation was also undertaken at
Renmark.\fn{J.\,E.\,M.~Vincent, \textsl{The Australian Irrigation
Colonies}, 1889, p.\,127.}

At Mildura George Chaffey made use of imported triple-expansion steam
engines coupled with centrifugal pumps; some were delivered in 1888
and the most powerful were made by Tangyes in England to his own
design and supplied in 1889.  Less powerful engines and pumps were
provided at Renmark to cope with the smaller area involved and the
lower lifts.  Generation of steam by boilers at the pump sites
required substantial supplies of local firewood.\fn{S.~Wells, 1986,
pp.\,114--115.}  The Mildura Irrigation Company and the Renmark
Irrigation Company, corresponding to the mutual irrigation companies
introduced in California, were formed in 1888 and 1889 to deal
specifically with the supply of irrigation water.\fn{S.~Wells, 1986,
p.\,127.}

Sales of land at the settlements were promoted by imaginative
advertisements and special excursions from Melbourne and Adelaide.
One particularly effective publication was the book entitled `The
Australian Irrigation Colonies', known widely as the Red Book from its
coloured cover.  This appeared in 1888, with a further edition in
1889; many settlers came from Britain in response to its advice on
opportunities to profit from production of oranges, wine, raisins,
olives and fruit from deciduous trees.\fn{J.\,E.\,M.~Vincent,
\textsl{The Australian Irrigation Colonies}, 1889. }

Changes were made during 1888 in the control of the main Chaffey
company.  Stephen Cureton, whose involvement in the irrigation scheme
had been important at its inception, retired as a director of Chaffey
Bros Ltd, took out a large sum of money as his share in the company
assets and was replaced by Jonas Levien (1840--1906), a Victorian
politician with horticultural experience and a reputation for ability
to arrange financial backing for his enterprises.  Levien soon
replaced George Chaffey as chairman of the board of directors.

With the successful testing of the powerful Chaffey pumping engine
early in 1890, George Chaffey completed his major contribution to
irrigation at Mildura.  Then troubles developed over management of
irrigation there.  W.\,M.~Paterson, the Mildura manager of Chaffey
Bros Ltd and a director of the Mildura Irrigation Co., was replaced as
manager by W.\,J.~Waddingham, a relative of the Chaffeys.\fn{S.~Wells,
1986, pp.\,150--51.}  Paterson then became involved in local
discontent during 1891 and 1892 over charges to be made by the
Irrigation Company: there were claims by some irrigators that no
charges should be made for the supply of water, and in fact no charges
were made for irrigation water before 1890.  The Victorian government
appointed A.\,W.~Howitt in 1892 to make enquiries, received his report
upholding a rating system for Mildura, and subsequently introduced
legislation which became the subject of another enquiry by a Select
Committee in 1892.  The outcome was the Mildura Rating Act.\fn{VicPP
1892--93. V\&P LA vol.\,1 2nd Sess., Rept Select Comm. LA upon the
Mildura Settlement \& S.~Wells, 1986, pp.\,143--44.}

Irrigators at Renmark were also concerned about management of the
local irrigation company, then virtually controlled by Chaffey
Brothers.  However, in 1892 after transferring control to the settlers
the Chaffeys contracted to operate the irrigation system.  Next year
the irrigation company was replaced by the Renmark Irrigation Trust.
This was achieved through legislation in 1893 which provided for
management of the trust by a board of seven elected
irrigators.\fn{S.~Wells, 1986, p.\,149}

Despite many difficulties in developing the irrigation colonies,
progress was achieved in the extension of cultivation and the growth
of population.  In 1891 the Chaffeys had sold 17\,000\,ac at Mildura
and 4500\,ac at Renmark; their settlers had planted 6500\,ac at
Mildura and about 1500\,ac at Renmark; and populations were 4000
and 600 respectively.  Mildura in 1894 had more than 4000\,ac
planted with fruit trees including apricots, citrus, peaches and
olives, less than 3500\,ac with vines mainly for raisins, and
700\,ac with annual or green crops.\fn{CSIR Bulletin 133, A soil
survey of the Mildura irrigation settlement, 1940, \& S.~Murray,
\textsl{Irrigation in Victoria}, 1892 p.\,6.}  Renmark in 1895
had 2864\,ac under irrigation with almost half that area planted
with deciduous fruit trees and citrus and the balance made up almost
equally by vines and fodder crops.\fn{S.~Wells, 1986, p.\,157.}

Transport became more of a problem when the settlements reached the
stage of producing goods for sale.  Despite encouraging remarks from
Richard Speight, chairman of the Victorian Railway Commissioners, on
his visit to Mildura in November 1888, nothing had been done to
provide a railway to that settlement and it remained dependent on
river transport.\fn{J.\,E.\,M.~Vincent, 1889, p.\,120.}  This was
reliable except during periods of low river levels and these
unfortunately occurred in several years during summer and autumn when
fresh fruit awaited transport to markets.\fn{S.\,Wells, 1986,
p.\,265.}  George Chaffey tried to cope with the problem of low river
levels by importing an American boat, the stern-wheel steamer
\textsl{Pearl}, with low draft but it was not a success.\fn{S.~Wells,
1986, pp.\,88--89.}  The rail connection to Melbourne was not achieved
until late in 1903.  Renmark was better placed except in exceptional
droughts such as that of 1914 when the river virtually dried up.
Citrus growers had more success: in June 1892 the first exports to
England were made, and in 1893 hundreds of cases of oranges and lemons
reached England and America.\fn{Ernestine Hill, \textsl{Water into
Gold}, 1937, p.\,125.}  In order to cope with marketing stone fruit
and vine fruits it was necessary to resort to drying and success with
this process firmly established the dried fruit industry in the two
irrigation settlements.

\section*{Waterlogging and salinity}

After irrigation was provided for the orchards and vineyards, many
settlers became aware of unexpected problems associated with watering.
The unlined earthen channels were subject to seepage or even collapse
after burrowing by freshwater crayfish (`yabbies') introduced from the
river.  This loss of water added to the expense of pumping from the
river and it also involved damage to fruit trees and vines from
waterlogging and salinity.  Seepage was detected at Mildura in 1892
when it extended 200\,ft from the banks of certain
channels.\fn{J.\,A.~Alexander, 1928, p.\,198.}  At Renmark the loss
from supply channels was estimated at 20 per cent before 1895 while at
Mildura where the length of main channels exceeded 100 miles in 1892,
the loss was reported to be more than half.\fn{SAPP no.\,113 of
1895. Select Committee Village Settlements etc., Rept. \& VictPP
no.\,19 of 1896, RC~Mildura, Rept, p.\,vi.}  Seepage from channels
can be reduced or averted by lining the earthen structures with
impermeable material such as concrete.

Excessive salinity in the soil was recognised soon after irrigation at
Mildura and Renmark.  This feature might have been anticipated from
soil analyses and it is possible that the Chaffeys had relevant
information when they expressed their interest in soils during an
interview, probably in 1887:
\begin{quote}
	\ldots we have had analyses made of the soil and compared them with
	the results of analyses made of soil from
	successfully-irrigated colonies in
	America.\fn{J.\,E.\,M.~Vincent, 1889, p.\,98, quotation from
	Melbourne Argus.}
\end{quote}
George Chaffey told the Victorian Select Committee on Mildura in 1892
that he examined the soils before irrigation channels were made and
found there was an `immense quantity of alkalis' which were `not
injurious to plant life'.\fn{VicPP V\&P LA vol.\,1, 1892--93, 2nd
sess.\ Select Comm.\ Mildura Settlement. Rept \& MoE, Q818--819.}

The first indication that salinity could be a disadvantage to
Australian irrigators came soon after irrigation at Mildura in 1888.
Land had been cleared and cultivated in 1887 and in 1888 irrigation
was begun though only 120\,ac had been planted.\fn{S.\,Wells, 1986,
p.\,73.}  In February 1889 four men from the Western Wimmera
Irrigation Trust visited Mildura and reported the existence of a white
salty crust on land that had been watered.\fn{Silenus (A.\,S.~Kenyon),
Early Mildura as others saw it. Sunraysia Daily, 19 Aug.\ 1921.}  A
few years later there was more definite evidence that salinity was a
problem for irrigators at Mildura. James Henshilwood had taken up a
block of 40\,ac on land previously covered by mallee vegetation.  He
had planted fruit trees of various types in 1890.  In the next two
years some trees failed to make satisfactory growth and were replaced.
Samples of his soil were sent to Melbourne for analysis, which showed
that one contained `a great excess of salt'.  He was advised to
install deep open drains and wash out the salt.  He had further
trouble with citrus though stone fruit did better.  Apparently
uncertain of the diagnosis given by the Melbourne analyst, Henshilwood
wrote to Professor Hilgard in California in 1895 for advice on
treatment of his soil.  Hilgard's reply was directed to finding if the
soil also had a problem of alkalinity and asked for samples of soil.
When these were sent in 1895, the results of analysis showed no
alkalinity and the very low concentration of salt could be explained
as the result of the high rainfall (30\,in) in that year.\fn{VicPP
no.\,19 of 1896, RC~Mildura, Appendix~A2, pp.\,277--279.}

Before 1900 settlers at Mildura and Renmark were familiar with the
appearance of waterlogging, seepage, and the effects of soil salinity
or alkalinity, all having some ill effects on fruit trees and vines.
The Chaffey brothers apparently had no experience of these problems in
their Californian settlements, but Professor Hilgard's advice on
under-drainage of Californian soils affected by salinity or alkalinity
after irrigation was taken up there by several landholders.  By 1907
individual landholders at Mildura had installed tile drains which
discharged via shafts or bores to underlying beds of sand at depths of
80 to 90\,ft.\fn{A.\,S.~Kenyon, Drainage and irrigation, J.\,Agric.\
Vic.\ 1907, vol.\,5, p.\,206, \& CSIR Bull.\ no.\,133, 1940, Soil
survey of the Mildura irrigation settlement.}  This method was less
reliable at Renmark.\fn{CSIR Bull.\ no.\,56, Soil survey of Renmark
irrigation district, South Australia.}

\section*{Chaffey Bros Ltd becomes insolvent}

While various technical and administrative problems of the two
settlements were being dealt with, a major financial crisis for the
Chaffeys was looming.  Much of their success had occurred during the
prosperity enjoyed in Victoria in the late 1880s following British
investments.  But in the eight months from July 1891 twenty major
financial institutions closed and British investments were
curtailed.\fn{M.~Cannon, \textsl{The Land Boomers}, 1986, p.\,26.}
The Chaffeys tried to avoid involvement by making personal appearances
in England to raise funds but failed after reports were circulated
there of the difficulties confronting settlers in the irrigation
colonies.  There was worse to come in 1893 when banks in Melbourne
suspended trading and the boom collapsed.  Some settlers left Mildura
and Renmark, seeking work even as far away as the new goldfields in
Western Australia.  Then Chaffey Bros Ltd went into liquidation in
1895 and legislation was passed enabling the establishment of the
First Mildura Irrigation Trust with board members elected by settlers
in much the same way as for the Renmark Irrigation Trust.  A few
months later a Royal Commission was appointed to enquire into
circumstances of the Mildura settlement.\fn{M.~Cannon, 1986, p.\,81.}
It was followed in South Australia by a similar enquiry concerning
Renmark and other settlements along the river.

The report of the Victorian royal commission found the Chaffeys and
the Victorian government had contributed to the debacle at
Mildura.\fn{VicPP no.\,19 of 1896, RC~Mildura, Rept.}  Many
witnesses, including George Chaffey, Alfred Deakin, and Stephen
Cureton had been called to give evidence before the commissioners.
They found that several provisions of the indenture of May 1887 had
not been met, though these deficiencies did not include the
expenditures of money required to obtain the crown grants of land.
The enquiry gave attention particularly to the arrangements for
irrigation, which were then in a critical state; as a result
recommendations were made in favour of a loan to ensure the lining of
existing channels, to improve the pumping plant and to make sure of a
water supply during the coming irrigation season.

One outcome of the debacle involving the Chaffey irrigation colonies
on the Murray was the return of George Chaffey to America after his
financial ruin and criticism of his irrigation development.  An
extensive defence of his work was given much later by his
biographer.\fn{J.\,A.~Alexander, 1928.} W.\,B.~Chaffey remained a
respected figure at Mildura where his fortunes gradually improved.

The South Australian enquiry took evidence from many Renmark settlers
and some members of the managerial staff, with general agreement that
serious seepage from unlined earthen channels had occurred.
Recommendations in the progress report of 1899 were concerned mainly
with maintenance of the water supply channels, including provision of
funds for the purpose.\fn{SAPP no.\,37 of 1899, RC~Renmark and
Murray River Settlements, Progress Rept and MoE.}

Improvements in the irrigation systems of the settlements were made
after the two irrigation trusts became responsible, but there were
recurrent difficulties with water supplies when the river flow was
seriously diminished in periods of drought.  Some future improvement
in water supplies from the river was indicated when the Commonwealth
and three States reached agreement in 1915 on management of the river
by provision of weirs and locks and construction of storages.

Despite the reduction in seepage losses from channels as their lining
with cement extended, waterlogging due to irrigation forced many
settlers to develop individual drainage systems.  The horticultural
industry continued to be based on production of dried fruit, which
gave good financial returns during the 1914--18 war.  Plantings of
wine grapevines began early in the development of the irrigation
colonies and led to establishment of wineries and distilleries at
Mildura and Renmark.\fn{S.~Wells, 1986, pp.\,123 \& 236.}

\section*{Conclusions}

The Mildura and Renmark colonies transferred much of the Californian
methods of irrigation for establishment of horticulture and
viticulture in a region regarded previously by many Australians as
almost unproductive.  The venture by the Chaffeys captured popular
attention at a time of prosperity but almost collapsed ten years after
it began.  The Chaffeys followed the same procedure as they had used
for promotion of their earlier settlements in California but their
Australian ventures were more ambitious and lacked the access to
reliable transport and extensive markets that enabled their earlier
ventures to flourish.  George and W.\,B.~Chaffey were an impressive
pair but they took on responsibilities beyond their capacity to
maintain.  When the major engineering problems of the two irrigation
settlements had been attended to, George was tempted by the scope for
irrigation in other parts of Australia and W.\,B.~Chaffey had to spend
time abroad after personal tragedy.  Management of affairs at Renmark
was handled successfully for a time by the youngest brother, Charles,
while serious problems at Mildura were proving beyond the capacity of
new managers there.

A basic problem of these ventures was that financing development
depended largely on sales of land.  This method succeeded initially
but could not cope with problems encountered early in the 1890s when
sales declined.

The lack of reliable transport to Mildura was a serious disadvantage
which forced dependence on production of dried or canned fruit but the
Chaffeys failed to develop a canning facility there.  Apart from a few
Americans brought to Mildura and Renmark by the Chaffeys, the settlers
had no prior experience of irrigation.

%\section*{References}
%1. O.O.Winther, The Colony System Of Southern California, Agric.History, 
%     1953, vol.27, pp. 94-102.
%2. D.G.Harris, Irrigation In India, 1923, London.
%3. T.Patterson, A Colony For California, Riverside's First Hundred Years, 
%     1971.
%4. E.Mead, Rise And Future Of Irrigation In The United States, p. 591 in 
%     USDA Yearbook of Agriculture 1899, Washington USA 1900.
%5. J.A.Alexander, The Life Of George Chaffey, 1928, \& R.L.Gentilcore, 
%    Agric.History, 1960, vol.34,p.77-87.
%6. VicPP No. 19A of 1885. R.C.Water Supply, Further Progress Rept. \&
%     J.E.M.Vincent, The Australian Irrigation Colonies On The River Murray In 
%     Victoria And South Australia, 1889, p.101.
%7. VicPP No. 19 of 1885, R.C.Water Supply.
%8. J.A.Alexander, 1928, p.91.
%9. Diary 1885, Deakin papers 1540/2/4, NL.
%10. J.A.Alexander, 1928, p.101, \& Diary 1886, Deakin papers 1540/2/5 NL.
%11. J.A.Alexander, 1928,p.98, R.L.Gentilcore 1960, \&
%      J.E.M.Vincent,1889, p.98, re equivalence of US\$5 to \pounds1.
%12. J.A.Alexander, p.108.
%13. J.A.Alexander, p.109, \& A.S.Kenyon, The Story Of The Mallee, Vict.Hist 
%     Mag. 1914-1915.
%14. S.Wells, Paddle Steamers To Cornucopia, The Renmark-Mildura 
%       Experiment of 1887, 1986, p.40.
%15. J.A.Alexander, 1928, p.108.
%16. Queenslander, 1/11/1890.
%17. Diary 1886, Deakin Papers 1540/2/5, NL.
%18. L.J.Blake, The Land Of The Lowan, 100 Years In Nhill And West 
%     Wimmera, 1976, p.100.
%19. J.A.Alexander, 1928, p.118.
%20. J.A.Alexander, 1928, p.119.
%21. Kapunda Herald, 18/1/1887 \&  S.Wells,1986, pp. 48-49,65.
%22. S.Wells, 1986, p.118.
%23. S.Wells, 1986, p.69.
%24. S.Wells, 1986, p.56.
%25. B.E.Butler, Aeolian Landscape Materials And Forms: Examples From The
%      Mildura Region. , p.40, in Aeolian Landscapes In The Semi-arid Zone Of
%      South-Eastern Australia. Procs Mildura Conf. 1979, 1980, Wagga Wagga.   
%26. D.A.Cumming and G.Moxham, They Built South Australia, p.35,1986, 
%      Adelaide.
%27. S.Wells, 1986,pp84-86.
%28. J.E.M.Vincent, The Australian Irrigation Colonies, 1889, p.127.
%29. S.Wells, 1986, pp114-115.
%30. S.Wells, 1986, p.127.
%31. J.E.M.Vincent, The Australian Irrigation Colonies 1889. 
%32. S.Wells, 1986, p.150-51.
%33. VicPP 1892-93. V.\&P LA Vol.1 2nd Sess., Rept Select Comm.
%      LA upon the Mildura Settlement \& S.Wells, 1986, pp 143-44.
%34. S.Wells, 1986, p.149
%35. CSIR Bulletin 133, A Soil Survey Of The Mildura Irrigation Settlement,
%      1940, \& S. Murray, Irrigation In Victoria, 1892 p.6.
%36. S.Wells, 1986, p.157.
%37. J.E.M.Vincent, 1889, p.120.
%38. S.Wells, 1986, p.265.
%39. S.Wells, 1986, p.88-89.
%40. Ernestine Hill, Water Into Gold, 1937, p.125.
%41. J.A.Alexander, 1928, p.198.
%42. SAPP No. 113 of 1895. Select Committee Village Settlements etc. Rept.
%      \&  VictPP No.19 of 1896, R.C.Mildura, Rept. p.vi.
%43.  J.E.M.Vincent, 1889, p.98 quotation from Melbourne Argus.
%44. VicPP V \& P LA Vol.1, 1892-93, 2nd sess. Select Comm. 
%       Mildura Settlement. Rept. \& MoE, Q818-819.
%45. S.Wells, 1986, p.73.
%46. Silenus(A.S.Kenyon),Early Mildura As Others Saw It. Sunraysia Daily,
%      19/8/1921.
%47. VicPP No.19 of 1896, R.C.Mildura, Appendix A2,pp. 277-279.
%48. A.S.Kenyon, Drainage And Irrigation, J.Agric.Vict. 1907, vol.5,
%      206, \& .  CSIR Bull. No.133, 1940, Soil Survey Of The Mildura
%      Irrigation Settlement.
%49. CSIR Bull. No.56, Soil Survey of . . . . Renmark Irrigation
%       District, South Australia.
%50. M.Cannon, The Land Boomers, 1986, p.26.
%51. M.Cannon, 1986, p.81.
%52. VicPP No.19 of 1896, R.C.Mildura, Rept.
%53. J.A.Alexander, 1928.
%54. SAPP No. 37 of 1899, R.C.Renmark and Murray River Settlements,
%      Progress Rept and MoE.
%55. S.Wells, 1986, pp 123 \& 236.

