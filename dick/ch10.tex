% $Id$
% CHAPTER TEN
% 6160 words at 2/5/99

\setcounter{endnote}{0}

\chapter{Proposals for Other Irrigation Colonies}
\index{irrigation colonies!proposed}
\label{ch:proposals}\addtoendnotes{\protect\subsection*{Chapter \thechapter}}
\markboth{Chapter \thechapter. Proposals for Other Colonies}%
{Pioneering Irrigation in Australia}

More than twenty irrigation colonies were proposed in Victoria, New
South Wales and South Australia, generally in response to progress
with those at Mildura and Rermark.  They were all initiated by private
companies which had to rely on governments or semi-governmental bodies
for access to a water supply. With one exception, the following
schemes were proposed in a period of about ten years from 1884.  Eight
colonies, all in Victoria, were the only ones to come into existence
and survive after 1920.

\section*{Murray Valley}

\subsection*{Morgan, SA} \index{Morgan, SA}

Frederick Ludwig Bevilaqua (1832--1913) \index{Bevilaqua, F.\,L.} came
to South Australia from Germany in 1854 and for many years was a
trader in the South-East and the Barossa
Valley. \index{valley!Barossa} His interest in irrigation at first in
the 1860s went as far as an unsuccessful attempt to use water from the
Murray River \index{river!Murray} near Mannum. \index{Mannum} In 1884
he launched his proposal for an irrigation settlement near Morgan to
be developed by the South Australian Irrigation Company \index{SA
Irrigation Co.} on 25\,000\,acres, part of which was to be subdivided
into holdings of 50 to 100\,acres irrigated from the Murray by means of
pumps.  The chosen area was approved by George Goyder,
Surveyor-General, \index{Goyder, G.}  and endorsed by John Custance,
\index{Custace, J.} Professor of Agriculture at the recently
established Roseworthy Agricultural College. \index{Roseworthy
Agricultural College, SA} Bevilaqua was aware of contemporary interest
in irrigation as expressed in Victoria by the Australian Irrigationist
and the newspaper reports of American irrigation provided by
T.\,K.~Dow.\index{Dow, T.\,K.}  However, despite further enquiries,
nothing came of Bevilaqua's proposal.\fn{F.~Bevilaqua,
\textsl{General Remarks and Detail Accounts of the South Australian
Irrigation Company}, 1884; SAPP no.\,158 of 1884; SAPP no.\,73 of
1885.}

\subsection*{Lake Bonney, SA} \index{lake!Bonney, SA}

Lake Bonney, maintained by inflow from the Murray River, lies about 20
miles west of Renmark.  It was mentioned in a scheme for irrigation
proposed in 1888 by business men in Melbourne and Adelaide.  They
sought the use of 150\,000\,acres of land, then held under pastoral lease
on the Cobdogla pastoral station. \index{station, Cobdogla} The group,
soon identified as the Lake Bonney Irrigation Company, \index{Lake
Bonney Irrigation Co.} began negotiations with South Australian
government officials in May 1888 and arranged surveys to determine
levels and the possible location of irrigation channels. Their
application went to the South Australian government and was publicised
in the Adelaide press, with some expressions of opposition from those
preferring development by the government rather than a company.  It
was intended that the company would spend
\pounds400\,000 on the scheme over 20 years.  The application was
considered by the South Australian parliament during September and
October but was not agreed to.  One stumbling block was the
government's refusal to allow subdivision of the irrigable land into
lots of 1000\,acres or more.\fn{\textsl{Adelaide Observer} 22 Sep.\ 1888
\& 29 Sep.\ 1888; SAPD 1888, p.\,1267, 9 Oct.\ 1888.}

One of the promoters was Jonas Levien \index{Levien, J.} who had
horticultural interests at Geelong and had been Minister for Mines and
Agriculture before opposing the Mildura agreement submitted by Deakin
to the Victorian Parliament in 1886.  Two others, Benjamin Fink
\index{Fink, B.} and
R.\,W.~Best, \index{Best, R.\,W.} were prominent in the Victorian land
boom.  It is surprising that Jonas Levien was so involved with this
scheme yet in August 1888, during the period of negotiations with the
SA government by the Lake Bonney Irrigation Company, he became
chairman of the board of directors of Chaffey Bros Ltd, \index{Chaffey
Bros Ltd} whose interests he had contested
previously.\fn{\citet{cannon1972}.}

Irrigation of at least part of the land near Lake Bonney was
sufficiently practicable and attractive to warrant further interest,
leading to important development in the district thirty years later.

\subsection*{Milang, SA}\index{Milang, SA}

Albert Henry Landseer \index{Landseer, A.\,H.} was engaged in trade
along the Murray River from his headquarters at Milang on Lake
Alexandrina.  \index{lake!Alexandrina, SA} He promoted an irrigation
scheme close to Milang, to be supplied with water pumped from the lake
to an area of more than 15\,000\,acres.  The area was inspected in May
1892 by Professor Lowrie \index{Lowrie, Prof} of Roseworthy College
\index{Roseworthy Agricultural College} 
who regarded the land as third-rate in quality and advised that only a
small scheme should be tried initially.  In July a parliamentary
delegation including two government ministers inspected the land under
consideration and the short-lived Holder government was apparently in
favour of a trial for the scheme.  Nothing came of this scheme but
interest in local irrigation was revived in 1893 when an English
syndicate was reported to be negotiating for irrigation of 2000\,acres of
land close to Milang.  This project also came to
nothing.\fn{\citet{stimson1983};
\textsl{Southern Argus}, 5 May 1892, 23 June 1892, 20 Oct.\ 1892 \&
2~Feb.\ 1893;
\citet{faull1981}.}

\subsection*{Reedy Creek, SA} \index{creek!Reedy, SA}

An irrigation project proposed in 1888 by J.\,D.~Derry, C.\,N.~Long,
and W.\,R. San\-do \index{Derry, J.\,D.} \index{Long, C.\,N.}
\index{Sando, W.\,R.} involved a storage reservoir with a capacity of
17\,000\,acre-feet to impound the waters of Reedy Creek, which
discharges into the Murray \index{river!Murray} near Mannum,
\index{Mannum, SA} and
construction of distribution channels for irrigation of 20\,000\,acres in
the Mannum Irrigation Colony. \index{irrigation colony!Mannum} The
promoters expected to spend
\pounds80\,000 on construction.  Their application was mentioned in
the 1890 report on South Australian public works; some details of the
scheme were provided later by Alfred Deakin.  Two years later two of
the promoters had withdrawn leaving the project entirely to
J.\,D.~Derry, the Victorian engineer involved with irrigation near
Horsham, but nothing eventuated before Derry's return to England in
1894.\fn{SAPP no.\,29 of 1890, no.\,29 of 1892,
\citet[p.\,92]{deakin1892};
\citet{blake1972}.}

\subsection*{Lake Boga Irrigation Company, Vic.} \index{lake!Boga, Vic}

Lake Boga is one of the larger freshwater lakes fed by overflow of the
Murray River along the Little Murray River \index{river!Little Murray}
in northwestern Victoria.  The Lake Boga Irrigation Co.\ \index{Lake
Boga Irrigation Co.} was concerned with development of irrigated
horticulture on land close to Lake Boga and the nearby smaller Long
Lake, \index{lake!Long, Vic} between which the railway had been
extended from Kerang \index{Kerang, Vic} to Swan Hill \index{Swan
Hill, Vic} in 1889.  Water for irrigation was to be pumped from the
lakes.\fn{\citet[p.\,77]{wells1986}.}

The scheme for irrigation was advertised in at least two district
newspapers and a Melbourne weekly in 1890 after formation of the
company.  Its prospectus showed an intention to sell township
allotments (20), horticultural allotments (135) ranging in size from 5
to 12\,acres and totalling 600\,acres, as well as an area of
106\,square miles to the south and west held under mallee pastoral
lease. Six men responsible for the project included three from St
Arnaud, and one each from Ballarat, Melbourne, and Torrumbarry.  Three
were engineers, including George Gordon, formerly of the Victorian
Water Supply Department.  Excursions to the area from Donald were
advertised in November 1890, after earlier efforts to sell allotments
by auction.  One advertisement stressed the accessibility of the
project area, with disparaging reference to the Mildura irrigation
settlement as `remote and outlandish'.\fn{G.~Gordon papers, File
H17329, LaTL; \textsl{Swan Hill Guardian} 19 Nov.\ 1890.}

By 1892 the company had sold 7590\,acres, mainly for dry-farm\-ing and
without much progress in selling the irrigable blocks.  The company
continued in existence for at least 10 more years but it failed to
make any significant development of irrigation.  Its difficulty arose
partly from the refusal of the Water Supply Department to allow
pumping from Lake Boga, a decision apparently prompted by government
support for the Mildura scheme.  This attempt at irrigation in the
vicinity of Lake Boga may have influenced a later effort there by
another company.\fn{G.~Gordon papers, File H17329, LaTL;
\citet[p.\,262]{barr1992}.}

\subsection*{Tresco, Vic.}

Irrigation at Tresco was promoted in 1913 by Australian Farms Ltd, a
company recently set up to establish farms in different parts of
Australia and New Zealand.  Its foundation directors were mainly
Victorian graziers, and it had the services of R.\,V.~Billis, formerly
a public servant dealing with projects for closer settlement and
immigration.

The company first developed irrigation at the margin of the riverine
tract between Kerang and Swan Hill.  A large area of wheat-farming
land adjacent to the Swan Hill railway line was purchased from a local
resident (Mr~Cornish) and a scheme was promoted for irrigation of
fruit trees and vines on about 4000\,acres divided into small allotments,
with water to be pumped from Lake Boga, filled by overflow from the
Murray River.  Sales of land in this area known as Tresco were
enhanced by the scope for speculators to purchase land without being
obliged to reside there.  Encouraged by the response, the company
acquired more land, firstly an estate adjoining the southern part of
Tresco and known as Mystic Park, and in 1919 land to the west of
Tresco, known as Tresco West.  Tresco Irrigation Ltd was the company
which took over the plant belonging to Australian Farms Ltd at Lake
Boga and Tresco; it was the body which had to deal with the problem of
salinity which emerged by 1920.\fn{\citet[p.\,328]{barr1992}.}

The settlement suffered its first setback from salinity about five
years after its establishment in 1913.  Citrus and vines had been
planted over about 2500\,acres, including two ridges with sandy topsoil
and extensive flats with loamy soils.  Salt damage first became
evident on the flats after a water-table developed close to the
surface and salt incrustations were noticed on the surface of the
ground.  Drains were installed but their outfall to Lake Boga probably
added to the salt content of its water, which was apparently excessive
before 1920. The damage extended to all the flat land; satisfactory
citrus orchards remained only on the high sandy
ridges.\fn{F.\,M.~Read, J.\,Agric.\ Vic.\ 1930, \textbf{28}, p.\,65 \&
1931 \textbf{29}, pp.\,551--563.}

The Tresco settlement had attracted investment by several men
pro\-minent in public affairs.  They found results at Tresco so
disappointing that in 1920 they looked for an alternative site for
irrigation development and chose land to the west of Kangaroo Lake,
south of the Mystic Park area.\fn{\citet{trengrove1969}.}

\subsection*{Torrumbarry North, Vic.}

An area of almost 20\,000\,acres near the Murray River between Echuca and
Kerang was acquired by a company in 1888 with the intention of
subdivision into small holdings to be irrigated for
fruit-growing.  Negotiation with the Government led to
the constitution in March 1889 of the Torrumbarry North Irrigation
Trust which was granted by the government a loan of
\pounds12\,300, spent on machinery and an engine house.  The company
first involved with the scheme went into liquidation, apparently after
the collapse of the Victorian land boom, and the trust recorded
irrigation only in 1891, of 17\,acres under fruit trees and vines.  In
1896 the irrigation trust was practically non-existent and the land
was owned by the Murray River Irrigation Estate Co Ltd.\fn{VicPP
nos.\,20 \& 21 of 1896, RC~Water Supply, Rept pp.\,176--180, MoE
p.\,438 \& Appendix~C.}

\section*{Goulburn River Valley}

\subsection*{Ardmona, Vic.}

The Ardmona irrigation settlement was the first established in the
valley of the Goulburn River, the major Victorian tributary of the
Murray.  Its development took place on small holdings formed over a
period of years by successive subdivision of larger holdings.  Ardmona
was the name given to one of these holdings by its owners, G.~and
L.~McDonald, who came to the district along with many other settlers.
Cereal growing was undertaken with reasonable harvests in the 1870s
but droughts in the early 80s caused hardship.  Adequate supplies of
water were needed but settlers had to depend on wells or cartage from
the Goulburn River.  

Legislation in 1881 providing for establishment of waterworks trusts
facilitated a scheme for distribution of river water by open channels
to the many farms in the region lying between the Goulburn, the
Campaspe, and the Murray Rivers.  The principal aim of the scheme was
to provide water for stock and domestic purposes; irrigation was a
minor consideration.  A weir on the Goulburn would be needed to divert
water into the channel system and its construction, under
consideration from 1884, was started in 1887 and did not provide water
for irrigation until the spring of 1891. Meanwhile a pumping plant was
installed and began operation in September 1885 to supply water to the
distribution channels then
completed.\fn{\citet[pp.\,105--106]{bossence1965}; \textsl{Aqua},
Nov.\ 1968, p.\,53.}

A few landholders near Mooroopna on the west bank of the Goulburn were
interested in horticulture and viticulture before water supplies
became available.  They together took steps to create small holdings
intended for fruit and vines.  A portion of W.~Davis's farm was
acquired by Fred J.~Young, a local farmer since 1870, who subdivided
it in 1886 to provide five allotments varying in size from 8 to
30\,acres. One was bought by John West, editor since 1883 of the local
paper, \textsl{Goulburn Valley Yeoman}.  An adjoining farm of
733\,acres, known as Ardmona, was purchased in 1886 by five members of
a syndicate: West, Michael Kavanagh, Fred Young, M.~Cussen, and
A.\,B.~Patterson, with the intention of subdivision.  Then George
Pagan bought an adjoining property of 329\,acres for subdivision and
began selling allotments a few years later.\fn{\citet{steven1990};
\citet[pp.\,57--58]{martin1955}.}

Vines and fruit trees planted earlier by Michael Kavanagh, a Canadian,
on his `Lake Erie' farm near Ardmona, and by Fred Young, produced
grapes for wine or raisins and currants as well as fresh fruit for
sale.  John West began a plant nursery and orchard on his block.  All
three were able to continue production without irrigation but they
were ready to make use of it when water became available in 1886.  In
that year the rainfall in the district was low in autumn but otherwise
normal (18.6\,inches at Shepparton) and in 1887 rains were more than
adequate (26.3\,inches).  The next year brought a drought and there
was insufficient water to allow irrigation.  Despite that handicap,
John West gained first prize in the government competition for the
best variety of irrigated crops,. His farm, named Milvina, was
reported as being a portion of `Fresno'
colony.\fn{\textsl{Australasian}, 18 Feb.\ 1888, p.\,354.}

In 1890 the Minister for Water Supply, Alfred Deakin announced the
Cabinet decision to appoint John West as an expert with practical
knowledge of irrigation to teach farmers; he was sent to California
and Colorado for several months to study irrigation methods. On return
to Victoria he gave a number of lectures on irrigation at towns in the
Goulburn Valley.\fn{\textsl{Goulburn Valley Yeoman}, 2 May 1890.}

The extent of irrigation at Ardmona was reported as about 300\,acres in
1892; it increased gradually later and involved 4000\,acres by
1916.\fn{\citet{murray1892}; Dept.\ Agric.\ Vic.\ Tech.\ Bull.\
no.\,3, 1944.}

\subsection*{Eshcol, Vic.}

The Eshcol irrigation settlement was promoted by J.\,B.~Campbell who
in 1894 issued a colourful prospectus indicating that the village of
Esh\-col would include a cannery, creamery, and drying houses, as well
as a library, school, and post office.  The proposed location was to
be west of Ardmona and three miles north of Tatura but there is no
evidence that such an irrigation settlement was ever
achieved.\fn{\citet{bossence1969}.}

\subsection*{Mount Scobie, Vic.}

Mount Scobie, a solitary hill south of Kyabram in the Goulburn Valley
was the name given to an estate of 9000\,acres bought for \pounds90\,000
by James Mirams, former politician turned land agent.  Mirams
advertised allotments for sale in 1890 as the Mt Scobie irrigation
colony, with minimum holdings of 10\,acres.  Later in the year, Mirams, a
prominent figure in the Victorian land boom, was convicted of issuing
a false balance sheet for a business and was sentenced to a year's
imprisonment.  Though some blocks of land were apparently sold at
Mount Scobie, there is no evidence that an irrigation colony was
realised there.\fn{\citet[p.\,150]{cannon1972}; \textsl{Australasian},
1 Mar.\ 1890; \citet{ingham1974b}.}

\subsection*{Mooroopna Irrigation Company, Vic.}

An article published in 1890 referred to the Mooroopna irrigation
colony three miles north of Ardmona and mentioned the involvement of
John West, Mich\-ael Kavanagh, M.~Cussen, W.Burnett, and Strachan~\&
Co.  This followed an advertisement in the same paper for the
Mooroopna Irrigable Lands Investment Co Ltd. The outcome of this
venture is not clear; it may have become part of the Ardmona
settlement.\fn{\textsl{Australasian}, 25 Oct.\ 1890 \& 8 Nov.\ 1890,
p.\,878.}

\section*{Campaspe River Valley}

\subsection*{Echuca Irrigation and Freehold Land Company, Vic.}

Efforts to develop irrigation near the downstream course of the
Campaspe River, west of Echuca, began in 1888--89 when a group of
investors arranged to buy three pastoral properties\,---\,Restdown,
Mara\-thon, and Wharparilla\,---\,with a combined area of almost
40\,000\,acres.\fn{VicPP no.\,21 of 1896, RC~Water Supply, MoE,
p.\,436.}

James Shackell, prominent in Echuca as an auctioneer and politician,
took a leading part in the scheme to irrigate that land.  He had been
a member of the 1884 royal commission on water supply and knew of the
possibility of bringing irrigation water to the area as indicated by
Gordon and Black in their second report on irrigation.  The Millewa
Irrigation Trust was formed in 1890 to undertake irrigation on
approximately 31\,000\,acres of the land purchased by the investment
company.  When the affairs of the trust were examined by the 1896
royal commission on water supply, Shackell stated that the scheme,
involving the formation of an irrigation colony, relied on the
expectation of getting gravity water `from the Goulburn channel that
was to be constructed at the expense of the government and deemed to
be national works'.\fn{G.~Gordon \& A.~Black, Second Report on
Irrigation, 1884; VicPP no.\,21 RC~Water Supply, MoE p.\,436.}

The scheme was in jeopardy, however, when the purchase of land by the
investment company could not be completed and Shackell went to London
to arrange alternative finance.  He then promoted the Echuca
Irrigation and Freehold Land Company Ltd to complete purchase of the
land for irrigation with subdivision of the three properties into
blocks varying in size from 5 to 25\,acres.  Shackell's rescue operation
failed and in 1892 the land reverted to the previous owners and the
Millewa trust ceased operations without ever having undertaken
irrigation.\fn{\citet[pp.\,209--210]{coulson1995}.}

The Millewa trust was one of a few Victorian irrigation trusts formed
not by a group of landholders but with only one owner\,---\,a syndicate
or private company.  Those trusts were known as private irrigation
trusts and were later considered to be promoted for speculative
purposes.

\section*{Wimmera River Valley}

\subsection*{Young Brothers Irrigation Colony, Vic.}

The first irrigation colony in the Wimmera district was at Horsham,
where the mean annual rainfall is 17.5\,inches and summers are dry and
hot.  Irrigation was promoted mainly by Thomas and George Young, stock
and station agents, who were involved with irrigation at two places
near Horsham about 1890, firstly at Dooen, five miles NE of Horsham,
and later within two miles of Horsham.  Their enterprise at Dooen has
been referred to as an irrigation colony although in 1892 Stuart
Murray reported only two holdings there\,---\,one for Young Bros and
another for Mr~Stephens.  The Young brothers were irrigating 10\,acres
at Dooen in 1885 with water drawn from a supply reservoir fed from the
Wimmera River.  The Young Brothers Irrigation Colony Ltd was in
existence on the north side of the Wimmera River in 1891 and in 1892
the colony involved 260\,acres all with irrigation channels.  At that
time 160\,acres had been sold and all but 60\,acres of the area had
been planted with fruit trees and vines.\fn{\citet{murray1892}; VicPP
no.\,53 of 1885, RC~Water Supply, Further Progress Rept MoE, p.\,124.}

\subsection*{Horsham Irrigation Colonising Company, Burnlea, Vic.}

The Horsham Irrigation Colonising Company began the Burnlea colony in
1890 on the south side of the Wimmera River at about one mile from
Horsham.  The company was formed by Thomas Young, James Brake,
Frederick Hagelthorn, and two engineers John D.~Derry and Percy
Learmonth.  By 1892 its extent was 550\,acres, all with access to
irrigation channels, and about 100\,acres had been sold with 50\,acres
planted to fruit trees.\fn{\citet[p.\,35]{blake1973};
\citet{murray1892}.}

\subsection*{American Colonisation Company, Riverside, Vic.}

In 1892 the American Colonisation Company promoted Riverside, another
colony on the south side of the river, about two miles from Horsham.
Its 430\,acres were provided with irrigation channels, all allotments
were sold, and 150\,acres had been planted with vines and fruit trees
before the end of 1892.\fn{\citet{murray1892}.}

\subsection*{Quantong Irrigation Colony, Vic.}

An irrigation colony west of Horsham near the Wimmera River was first
proposed in 1890 by Brake and Derry when they bought 2253\,acres.  In
1891 the colony was being promoted by the Cooperative Irrigation and
Mercantile Society of Australia Ltd which advertised more than 80
horticultural allotments varying in size from 5 to 20\,acres and a
large number of township allotments.  The promoting company had
Alexander Baird as its secretary in Melbourne and many Quantong
settlers had dealings with this successful business man whose
interests later included mining, brewing, and cement.  By about 1895
there were about 60 settlers in the area.  Although the scheme was
promoted as an irrigation colony, there was no supply of water until
1905.  The earliest products of the colony were dried fruit, including
raisins, currants, and sultanas; fruit trees came into bearing before
1920.\fn{\citet[pp.\,35, 98]{blake1973}; \citet{horsham1991};
\citet[p.\,28]{gibbney1987}.}

\subsection*{Mount Arapiles, Vic.}

Near Mt Arapiles another colony was promoted in the early 90s; it was
reported in 1892 as involving 640\,acres, with 200\,acres sold in blocks of
10\,acres or more.  It was expected to be irrigated by water from the
Grampians supplied by the channel system.\fn{\citet{murray1892}.}

\subsection*{Haven, Vic.}

Haven was the name given to the last colony established near the
Wimmera River.  It was not begun until 1894 apparently as an outcome
of the village settlement south of Horsham associated with Rev.~Horace
Finn Tucker.

Some of these colonies in the Wimmera district were apparently
receiving irrigation in 1892.  By 1893 a few hundred acres were
recorded as receiving irrigation in the Western Wimmera Irrigation
Trust area. By 1907 the irrigated area in the district did not exceed
a thousand acres and mainly concerned vineyards and orchards. Quantong
became the largest irrigation enterprise in the district.  Irrigation
of these colonies generally depended on supplies from storages
dependent on runoff from the nearby Grampians mountains, with gravity
supply of water by streams and artificial channels developed by the
local trust responsible for water supply. John Derry, an engineer with
experience of irrigation in India, came to the Wimmera in the early
80s, visited USA with Alfred Deakin for the Royal Commission on Water
Supply, and was closely involved with the Burnlea and Quantong
colonies.\fn{\citet{murray1892}; VicPP no.\,20 of 1896, RC~Water
Supply, Appendix~C; VicPP no.\,38 of 1907, Annual Rept SRWSC, p.\,8.}

\section*{Werribee River Valley}

\subsection*{Werribee Irrigation Colony, Vic.}

The plain traversed by the Werribee River on its way to Port Phillip
Bay remained for many years under the control of pastoralists,
especially the Chirnside family which in 1875 held 85\,000\,acres
freehold and grazed 80\,000 sheep.  Agriculture involved less than 5
per cent of the area although rainfall, on the average 20\,inches in
the year, was sufficient to produce good crops of cereal hay and
grass.  In 1880 less than 100\,acres were devoted to vegetables,
orchards, and gardens.\fn{\citet[pp.\,162--63]{peel1974}.}

The Werribee River, a relatively small stream by Victorian standards,
occupies a deep valley and changed seasonally from a torrent to a
series of pools until dams were constructed in its upper reaches.  Its
tidal reach was used for some years by paddle-steamers travelling five
miles from the Bay.

When public interest in irrigation was stimulated by Deakin's visit to
America, enactment of his irrigation legislation, and the arrival of
the Chaffey brothers to promote irrigation, the Werribee people were
not backward in responding: they held a public meeting in May 1886 to
register interest in local irrigation.  The provisions of legislation
that year for the development of irrigation by irrigation trusts
supported by loans from the government led to the formation of the
Werribee Irrigation Trust in November 1888, a time of low rainfall at
Werribee where only 14\,inches were recorded that year.  The trust
concerned an area of about 1450\,acres acquired from a local resident
by the Werribee Irrigation and Investment Co which initiated formation
of the trust.  This company was launched in October 1888 and a few
months later issued 100\,000 shares of \pounds1 each.  George Chaffey
became its chairman of directors and E.\,C.~De Garis was the managing
director.  These men were also respectively the chairman and secretary
of the irrigation trust.\fn{\textsl{Leader}, 29 May 1886; VicPP
no.\,20 of 1896, RC~Water Supply, Rept pp.\,179--180, Evidence
pp.\,425--433; \citet{ronald1978}; \citet[p.\,192]{alexander1928}.}

The company intended that the property would be subdivided into small
holdings provided with water stored on the river by a weir to be
erected a few miles upstream from the irrigation colony.  A temporary
pumping plant was being installed on the river to allow a considerable
portion of the land to be irrigated before completion of the weir.
The site involved was on the east side of the river, within a few
miles of the Werribee township.\fn{\textsl{Mildura Cultivator}, 14
Feb.\ 1889.}

Subdivision of the area went ahead, the pumping plant was installed on
the river, and an irrigation channel was made.  Boundaries of
allotments were planted with trees and advertisements enticing
settlers with the slogan `Fortunes in Vegetables' were made late in
1890 with advice that payment for allotments could be spread over 15
years.\fn{\textsl{Australasian}, 25 Oct.\ 1890, p.\,804.}

The irrigation company wished to install a pipeline in order to
irrigate more land; this depended on agreement with Percy Chirnside,
the owner of land to be crossed by the pipeline.  In order to
propitiate him, the Company proposed calling their irrigation colony
`Chirnside', but the stumbling block of compensation persisted and
delayed further action.  Whether this pipeline was to bring water from
the proposed weir or for delivery of supplies from the temporary
pumping plant is not clear, but there is no record of construction of
the weir or installation of the pipeline.\fn{\citet{ronald1978}.}

Within two years of formation, some progress had been made with the
irrigation colony.  The Melbourne Nursery and Poultry Farm Company had
purchased 96\,acres of land for \pounds7500, the company balance sheet
showed a profit, and buildings were going up on the land.  One was the
expensive residence for George Chaffey, reputed to have cost
\pounds3000 and known as Manchester Park.  It was later let to 
Percy Chirnside while his mansion, Manor Park, was being built just
across the river.  Manchester Park burnt to the ground in
1924.\fn{\citet[p.\,172]{ronald1978}.}

In 1891 the Trust was responsible for irrigating no more than 200\,acres
and when the company went into liquidation that year, its
responsibility was assumed by George Chaffey who also failed to carry
on the undertakings and the land reverted to its previous owner,
Thomas Agar, who was elected by the remaining settlers as chairman of
the trust.  De Garis, who had lived in the district for a few years
while active in the irrigation development, moved to Mildura late in
1891, while George Chaffey and his wife made the same change in 1892.
A few settlers continued with irrigation of vegetables and fruit
trees.\fn{VicPP no.\,20 of 1896, RC~Water Supply, Rept pp.\,179--80,
Evidence pp.\,425--433.}

The Werribee trust was one of three private irrigation trusts which
had been assisted by the government with loans not repaid by 1896 when
the report by the royal commission found that:
\begin{quote}
	the promoters of these trusts had in each case conditionally
	purchased an area of lands which they desired to form into an
	irrigation trust, not for the purpose of working the land
	themselves, but with the object of cutting it up and selling
	it to others.\fn{VicPP no.\,20 of 1896, RC~Water Supply,
	Rept pp.\,176--80.}
\end{quote}
After this finding, the commissioners expressed the opinion that `the
State money had been advanced to irrigation trusts in a most lavish
and prodigal manner, and without careful investigation'. The
commission also found that in December 1889 George Chaffey had
obtained an advance of \pounds3000 from the government without
security and that the Minister, Alfred Deakin, was at fault in making
the loan.\fn{VicPP no.\,20 of 1896, Rept pp.\,185, 190.}

Deakin's speech in 1896 in response to the report of the Commission,
which touched on so much of his responsibility for water supply in the
period 1886--90, was notable for its length and the fact that he was
obliged to acknowledge his personal responsibility for the loan to
George Chaffey.\fn{VicPD, 18 Dec.\ 1896 vol.\,84, p.\,4738.}

\section*{Nepean River Valley}

\subsection*{Mulgoa Irrigation Company, NSW}

Irrigation of fruit trees and vines on an area of more than
10\,000\,acres lying 30 miles west of Sydney, extending southwest from
St Marys to the Nepean River was promoted by George Chaffey and Henry
Gorman in the late 1880s.  At that time George Chaffey had experience
of developing irrigation settlements in areas with very low rainfall
at Mildura and Renmark, and at Werribee with moderate rainfall.  The
Mulgoa scheme was unusual because its area had an average rainfall of
27\,inches.

The scheme was probably initiated in 1888, judging by a published
reference then to proposed irrigation near Penrith involving the
Chaffeys.  Promotion in 1891 apparently involved several men including
E.\,C.~De Garis in addition to the Chaffey brothers.  The Mulgoa
Irrigation Bill came before the New South Wales parliament in 1890 and
was considered by a select committee.  Several witnesses appeared
before that committee including the principal promoters, G.~Chaffey
and Henry Gorman, and H.\,G.~McKinney, chief engineer for water
conservation.  The Bill was passed late in 1890.\fn{\textsl{Mildura
Cultivator}, 20 Dec.\ 1888; \citet{alexander1928}; NSW LC Journal
vol.\,47, pt.\,2 1890, Rept Select Comm.\ on the Mulgoa Irrigation
Bill.}

The Mulgoa scheme involved (i) purchase of 16\,000\,acres from several
more or less contiguous properties in the Penrith district and
division into allotments of three types: town, suburban villas
(2--5\,acres), and horticultural (10--20\,acres); (ii) supply of water
for irrigation by lifting water 130\,feet by pumping from the Nepean
River to a reservoir, reticulation by gravity via earthen channels,
and a subsidiary lift of a further 50\,feet to water the higher parts
of the area; and (iii) provision of weirs on the Nepean and/or its
tributary the Warragamba, to ensure adequate supplies.

Two men responsible for the engineering design were George Chaffey and
his associate J.\,G.~Starr.  The scheme was publicised early in 1891 by
articles in several Sydney newspapers and the Mulgoa Irrigation Co Ltd
was subsequently formed, with A.\,W.~Stephen as managing director.
The company then invited applications for purchases of the different
types of holdings and stated the arrangements for water supply, giving
each settler `a perpetual claim upon the company for water on the
specified terms' without the establishment of a water
company.\fn{\textsl{Sydney Morning Herald}, 3 Jan.\ 1891;
\textsl{Mulgoa Irrigation Settlement}, brochure 1892.}

The company apparently made little progress with the irrigation
sch\-eme, and the government was invited to take over the scheme in
1896 when legislation on water rights gave the opportunity for greater
state involvement in irrigation.  This proposition was considered by
Colonel Home, the irrigation engineer from India invited to report to
the government on irrigation prospects in the colony.  His report
showed that the Mulgoa scheme would be `far from remunerative'.  There
was no subsequent development of the Mulgoa irrigation scheme.\fn{NSW
V\&P 1897, vol.\,5, Rept on Prospects of Irrigation and Water
Conservation in New South Wales, by Colonel F.\,J.~Home.}

\section*{Hunter River Valley}

\subsection*{Segenhoe Estate, NSW}

Segenhoe was the name of an agricultural settlement which flourished
in the Hunter valley before 1830.  Prospective irrigation of the
Segenhoe estate, which involved 25\,000\,acres, was reported on by a
select committee of the New South Wales Legislative Assembly in 1891
when it considered the Segenhoe Estate Irrigation Bill then before
parliament.  William Harris told the committee that he was managing
director of the Land Company of Australasia Ltd, which owned the
property and proposed raising funds from abroad to impound at least
800 million gallons of flood waters of the Hunter River system
upstream from Aberdeen and Muswellbrook.  F.\.B.~Gipps was the civil
engineer responsible for the scheme which involved gravitation of
supplies to pastoralists acquiring portions of the property.  Other
witnesses included local pastoralists and townspeople from
Muswellbrook which might incur some disadvantage to water supply.

After the select committee reported to the legislative assembly the
proposed legislation apparently lapsed.\fn{NSW V\&P LA 1891--92
vol.\,7, pp.\,713--742, Rept Select Comm.\ on the Segenhoe Estate
Irrigation Bill.}

\section*{Darling River}

\subsection*{Menindie Irrigation Scheme, NSW}

An area of 25\,000\,acres extending northwards from Lake Menindie on
the Darling River was proposed for irrigation in 1892 by the Menindie
Irrigation Settlement Co Ltd with capital of \pounds50\,000.  The
scheme was considered by a select committee of New South Walews
parliament in 1893 when it was called on to report on the Menindie
Irrigation Bill.  Witnesses who gave evidence to the committee
included H.\,G.~McKinney, chief engineer for water conservation,
C.\,W.~Darley, engineer in chief for harbours and rivers, C.\,E.~Hogg,
engineer to the company involved, and J.\,M.~Purves, a land agent from
Sydney who represented the company.  This scheme was associated with
the projected Menindie and Broken Hill tramway, also the subject of a
report by a parliamentary select committee.  Neither of these projects
was realised.\fn{NSW V\&P LA 1892--93, vol.\,8, Rept Select Comm.\ on
the Menindie Irrigation Bill.}

\section*{Gawler River}

\subsection*{Gawler Plains Irrigation Colony, SA}

A scheme to provide irrigation on the Gawler plains north of Adelaide
was considered in 1889 by the Barossa Water Commission as one option
associated with construction of a reservoir on a tributary of the
Gawler River intended to augment supplies to the Adelaide metropolis.
Apparently inspired by the development of the Chaffey irrigation
colonies on the Murray River, the establishment of an irrigation
colony involving 12\,000\,acres near Smithfield was supported by several
prominent agriculturists but was viewed unfavourably by W.~Culcheth, a
Victorian irrigation engineer, and by the Commission.\fn{SAPP no.\,25
of 1889, Rept of the Barossa Water Commission.}

The scheme was revived later by two men: Packard, a surveyor, and
Lutz, an engineer.  In 1894 they were authorised by the South
Australian government to proceed with their plan for irrigation on the
Gawler plains using water from a reservoir to be constructed on the
South Para River.  Their project, referred to in an official report as
the Gawler Plains Irrigation Colony was expected to involve
subdivision of 10\,000\,acres between Smithfield and Salisbury into
blocks of 5 to 10\,acres for `intense irrigated culture of vineyards,
orchards, orangeries, etc', and into larger blocks for growth of
fodder for dairying and grazing purposes.  Water was to be provided
from a storage reservoir on the South Para River, with an embankment
near Williamstown.\fn{SAPP no.\,111 of 1896, The Proposed Gawler
Plains Irrigation Scheme.}

As with other promoters of irrigation colonies in Australia during the
1890s, Packard and Lutz failed to secure adequate financial backing
from their associates in England.  However, they prevailed on the
South Australian government to give limited financial help, by
guaranteeing a small amount of money, so that the work could proceed.
This help was authorized by the Gawler Plains Irrigation Act of 1896
but it failed to bring the project any nearer to realisation.
Eventually the Barossa Reservoir was started in 1898 by the government
to provide water supply to Gawler and settlements to the north.

\closure
Many of the schemes were never tested in practice after failing to
attract sufficient support at the outset. This lack of support
indicates their difficulty in competing with many other speculative
ventures proposed during the land boom which affected several
Australian colonies. Except for the Tresco settlement which started
much later, the irrigation colonies which came into existence were
relatively small ventures in the Wimmera and Goulburn Valley of
Victoria.

%\section*{References}
%1. F.Bevilaqua, General Remarks And Detail Accounts Of The South Australian
%    Irrigation Company, 1884.
%2. SAPP No.158 of 1884 \& SAPP No.73 of 1885.
%3. Adelaide Observer 22/9/1888 \& 29/9/1888.
%4. SAPD 1888, P.1267, 9/10/1888.
%5. M.Cannon, The Land Boomers, 1986.
%6. A.J.Stimson, ADB Vol.9, p.655.
%7. Southern Argus, 5/5/1892.
%8. Southern Argus, 23/6/1892.
%9. Southern Argus, 20/10/1892.
%10. Southern Argus, 2/2/1893, \& J.Faull, Alexandrina's Shore, A
%      History Of The Milang district, 1981.
%11. SAPP No.29 of 1890, A.Deakin, Irrigation In Australia, 1892, p.92.
%12. SAPP No.29 of 1892, L.J.Blake, ADB vol.4,pp.58-59.
%13. S.Wells, Paddle Steamers to Cornucopia, 1986, p.77.
%14. G.Gordon papers, File H17329, La TL.
%15. Swan Hill Guardian 19/11/1890.
%16. G.Gordon papers, File H17329, La TL.
%17. N.Barr \& J.Cary, Greening A Brown Land, 1992, p.262.
%18. N.Barr \& J.Cary, 1992. p.328.
%19. F.M.Read, J.Agric.Vict. 1931 vol.29, p.551.
%20. F.M.Read, J.Agric.Vict. 1930, vol.28,p.65.
%21. F.M.Read, 1931, p.563.
%22. A.Trengrove, John Grey Gorton, 1969.
%23. VicPP No.20 \& 21 of 1896, R.C.Water Supply, Rept.  pp.176-180 \& 
%      MoE p.438.
%24. VicPP No 20 of 1896, Rept.  pp. 176-180 \& Appendix C.
%25. VicPP No.21 of 1896, MoE p.438
%26. W.H.Bossence, Murchison, 1965, pp.105-106.
%27. Aqua, Nov. 1968, p.53.
%28. Margaret Steven, ADB vol.12, 1990, p.445.
%29. C.S.Martin, Irrigation And Closer Settlement In The Shepparton
%    District 1836--1905, 1955, pp. 57-58.
%30. Australasian, 18/2/1888, p.354.
%31. Goulburn Valley Yeoman, 2/5/1890.
%32. S.Murray, Irrigation In Victoria, 1892, \&
%    Dept.Agric.Vict. Tech. Bull. No.3,1944.
%33. W.H.Bossence, Tatura And The Shire Of Rodney, 1969.
%34. M.Cannon, The Land Boomers, 1986, p.150.
%35. Australasian, 1/3/1890.
%36. S.M.Ingham, ADB, vol.5, 1974, p.258.
%37. Australasian, 8/11/1890, p.878.
%38. Australasian, 25/10/1890.
%39. VicPP No.21 of 1896, R.C.Water Supply, MoE, p.436.
%40. G.Gordon \& A.Black, Second report on irrigation, 1884.
%41. VicPP No.21 R.C.Water Supply, MoE p.436.
%42. Helen Coulson, Echuca -Moama On The Murray, 1995, pp.209-210.
%43. S.Murray, Irrigation In Victoria, 1892.
%44. VicPP No.53 of 1885, R.C.Water Supply, Further Progress Rept.
%      MoE, p.124.
%45. S.Murray, Irrigation In Victoria, 1892.
%46. L.J.Blake, Wimmera, 1973, p.35.
%47. S.Murray, Irrigation In Victoria, 1892.
%48. S.Murray, Irrigation In Victoria, 1892.
%49. L.J.Blake, Wimmera, 1973, p.35.
%50. Horsham \& District Historical Soc., Memories of Quantong, 1991.
%51. H.J.Gibbney \& Ann G.Smith, 1987, A Biographical Register
%      1788-1939, vol. 1 p.28.
%52. L.J.Blake \& K.H.Lovett, Wimmera Shire Centenary, 1962, p.98.
%53. Horsham \& District Historical Soc., Memories of Quantong 1991.
%54. S.Murray, Irrigation In Victoria, 1892.
%55. S.Murray, 1892.
%56. VicPP No.20 of 1896, R.C. Water Supply,Appendix C.
%57. VicPP No 38 of 1907, Annual Rept SRWSC,p.8.
%58. Lynette J.Peel, Rural Industry In The Port Phillip Region
%      1835-1880, 1974, pp. 162-163.
%59. Leader, 29/5/1886.
%60. VicPP No.20 of 1896, R.C.Water Supply, Rept.pp.179-180.
%61. Heather B.Ronald, Wool Past The Winning Post, 1978.
%62. J.A.Alexander, The Life Of George Chaffey, 1928, p.192.
%63. VicPP No.20 of 1896, R.C.Water Supply, Evidence pp. 425-433.
%64. Mildura Cultivator, 14/2/1889.
%65. Australasian, 25/10/1890, p.804.
%66. Heather B. Ronald, 1978.
%67. Heather B.Ronald, 1978, p.172.
%68. VicPP No.20 of 1896, R.C.Water Supply, Rept., pp.179-80, \& 
%      Evidence pp. 425-433.
%69. VicPP No.20 of 1896, R.C.Water Supply, Rept. pp.176-80.
%70. VicPP No. 20 of 1896, Rept. p.185.
%71. VicPP No 20 of 1896, Rept. p.190.
%72. VicPD, 18/12/1896 vol.84, p.4738.
%73. Mildura Cultivator, 20/12/1888.
%74. J.A.Alexander, The Life Of George Chaffey, 1928.
%75. NSW LC Journal Vol.47, Pt.2 1890,Rept Select Comm.on the Mulgoa
%      Irrigation Bill.
%76. Sydney Morning Herald, 3/1/1891.
%77. Mulgoa Irrigation Settlement, brochure 1892.
%78. NSW V\&P 1897, vol.5, Rept on Prospects of Irrigation and Water
%      Conservation in New South Wales, by Colonel F.J.Home.
%79. NSW V\&P LA 1891-92 vol. 7,pp.713-742, Rept. Select Comm.on the
%      Segenhoe Estate Irrigation Bill.
%80. NSW V\&P LA 1892-93, vol.8, Rept Select Comm. on the Menindie
%      Irrigation Bill.
%81. SAPP No.25 of 1889, Rept of the Barossa Water Commission.
%82. SAPP No.111 of 1896, The Proposed Gawler Plains Irrigation Scheme.

