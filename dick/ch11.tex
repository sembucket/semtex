% $Id$
% CHAPTER ELEVEN
% 2495 words at 22/4/99

\chapter{Irrigation of Group Settlements from 1891}

\index{cooperative settlements|(}
\label{ch:groups}\addtoendnotes{\protect\subsection*{Chapter \thechapter}}
\fancyhead[RE]{\sffamily \small Chapter \thechapter.\ %
               Group Settlements}

\setcounter{endnote}{0}

Village settlements were the common form of group settlements
established in all colonies but Western Australia during the economic
depression of the 1890s.  Irrigation was undertaken at most of these
settlements in South Australia, at several in Victoria, and at one in
Queensland.

The land settlement schemes sought in Australia from the late 1880s
included proposals for group settlements to provide either communal
arrangements for land use or allotments to individuals.  The communal
types of settlements were favoured in Queensland by the utopian
William Lane \index{Lane, W.} and his associates and in South
Australia by the single tax leagues following the ideas of Henry
George, \index{George, H.} the \index{American}American land reformer.
William Lane's decision in 1891 to promote the New Australia communal
colony in South \index{America!South}America gained support in some
Australian colonies and was countered by government schemes for
communal land settlement in Queensland and South Australia.  The
legislation achieved in those two colonies in 1893 was intended to
cater for prospective settlers with limited capital, who might
otherwise be tempted by the New Australia scheme.  In other colonies
the government schemes in 1893 for group settlement were designed to
cater particularly for unemployed workers, who thereby generally
obtained individual allotments of land.

\section*{Queensland}
\index{Queensland}

Apparently the only group settlement in Queensland to make use of
irrigation was the Alice River Cooperative Settlement
\index{river!Alice} formed by more than 70 men who had been on strike
during the important pastoral dispute of 1891.  Their enterprise was
four miles from Barcaldine, \index{Barcaldine, Qld} then supplied with
water from bores in the Great Artesian Basin.  Although legislation
for village settlements had existed in the colony for a few years, the
shearers occupied 3500\,acres provided by the government under an
occupation licence.  A bore to a depth of 1333\,feet was
\index{finance}financed by the settlers to provide artesian water to
raise vegetables and maintain an orchard.  Support for the Alice River
settlement came from many parts of the region as gifts of livestock,
equipment and farm implements. Some settlers left to join William
Lane's colony in South \index{America!South}America and within a few
years the Alice River settlement had few residents. In 1903 there were
only eight single men in occupation, with a vegetable farm of five
acres and an orchard of five acres.\fn{\textit{Qld Agr.\ J}.,
vol.\,12, (1903), p.\,167; \cite[p.\,20]{metcalf1995}; G.~Lewis,
\textit{Aust.\ J.\,Pol.\ Hist.}, vol.\,19, (1973), p.\,352;
\cite{svensen1989}; A.\,H.~Boyd, \textit{Qld Agr.\,J.}, vol.\,12,
(1903), p.\,167; Qld PD 1893, vol.\,lxx, 10Aug.\ 1893, pp.\,402--405,
416.}

\section*{South Australia}
\index{South Australia}

The Kingston government introduced legislation for communal land
settlements, known as village settlements, in October 1893, with the
aim of providing for people of limited means.  Lands could be made
available not to individuals but to associations, which could lease an
area representing no more than 160\,acres per member, with need for
payment of rent after the first year and for improvements to the land.
Each association would make its own rules for management of its
settlement, with provision for collective working of the land or for
individual allotments.  Loans to associations of not more than
\pounds50 per member could be made by the government in respect to
half the value of improvements already undertaken.  Lands available
for such settlements were reported to the government by G.\,W.~Goyder,
\index{Goyder, G.\,W.} 
the surveyor-general, who included in his list approximately
56\,000\,acres near the River Murray.\fn{SA PP no.\,154 of 1893.}

Thirteen village settlements were established in 1894 after passage of
the enabling legislation in the Crown Lands Amendment Act of December
1893. \index{legislation!SA!Crown Lands Amend.\ Act 1893} Two
settlements were formed by associations with some
\index{finance}financial resources while others involved only
unemployed workers who were then numerous in Adelaide and able to
persuade the government to make special financial assistance for their
participation in the scheme.  Eleven settlements located on the banks
of the River Murray \index{river!Murray} upstream from Morgan
\index{Morgan, SA} had a population of more than 1600 in 1895 and were
the only ones attempting irrigation.

The choice of the river as a location for settlements and interest in
their irrigation indicates the Renmark \index{Renmark, SA} irrigation
settlement with its \index{fruit}fruit trees and
\index{vineyards}grape vines as the model for these new ventures.  The
Lyrup \index{Lyrup, SA} settlement was the first to be established on
the river in February 1894.  Its site was selected after advice from
people at the nearby Renmark irrigation settlement following rejection
of land near Moorlands found unsuitable for irrigation.  Lyrup and
Renmark both utilise terraces that occur in the local river valley and
provide significant areas of land suitable for irrigation without
high-lift pumping.

The second settlement on the river at Waikerie \index{Waikerie, SA}
started several days later and at a considerable distance downstream
from Lyrup, in that distinctive part of the river valley which
provides much less terrace land above \index{flood}flood level.  Later
settlements also failed to secure land with the advantages at Lyrup
and had to utilise ground relatively high above the river, covered
with trees or scrub, and providing sandy soils.  Probably the worst
site was one with a vertical cliff reaching more than 100\,feet above
the river, thus forcing the settlement to develop on the undulating
mallee country fringing the river valley.

The village settlers had to cope with their general inexperience of
farming, the disadvantages of communal organisation, their own
pov\-er\-ty, and the limited advice provided by the government.  Their
preparations for irrigation at the settlements included provision of
pumps and steam engines, the layout of an irrigation system, and
establishment of \index{fruit}fruit trees and \index{vineyards}vines.
They also grew \index{wheat}wheat and \index{fodder}hay, erected
buildings, and put up fences to exclude rabbits.  These tasks were too
much for some members of all settlements, leading to resignations and
the collapse of three settlements within three years.  The Lyrup
settlement was favoured in several ways by its proximity to Renmark
where pruning by Lyrup men was exchanged for trees and vines provided
by \index{Chaffey Bros} Chaffey
Brothers.\fn{\cite[p.\,93]{jones1994}.}

In the first year the settlements on the river had about 500 members
of associations and a total population of 1600.  By 1900 some
settlements had been abandoned and only 95 members and a total
population of 513 lived on the seven remaining ones, only four of
these lasted for ten years.  Their relative failure has been
attributed by participants in the scheme and by others to the
communistic or communal organisation.  The rules governing the
settlements were amended by legislation following official inquiries
in 1895 and in 1899--1900 , and the maximum credit per settler was
increased in 1895 from \pounds50 to \pounds100.  Apparently the
official inquires concentrated on the relations between settlers and
their occupation of land, with little attention to the problems of
irrigation.  Much detailed information on this matter has been
collated, showing that progress with irrigation was generally hampered
by the initial provision of \index{technology!pump}pumping
equipment inadequate for the extension of irrigation needed to make
the settlements \index{finance}financially successful.  The
conspicuous exception was at Lyrup where most progress was made with
irrigation.\fn{SA PP no.\,113 of 1895 Parliamentary Select Committee
Rept on Village Settlements 1895; SA Select Committee Rept 1895; SA PP
no.\,37 of 1900, RC~on Renmark and Murray River Settlements, Final
Rept; \cite{mack1994}.}

Most settlements initially obtained transportable centrifugal pumps
\index{technology!pump!centrifugal} and steam engines,
apparently with help from two government officials detailed to give
advice on engineering and agriculture.  Under the best conditions they
could lift water no more than 30\,feet and were therefore suitable for
the extensive low-lying land of the Lyrup\index{Lyrup, SA} settlement,
but only for the more restricted riverine fringe of other settlements
with land rising 100\,feet or more above the river.  The Waikerie
\index{Waikerie, SA} settlement was the only one which from the outset
used a high-lift pump. \index{technology!pump!high-lift}
As long as the other settlements were dependent solely on the output
of the centrifugal pumps then available they could irrigate only their
limited areas of lower land.

There was gradual change to high-lift pumps, except at Lyrup, from
late 1895.  These allowed settlers to water \index{fruit}fruit trees
and \index{vineyards}vines on the higher land but at the best the more
suitable pumps lacked the capacity to supply a sufficiently large area
under irrigation to make the settlements successful.  Lyrup coped to
some extent with this problem by acquiring a second centrifugal pump
\index{technology!pump!centrifugal} and eventually had more
than 200\,acres of fruit trees, vines, \index{vegetables}vegetables
and \index{fodder}lucerne under irrigation.  Even so, it was claimed
in 1895 that Lyrup would have needed at least 700\,acres under
irrigation to support its population.\fn{\cite{jones1994};
\cite[p.\,39, from Select Comm.\ Rept 1895 p.\,37.]{mack1994}}

Another complication concerning irrigation in the 1890s from the
Murray River \index{river!Murray} was that its level could vary by as
much as 20\,feet within 12 months.  If a permanent site for a pump was
carefully placed high enough to avoid damage from \index{flood}floods,
the seasonal fall of the river might render pumping ineffective
because of too great a suction lift for the pump or from the intake
pipe being no longer under water.  Unfortunately the settlers were
initially unaware of the natural variation in river flow.  Their first
year was far from characteristic: rainfall was exceptionally high and
so was the river level until the autumn of 1895.\fn{\cite{mack1994}}

Following an official inquiry in 1895 the government appointed
Sam\-uel McIntosh \index{McIntosh, S.} as the Village Settlement
Expert, in order to provide advice to settlers on irrigation,
horticulture and agriculture. He had been responsible since 1892 for
horticultural planting and irrigation at Renmark and became familiar
with the Lyrup settlement soon after its foundation.  After two years
in an advisory capacity, McIntosh was put in control of production by
the settlements.\fn{SA PP no.\,113 of 1895, Select Comm.\ Village
Settlements;
\cite[p.\,46--8]{jones1994}.}

Development of irrigation on the settlements was variable and
no\-where sufficient to secure their economic success.  Four
settlements apparently failed to irrigate and each of the others
except Lyrup
\index{Lyrup, SA} failed to irrigate more than 100\,acres. The total 
area of \index{fruit}fruit trees, \index{vineyards}vines and
\index{vegetables}vegetables under irrigation apparently amounted to
little more than 400\,acres on six settlements, with almost half due
to progress at Lyrup.\fn{SA PP no.\,113 of 1895, Select Comm.\ Rept,
quoted B.\,J.~Menzies \& P.\,N.~Gray, \textit{Irrigation and
Settlement in the South Australian Riverland}, SA Dept Agric.\ Tech.\
Paper no.\,7, (1983), p.\,186;
\cite{mack1994}}

Some of the settlements were abandoned after a few years and all came
to an end in 1901 with repeal of the legislative provisions.
Nevertheless a minority of settlers remained at several of the
riverside locations where allotments became available on perpetual
lease.  One settlement survived in a different form and irrigation was
resumed after an interval at others on the Murray.  According to one
review of the village settlements:
\begin{Quote}
	\ldots these settlements were a new experiment in land
	settlement.  The sandy slopes adjacent to the river were
	untried for intensive culture.  The results demonstrated the
	possibility for future development and justified the
	government in establishing a pioneer chain of settlements and
	also establishing the Irrigation Department in
	1910\ldots\fn{\cite[p.\,21, quoting SA Dept of Lands File
	no.\,DL/1747/1950]{mack1994}.}
\end{Quote}

Another significant outcome of the village settlements was the wide
experience gained by Samuel McIntosh who was subsequently given
considerable responsibility for irrigation development in South
Australia.

\section*{Victoria}
\index{Victoria}

In Victoria, Rev.~Horace Finn Tucker \index{Tucker, H.\,F.} launched
village settlements on a charity basis to help the large number of
people rendered destitute with the collapse of the Victorian land boom
in the 1890s.  Tucker's endeavour began early in 1892 and had support
from other clergymen and business people including W.\,W.~Culcheth,
\index{Culcheth, W.\,W.}  an irrigation engineer.  Tucker had
experience in northern Victoria and favoured use of irrigation, as
shown in his novel \textit{The New Arcadia}, but the Tucker
settlements lacked resources for its use.  Their plight was moderated
by legislation in 1893\,---\,The Settlement On Lands
Act\,---\,which\index{legislation!Vic.!Settlements Lands Act 1893}
provided land and credit for the creation of village communities,
homestead associations, and labour colonies.  It was apparently
expected that except for the labour colonies settlers under this
scheme would work for up to four months of the year elsewhere than on
a settlement.  There was no requirement for communal organisation as
in the comparable Queensland legislation, except possibly in the
labour colonies.  For the village communities there were allotments no
larger than 20\,acres for each settler; for homestead associations the
allotments could be no larger than 50\,acres.\fn{\cite{badger1971};
L.\,J.~Blake, \textit{Vic.\ Hist.\ Mag}., vol.\,37, (1966), p.\,191;
\cite{tucker1894}.}

About 80 of the different kinds of settlements started under the 1893
legislation.  Most were in southern Victoria; others were sprinkled
thr\-ough the Wimmera, \index{district!Wimmera} the Goulburn Valley
\index{river!Goulburn} and along the riverine fringe of the mallee
country.  Irrigation was used on a few settlements with a good supply
of water from the Murray or Goulburn rivers.  Much of the settlement
was undertaken before the end of 1893.\fn{L.\,J.~Blake, \textit{Vic.\
Hist.\ Mag.}, vol.\,37, (1966), p.\,195.}

Several miles downstream from Swan Hill, \index{Swan Hill, Vic.} two
settlements were made in 1893, one known as Wood Wood \index{Wood
wood, Vic.} and the other as Tyntynder. \index{Tyntynder, Vic.} Both
were close to the river and attracted several settlers from Mildura
\index{Mildura, Vic.} who were interested in irrigating
\index{fruit}fruit trees and \index{vineyards}vines, whose
establishment in Victoria was then the subject of a government bonus.
The Tyntynder settlement was about 70\,feet above river level and its
pumping plant supplied only domestic requirements; it was many years
before irrigation of orchards and vineyards was undertaken on the
enlarged settlement then known as
Nyah.\fn{\cite[pp.\,104--106]{scholes1989};
\cite[p.\,78]{martin1955}.}

A similar settlement was begun just north of Shepparton
\index{Shepparton, Vic.} in 1893, with 44 settlers occupying
220\,acres; by 1897 it occupied 500\,acres and held 183 people with
half the area under cultivation.  Water was raised by
\index{technology!wind-mill}wind-mills from the river to the sandhills
occupied by the settlers and though irrigation was undertaken, the
supply of water would not have allowed much more than the domestic
requirements.  At Echuca, \index{Echuca, Vic.} a village settlement
was begun in 1894 with the allotment of 20-acre blocks on land east of
the township and near the river.  The prospect of irrigation attracted
settlers but \index{technology!channel}channels were not provided for
a few years.\fn{\cite[p.\,75]{martin1955};
\cite[pp.\,166--68]{coulson1995}; \cite{priestley1965}; L.\,J. Blake,
\textit{Vic.\ Hist.\ Mag.}, vol.\,37, (1966), p.\,195.}

\closure
The irrigation attempted on the most village settlements in the 1890s
required equipment to raise water from nearby streams but the
combination of inexperience of would-be irrigators and the difficulty
in obtaining suitable equipment meant there was little success with
the efforts.  However, once these weaknesses were overcome, the
experience made it possible to develop irrigation successfully at most
of the sites under different conditions of \index{cooperative
settlements|)}settlement.

%\section*{References}
%1.  Qld Agr.J. 1903, vol.12 , p.167.
%2.  W.Metcalf, From Utopian Dreaming To Communal Reality, 1975, p.20.
%3.  G.Lewis Aust.J.Pol.Hist.1973 v.19 p.352, \& S.Svensen, The 
%     Shearers' War, 1989,  A.H.Boyd, Qld Agr.J. 1903 v.12 
%    p.167; W.Metcalf 1995, QldPD 1893,vol.lxx,10/8/1893, p.402-405,416.
%4.  SAPP 154 of 1893.
%5.  A.Jones, Lyrup Village, 1994, p.93
%6.  SAPP 113 of 1895  Parliamentary Select Committee rept on village
%      settlements 1895.
%7.  SA Select Committee  rept. 1895  \& SAPP 37 of 1900. Royal
%     Commission  on Renmark and Murray River Settlements. Final Rept.
%8. D.Mack, The Village Settlements On The River Murray In South Australia 
%    1894-1909, 1994.
%9.  A.Jones, 1994.
%10. D.Mack, 1994 p.39, from Select Comm.Rept 1895 p.37.
%11. D.Mack 1994.
%12. SAPP No.113 of 1895, Select Comm.Village Settlements, \&
%     A.Jones,1994, p.46.
%13. A.Jones, 1994 p.48.
%14. SAPP No. 113 of 1895, Select Comm.Rept, quoted B.J.Menzies \&
%      P.N.Gray, Irrigation And Settlement In The South Australian Riverland,
%      SADept.Agric.Tech.Pap. No.7, 1983, p 186,\& indicated by D.Mack 
%      1994.
%15. SA Dept of Lands File No.DL/1747/1950, quoted by D.Mack, 1994 p.21.
%16. C.R.Badger , The Reverend Charles Strong And The Australian Church,
%      1971, \& L.J.Blake, Vict.Hist.Mag.,1966 ,vol. 37,p.191.
%17. H.F.Tucker, The New Arcadia,  1894.
%18. L.J.Blake Vict.Hist.Mag. 1966, vol.37 p.195.
%19. Lesley Scholes, A History Of The Shire Of Swan Hill,  1989, pp.104-106.
%20. C.S.Martin , Irrigation And Closer Settlement In The Shepparton District 
%     1836-1906, 1955, p.78.
%21. C.S.Martin , 1955, p.75. 
%22.  Helen Coulson, Moama-Echuca 1995, pp.166-68, Susan Priestley, Echuca,
%      1965,  L.J. Blake , Vict.Hist.Mag. 1966, vol.37 p.195.

