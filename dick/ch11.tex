% $Id$
% CHAPTER ELEVEN
% 2495 words at 22/4/99

\chapter{Irrigation of Group Settlements from 1891}

Village settlements were the common form of group settlements
established in all colonies but Western Australia during the economic
depression of the 1890s.  Irrigation was undertaken at most of these
settlements in South Australia, at several in Victoria, and at one in
Queensland.

The land settlement schemes sought in Australia from the late 1880s
included proposals for group settlements to provide either communal
arrangements for land use or allotments to individuals.  The communal
types of settlements were favoured in Queensland by the utopian
William Lane and his associates and in South Australia by the single
tax leagues following the ideas of Henry George, the American land
reformer.  William Lane's decision in 1891 to promote the New
Australia communal colony in South America gained support in some
Australian colonies and was countered by government schemes for
communal land settlement in Queensland and South Australia.  The
legislation achieved in those two colonies in 1893 was intended to
cater for prospective settlers with limited capital, who might
otherwise be tempted by the New Australia scheme.  In other colonies
the government schemes in 1893 for group settlement were designed to
cater particularly for unemployed workers, who thereby generally
obtained individual allotments of land.

\section{Queensland}

Apparently the only group settlement in Queensland to make use of
irrigation was the Alice River Cooperative Settlement formed by more
than 70 men who had been on strike during the important pastoral
dispute of 1891.  Their enterprise was four miles from Barcaldine,
then supplied with water from bores in the Great Artesian Basin.
Although legislation for village settlements had existed in the colony
for a few years, the shearers occupied 3500\,ac provided by the
government under an occupation licence.  A bore to a depth of 1333\,ft
was financed by the settlers to provide artesian water to raise
vegetables and maintain an orchard.  Support for the Alice River
settlement came from many parts of the region as gifts of livestock,
equipment and farm implements.1 Some settlers left to join William
Lane's colony in South America and within a few years the Alice River
settlement had few residents.2 In 1903 there were only eight single
men in occupation, with a vegetable farm of 5\,ac and an orchard of
5\,ac.3

\section{South Australia}

The Kingston government introduced legislation for communal land
settlements, known as village settlements, in October 1893, with the
aim of providing for people of limited means.  Lands could be made
available not to individuals but to associations, which could lease an
area representing no more than 160\,ac per member, with need for
payment of rent after the first year and for improvements to the land.
Each association would make its own rules for management of its
settlement, with provision for collective working of the land or for
individual allotments.  Loans to associations of not more than
\pounds50 per member could be made by the government in respect to
half the value of improvements already undertaken.  Lands available
for such settlements were reported to the government by G.\,M.~Goyder,
the surveyor-general, who included in his list approximately
156,000\,ac near the River Murray.4

Thirteen village settlements were established in 1894 after passage of
the enabling legislation in the Crown Lands Amendment Act of December
1893.  Two settlements were formed by associations with some financial
resources while others involved only unemployed workers who were then
numerous in Adelaide and able to persuade the government to make
special financial assistance for their participation in the scheme.
Eleven settlements located on the banks of the River Murray upstream
from Morgan had a population of more than 1600 in 1895 and were the
only ones attempting irrigation.

The choice of the river as a location for settlements and interest in
their irrigation indicates the Renmark irrigation settlement with its
fruit trees and grape vines as the model for these new ventures.  The
Lyrup settlement was the first to be established on the river in
February 1894.  Its site was selected after advice from people at the
nearby Renmark irrigation settlement following rejection of land near
Moorlands found unsuitable for irrigation.  Lyrup and Renmark both
utilise terraces which occur in that part of the river valley and
provide significant areas of land suitable for irrigation without
high-lift pumping.

The second settlement on the river at Waikerie started several days
later and at a considerable distance downstream from Lyrup, in that
distinctive part of the river valley which provides much less terrace
land above flood level.  Later settlements also failed to secure land
with the advantages at Lyrup and had to utilise ground relatively high
above the river, covered with trees or scrub, and providing sandy
soils.  Probably the worst site was one with a vertical cliff reaching
more than 100\,ft above the river, thus forcing the settlement to
develop on the undulating mallee country fringing the river valley.

The village settlers had to cope with their general inexperience of
farming, the disadvantages of communal organisation, their own
poverty, and the limited advice provided by the government.  Their
preparations for irrigation at the settlements included provision of
pumps and steam engines, the layout of an irrigation system, and
establishment of fruit trees and vines.  They also grew wheat and hay,
erected buildings, and put up fences to exclude rabbits.  These tasks
were too much for some members of all settlements, leading to
resignations and the collapse of three settlements within three years.
The Lyrup settlement was favoured in several ways by its proximity to
Renmark where pruning by Lyrup men was exchanged for trees and vines
provided by Chaffey Brothers.5

In the first year the settlements on the river had about 500 members
of associations and a total population of 1600.6 By 1900 some
settlements had been abandoned and only 95 members and a total
population of 513 lived on the seven remaining ones, only four of
these lasted for ten years.  Their relative failure has been
attributed by participants in the scheme and by others to the
communistic or communal organisation.  The rules governing the
settlements were amended by legislation following official inquiries
in 1895 and in 1899--1900 , and the maximum credit per settler was
increased in 1895 from \pounds50 to \pounds100.7 Apparently the
official inquires concentrated on the relations between settlers and
their occupation of land, with little attention to the problems of
irrigation.  Much detailed information on this matter has been
collated, showing that progress with irrigation was generally hampered
by the initial provision of pumping equipment inadequate for the
extension of irrigation needed to make the settlements financially
successful.8 The conspicuous exception was at Lyrup where most
progress was made with irrigation.

Most settlements initially obtained transportable centrifugal pumps
and steam engines, apparently with help from two government officials
detailed to give advice on engineering and agriculture.  Under the
best conditions they could lift water no more than 30\,ft and were
therefore suitable for the extensive low-lying land of the Lyrup
settlement, but only for the more restricted riverine fringe of other
settlements with land rising 100\,ft or more above the river.  The
Waikerie settlement was the only one which from the outset used a
high-lift pump.  As long as the other settlements were dependent
solely on the output of the centrifugal pumps then available they
could irrigate only their limited areas of lower land.

There was gradual change to high-lift pumps, except at Lyrup, from
late 1895.  These allowed settlers to water fruit trees and vines on
the higher land but at the best the more suitable pumps lacked the
capacity to supply a sufficiently large area under irrigation to make
the settlements successful.  Lyrup coped to some extent with this
problem by acquiring a second centrifugal pump and eventually had more
than 200\,ac of fruit trees, vines, vegetables and lucerne under
irrigation.9 Even so, it was claimed in 1895 that Lyrup would have
needed at least 700\,ac under irrigation to support its population.10

Another complication concerning irrigation in the 1890s from the
Murray River was that its level could vary by as much as\,20 ft within
12 months.  If a permanent site for a pump was carefully placed high
enough to avoid damage from floods, the seasonal fall of the river
might render pumping ineffective because of too great a suction lift
for the pump or from the intake pipe being no longer under water.
Unfortunately the settlers were initially unaware of the natural
variation in river flow.  Their first year was far from
characteristic: rainfall was exceptionally high and so was the river
level until the autumn of 1895.11

Following an official inquiry in 1895 the government appointed Samuel
McIntosh as the Village Settlement Expert, in order to provide advice
to settlers on irrigation, horticulture and agriculture.12 He had been
responsible since 1892 for horticultural planting and irrigation at
Renmark and became familiar with the Lyrup settlement soon after its
foundation.  After two years in an advisory capacity, McIntosh was put
in control of production by the settlements.13

Development of irrigation on the settlements was variable and nowhere
sufficient to secure their economic success.  Four settlements
apparently failed to irrigate and each of the others except Lyrup
failed to irrigate more than 100\,ac. The total area of fruit trees,
vines and vegetables under irrigation apparently amounted to little
more than 400\,ac on six settlements, with almost half due to progress
at Lyrup.14

Some of the settlements were abandoned after a few years and all came
to an end in 1901 with repeal of the legislative provisions.
Nevertheless a minority of settlers remained at several of the
riverside locations where allotments became available on perpetual
lease.  One settlement survived in a different form and irrigation was
resumed after an interval at others on the Murray.  According to one
review of the village settlements:
\begin{quote}
	\ldots these settlements were a new experiment in land
	settlement.  The sandy slopes adjacent to the river were
	untried for intensive culture.  The results demonstrated the
	possibility for future development and justified the
	government in establishing a pioneer chain of settlements and
	also establishing the Irrigation Department in 1910\ldots15
\end{quote}

Another significant outcome of the village settlements was the wide
experience gained by Samuel McIntosh who was subsequently given
considerable responsibility for irrigation development in South
Australia.

\section{Victoria}

In Victoria, Rev.~Horace Finn Tucker launched village settlements on a
charity basis to help the large number of people rendered destitute
with the collapse of the Victorian land boom in the 1890s.  Tucker's
endeavour began early in 1892 and had support from other clergymen and
business people including W.\,W.~Culcheth, an irrigation engineer.16
Tucker had experience in northern Victoria and favoured use of
irrigation, as shown in his novel The New Arcadia, but the Tucker
settlements lacked resources for its use.17 Their plight was moderated
by legislation in 1893 --- the Settlement on Lands Act --- which
provided land and credit for the creation of village communities,
homestead associations, and labour colonies.  It was apparently
expected that except for the labour colonies settlers under this
scheme would work for up to four months of the year elsewhere than on
a settlement.  There was no requirement for communal organisation as
in the comparable Queensland legislation, except possibly in the
labour colonies.  For the village communities there were allotments no
larger than 20\,ac for each settler; for homestead associations the
allotments could be no larger than 50\,ac.

About 80 of the different kinds of settlements started under the 1893
legislation.18 Most were in southern Victoria; others were sprinkled
through the Wimmera, the Goulburn Valley and along the riverine fringe
of the mallee country.  Irrigation was used on a few settlements with
a good supply of water from the Murray or Goulburn rivers.  Much of
the settlement was undertaken before the end of 1893.

Several miles downstream from Swan Hill, two settlements were made in
1893, one known as Wood Wood and the other as Tyntynder19.  Both were
close to the river and attracted several settlers from Mildura who
were interested in irrigating fruit trees and vines, whose
establishment in Victoria was then the subject of a government
bonus.20 The Tyntynder settlement was about 70\,ft above river level
and its pumping plant supplied only domestic requirements; it was many
years before irrigation of orchards and vineyards was undertaken on
the enlarged settlement then known as Nyah.

A similar settlement was begun just north of Shepparton in 1893, with
44 settlers occupying 220\,ac; by 1897 it occupied 500\,ac and held
183 people with half the area under cultivation.  Water was raised by
wind mills from the river to the sandhills occupied by the settlers
and though irrigation was undertaken, the supply of water would not
have allowed much more than the domestic requirements.21 At Echuca, a
village settlement was begun in 1894 with the allotment of 20\,ac
blocks on land east of the township and near the river. The prospect
of irrigation attracted settlers but channels were not provided for a
few years.22

\section{Conclusion}

The irrigation attempted on the most village settlements in the 1890s
required equipment to raise water from nearby streams but the
combination of inexperience of would-be irrigators and the difficulty
in obtaining suitable equipment meant there was little success with
the efforts. However, once these weaknesses were overcome, the
experience made it possible to develop irrigation successfully at most
of the sites under different conditions of settlement.

\section{References}

1.  Qld Agr.J. 1903, vol.12 , p.167.

2.  W.Metcalf, From Utopian Dreaming To Communal Reality, 1975, p.20.

3.  G.Lewis Aust.J.Pol.Hist.1973 v.19 p.352, \& S.Svensen, The 
     Shearers' War, 1989,  A.H.Boyd, Qld Agr.J. 1903 v.12 
    p.167; W.Metcalf 1995, QldPD 1893,vol.lxx,10/8/1893, p.402-405,416.

4.  SAPP 154 of 1893.

5.  A.Jones, Lyrup Village, 1994, p.93

6.  SAPP 113 of 1895  Parliamentary Select Committee rept on village
      settlements 1895.

7.  SA Select Committee  rept. 1895  \& SAPP 37 of 1900. Royal
     Commission  on Renmark and Murray River Settlements. Final Rept.

8. D.Mack, The Village Settlements On The River Murray In South Australia 
    1894-1909, 1994.

9.  A.Jones, 1994.

10. D.Mack, 1994 p.39, from Select Comm.Rept 1895 p.37.

11. D.Mack 1994.

12. SAPP No.113 of 1895, Select Comm.Village Settlements, \&
     A.Jones,1994, p.46.

13. A.Jones, 1994 p.48.

14. SAPP No. 113 of 1895, Select Comm.Rept, quoted B.J.Menzies \&
      P.N.Gray, Irrigation And Settlement In The South Australian Riverland,
      SADept.Agric.Tech.Pap. No.7, 1983, p 186,\& indicated by D.Mack 
      1994.

15. SA Dept of Lands File No.DL/1747/1950, quoted by D.Mack, 1994 p.21.

16. C.R.Badger , The Reverend Charles Strong And The Australian Church,
      1971, \& L.J.Blake, Vict.Hist.Mag.,1966 ,vol. 37,p.191.

17. H.F.Tucker, The New Arcadia,  1894.

18. L.J.Blake Vict.Hist.Mag. 1966, vol.37 p.195.

19. Lesley Scholes, A History Of The Shire Of Swan Hill,  1989, pp.104-106.

20. C.S.Martin , Irrigation And Closer Settlement In The Shepparton District 
     1836-1906, 1955, p.78.

21. C.S.Martin , 1955, p.75. 

22.  Helen Coulson, Moama-Echuca 1995, pp.166-68, Susan Priestley, Echuca,
      1965,  L.J. Blake , Vict.Hist.Mag. 1966, vol.37 p.195.
