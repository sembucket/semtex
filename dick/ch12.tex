% $Id$
% CHAPTER TWELVE 
% 2108 words at 23/4/99

\setcounter{endnote}{0}

\chapter{Sewage Irrigation 1880--1920}
\index{sewage irrigation}
\label{ch:sewage}\addtoendnotes{\protect\subsection*{Chapter \thechapter}}
\markboth{Chapter \thechapter. Sewage Irrigation}%
{Pioneering Irrigation in Australia}

Irrigation with sewage was introduced before 1900 by sanitary
authorities in three Australian capital cities.  The sewage farms in
two cities operated after 1920 but in Sydney the practice was
abandoned earlier.

Application of sewage to land had begun near several European cities
in the 19th century; its use in Australia was recommended by 1865.  By
that time the larger cities\,---\,Melbourne, Sydney, and
Adelaide\,---\,had municipal water supplies piped to many houses, and
the volume of waste water, including human excrement, increased to the
point where the cities needed comprehensive sewerage schemes to
minimise pollution and reduce the toll of typhoid fever.  Remedial
action was delayed by conflicts between colonial governments and civic
authorities, and by the expense involved.  As late as the early 1880s,
Melbourne's open sewers continued to be a noisome feature of the city,
and though underground sewers improved the atmosphere in the city of
Sydney, their outfalls to the Harbour made its waters unhealthy for
mariners.\fn{\citet{twopenny1883}; \citet[p.\,140, quoting NSW PP
1875--78]{lloyd1988}.}

The first use of sewage irrigation was in Adelaide in 1881, followed
by Sydney in 1890 and Melbourne in 1897.

\section*{Adelaide}
\index{Adelaide, SA}

Development of a comprehensive sewerage system for the City of
Adelaide, with use of sewage for irrigation, was first proposed in
1864 by two entrants in the competition arranged by the city council
for a prize essay on drainage and sewerage.  At that time Adelaide was
provided with water from its first reservoir at Thorndon Park;
\index{reservoir!Thorndon Park} water flowed into the city in
December 1850 and by 1867 more than 20\,000 consumers in Adelaide and
Port Adelaide had piped
supplies.\fn{\citet[p.\,24--25]{hammerton1986}.}

The existence of a reticulated water supply encouraged attention to
replacing the cesspits needing periodical emptying and transfer of
their contents to burial pits in parks or disposal in the River
Torrens.  \index{river!Torrens} The essay competition brought the
first prize to R.\,G.~Thomas, \index{Thomas, R.\,G.}  architect and
civil engineer, for his scheme involving sewage irrigation as
practiced in British and European cities.  No development of a
sewerage scheme occurred for more than ten years; in 1876 a commission
was appointed by the government `to inquire into the whole question of
sanitation for the province'.  It recommended a system of deep
drainage and disposal of sewage by irrigation.  In 1878, a noted
English hydraulic engineer, William Clark, \index{Clark, W.}  made a
report to the City Council on drainage and sewage disposal, with a
finding in favour of deep drainage, a sewage farm about three miles
northwest of the city, and discharge of the city's main sewer into the
sea.  Both government and civic authorities were then agreed that
Adelaide needed the underground sewerage system generally referred to
as deep drainage, but the disposal of sewage was undecided until the
government accepted the report of a commission appointed to consider
the best location for a sewage farm.  The inquiry, chaired by the
Victorian engineer Thomas Higinbotham, \index{Higinbotham, T.}
recommended the location of a sewage farm near Islington, some miles
north of the city.  The government then acquired 160\,acres of land
near Islington in readiness for preparation of a sewage farm. Later
the farm was extended to cover 470\,acres.\fn{\citet{thomas1865}; SAPP
nos.\,18 of 1876, 174 of 1879;
\citet[p.\,39]{hammerton1986};
\citet[pp.\,221--24]{lewis1985}.}

Work started on construction of deep drains for water-borne conveyance
of sewage by gravity to the sewage farm of Islington, \index{sewage
farm!Islington, SA} with provision for effluent to be led by an open
drain a distance of 4.5 miles for discharge into the tidal North Arm
Creek. \index{creek!North Arm}

The sewage farm received sewage in 1881.  It was filtered before use
to irrigate paddocks with a fall of 12\,feet to the drain at the
north.  Initially the main purpose of irrigation was to grow fodder
for livestock, including dairy cows, sheep, and pigs, and there was
also some production of vegetables and root crops for stock fodder.
The dairy production supplied milk for consumption by Adelaide
residents, and butter was also marketed.  The piggery was intended to
supply meat for treatment in a smoke-house erected on the property.
Following a few cases of typhoid in 1884 attributed to farm produce,
the dairy was closed but butter was still made and sold.  Gradually
the main use of the farm became grazing livestock on agistment from
owners of small properties.  Some of these animals were dairy cows.
Eventually the grazing of dairy cows was discontinued, but vegetables
were still grown.\fn{\citet{hammerton1986};
\citet[p.\,221]{lewis1985}.}

The Islington farm continued to receive most of the sewage from the
Adelaide metropolis for many years after 1920.  The sewerage system
served most of the urban area by gravitation at first; The city lies
at the edge of a fault block, from which the land drops about
100\,feet to the plain with the sewage farm.  The scheme avoided the
discharge of raw sewage into the tidal waters of the Gulf St~Vincent,
\index{Gulf St Vincent} and the production from the farm helped reduce
the costs of sewerage.

The Adelaide system of sewage irrigation attracted Victorian attention
during the 1890s: two Victorian Royal Commissions took evidence from
South Australian witnesses with experience of the farm's
value.\fn{VicPP no.\,27 of 1889, Progress Rept of RC~Sanitary
Condition, MoE; 4th Progr. Rept Vic. RC~Vegetable Products, MoE
p.\,115; \citet{lewis1985}.}

\section*{Sydney}
\index{Sydney, NSW}

Human excrement in Sydney at first accumulated in cesspits and middens
excavated from the sandstone on which the city rests.  Regular
cleansing of the cesspits was undertaken by the 1850s and their
contents were disposed of by burial in parks, sale to market
gardeners, and dumping in the harbour.\fn{\citet[p.\,140]{lloyd1988}.}

In 1875 the government appointed the Sydney City Sewage and Hea\-l\-th
Board to investigate the problem and make recommendations for
improvement.  The Board favoured a system of sewers at the north to
channel sewage into the ocean and a southern system with outfall to a
sewage farm on Botany Bay near Cooks River.  These improvements
depended on a better water supply for Sydney, finally achieved on
completion of the Nepean River \index{river!Nepean} scheme in 1888; it
had been recommended in 1869 by the Commission appointed to inquire
into the Sydney Water Supply.\fn{\citet[pp.\,143--48]{lloyd1988}.}

An area of 309\,acres for a sewage farm at the mouth of Cooks River
\index{river!Cooks} \index{sewage farm!Cooks River} was
obtained by the government in 1882.  Work then began for the transfer
of sewage from southern suburbs to the farm.  This involved drainage
of sewage to the north bank of the river, where it was transferred by
a light railway bridging the river; on the south side the sewage was
discharged to bays on the sandy soil near the river mouth.

At first in 1890 the farm had about five acres under cultivation of
cabbages, tur\-nips, lucerne and sorghum for sale.  Then cows and pigs
were bought to consume unsold produce.  Later, stock-raising proceeded
satisfactorily for some years, and agistment of stock was also carried
on.\fn{\citet[p.\,138]{aird1961}.}

By 1897 the farm area was increased in anticipation of the western
suburbs sewerage scheme by resumption of an additional 311\,acres to
the south of the original area, but only 96\,acres were in use as
filter beds.

Complaints by local residents after the farm had been operating for 15
years led to experiments in 1898 with biological treatment of sewage
at the farm.  That method would have become general practice at the
farm but for an expected increase in sewage from southern suburbs
along the Illawarrra railway.  With this development in view, work was
begun to change from irrigation at Botany Bay to discharge of sewage
from all southern and western suburbs into the ocean at Long Bay
\index{Malabar, NSW} (Malabar).\fn{\citet{aird1961}.}

An outbreak of swine fever at the farm in 1905 put an end to
pig-raising.  In 1906 crop production declined as the volume of sewage
increased so much that cultivation areas were flooded continuously.
Crop production ceased about 1910.

With completion of the outfall sewer for southern and western suburbs
in 1916, the sewage farm ceased operations.  Eventually an area of
600\,acres formerly part of the farm was put to use for sporting
purposes, parks and residences.  Thereafter all sewage from the Sydney
area was discharged to the ocean from one point on the north shore and
two (Bondi and Malabar) \index{Bondi, NSW} on the south
coast.\fn{\citet{lloyd1988};
\citet{aird1961}.}

\section*{Melbourne}
\index{Melbourne, Vic.}

Melbourne at first relied on cesspits for collection of human
excrement and other wastes.  These were cleaned out periodically and
the contents were dumped at a manure depot at the outskirts of the
town.  In 1866 the city authorities began the weekly removal of
night-soil from homes for burial in two parks.  They also began the
sale of nightsoil to market gardeners.  These measures reduced the
pollution of streams but open sewers in the city continued to attract
criticism during the 1880s.\fn{\citet[pp.\,140--45]{cornish1880};
\citet{twopenny1883}.}

In 1880, the year of the Melbourne International Exhibition,
\index{Melbourne International Exhibition 1880} the
Melbourne City Council held a competition for the best essay on
underground drainage of the city.  The first prize of \pounds200 was
awarded to George Gordon, \index{Gordon, G.} formerly chief hydraulic
engineer of Victoria.  His proposal included transfer of sewage for
irrigation on a site near Melbourne, his first preference being
Sandridge (Fishermans Bend), but in 1889 he favoured transport of
sewage to Laverton or Mordialloc.\fn{\citet[p.\,260]{dunstan1984};
VicPP no.\,112 of 1889, 3rd Prog.\ Rept RC~Sanitary Conditions of
Melbourne, MoE pp.\,8--9,~46.}

For several years there was no improvement in sewerage and the
incidence of typhoid increased alarmingly.  The City Council tried to
establish a Metropolitan Board of Works to control water supply and
sewerage, but negotiations with other municipalities were protracted
and eventually the Government early in 1888 announced a Royal
Commission to inquire into the sanitary condition of the
metro\-po\-lis.  This Commission consisted of eminent representatives
of public health, engineering, and commerce, without any from
municipal authorities.  It strongly recommended an underground system
of drainage and in 1889 sought advice from J.\,M.~Mansergh,
\index{Mansergh, J.\,M.} a British hydraulic engineer.  He prepared
plans and specifications for a system of underground
sewers.\fn{\citet[p.\,274]{dunstan1984}.}

Following the Royal Commission inquiry, the Government secured
legislation to establish a Melbourne and Metropolitan Board of Works
(MMBW) \index{Melbourne Metropolitan Board of Works} in 1890.  This
body was responsible for the recommendations of the Commission.  The
Board first met in 1891 and proceeded to choose the area for disposal
of Melbourne sewage, either on the Werribee Plains at the west side of
Port Phillip Bay or the Mordialloc area on the east of the Bay.  The
decision was in favour of Werribee, \index{sewage farm!Werribee}
where the land would be much cheaper to buy; an area of 8847\,acres
was purchased from the Chirnside brothers, owners of a large area
there.\fn{\citet[p.\,63]{james1985}.}

Construction of sewers proceeded for several years while part of the
sewage farm was used by tenant farmers for production of hay and
cereals.  Preparation of the land for irrigation by grading and
cultivation was also undertaken.  Apparently the original intention
was to undertake dairy farming and market gardening on the sewage
farm, and to ship the produce to Melbourne by steamer. The sewage farm
was bounded at the north by the Werribee River, \index{river!Werribee}
accessible to steamers which had previously shipped local produce to
the city.  In order to facilitate marine transport from the sewage
farm, a jetty was installed on the bay but it had little use.
Eventually grazing of beef cattle became the main use of the land
irrigated with sewage.\fn{\citet[pp.\,63--65]{james1985}.}

Two trunk sewers draining from northern and southern parts of the
metro\-po\-lis joined together on the west side of the Yarra
\index{river!Yarra} at Spotswood \index{Spotswood, Vic.} 
where their contents were raised 125\,feet by pumping over nearly
three miles for discharge into a single channel with gravity flow to
the land at Werribee.  This slopes from north to south and as sewage
entered the area at the lower part, further pumping was necessary, The
farm operated mainly by irrigation of 20-acre blocks, with filtered
sewage discharging into Port Phillip Bay.  The main system used was
land filtration, also grass filtration, with recourse to lagooning
only at periods of peak discharge.\fn{M.\,M.~Wilson, ANZAAS Hdbk Vic.\
1955 p.\,281--83.}

%\section*{References}
%1. R.Twopeny, Town Life In Australia,1883 \& C.J.Lloyd, Either Drought Or 
%    Plenty, 1988, p.140 quoting NSW PP 1875-78.
%2. Marianne Hammerton, Water South Australia, 1986, p.24-25.
%3. R.G.Thomas et al., Prize Essays On The Drainage And Sewerage Of
%    Adelaide, 1865. 
%4. SAPP No. 18 of 1876. 
%5. Marianne Hammerton 1986 p. 39.
%6. SAPP 174 of 1879.
%7. H.J.Lewis, Enfield And The Northern Villages, 1985 ,pp. 221-4.
%8. Marianne Hammerton 1986 \& H.J.Lewis 1985, p.221.
%9. VicPP No.27 of 1889, Progress Rept. of Royal Commission on Sanitary 
%     Condition, MoE, \& 4th Progr. Rept Vic Royal Commission Vegetable 	
%     Products, MoE p.115, \& H.J.Lewis, 1985.  	
%10. C.J.Lloyd , 1988, p.140.
%11. C.J.Lloyd, 1988,  p.143.
%12. C.J.Lloyd, 1988, p.145-8.
%13. W.V.Aird , The Water Supply, Sewerage And Drainage Of Sydney,1961,
%       p.138.
%14. W.V.Aird 1961.
%15. C.J.Lloyd, 1988 and W.V.Aird 1961.
%16. H.Cornish ,1880, Under The Southern Cross, pp.140-145.
%17. R.Twopeny, Town Life in Australia,1883.
%18. D.Dunstan, Governing The Metropolis, 1984, p.260.
%19. VicPP No.112 of 1889, 3rd Prog.Rept. R.C.Sanitary  Conditions
%      of Melbourne, MoE pp.8-9,46. 
%20. D.Dunstan 1984 p.274.
%21. K.N.James, Werribee, 1985, p.63.
%22. K.N.James 1985 p.63.
%23. K.N.James p.65.
%24. M.M.Wilson, ANZAAS Hdbk Vic.1955 p.281-3.

