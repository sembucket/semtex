% $Id$
% CHAPTER THIRTEEN
% 3241 words at 2/5/99

\setcounter{endnote}{0}

\chapter{Use of the River Murray for Navigation and Irrigation}
\index{river!Murray}
\index{navigation}
\label{ch:murray}\addtoendnotes{\protect\subsection*{Chapter \thechapter}}
%\markboth{Chapter \thechapter. The River Murray}%
%{Pioneering Irrigation in Australia}

\fancyhead[RE]{\sffamily \small Chapter \thechapter.\ %
               The River Murray}

The first interest of the three south-eastern Australian colonies in
the River Murray system concerned the limitation of navigation by
debris accumulated in the streams.  Snagging operations commenced in
the 1850s and were paid for by the three colonies.  Their mutual
interest was expressed in a resolution of the 1863 inter-colonial
conference in Melbourne:
\begin{Quote}
	That \ldots the commerce, population and wealth of Australia
	can be largely increased by rendering navigable and otherwise
	utilising the great rivers of the interior such as the Murray,
	Edward, Murrumbidgee and Darling; and that the obligation of
	carrying into effect the necessary works to accomplish these
	objects devolves primarily upon the respective Governments
	having jurisdiction over such rivers.\fn{\citet[pp.\,90, 152,
	citing Aqua 1958 \textbf{9}(8) p.\,164, \& Glynn
	1902]{lloyd1988}.}
\end{Quote}

Transport by paddle-steamers \index{paddle-steamer} and barges was at
first based on Gool\-wa, \index{Goolwa, SA} the river port near the
Murray mouth in South Australia.  Supplies for inland goldmining towns
and for pastoral stations along the Murray and its larger tributaries
were shipped from South Australia which duly received and exported
much of the wool from a vast pastoral area.  The monopoly of river
trade held by that colony ended in 1864 when Echuca \index{Echuca,
Vic.} became a major river port after its railway connection to
Melbourne.  In 1880 the steamers and barges on the river carried about
10 per cent of the Australian wool clip.  By 1893, after more ports on
the rivers gained railway connections with Melbourne or Sydney, half
of New South Wales west of longitude\,148 still remained dependent on
river transport from South Australia and Victoria while the balance
was served by rail from Sydney.\fn{\citet[p.\,88]{painter1979};
\citet[p.\,11]{hall1931}.}

Development of irrigation was seen by South Australians as likely to
further reduce their use of the Murray river system for trade.
Victorians had by 1886 begun irrigation in northern Victoria, based on
supplies from the Murray and tributaries.  When an agreement providing
for irrigation development at Renmark \index{Renmark, SA} by the
Chaffey Brothers \index{Chaffey Bros} was debated in the South
Australian parliament in 1887, Patrick Glynn \index{Glynn, P.\,McM.}
opposed the proposal on the grounds that its use of river water would
jeopardise river navigation. By 1900 it was estimated that diversions
for irrigation from the Murray river system were equivalent to an
annual flow at the rate of 4942 cubic feet per second. This represents
approximately 20 per cent of the estimated yield of the
Murray.\fn{SAPD 19 July 1887 p.\,259; H\,.G.~McKinney, J.\,Roy.\ Soc.\
NSW, 190; \textsl{Review of Aust.\ Water Resources} 1975.}

\section*{Colonial negotiations on irrigation and \\
navigation affecting Federation}
\index{Federation}

A possible development of irrigation in New South Wales, considered by
its 1884 Royal Commission on Water Conservation, \index{NSW RC Water
Conservation 1884} led to the proposed agreement framed in 1886 with
its counterpart in Victoria where\-by the two colonies would share the
waters of the River Murray. This became known to the South Australian
government after James Service, \index{Service, J.} the Victorian
Premier, suggested in July 1885 to the South Australian Premier,
J.\,W.~Downer, \index{Downer, J.\,W.} that a joint Royal Commission
should be appointed to consider navigation and utilisation of River
Murray waters for irrigation.  South Australian indignation at the
proposed agreement finally brought a conciliatory response from the
other colo\-nies and Downer in 1886 expressed continued interest in a
joint Royal Commission.  Thus when the SA Royal Commission on Murray
River Waters Utilisation was appointed in 1887, \index{SA RC Murray
Waters Utilisation 1887} one of its tasks was `to confer and consult
with similar commissions in New South Wales and Victoria.  This body
was dissolved in 1894, having failed in its purpose, mainly due to the
attitude taken by Sir Henry Parkes, \index{Parkes, H.} who became
Premier of New South Wales early in 1887.  He promptly dissolved the
Royal Commission on Water Conservation, emphasised his State's
ownership of the Murray, and even threatened the Chaffey Bros, the
Mildura irrigation promoters, with trespass for pumping from the
river.\fn{2nd Rept RC Water Conservn NSW, 1886; SAPP
nos.\,59,\,59a,\,59b of 1886, 34 of 1890;
\citet[pp.\,157--58]{lloyd1988}.}

The Victorian position regarding use of the Murray was similar to that
of South Australia in respect to navigation, though assertive about
its interest in irrigation dependent on the Murray and its Victorian
tributaries, and defensive towards the New South Wales government
which was then affirming its claim on the river upstream from the
South Australian boundary.  The Victorian government felt relatively
secure with plans for irrigation from Victorian tributaries of the
Murray even though South Australian claims concerning navigation
involved all parts of the river system and could involve an appeal on
the issue from the South Australian parliament to the Imperial
government.  These conflicts between the three colonies were among the
many complications in negotiations leading to their Federation.  These
matters involved prolonged consideration at the Second National
Australasian Convention of 1897--98 before the framing of sections of
the Australian Constitution \index{Australian Constitution} dealing
with irrigation and navigation and means of settling any future
conflicts.  South Australian insistence on a right to navigation
\index{navigation} on the Murray gained some recognition in Section~98
of the proposed constitution agreed to in 1898 and subsequently
adopted.\fn{\citet[S.\,D.~Clark \& I.\,A.~Renard]{frith1974};
\citet[pp.\,159--60]{lloyd1988}.}  This section reads:
\begin{Quote}
	The power of the Parliament to make laws with respect to trade
	and commerce extends to navigation and shipping, and to
	railways the property of any State.\fn{\citet[p.\,289]{frith1974}.}
\end{Quote}

The two upstream colonies concerned more with promotion of irrigation
saw that object safeguarded by Section 100, which states:
\begin{Quote}
	The Commonwealth shall not, by any law or regulation of trade
	or commerce, abridge the right of a State or of the residents
	therein to the reasonable use of the waters of rivers for
	conservation or irrigation.
\end{Quote}

The Constitution makes no provision for regulation of the Murray for
irrigation or navigation, but its sections 101, 102, 103 and 104
relating to an Inter-State Commission might have provided for
consideration of disputes concerning use of the Murray waters.
Section 101 states:
\begin{Quote}
	There shall be an Inter-State Commission with such powers of
	adjudication and administration as the Parliament deems
	necessary for the execution and maintenance, within the
	Commonwealth, of the provisions of this Constitution relating
	to trade and commerce, and of all law thereunder.
\end{Quote}
Federation had no immediate effect on use of the Murray River system
as the first Commonwealth government took no steps to legislate on
riparian navigation and contemporary moves to set up an Inter-State
Commission under the constitution were
rejected.\fn{\citet[p.\,630]{lanauze1979}.}

\section*{Response to serious drought in 1902}
\index{drought}

Severe drought experienced for several years in the three neighbouring
States reached a climax late in 1902 when flow of the Murray
\index{river!Murray} at Mildura declined to six per cent of normal. It
led to a renewal of calls from New South Wales for irrigation in the
Riverina.  Schemes had been proposed over several years for diversion
of Murray water to irrigate land there.  Ultimately the Murray River
Main Canal League promoted the Corowa conference in April 1902,
attended by the Prime Minister, the Premiers of three States, and
other prominent people.  One important result was the announcement
that a Royal Commission representing three States would investigate
and report on the Murray waters:\fn{W.\,J.~Gibbs \& J.\,V.~Maher
(1967), CBM Bull.\ no.\,48;
\citet[p.\,73]{keating1992}; \citet[p.\,182]{burton1973};
\citet[p.\,21]{evatt1938}.}
\begin{Quote}
	The Interstate Royal Commission on the River Murray
	\index{Interstate RC River Murray 1902} was appointed without
	delay to inquire and report on the conservation and
	distribution of the waters of the Murray and its tributaries
	for the purposes of irrigation, navigation and water supply.
\end{Quote}

Its three members were Joseph Davis, \index{Davis, J.} civil engineer
and secretary of the New South Wales public works department, Stuart
Murray, \index{Murray, S.} engineer and secretary of the Victorian
water supply department, and Frederick Burchell, \index{Burchell, F.}
engineer of the South Australian government.  In May 1902, the
Commission began a series of interviews aboard a steamer moving
upstream on the Murray and completed its report in December of that
year.\fn{VicPP no.\,35 of 1902--3 vol.\,3.}

The Commission recommended the establishment of a storage on the upper
Murray (Cu\-m\-ber\-oona), another on Lake Victoria
\index{lake!Victoria} at the extreme south-west of New South Wales, a
series of weirs
\index{weirs} and locks on the Murray \index{river!Murray} from
Blanche\-town
\index{Blanchetown, SA} to
Echuca, \index{Echuca, Vic.} and reservation of 1\,380\,000\,acre-feet
to South Australia in a normal years after the river was locked. It
supported the use of the river for navigation but acceptance of the
view that irrigation should be the major gift of the Murray caused the
South Australian member to give a dissenting opinion.  Its
recommendations formed the basis for unsuccessful efforts over several
years for agreement by the three States involved but the work of the
commission had many long-term effects, including attention to
pollution of the river by sewage and mining wastes.\fn{VicPP no.\,35
of 1902--3; Aqua 1958 p.\,165; Vic RC of 1909;
\citet[p.183]{lloyd1988};
\citet[p.\,21]{evatt1938}; \citet[S.\,D.~Clark \&
I.\,A.~Renard]{frith1974}.}

\section*{Renewed interest in navigation and\\ irrigation}

After several years there were new developments. One was the
initiative of the South Australian government which in 1909 sought
advice from the Victorian engineer Stuart Murray \index{Murray, S.}
on utilization of the Murray waters for navigation and irrigation and
the development of the resources of the river valley.  His report
considered the benefits resulting from storage at Lake Victoria
\index{lake!Victoria} in New
South Wales, exclusion of sea water from the terminal lakes, and
possible reclamation of lake beds.  A new government then decided to
act alone by arranging for construction of a weir \index{weirs} and
lock for navigation on the Murray River in South Australia and of
storage at Lake Victoria, both at the expense of South Australia,
under the South Australian Murray Works Construction Act of
1910. \index{SA Murray Works Const.\ Act 1910} Captain E.\,N.~Johnston
\index{Johnston, E.\,N.} of the USA was engaged to advise on river
control by locks.  His report in 1912 led to the start of work on the
Blanchetown \index{Blanchetown, SA} weir and lock in 1913.  At the same
time the government made the first move in its plan to install
barrages to protect the terminal Lakes Alexandrina
\index{lake!Alexandrina} and Albert \index{lake!Albert} from entry of
sea water when the Murray flow was insufficient\,---\,as during
droughts.  The Mundoo channel, \index{channel!Mundoo} one of the five
allowing flow from sea to river, or vice versa, was closed by a
temporary dam about 1913.\fn{SAPP nos.\,29 of 1910, 116 of 1913;
\citet[p.\,289]{frith1974}; \citet[p.\,150]{williams1974};
\citet[p.\,153]{hammerton1986}.}

In Victoria a Royal Commission was set up in 1909 by the Victorian
government to consider the use of the Murray River and tributaries by
the three south-eastern States.  Its wide-ranging inquiry involved
meetings in three States, review of earlier technical
investigations,and proposals for storage, a system of locks,
allocation of flow to South Australia, and a Murray River Inter-State
Board for control irrigation and navigation.\fn{VicPP no.\,7 of 1910;
\citet[p.\,141]{powell1989}.}

The possible involvement of the Commonwealth Government in the
long-standing problem of reconciling navigation and irrigation could
have followed the renewal in 1909 of moves to establish the
Inter-State Commission provided for under the Commonwealth
Constitution.  The reference in Section 100 of the Constitution
\index{Australian Constitution} to `reasonable use of the waters of
rivers for conservation or irrigation' could have been an issue for
consideration by this Commission.  Perhaps coincidentally, the
Inter-State Commission Act \index{Commonwealth Inter-State Comm.\ Act
1912} was passed by the Commonwealth in 1912 and the Fisher Government
established the Commission in 1913.  When Deakin
\index{Deakin, A.}  declined
to become its chief commissioner, that position was filled by
A.\,B.~Piddington, KC; \index{Piddington, A.\,B.} other members being
G.~Swinburne, \index{Swinburne, G.} industrial engineer and former
Victorian Minister for Agriculture and Water Supply, and
N.\,C.~Lockyer, \index{Lockyer, N.\,C.} a Commonwealth customs
authority.  At the time this Commission was regarded as a very
important body; its existence and wide powers possibly encouraged the
three States to seek an early compromise on control of the Murray,
thus avoiding intervention by the Commission.\fn{\citet[note,
p.\,630]{lanauze1979};
\citet[pp.\,219--29]{roe1984}.}

\section*{The River Murray Agreement}
\index{River Murray Agreement}
\index{river!Murray}

The next significant move in joint action by the States came when the
Melbourne conference of Premiers of New South Wales, Victoria and
South Australia in 1911 \index{Premiers Conference 1911} agreed that
South Australia should construct the Lake Victoria
\index{lake!Victoria} storage and decided to refer the use of the
waters of the Murray and tributaries to engineers of the three States:
E.\,M.~De Burgh \index{De Burgh, E.\,M.} from New South Wales,
J.\,S.~Dethridge \index{Dethridge, J.\,S.} from Victoria and
G.~Stewart \index{Stewart, G.} from South Australia.  Their report in
July 1913 was important in reconciling differences in river gaugings
made by three States and producing an authoritative statement of the
water available in the Murray system.  They could not agree about the
relative importance of irrigation and navigation but they suggested
means for control of the water and emphasised the need for
storages.\fn{\citet[p.\,132]{hammerton1986}; Aqua April 1958,
pp.\,163--66; SAPP 21 of 1913;
\citet[K.\,E. Johnson]{frith1974};
\citet[p.\,23]{evatt1938}.}

At last, during a serious drought, the three State Premiers and the
Prime Minister in September 1914 drew on the Engineers' Report when
they completed the River Murray Waters Agreement leading to a River
Murray Commission \index{River Murray Commission} to satisfy claims
for navigation and irrigation. The 1914 achievement was partly due to
the efforts of P.~McM.~Glynn, \index{Glynn, P.\,McM.} then a South
Australian Minister in the Federal Cabinet.\fn{\citet[p.\,132
\& p.\,133 re refusal of upper States to give greater control to
RMC.]{hammerton1986}.} The principles of the Agreement were:
\begin{Quote}
	\ldots flow at Albury \index{Albury, NSW} is shared equally
	between New South Wales and Victoria; Victoria and New South
	Wales retain control of their tributaries below Albury;
	Victoria and New South Wales supply South Australia with a
	guaranteed minimum quantity of water or
	`entitlement'.\fn{\citet[p.\,40]{jacobs1990}.}
\end{Quote}

The agreement also provided for construction of a storage on the upper
Murray; another storage at Lake Victoria; \index{lake!Victoria} 24
locks and weirs \index{weirs} below Echuca; \index{Echuca, Vic.} and
nine locks and weirs on the Murrumbidgee or Darling
River. \index{river!Darling} \index{river!Murrumbidgee} The cost of
any works was to be shared equally by the three States, with a
Commonwealth contribution of up to \pounds1\,000\,000. After
ratification of the agreement by the four parties in 1915, the River
Murray Commission was constituted and held its first meeting in
February 1917, its president, Western Australian Senator P.\,J.~Lynch,
\index{Lynch, P.\,J.}  represented the Commonwealth and three
engineers: H.\,H.~Dare (NSW), \index{Dare, H.\,H.}  J.\,S.~Dethridge
(Vic); \index{Dethridge, J.\,S.} and G.~Stewart~(SA), \index{Stewart,
G.}  represented the States.\fn{Details of River Murray Waters
Agreement are given as schedule to River Murray Waters Act of 1915,
passed by Commonwealth and by the three States ---\,see
\citet[p.\,280, note 18]{frith1974}.}

\section*{Benefits for irrigation}

\subsection*{Upper Murray storage}

Diversion of Murray water for irrigation in southern New South Wales
was advocated before the Lyne Royal Commission in the 1880s by
H.\,G. McKinney, \index{McKinney, H.\,G.} the irrigation engineer.  He
favoured a scheme for a weir on the Murray \index{river!Murray} near
Corowa \index{Corowa, NSW} for diversion into an irrigation canal, and
storage on the upper Murray near Talmalmo, \index{Talmalmo, NSW}
upstream from the Mitta Mitta River. \index{river!Mitta Mitta}
Victoria stood to gain from such a storage, in view of the scheme
developed in 1885 for the two colonies to share the flow of the
Murray.  The McKinney project went into temporary abeyance after the
visit in 1896 of Colonel Home who recommended that development of
irrigation in New South Wales should be based on the Murrumbidgee
River, \index{river!Murrumbidgee} with storage at Burrinjuck,
\index{Burrinjuck, NSW} rather than the Murray
River.\fn{\citet[p.\,182]{lloyd1988}.}

The 1902 drought and the Corowa Conference brought renewed attention
to use of the Murray for irrigation by New South Wales and Victoria;
the conference resolved that the New South Wales government should
empower the Commonwealth to provide storage on the Upper Murray as
proposed by McKinney.  However, that government in 1905 was reluctant
to abandon interest in Murrumbidgee irrigation, which would not be
complicated by joint action with
Victoria.\fn{\citet[p.\,187]{lloyd1988}.}

Nevertheless the upper Murray storage to provide one million acre-feet
was later accepted as part of the River Murray Waters agreement. The
site for the storage, which became known as the Hume Weir,
\index{weir!Hume} was chosen 10 miles above the town of Albury
\index{Albury, NSW} and below the confluence of the Mitta Mitta River
after two years of investigations including examination of 25 sites.
Work began in 1919.  The purpose was to conserve some of the peak
flows during periods of little demand for irrigation so that releases
of water could be made during periods of high demand and poor river
flow, as in droughts.\fn{\citet{burton1973}.}

\subsection*{Lake Victoria Storage}
\index{lake!Victoria}

The storage of 550\,000\,acre-feet at Lake Victoria, involving works
begun in 1915, was intended to maintain sufficient reserves to ensure
adherence to the annual `entitlement' of 1.25~million acre-feet of
river water for South Australia.  This would ensure continuity of
navigation in the South Australian sector, allow development of
irrigation on both sides of the river, and provide sufficient flow to
withstand incursions of the sea into Lakes Alexandrina and
Albert. \index{lake!Alexandrina}\index{lake!Albert} The allotment of
water for South Australia provided by the agreement was sufficient for
considerable development of irrigation in that State; it was estimated
as enough for 150\,000\,acres.\fn{J.\,H.\,O.~Eaton (1946), ANZAAS Hdbk
for South Australia, p.\,82.}

\subsection*{Weirs and Locks}
\index{weirs} \index{locks}

The agreement to provide 26 weirs and locks on the Murray between
Blanche\-town \index{Blanchetown, SA} and Echuca \index{Echuca, Vic.}
satisfied South Australian interest in the river trade; it also had
important advantages for irrigation.  One of these concerned Victoria
which needed a weir on the Murray at Torrumbarry \index{Torrumbarry,
Vic.} to ensure filling of the Kow Swamp \index{swamp!Kow} each year,
thus helping irrigation development near Cohuna \index{Cohuna, Vic.}
and Kerang. \index{Kerang, Vic.} Following the Murray Waters Agreement
\index{Murray Waters Agreement 1915} of 1915, the Torrumbarry weir,
representing Lock No.\,26 of those between Blanchetown and Echuca, was
begun in 1919. It was the first to be installed upstream from South
Australia and the first movable weir on the river.  According to
Elwood Mead \index{Mead, E.} `the construction of these locks will be
mainly to save for irrigation the water that would be lost by
maintaining navigation on an unlocked stream'.  In South Australia the
locks had definite advantages for irrigation, being located at such
intervals that besides ensuring navigation they provided pools to
supply irrigation on nearby banks of the
Murray.\fn{\citet[p.\,289]{frith1974}; E.~Mead's memo (1915) in Aqua
1958 p.\,169; Frith \& Sawer 1974, p.\,289;
\citet[p.\,22]{mccoy1988};
\citet[pp.\,150 and quotation p.\,152]{williams1974}.}

\closure
The conflicts between New South Wales, Victoria, and South Australia
concerning use of the River Murray waters were partly resolved by the
adoption of certain clauses in the Australian Constitution, followed
by decisions on the regulation of river flow by storage and weirs.
The River Murray waters agreement allowed the maintenance of
navigation sought by South Australia and supply of water for
irrigation in the three States.  One of its effects was to encourage
further expansion of irrigation in the Murray Valley.

Although some concern was expressed by authorities at pollution of the
river by mining and sewage, there was by 1920 nothing to indicate
pollution by saline drainage \index{salinity} from irrigated areas
along the Murray.

%\section*{References}
%1. C.J.Lloyd, Either Drought Or Plenty, 1988,p.90.
%2.  Aqua 1958  vol 9 No 8 p.164 and Glynn 1902 cited in C.J.Lloyd,1988,
%    p.152.
%3. Gwenda Painter, The River Trade, 1979 p.88.
%4. H.L.Hall 1931 Victoria's Part In The Australian Federation Movement 1849-
%    1900,  p.11.
%5. SAPD 19/7/1887 p.259.
%6. H.G.McKinney, J.Roy Soc NSW,190.
%7. Review of Aust.Water Resources 1975 or 24922 cu ft/sec.
%8. 2nd Rpt. RC Water Conservn NSW, 1886.
%9. SAPP 59,59a, 59b of 1886.
%10. SAPP 34 of 1890.
%11. C.J.Lloyd 1988, pp.157-8.
%12.  S.D.Clark \& I.A.Renard in Frith \& Sawer(eds), The Murray Waters,
%       1974,   p.266.
%13. S.D.Clark \& I.A.Renard in The Murray Waters, 1974 \&C.J. Lloyd, 1988,
%      pp.159-60.
%14. S.D.Clark \& I.A.Renard in The Murray Waters 1974, p.269.
%15. J.A.LaNauze, Alfred Deakin, 1965, p.630n.
%16. W.J.Gibbs \& J.V.Maher, CBM Bull.No 48, 1967 \& Jenny Keating, The
%      Drought Walked Through 1992, p.73.
%17.  B.Burton ,Flow Gently Past, 1973 p.182; G.J.Evatt ,Public 
%      Administration,1938 p.21.
%18. VicPP No 35 of 1902-03 vol.3.
%19. VicPP no 35 of 1902-1903, \& Aqua 1958 p.165.
%20. Vic R.C. of 1909, C.J.Lloyd,1988 p.183; G.J.Evatt ,1938 p.21; S.D. Clark 
%      \& I.A.Renard, in Frith \& Sawer 1974 p.270.
%21. SAPP No 29 of 1910.
%22. SAPP 116 of 1913; \& Frith \& Sawer, 1974 The Murray Waters, p.289.
%23. M.Williams, The Making Of The South Australian Landscape 1974 p.150 
%       \& Marianne Hammerton , Water South Australia, 1986 p.153.
%24. VicPP No. 7 of 1910, \& J.M.Powell,Watering The Garden State,1989, 
%       p.141.
%25. La Nauze, 1965 p.630note.
%26. M.Roe, Nine Australian Progressives, 1984,  pp.219-229.
%27. Marianne Hammerton , 1986 p.132, Aqua April 1958, pp.163-166,
%       G.J.Evatt 1938, K.E.Johnson in Frith \& Sawer,The Murray
%       Waters,1974.
%28. SAPP 21 of 1913 , Aqua April 1858,pp.163-166, Frith \& Sawer 1974, 
%      p.283.
%29. G.J.Evatt, 1938 p.23.
%30. Marianne Hammerton 1986  p.132 \& p.133 re refusal of upper States  
%      to give greater control to RMC.
%31. T.Jacobs,1990  'The Murray' p.40.
%32. Details of River Murray Waters Agreement are given as schedule to River 
%p      Murray Waters Act of 1915, passed by Commonwealth and by the three
%      States - See  Frith \& Sawer, 1974, p.280, note 18.
%33. C.J.Lloyd , 1988  p.182.
%34. C.J.Lloyd, 1988, p.187.
%35. B.Burton , Flow Gently Past, 1973.
%36. J.H.O.Eaton ,1946, ANZAAS Hdbk for South Australia, p.82.
%37. Frith \& Sawer,The Murray Waters, 1974 p.289.
%38. E.Mead's memo 1915 in   Aqua 1958 p.169 .
%39. C.G.McCoy, 1988, p.22.
%40. Aqua 1958 p.169.
%41. Frith \& Sawer1974, p.289; M.Williams 1974 p.150 and 152 quote.

