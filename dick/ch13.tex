% $Id$
% CHAPTER THIRTEEN
% 3241 words at 2/5/99

\setcounter{endnote}{0}

\chapter{Use of the River Murray for Navigation and Irrigation}

The first interest of the three south-eastern Australian colonies in
the River Murray system concerned the limitation of navigation by
debris accumulated in the streams.  Snagging operations commenced in
the 1850s and were paid for by the three colonies.\fn{C.\,J.~Lloyd,
\textsl{Either Drought or Plenty}, 1988, p.\,90.}  Their mutual
interest was expressed in a resolution of the 1863 inter-colonial
conference in Melbourne:
\begin{quote}
	That \ldots the commerce, population and wealth of Australia
	can be largely increased by rendering navigable and otherwise
	utilising the great rivers of the interior such as the Murray,
	Edward, Murrumbidgee and Darling; and that the obligation of
	carrying into effect the necessary works to accomplish these
	objects devolves primarily upon the respective Governments
	having jurisdiction over such rivers.\fn{Aqua 1958 vol.\,9
	no.\,8 p.\,164, \& Glynn 1902 cited in C.\,J.~Lloyd, 1988,
	p.\,152.}
\end{quote}

Transport by paddle-steamers and barges was at first based on Goolwa,
the river port near the Murray mouth in South Australia.  Supplies for
inland goldmining towns and for pastoral stations along the Murray and
its larger tributaries were shipped from South Australia which duly
received and exported much of the wool from a vast pastoral area.  The
monopoly of river trade held by that colony ended in 1864 when Echuca
became a major river port after its railway connection to Melbourne.
In 1880 the steamers and barges on the river carried about 10 per cent
of the Australian wool clip.\fn{Gwenda Painter, \textsl{The River
Trade}, 1979 p.\,88.}  By 1893, after more ports on the rivers gained
railway connections with Melbourne or Sydney, half of New South Wales
west of longitude148 still remained dependent on river transport from
South Australia and Victoria while the balance was served by rail from
Sydney.\fn{H.\,L.~Hall 1931 \textsl{Victoria's Part in the Australian
Federation Movement 1849--1900}, p.\,11.}

Development of irrigation was seen by South Australians as likely to
further reduce their use of the Murray river system for trade.
Victorians had by 1886 begun irrigation in northern Victoria, based on
supplies from the Murray and tributaries.  When an agreement providing
for irrigation development at Renmark by the Chaffey Brothers was
debated in the South Australian parliament in 1887, Patrick Glynn
opposed the proposal on the grounds that its use of river water would
jeopardise river navigation.\fn{SAPD 19 July 1887 p.\,259.}  By 1900
it was estimated that diversions for irrigation from the Murray river
system were equivalent to an annual flow at the rate of
4942\,cu\,ft/sec.\fn{H\,.G.~McKinney, J.\,Roy.\ Soc.\ NSW, 190.} This
represents approximately 20 per cent of the estimated yield of the
Murray.\fn{\textsl{Review of Aust.\ Water Resources} 1975.}

\section*{Colonial Negotiations on Irrigation and Navigation
affecting Federation}

A possible development of irrigation in New South Wales, considered by
its 1884 Royal Commission on Water Conservation, led to the proposed
agreement framed in 1886 with its counterpart in Victoria whereby the
two colonies would share the waters of the River Murray.\fn{2nd Rept
RC Water Conservn NSW, 1886.}  This became known to the South
Australian government after James Service, the Victorian Premier,
suggested in July 1885 to the South Australian Premier, J.\,W.~Downer,
that a joint Royal Commission should be appointed to consider
navigation and utilisation of River Murray waters for irrigation.
South Australian indignation at the proposed agreement finally brought
a conciliatory response from the other colonies and Downer in 1886
expressed continued interest in a joint Royal Commission.\fn{SAPP
59,\,59a,\,59b of 1886.}  Thus when the SA Royal Commission on Murray
River Waters Utilisation was appointed in 1887, one of its tasks was
`to confer and consult with similar commissions in New South Wales and
Victoria.\fn{SAPP 34 of 1890.}  This body was dissolved in 1894,
having failed in its purpose, mainly due to the attitude taken by Sir
Henry Parkes, who became Premier of New South Wales early in 1887.  He
promptly dissolved the Royal Commission on Water Conservation,
emphasised his State's ownership of the Murray, and even threatened
the Chaffey Bros, the Mildura irrigation promoters, with trespass for
pumping from the river.\fn{C.\,J.~Lloyd 1988, pp.\,157--8.}

The Victorian position regarding use of the Murray was similar to that
of South Australia in respect to navigation, though assertive about
its interest in irrigation dependent on the Murray and its Victorian
tributaries, and defensive towards the New South Wales government
which was then affirming its claim on the river upstream from the
South Australian boundary.  The Victorian government felt relatively
secure with plans for irrigation from Victorian tributaries of the
Murray even though South Australian claims concerning navigation
involved all parts of the river system and could involve an appeal on
the issue from the South Australian parliament to the Imperial
government.\fn{S.\,D.~Clark \& I.\,A.~Renard in Frith \& Sawer (eds),
\textsl{The Murray Waters}, 1974, p.\,266.}  These conflicts between
the three colonies were among the many complications in negotiations
leading to their Federation.  These matters involved prolonged
consideration at the Second National Australasian Convention of
1897--98 before the framing of sections of the Australian Constitution
dealing with irrigation and navigation and means of settling any
future conflicts.\fn{S.\,D.~Clark \& I.\,A.~Renard in \textsl{The
Murray Waters}, 1974 \& C.\,J.~Lloyd, 1988, pp.\,159--60.} South
Australian insistence on a right to navigation on the Murray gained
some recognition in Section 98 of the proposed constitution agreed to
in 1898 and subsequently adopted. This section reads:
\begin{quote}
	The power of the Parliament to make laws with respect to trade
	and commerce extends to navigation and shipping, and to
	railways the property of any State.fn{S.\,D.~Clark \&
	I.\,A.~Renard in \textsl{The Murray Waters} 1974, p.\,269.}
\end{quote}

The two upstream colonies concerned more with promotion of irrigation
saw that object safeguarded by Section 100, which states:
\begin{quote}
	The Commonwealth shall not, by any law or regulation of trade
	or commerce, abridge the right of a State or of the residents
	therein to the reasonable use of the waters of rivers for
	conservation or irrigation.
\end{quote}

The Constitution makes no provision for regulation of the Murray for
irrigation or navigation, but its sections 101, 102, 103 and 104
relating to an Inter-State Commission might have provided for
consideration of disputes concerning use of the Murray waters.
Section 101 states:
\begin{quote}
	There shall be an Inter-State Commission with such powers of
	adjudication and administration as the Parliament deems
	necessary for the execution and maintenance, within the
	Commonwealth, of the provisions of this Constitution relating
	to trade and commerce, and of all law thereunder.
\end{quote}
Federation had no immediate effect on use of the Murray River system
as the first Commonwealth government took no steps to legislate on
riparian navigation and contemporary moves to set up an Inter-State
Commission under the constitution were rejected.\fn{J.\,A.~LaNauze,
\textsl{Alfred Deakin}, 1965, p.\,630.}

\section*{Response to Serious Drought in 1902}

Severe drought experienced for several years in the three neighbouring
States reached a climax late in 1902 when flow of the Murray at
Mildura declined to 6 per cent of normal.\fn{W.\,J.~Gibbs \&
J.\,V.~Maher, CBM Bull.\ no.\,48, 1967, \& Jenny Keating, \textsl{The
Drought Walked Through} 1992, p.\,73.}  It led to a renewal of calls
from New South Wales for irrigation in the Riverina.  Schemes had been
proposed over several years for diversion of Murray water to irrigate
land there.  Ultimately the Murray River Main Canal League promoted
the Corowa conference in April 1902, attended by the Prime Minister,
the Premiers of three States, and other prominent people.  One
important result was the announcement that a Royal Commission
representing three States would investigate and report on the Murray
waters:\fn{B.~Burton, \textsl{Flow Gently Past}, 1973 p.\,182, \&
G.\,J.~Evatt,
\textsl{Public Administration}, 1938 p.\,21.}
\begin{quote}
	The Interstate Royal Commission on the River Murray was
	appointed without delay to inquire and report on the
	conservation and distribution of the waters of the Murray and
	its tributaries for the purposes of irrigation, navigation and
	water supply.
\end{quote}

Its three members were Joseph Davis, civil engineer and secretary of
the New South Wales public works department, Stuart Murray, engineer
and secretary of the Victorian water supply department, and Frederick
Burchell, engineer of the South Australian government.  In May 1902,
the Commission began a series of interviews aboard a steamer moving
upstream on the Murray and completed its report in December of that
year.\fn{VicPP no.\,35 of 1902--3 vol.\,3.}

The Commission recommended a storage on the upper Murray (Cumberoona),
another on Lake Victoria at the extreme south-west of New South Wales,
a series of weirs and locks on the Murray from Blanchetown to Echuca,
and reservation of 1\,380\,000\,acre-foot (1.7 million\,ML) to South
Australia in a normal years after the river was locked.\fn{VicPP
no.\,35 of 1902--3, \& Aqua 1958 p.\,165.}  It supported the use of
the river for navigation but acceptance of the view that irrigation
should be the major gift of the Murray caused the South Australian
member to give a dissenting opinion.  Its recommendations formed the
basis for unsuccessful efforts over several years for agreement by the
three States involved but the work of the commission had many
long-term effects, including attention to pollution of the river by
sewage and mining wastes.\fn{Vic RC of 1909, C.\,J.~Lloyd, 1988
p.\,183; G.\,J.~Evatt, 1938 p.\,21; S.\,D.~Clark \& I.\,A.~Renard, in
Frith \& Sawer 1974 p.\,270.}

\section*{Renewed Interest in Navigation and Irrigation}

After several years there were new developments. One was the
initiative of the South Australian government which in 1909 sought
advice from the Victorian engineer Stuart Murray on utilization of the
Murray waters for navigation and irrigation and the development of the
resources of the river valley.  His report considered the benefits
resulting from storage at Lake Victoria in New South Wales, exclusion
of sea water from the terminal lakes, and possible reclamation of lake
beds.\fn{SAPP no.\,29 of 1910.}  A new government then decided to act
alone by arranging for construction of a weir and lock for navigation
on the Murray River in South Australia and of storage at Lake
Victoria, both at the expense of South Australia, under the South
Australian Murray Works Construction Act of 1910.  Captain
E.\,N.~Johnston of the USA was engaged to advise on river control by
locks.  His report in 1912 led to the start of work on the Blanchetown
weir and lock in 1913.\fn{SAPP 116 of 1913, \& Frith \& Sawer, 1974
\textsl{The Murray Waters}, p.\,289.}  At the same time the government
made the first move in its plan to install barrages to protect the
terminal Lakes Alexandrina and Albert from entry of sea water when the
Murray flow was insufficient---as during droughts.  The Mundoo
channel, one of the five allowing flow from sea to river, or vice
versa, was closed by a temporary dam about 1913.\fn{M.~Williams,
\textsl{The Making of the South Australian Landscape} 1974 p.\,150
\& Marianne Hammerton, \textsl{Water South Australia}, 1986 p.\,153.}

In Victoria a Royal Commission was set up in 1909 by the Victorian
government to consider the use of the Murray River and tributaries by
the three south-eastern States.  Its wide-ranging inquiry involved
meetings in three States, review of earlier technical
investigations,and proposals for storage, a system of locks,
allocation of flow to South Australia, and a Murray River Inter-State
Board for control irrigation and navigation.\fn{VicPP no.\,7 of 1910,
\& J.\,M.~Powell, \textsl{Watering the Garden State}, 1989, p.\,141.}

The possible involvement of the Commonwealth Government in the
long-standing problem of reconciling navigation and irrigation could
have followed the renewal in 1909 of moves to establish the
Inter-State Commission provided for under the Commonwealth
Constitution.\fn{LaNauze, 1965 p.\,630 note.}  The reference in
Section 100 of the Constitution to `reasonable use of the waters of
rivers for conservation or irrigation' could have been an issue for
consideration by this Commission.  Perhaps coincidentally, the
Inter-State Commission Act was passed by the Commonwealth in 1912 and
the Fisher Government established the Commission in 1913.  When Deakin
declined to become its chief commissioner, that position was filled by
A.\,B.~Piddington, KC; other members being G.~Swinburne, industrial
engineer and former Victorian Minister for Agriculture and Water
Supply, and N.\,C.~Lockyer, a Commonwealth customs
authority.\fn{M.~Roe, \textsl{Nine Australian Progressives}, 1984,
pp.\,219--229.}  At the time this Commission was regarded as a very
important body; its existence and wide powers possibly encouraged the
three States to seek an early compromise on control of the Murray,
thus avoiding intervention by the Commission.

\section*{The River Murray Agreement}

The next significant move in joint action by the States came when the
Melbourne conference of Premiers of New South Wales, Victoria and
South Australia in 1911 agreed that South Australia should construct
the Lake Victoria storage and decided to refer the use of the waters
of the Murray and tributaries to engineers of the three States:
E.\,M.~De Burgh from New South Wales, J.\,S.~Dethridge from Victoria
and G.~Stewart from South Australia.\fn{Marianne Hammerton, 1986
p.\,132, Aqua April 1958, pp.163--6, G.\,J.~Evatt 1938, K.\,E.~Johnson
in Frith \& Sawer, \textsl{The Murray Waters}, 1974.}  Their report in
July 1913 was important in reconciling differences in river gaugings
made by three States and producing an authoritative statement of the
water available in the Murray system.\fn{SAPP 21 of 1913, Aqua April
1858, pp.\,163-6, Frith \& Sawer 1974, p.\,283.}  They could not agree
about the relative importance of irrigation and navigation but they
suggested means for control of the water and emphasised the need for
storages.\fn{G.\,J.~Evatt, 1938 p.\,23.}

At last, during a serious drought, the three State Premiers and the
Prime Minister in September 1914 drew on the Engineers' Report when
they completed the River Murray Waters Agreement leading to a River
Murray Commission to satisfy claims for navigation and irrigation. The
1914 achievement was partly due to the efforts of P.~McM.~Glynn, then
a South Australian Minister in the Federal Cabinet.\fn{Marianne
Hammerton 1986 p.\,132 \& p.\,133 re refusal of upper States to give
greater control to RMC.} The principles of the Agreement were:
\begin{quote}
	\ldots flow at Albury is shared equally between New South
	Wales and Victoria; Victoria and New South Wales retain
	control of their tributaries below Albury; Victoria and New
	South Wales supply South Australia with a guaranteed minimum
	quantity of water or `entitlement'.\fn{T.~Jacobs, 1990
	\textsl{The Murray}, p.\,40.}
\end{quote}

The agreement also provided for construction of a storage on the upper
Murray; another storage at Lake Victoria; 24 locks and weirs below
Echuca; and 9 locks and weirs on the Murrumbidgee or Darling River.
The cost of any works was to be shared equally by the three States,
with a Commonwealth contribution of up to \pounds1\,000\,000. After
ratification of the agreement by the four parties in 1915, the River
Murray Commission was constituted and held its first meeting in
February 1917, its president, Western Australian Senator P.\,J.~Lynch,
represented the Commonwealth and three engineers: H.\,H.~Dare (NSW),
J.\,S.~Dethridge (Vic); and G.~Stewart (SA), represented the
States.\fn{Details of River Murray Waters Agreement are given as
schedule to River Murray Waters Act of 1915, passed by Commonwealth
and by the three States---see Frith \& Sawer, 1974, p.\,280, note 18.}

\section*{Benefits for Irrigation}

\subsection*{Upper Murray Storage}

Diversion of Murray water for irrigation in southern New South Wales
was advocated before the Lyne Royal Commission in the 1880s by
H.\,G.~McKinney, the irrigation engineer.  He favoured a scheme for a
weir on the Murray near Corowa for diversion into an irrigation canal,
and storage on the upper Murray near Talmalmo, upstream from the Mitta
Mitta River.\fn{C.\,J.~Lloyd, 1988 p.\,182.}  Victoria stood to gain
from such a storage, in view of the scheme developed in 1885 for the
two colonies to share the flow of the Murray.  The McKinney project
went into temporary abeyance after the visit in 1896 of Colonel Home
who recommended that development of irrigation in New South Wales
should be based on the Murrumbidgee River, with storage at Burrinjuck,
rather than the Murray River.

The 1902 drought and the Corowa Conference brought renewed attention
to use of the Murray for irrigation by New South Wales and Victoria;
the conference resolved that the New South Wales government should
empower the Commonwealth to provide storage on the Upper Murray as
proposed by McKinney.  However, that government in 1905 was reluctant
to abandon interest in Murrumbidgee irrigation, which would not be
complicated by joint action with Victoria.\fn{C.\,J.~Lloyd, 1988,
p.\,187.}

Nevertheless the upper Murray storage to provide 1~million\,ac\,ft was
later accepted as part of the River Murray Waters agreement. The site
for the storage, which became known as the Hume Weir, was chosen 10
miles above the town of Albury and below the confluence of the Mitta
Mitta River after two years of investigations including examination of
25 sites.\fn{B.\,Burton, \textsl{Flow Gently Past}, 1973.}  Work began
in 1919.  The purpose was to conserve some of the peak flows during
periods of little demand for irrigation so that releases of water
could be made during periods of high demand and poor river flow, as in
droughts.

\subsection*{Lake Victoria Storage}

The storage of 550\,000\,acre-foot at Lake Victoria, involving works
begun in 1915, was intended to maintain sufficient reserves to ensure
adherence to the annual `entitlement' of 1.25~million\,ac\,ft of river
water for South Australia.  This would ensure continuity of navigation
in the South Australian sector, allow development of irrigation on
both sides of the river, and provide sufficient flow to withstand
incursions of the sea into Lakes Alexandrina and Albert.  The
allotment of water for South Australia provided by the agreement was
sufficient for considerable development of irrigation in that State;
it was estimated as enough for 150\,000\,ac.\fn{J.\,H.\,O.~Eaton,
1946, ANZAAS Hdbk for South Australia, p.\,82.}

\subsection*{Weirs and Locks}

The agreement to provide 26 weirs and locks on the Murray between
Blanche\-town and Echuca satisfied South Australian interest in the
river trade; it also had important advantages for irrigation.\fn{Frith
\& Sawer, \textsl{The Murray Waters}, 1974 p.\,289.}  One of these
concerned Victoria which needed a weir on the Murray at Torrumbarry to
ensure filling of the Kow Swamp each year, thus helping irrigation
development near Cohuna and Kerang.\fn{E.~Mead's memo 1915 in Aqua
1958 p.\,169.} Following the Murray Waters Agreement of 1915, the
Torrumbarry weir, representing Lock No.\,26 of those between
Blanchetown and Echuca, was begun in 1919.\fn{C.\,G.~McCoy, 1988,
p.\,22.} It was the first to be installed upstream from South
Australia and the first movable weir on the river.  According to
Elwood Mead `the construction of these locks will be mainly to save
for irrigation the water that would be lost by maintaining navigation
on an unlocked stream'.\fn{Aqua 1958 p.\,169.}  In South Australia the
locks had definite advantages for irrigation, being located at such
intervals that besides ensuring navigation they provided pools to
supply irrigation on nearby banks of the Murray.\fn{Frith \& Sawer
1974, p.\,289; M.~Williams 1974 pp.\,150 and 152 quote.}

\section*{Conclusion}

The conflicts between New South Wales, Victoria, and South Australia
concerning use of the River Murray waters were partly resolved by the
adoption of certain clauses in the Australian Constitution, followed
by decisions on the regulation of river flow by storage and weirs.
The River Murray waters agreement allowed the maintenance of
navigation sought by South Australia and supply of water for
irrigation in the three States.  One of its effects was to encourage
further expansion of irrigation in the Murray Valley.

Although some concern was expressed by authorities at pollution of the
river by mining and sewage, there was by 1920 nothing to indicate
pollution by saline drainage from irrigated areas along the Murray.

%\section*{References}
%1. C.J.Lloyd, Either Drought Or Plenty, 1988,p.90.
%2.  Aqua 1958  vol 9 No 8 p.164 and Glynn 1902 cited in C.J.Lloyd,1988,
%    p.152.
%3. Gwenda Painter, The River Trade, 1979 p.88.
%4. H.L.Hall 1931 Victoria's Part In The Australian Federation Movement 1849-
%    1900,  p.11.
%5. SAPD 19/7/1887 p.259.
%6. H.G.McKinney, J.Roy Soc NSW,190.
%7. Review of Aust.Water Resources 1975 or 24922 cu ft/sec.
%8. 2nd Rpt. RC Water Conservn NSW, 1886.
%9. SAPP 59,59a, 59b of 1886.
%10. SAPP 34 of 1890.
%11. C.J.Lloyd 1988, pp.157-8.
%12.  S.D.Clark \& I.A.Renard in Frith \& Sawer(eds), The Murray Waters,
%       1974,   p.266.
%13. S.D.Clark \& I.A.Renard in The Murray Waters, 1974 \&C.J. Lloyd, 1988,
%      pp.159-60.
%14. S.D.Clark \& I.A.Renard in The Murray Waters 1974, p.269.
%15. J.A.LaNauze, Alfred Deakin, 1965, p.630n.
%16. W.J.Gibbs \& J.V.Maher, CBM Bull.No 48, 1967 \& Jenny Keating, The
%      Drought Walked Through 1992, p.73.
%17.  B.Burton ,Flow Gently Past, 1973 p.182; G.J.Evatt ,Public 
%      Administration,1938 p.21.
%18. VicPP No 35 of 1902-03 vol.3.
%19. VicPP no 35 of 1902-1903, \& Aqua 1958 p.165.
%20. Vic R.C. of 1909, C.J.Lloyd,1988 p.183; G.J.Evatt ,1938 p.21; S.D. Clark 
%      \& I.A.Renard, in Frith \& Sawer 1974 p.270.
%21. SAPP No 29 of 1910.
%22. SAPP 116 of 1913; \& Frith \& Sawer, 1974 The Murray Waters, p.289.
%23. M.Williams, The Making Of The South Australian Landscape 1974 p.150 
%       \& Marianne Hammerton , Water South Australia, 1986 p.153.
%24. VicPP No. 7 of 1910, \& J.M.Powell,Watering The Garden State,1989, 
%       p.141.
%25. La Nauze, 1965 p.630note.
%26. M.Roe, Nine Australian Progressives, 1984,  pp.219-229.
%27. Marianne Hammerton , 1986 p.132, Aqua April 1958, pp.163-166,
%       G.J.Evatt 1938, K.E.Johnson in Frith \& Sawer,The Murray
%       Waters,1974.
%28. SAPP 21 of 1913 , Aqua April 1858,pp.163-166, Frith \& Sawer 1974, 
%      p.283.
%29. G.J.Evatt, 1938 p.23.
%30. Marianne Hammerton 1986  p.132 \& p.133 re refusal of upper States  
%      to give greater control to RMC.
%31. T.Jacobs,1990  'The Murray' p.40.
%32. Details of River Murray Waters Agreement are given as schedule to River 
%p      Murray Waters Act of 1915, passed by Commonwealth and by the three
%      States - See  Frith \& Sawer, 1974, p.280, note 18.
%33. C.J.Lloyd , 1988  p.182.
%34. C.J.Lloyd, 1988, p.187.
%35. B.Burton , Flow Gently Past, 1973.
%36. J.H.O.Eaton ,1946, ANZAAS Hdbk for South Australia, p.82.
%37. Frith \& Sawer,The Murray Waters, 1974 p.289.
%38. E.Mead's memo 1915 in   Aqua 1958 p.169 .
%39. C.G.McCoy, 1988, p.22.
%40. Aqua 1958 p.169.
%41. Frith \& Sawer1974, p.289; M.Williams 1974 p.150 and 152 quote.

\theendnotes
