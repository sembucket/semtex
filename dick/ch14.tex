% $Id$
% CHAPTER FOURTEEN
% 9056 words at 3/5/99

\chapter{State Intervention in Irrigation 1900--1920}

Despite the success obtained by many who independently took up
irrigation and the enthusiasm displayed by those who regarded its
development as the much-needed high technology for the countryside,
little progress had been made by the end of the century.  On the
mainland Victoria and New South Wales had almost extinguished riparian
rights, an important prelude to government control of irrigation, but
efforts to foster the practice by establishment of irrigation trusts
or help for speculative promoters of irrigation colonies had little
success.

Direct government involvement with irrigation in Victoria and New
South Wales followed the widespread drought of 1902 and was followed
later in South Australia and Western Australia.  Legislation was
passed in Queensland to provide for state control of water resources
but no government development of irrigation occurred before 1920.  In
Tasmania riparian rights remained in force and governments refused to
become actively involved in irrigation.

\section{Victoria}

As related in a previous chapter, a succession of governments failed
to solve problems of the numerous irrigation trusts which had nominal
control of more than 700,000\,ac, mostly suitable for irrigation, but
never managed to water more than 120,000\,ac in a year.1 Irrigation
trusts generally proved unable to meet their debts by collection of
rates from landholders, many being disinclined to adopt irrigation.

The Water Act of 1905 provided for establishment of a new authority
--- the State Rivers and Water Supply Commission(SRWSC) --- to control
surface water resources of Victoria other than those of the Melbourne
metropolis and use them to provide water for domestic use and
livestock and for irrigation.  The Commission was controlled by three
men appointed by the government; they had power to construct
headworks, levy rates, and make charges for delivery of water.  Though
expected to operate like a commercial business and to be free from
political influence, the Commission was not expected to make a profit;
it depended on allocation of funds by government for capital works and
on the rates paid by irrigators.  As far as irrigation was concerned,
the 1905 legislation had some important features.

State control of water was extended to embrace all streams.  This was
an advance on the Irrigation Act of 1886 which left intact those
riparian rights established previously and thus limited the power of
the government to make reservoirs in the head waters of any Victorian
stream, a matter referred to earlier by Stuart Murray.2 The extent of
riparian rights had been limited by reservations made in 1881 of all
frontages to watercourses adjoining crown land, but those rights
apparently remained in substantial parts of Victoria previously
alienated from the crown.3 When the 1905 Water Bill was debated strong
opposition to its blanket provisions regarding control of stream
waters came from one parliamentarian from the Western District, where
many pastoralists had long-established riparian rights.4 One river
apparently then subject to those rights along parts of its course was
the Glenelg flowing from the Grampians mountains in western Victoria
to the coast.  The diversion of its headwaters northwards to improve
supply for irrigation and water supply in the Wimmera was considered
before 1892 but not undertaken then, possibly because of riparian
rights and expense.5 Existing riparian rights were reduced by the 1905
Act to allow landholders involved to make only limited use of stream
water, generally for stock and domestic use.

After establishment of the SRWSC the maintenance of irrigation was
continued temporarily by the irrigation trusts. In 1906 the area
irrigated in Victoria was reported by the Commission as approximately
104,000\,ac including the Mildura settlement.6 Storage capacity,
required for irrigation and for stock and domestic supply, was then
about 370,000\,acre-foot including 200,000\,acre-foot recently added
on completion of the initial design of Waranga Basin.  Additional
supplies were obtained from the Murray River by pumping and many
permits and licences were issued by the Commission for private
diversion from different streams.  Between 1910 and 1915, the
irrigation annually dependent on these diversions ranged from 14,000ac
to 24,000\,ac.7

The provision of stock and domestic supplies by the SRWSC became a
major undertaking as an extensive system of earthen channels was
developed by 1914 to serve wheat-growing farms in northwestern
Victoria, an area largely devoid of useful streams.8 That system was
for many years largely dependent on supplies from reservoirs filled
from the Grampians, with the consequence that irrigation could not
then be extended further in the nearby Wimmera district.

\subsection{Elwood Mead's Innovations}

Continuity with the previous management of irrigation was ensured by
the appointment of Stuart Murray as first chairman of SRWSC. On his
retirement in 1907, the government appointed Elwood Mead to the
position.  Mead, an irrigation engineer and professor of irrigation in
California, had spent ten years with the US Department of Agriculture
in charge of irrigation investigations.  In Victoria he sought
increased use of irrigation by a new rating system and establishing
new irrigation settlements.

The SRWSC had initially fixed rates for water supply on the basis of
land values but in 1909 the Water Act was amended to provide a scale
of water-rights according to use of the irrigable land, and a
proportionate charge for water to be met by landholders whether or not
it was used for irrigation.  For each irrigation district the rate of
payment would be fixed to cover the maintenance and management of the
works supplying irrigation water, the interest on the cost of such
works and a redemption fund on the cost of such works.  No charges
were to be made in respect to what were designated as `free headworks'
--- their identification in the relevant statute shows that they
included the national works provided by the State before 1905.9 This
system of charging irrigators remained in force for many years.

The new settlements sought by Mead were established on properties
hitherto unirrigated. They were to be subdivided into holdings ranging
in area from 20 to 200\,ac depending on soils, access to markets, and
nature of intended crops.  They would be provided with channels and
works necessary for irrigation of the new farms.  The settlers would
preferably be residents of Victoria but if their response was not
adequate steps should be taken to attract settlers from Europe and
America.10

Mead's proposals came at a time when Victorian governments were trying
to redress the loss of almost 10 per cent of the population during the
depression of the 1890s, mainly to Western Australia, by attracting
migrants to country districts.11 A Closer Settlement Board had been
set up a few years earlier to establish new farms, generally by
subdivision of sizable holdings purchased by the government.  During a
period of about ten years the Board acquired more than 500,000\,ac of
land it regarded as suitable for farming; the SRWSC became responsible
for developing irrigated agriculture on no more than 100,000\,ac.  The
first involvement of the SRWSC in this activity related to the
establishment of new irrigation settlements at Merbein and Nyah, both
to be supplied with water by pumping from the nearby River Murray.

The Merbein settlement, originally known as White Cliffs, was
established on crown land covered by mallee scrub and woodland
alongside the northwest boundary of the concession obtained by Chaffey
Brothers for irrigation development at Mildura.  Plans for subdivision
were made in 1908 and next year the SRWSC offered 150 blocks for
selection in an area of 5000\,ac.12 The Commission intended dairying
to be the main industry, based on lucerne irrigated with water lifted
about 90\,ft from the River Murray.  By 1914 there were 182 blocks
occupied by settlers; the average block size was 32\,ac.13
Horticultural crops were soon established and with the increased price
for dried fruit during the 1914--18 war this commodity became the main
product of the settlement.  Extensions to the irrigated area began in
1917 for settlement of war veterans, including one portion known as
Birdwoodton which occupied part of the concession area granted to the
Chaffey Brothers in the 1880s at Mildura.14

The Nyah settlement compared with Merbein in respect to topography,
vegetation and supply of irrigation water by pumping from the
Murray. It was developed by the Commission in 1909 on an area of
3000\,ac, which was subdivided into 100 horticultural holdings with an
average size of 29\,ac but many blocks were unoccupied in 1914.15 An
additional 45 blocks became available to settlers in 1919 in the areas
known as Vinifera and North Nyah Extension.  The main produce of the
settlement was dried fruit.  Irrigation depended on pumping from the
river to a height of 80\,ft.16

Experiments significant for Australian irrigation of rice though
apparently started without provision of irrigation by the SRWSC were
conducted for several years by the Japanese immigrant Jo~Takasuka near
the Murray river between Swan Hill and Nyah.  In 1908 he obtained
occupation of 200\,ac of land subject to flooding and ultimately
succeeded in 1911 in raising rice from seed sown directly in the
ground.17 Seed that he had raised was apparently used by another
grower at Koyuga, where the crop sown in November 1915 and watered
weekly from a special dam filled from irrigation channels gave a
marketable yield of grain impressive enough for others in the district
to start sowing rice.18

The major implementation of Mead's plans took place in northern
Victoria, where several holdings with little or no previous use for
irrigation were purchased for closer settlement and subdivided to
provide for intensive production on irrigated holdings.  Most were to
be watered by gravitation from the Goulburn River via the Waranga
Basin and the Western Waranga Main Channel, then being extended
towards the Loddon Valley which it reached at Serpentine Creek in
1912.19 The new irrigation settlements stood like islands in a region
largely used for cereal growing and pastures with supplementary
irrigation sought only in dry seasons for a proportion of a holding.
By 1914 more than 800 holdings varying in size from about 30\,ac to
more than 80\,ac had become available in northern Victoria by
subdivision of estates for irrigation of orchards and lucerne near
Shepparton, Kyabram, Tongala, and Rochester.20

Other additions to the area under irrigation were achieved at Cohuna,
also in northern Victoria, relying on water drawn from the Murray
along Gunbower Creek; near Swan Hill using water pumped from the
Murray; and at Werribee in central Victoria where an irrigation
settlement occupying 6000\,ac and supplied with water from the Werribee
River was opened in 1912, following construction of a reservoir at
Pykes Creek in 1911.21

Some of the new irrigation settlements failed to attract a sufficient
number of applicants by 1910, a situation prompting efforts to attract
new settlers and agricultural labourers from the northern hemisphere.
Mead and Hugh McKenzie, Minister of Lands, visited Europe and North
America for six months intending to recruit six thousand immigrants
for the new irrigation farms.22 Their mission visited Italy, Sweden,
Denmark, the United Kingdom and the United States, with the result
that almost 200 migrant settlers took up blocks before 1912.23 Further
efforts were made in 1912 by William Cattanach of the SRWSC to recruit
settlers from the United States and by R.\,B.~Rees, a parliamentarian,
to persuade Welsh settlers in Patagonia to migrate to Victoria.24
Irrigation was also taken up by Spaniards from Catalonia at Echuca and
Cohuna and by the colony of Russian Jews near Shepparton, but in 1914
there were still many closer settlement blocks without occupants.25
While Mead was prominent in the SRWSC the irrigated area in Victoria
increased following the creation of new irrigation settlements and
improvements in the areas taken over from the irrigation trusts.
Nevertheless there was criticism of closer settlement by many settlers
and questions were raised about purchases of land in electorates of
government ministers.  Eventually in 1914 the government ordered a
royal commission on closer settlement, leading to separate reports on
dryland and irrigated settlement.26 The association of the Closer
Settlement Board and SRWSC ended after the Commission in 1916 found
evidence of some earlier purchases of land unsuitable for irrigation.
One instance related to an area of 11,000\,ac in the Cohuna district,
of which 1200\,ac in the locality known as Mead had been so affected
by salinity as to render the land temporarily unfit for production of
any kind.  The problem was investigated by the Victorian Department of
Agriculture in 1913, with the conclusion that hundreds of acres had
been rendered infertile by the accumulation of chlorides and sulphates
mainly in the past two years following the extension of irrigation.27
Drainage was held to offer the only effectual cure for the salt
problem.28 The widespread and persistent drought of 1914 showed that
the success of irrigation depended on the reliability of water
supplies, which were maintained from the Waranga Reservoir thoughout
much of 1914 but failed early in 1915.  One outcome was the decision
in 1915 to increase its capacity and to build a new dam on the upper
Goulburn River.29 In addition, the supply of irrigation water from the
Murray was to become more assured following the River Murray waters
agreement of 1915 and the ensuing work of the River Murray Commission
which from 1919 included construction on the Murray of the Hume Dam
and the Torrumbarry Weir.

The first world war and the departure of Elwood Mead in 1915 led to
some reduction of efforts by the SRWSC until in 1918, following the
Closer Settlement Act, it became responsible for land purchase and
settlement in irrigation areas.30 Additions to some irrigated
settlements were then begun to provide for returned servicemen.  Work
also began on a new settlement known as Red Cliffs, for which
A.\,S.~Kenyon at the end of 1918 proposed the development by the SRWSC
of 10,000\,ac under irrigation near Mildura.31 The scheme depended on
purchasing portion of the Chaffey concession land then held by
creditors of the Chaffeys.  This was finalised to allow survey work to
begin there early in 1920 and deliveries of water pumped from the
river began in 1921.  Red Cliffs later included more than 700
allotments held mainly by discharged soldiers for production of dried
fruit.32

Ex-servicemen were also settled on many of the vacant allotments in
closer settlement areas developed by the SRWSC after Mead's
appointment.  In 1914 there were almost 400 such allotments with a
total area of 19,000\,ac in 10 settlements.33 The outcome of this
post-war settlement is reported by the Victorian Royal Commission on
Soldier Settlement in 1925.34

The SRWSC relied entirely on earthen channels for distribution of
water to irrigators, thus efficiency was lost as a result of seepage
which in some places contributed later to waterlogging with harmful
effects for irrigated crops.  The system of varying water rights for
different classes of irrigable land required some means of measuring
the use of water by each irrigator; after testing different metering
devices the Commission in 1911 adopted one invented by John Dethridge,
a member of the Commission, but it was not in general use before
1920.35

By 1920 the Victorian area under irrigation, including Mildura,
increased to about 370,000\,ac from 104,000 in 1906--07.36 Pastures
and lucerne accounted for most of the irrigation by 1920, with vines,
orchards and vegetables covering less than a quarter of the irrigated
area. Considering that the area irrigated by private licensed
diversionsof water accounted for about 20,000\,ac in 1914 and
17,000\,ac in 1917--18 while Mildura accounted for a smaller area, the
Victorian achievement of irrigation was due mainly to government
intervention.37

\section{New South Wales}

The entry of government into the field of irrigation came after its
decision to establish a major water storage in headwaters of the
Murrumbidgee River as a prelude to delivery of supplies to plains
traversed by the river downstream.

The first step taken towards government intervention was in 1896 when
the Water Rights Act vested water rights in the Crown and instituted a
system of licences for dams and diversions from streams for irrigation
by individual landholders.  The immediate effect of the legislation
was to recognise a practice which had been in existence for some time,
often with considerable friction between water-users.38 Two
landholders who made early use of the provisions of the Act to divert
water for irrigation were Samuel McCaughey in the Riverina, and Thomas
Ellis on the Hunter River in north of the State.39 In the longer term,
however, the Act provided the basis for further intervention by the
state.

Another important step by the government in 1896 was to call on
Colonel Home, an irrigation engineer recently retired from service in
India, to report on prospects for irrigation in New South Wales.  Home
spent 12 months in the colony, concentrating attention on the inland
rivers and finding only the Murray and Murrumbidgee offered potential
for major irrigation development.  He reviewed the earlier proposals
made by McKinney for development from these rivers and favoured use of
the Murrumbidgee because it was entirely within the colony.40 Neither
river could sustain major extension of irrigation without a storage
reservoir; he favoured the Barren Jack site, southwest of Yass, as
most appropriate for storage on the Murrumbidgee.41

The serious drought which culminated in 1902 led to a renewal of
interest in irrigation, expressed in several ways. Passage of the
Water and Drainage Act 1902 provided funds for waterworks, established
the State as a constructing body and authorised establishment of
trusts to administer waterworks.42 Irrigation leagues in the Riverina
sought diversion of Murray water by a canal.  Their activity led to
the notable Corowa Conference in 1902, which attracted support from
beyond New South Wales and led to the establishment by three States of
the Inter-State Royal Commission on the Murray Waters.43 That inquiry
collected much evidence on development of irrigation as well as use of
the river system for navigation.44 Samuel McCaughey's success with
irrigation near the Murrumbidgee River was reported by him to the
Commission and gained wide publicity. After becoming owner of the
North Yanco estate of 100,000\,ac in 1899, he had begun extensive
irrigation, including 750\,ac of lucerne and 250\,ac of sorghum.45

Another important proposal at this time was made by Robert Gibson of
Hay and Hugh McKinney, the former chief irrigation engineer, for
private irrigation of a large tract of land north of the Murrumbidgee
River. This was identifed at one time as the Gunbar scheme, referring
to the destination of a proposed major canal to end 60\,miles
(100\,km) northeast of Hay. Gibson's scheme was to involve irrigation
of 150,000 ac of first-class land and limited (winter) irrigation of
up to 1,250,000\,ac of natural pasture.46 Gibson, a pastoralist,
irrigator and land agent, had options on 170,000\,ac of land in the
region.47He acknowledged financial backing from Victoria and England
and probably also had the promise of support from Samuel McCaughey.48
His special interest in the scheme was that of a land agent with a
speculative interest in the purchases to be made by would-be
irrigators.  Hugh McKinney was to be the irrigation engineer for the
project.  Arrangements were made to have Gibson's proposal submitted
to parliament for its approval.  Accordingly the Murrumbidgee Northern
Water Supply and Irrigation Bill was submitted to parliament in
September 1903 by Mr G.\,S.~Briner~MP.  Progress with the Bill was
impeded by changes to the scheme and a change of government. During
this delay it emerged that since 1901 the Public Works Department had
been planning irrigation in the same region; L.\,A.\,B.~Wade, its
engineer, explained that a major distinction from Gibson's scheme was
the consideration of irrigation both north and south of the river.49

While Gibson's proposal was still before parliament, the Minister for
Public Works, C.\,A.~Lee, convened a conference early in 1905 to
consider irrigation.  It was widely attended; among the delegates was
the engineer John Monash representing the Lower Murrumbidgee district.
The conference agreed that New South Wales should begin a
comprehensive scheme of water conservation and irrigation and left it
open as to whether a private or State scheme should be adopted.50 By
this time L.\,A.\,G.~Wade, the principal engineer of water supply and
sewerage in the public works department, had visited USA to study
irrigation and was preparing plans for the Burrinjuck reservoir on the
Murrumbidgee and the use of its water for irrigation beyond the
highlands.51 Both Gibson's and Wade's schemes for irrigation to the
north of the Murrumbidgee came before a parliamentary committee in
1906, which collected evidence in the district under consideration and
in Victoria at Mildura, Kerang, Tatura and Melbourne.  One point
already decided was that the government had taken responsibility for
building the storage dam.  The many witnesses included Samuel
McCaughey of North Yanco; Robert Gibson of Hay; several government
officers including L.\,A.\,B.~Wade and Joseph Davis; and Victorians
including Stuart Murray, first chairman of SRWSC; George Swinburne, MP
and Minister for Water Supply; John Monash, engineer; and
W.\,B.~Chaffey of Mildura.52 This committee recommended the
departmental scheme with some modifications due to Stuart Murray and
declined that proposed by Robert Gibson.  John Monash appeared for
pastoralists whose livelihood depended on periodic flooding of the
Lower Murrumbidgee below Hay; he voiced their claim that Gibson's
scheme would reduce that inundation. Wade's plans shown to the
committee indicated an area of first-class land north of Mirrool and
Wah Wah Creeks with 196,000\,ac, and 160,870\,ac of second-class land
stretching from Yanco westerly towards Hay. These could be served by
two canals diverging from Yanco: the Gunbar canal to the north and the
Hay canal to the west.53

The government acted promptly on the report of the Standing Committee
and successfully introduced legislation authorising construction of
the Burrinjuck dam and the necessary works for irrigation: the Barren
Jack Dam and Murrumbidgee Canals Construction Act 1906.  The Public
Works Department was to be the constructing authority.54 Work on the
dam commenced in 1908 and the first supply of water for irrigation was
expected after the dam reached a height of 70\,ft.

In 1910 the Murrumbidgee Resumption Act and the Murrumbidgee
Irrigation Act authorised the resumption of 1,668,000\,ac of land for
the scheme and its management by a Trust consisting of the Secretaries
(Ministers) for Public Works and for Lands together with the Minister
for Agriculture.55 This arrangement lasted until the Irrigation Act of
1912 which led to the appointment of a single Commissioner for Water
Conservation and Irrigation, a position awarded to L.\,A.\,B.~Wade.

Resumption of land started in 1911, when 60,000\,ac of McCaughey's
North Yanco estate was purchased.56 By 1913 more than 300,000\,ac had
been acquired to provide two irrigation areas to be named Yanco and
Mirrool close to the western edge of the highlands and served by a
main canal starting at the Berembed weir on the Murrumbidgee and
stretching more than 100 miles to its terminus in the Mirrool
Irrigation Area. The first water for irrigation in the Yanco area was
delivered in July 1912, and to the Mirrool area in October 1913.
Towns designed by the American architect Walter Burley Griffin were
established in 1914: Leeton in the Yanco area named for C.\,A.~Lee,
the Minister for Public Works first associated with the Murrumbidgee
Irrigation Area and Griffith in the Mirrool area for A.~Griffith, the
second Minister.

The Murrumbidgee Irrigation Area (MIA) was divided into holdings
varying in size from 2\,ac for workingmen's blocks, to 10\,ac for
horticulture and 50\,ac for mixed farming.  The largest blocks were
expected to be used for production of lucerne and sorghum.  Variation
in natural vegetation on the project area provided a guide to the
soils which ranged from deep sands to heavy clays.  The better-drained
and more porous soils of land regarded as of first-class quality were
intended for horticulture; they were generally associated with stands
of native pine.

Prospective settlers began selecting blocks when these became
available in 1912.  Many of the first settlers were miners from Broken
Hill, some with occupational pulmonary disorders.  Another group
included the distinctive Welsh settlers from Patagonia where they had
gained experience with irrigation; they had been sought initially by a
Victorian agent but were lured to the MIA.  One Italian, though
originally from Switzerland, was allotted a farm in 1914 and two
Spaniards obtained farms in 1916.57 A few Italians came to work as
labourers before 1920 and several irrigation blocks were allotted to
American migrants.  The well-known Australian writer Henry Lawson
spent a year in the Yanco area.58 Between July 1912 and April 1914 622
farms covering 24,000\,ac were allotted on perpetual lease.  At Leeton
a butter factory was opened in 1913, followed by a cannery in 1914.
Besides dairying on the larger holdings and fruit grown on
horticultural blocks, pigs and fat lambs were raised and production
was also diversified by growing tobacco and broom millet.59

Irrigators in the MIA were required to pay water rates sufficient `to
meet interest on the cost and upkeep of the storage dam and all main
channels for conveyance of water to the main parts of the settlement,
and also the cost of distribution of the water'. A concession provided
that `half water rates are charged for the first year, increasing
until full rates are paid in the sixth year.'60

Within a few years there was widespread dissatisfaction with the
results obtained by settlers.  The problem was most serious for those
on the non-horticultural blocks and arose from the failure to secure
yields of lucerne comparable with those which had been reported
earlier by Samuel McCaughey and provided the basis for farm size.
Inquiries were made in 1915 by A.\,C.~Carmichael, a Royal
Commissioner, and again in 1916 by Judge W.~Bevan, leading to a
revision of farm size.61 Another grievance was the claim that the rent
charged to landholders was based on too high a valuation of the land.

Irrigation in the MIA was needed to supplement the annual rainfall,
about 16\,in. on average, particularly during the long dry summers to
support production of fodder crops and fruit.  Initially it was
expected that the basic supply to allotments would be equivalent to a
further 12\,in. with provision for additional water for particular
crops.62 Irrigation brought problems in 1917 at Yanco when serious
waterlogging occurred and was the beginning of a long-held view among
settlers that it was the consequence of seepage from the earthen
channels.  However, this experience followed exceptionally high
rainfall in 1916 and could be connected in one locality with previous
irrigation over some years by Samuel McCaughey on land later
incorporated in the new irrigation settlement.63 Regulation of water
delivery to allotments was the responsibility of water bailiffs and
although some Dethridge meters were made available in 1911 soon after
their invention in Victoria they were apparently not used before
1920.64

A major development began during World War 1 for settlement of war
veterans in the MIA.  Plans were made for 1,500 blocks to be available
but little more than 800 were occupied in the early 1920s. Experience
with this scheme was disheartening and led later to several official
enquiries.65

The Curlwaa Irrigation Area was established on 10,550\,ac under the
Wentworth Irrigation Act of 1890 and in 1912 it came under the control
of the Water Conservation and Irrigation Commission (WCIC).66 Its
location is on low-level alluvial land between Tuckers Creek and the
Murray River. Much of the land was unsuitable for irrigation and in
1916 only 1356\,ac were held on irrigation leases by 67 settlers.
Their blocks ranged from 5 to 37\,ac and each lessee was entitled to
receive 30\,in. of water per acre annually by pumping from the Murray
River.  Irrigation was then used mainly for fruit growing.

The Hay Irrigation Area had an area of 3840\,ac, with almost 1000\,ac
held under irrigation leases by 81 settlers in 1916. Water was pumped
from the Murrumbidgee River to give each settler a right to 24\,in. of
water annually for each acre incurring the water rate.  Dairying was
the main occupation in 1916.67

By 1920 the area irrigated in New South Wales included 34,000\,ac in
settlements controlled by the WCIC and perhaps 50,000\,ac undertaken
by those licenced by the government to divert water for irrigation
from various streams in the State.  More than 1200 licences were in
force by 1918 for diversions from streams.68 The Murray River was the
major supplier for the private diverters, whose pumps in 1919--20 were
estimated as delivering up to 40,000\,acre-foot in New South Wales.69
Although details of irrigation by private licensees are not shown in
annual reports of the WCIC, the areas irrigated in 1920 from all
streams in the State are reported elsewhere as totalling more than
50,000\,ac, mostly supplied by the Murray, Murrumbidgee, Lachlan,
Hunter, and Namoi.70 Accordingly, irrigation by private diverters was
by 1920 much more widespread than that undertaken by government
enterprise.

\section{South Australia}

Entry by the South Australian government into irrigation development
was a consequence of policy in favour of closer settlement, subject to
the Crown Lands Act 1897 passed during the term of the Kingston
Government when Peter Gillen and Laurence O'Loughlin had terms as
Commissioner of Crown Lands.  Among the areas chosen for closer
settlement were the swamps along the Murray River downstream from
Mannum.  Earlier efforts at their reclamation included the protection
of a large swamp from flooding by an embankment erected in 1881 by Sir
William Jervois who acquired the Wellington pastoral property during
his term as Governor of the colony.  Irrigation was being undertaken
later on swamps after their reclamation by a few landowners including
J.~Cowan at Glen Lossie (700\,ac) and H.\,W.~Morphett at Woods Point
(650\,ac), who along with many others gave evidence in 1902 to the
Interstate Royal Commission on the River Murray aboard a paddle
steamer travelling along the river.71

The successful reclamation and irrigation of swamps on the lower
Murray led to a proposal for similar treatment for the freshwater Lake
Albert near the mouth of the Murray River.  The scheme was urged by
D.\,J.~Gordon in the press and in a government publication with the
claim that it would provide an additional 40,000\,ac of land.72
Reference to the proposal was made in 1910 in the report by Stuart
Murray, the Victorian engineer, to the South Australian government on
the utilization of the waters of the Murray River.73 There is no
indication of further official attention to this scheme before 1920.74
From 1904 the Surveyor-General's office became responsible for
reclamation and subdivision of several swamps near Murray Bridge and
the Department of Agriculture established an experimental farm on the
Mobilong swamp in 1906 with Samuel McIntosh as manager. He later
became government nominee and chairman of boards of management
controlling these swamps. Their reclamation relied on provision of
levee banks extending along the river between high land at each end of
the bow-shaped swamp. These banks were made high enough to give
protection against floods, so to cope with those of the magnitude
known in 1870, a height of almost 8\,ft was necessary at Murray Bridge
instead of the lower embankment used by Sir William Jervois.75
Material for the banks was taken from the adjacent highland and from
the swamps, whose highly organic or clayey soils contracted on drying
and enabled river water to leak into the swamps.  This defect was
corrected when embankments were properly cored.  Sluice gates
installed in the levee banks could be used to assist drainage during
periods of low river level or for irrigation when the river rose
adequately. The Surveyor-General's report for 1909--10 refers to the
difficulty of reclamation on certain swamps where the water let in by
settlers who opened sluice gates took days to remove by pumping.  The
main use of the swamps reclaimed for closer settlement was dairying.
Over the period 1904 to 1909 the government undertook reclamation of
the swamps known as Mobilong, Burdett, Long Flat, Monteith and
Mypolonga.  By 1920 their combined area of 3800\,ac provided 113
holdings varying in size up to 50\,ac.

The Irrigation and Reclamation Lands Act of 1908 enabled the
government to begin extension of irrigation along the Murray upstream
from Morgan.  The legislation provided for establishment of government
irrigation areas containing individual allotments under perpetual
leasehold tenure, with 50\,ac as the maximum irrigable area to be held
by one person.76 The government proceeded to use its powers in two
ways.  Irrigation still being carried on at remnants of village
settlements became the basis of irrigation areas under its control.
The Waikerie Irrigation Area was thus established in 1909 with 66
holdings including 42 new ones on adjacent highland where water supply
channels were made and supplied by improved pumping facilities.77
Similarly the Kingston and Moorook Irrigation Areas provided for
irrigators once involved in village settlements at these localities;
and the Cadell Irrigation Area (1916) was established on land briefly
occupied in the 1890s by the New Era village settlement.78

The other path taken by the government was to irrigate virgin land
near the river.  Extensive areas of lowland then held under pastoral
leasehold were available north of the river between Renmark and
Kingston.  One of these near Lake Bonney had been under consideration
from the 1880s for private or government irrigation.  A start was
made, however, at a river-bank locality now known as Berri which had
once been considered for a village settlement.  By 1908 when the new
enterprise began, a moderate area of lowland flood plain accommodated
a wood-lot to supply steamboats together with irrigated production of
fruit and vegetables for the passing traffic.  This segment of the
flood plain is backed by sandy slopes rising to an undulating plain
then covered by mallee vegetation.  Attention was first given to the
lower ground; it compared with land irrigated successfully at Renmark
and some village settlements by means of centrifugal pumps suitable
for the low lifts involved.  The survey work led to a start in 1909 on
the construction of irrigation channels.  Pumping of water began in
1910 when the Berri Irrigation Area was proclaimed.79 The irrigated
area spread northwest and west from Berri township before and during
the 1914-18 war.80 Preparations were made for the irrigation of
further allotments at a distance from the river; these were reserved
for returned servicemen.  The provision of additional irrigation
blocks involved clearing and subdivision of the higher country covered
by mallee vegetation before construction of water channels.  One
complication was the need to exclude certain tracts not irrigable by
gravitation from open channels at the highest levels commensurate with
pumping equipment to be installed; another was the requirement that
subdivision of the commandable area should provide allotments
equivalent in value rather than size.  This latter consideration was
important in view of the variations represented by the pine-clad sandy
slopes facing the river and by the pattern of east-west sandy ridges
alternating with loamy swales.

Another irrigation settlement started by the South Australian
government involved a large tract of land west of Berri used by John
Chambers as part of his vast Cobdogla pastoral run; it included
extensive flood plains along the river backed by higher land with
mallee scrub.  Surveyors moved to this area in 1911 after finishing
their work at Berri; they were followed by gangs clearing the land and
making channels.  Topographic surveys showed that tens of thousands of
acres, mainly in the scrub-covered highland rising more than 100 ft
above the river, were available for irrigation.  The first effort was
to make use of the flood plain between the river and Lake Bonney; it
could be supplied by a pump lift of 30 to 40\,ft and earthen
channels. The intention was to raise fodder crops and lucerne there as
the basis for dairying.  The Cobdogla Irrigation Area was proclaimed
in 1916 and land in this lower part, known as the Cobdogla Division,
became available in 1918 for allotment to settlers.81 Next year the
Cobdogla township was proclaimed.

Attention then turned to development of neighbouring areas for
occupation later by ex-servicemen.  One portion known as the Nookamka
Division was provided with a town known firstly as Lake Bonney South
and later as Barmera.  Another formed the Loveday Division in which
the Loveday township appeared later.  There was ample scope for
extension of irrigation to more than 10,000\,ac in the Weigall
Division at the south and the McIntosh Division lying north of Lake
Bonney.  The good returns available in the war period for dried fruit
supported optimism for expansion of irrigation in the Cobdogla
Irrigation Area for vines and fruit.

The Verran government installed in June 1910 took a very positive
attitude to use of the Murray River for both navigation and
irrigation.  Crawford Vaughan, the new Commissioner of Crown Lands
gave attention to irrigation by creating the Irrigation and
Reclamation Works department with Samuel McIntosh as director and
staff drawn from the Surveyor-General's office.  McIntosh then had
almost twenty years of experience along the river, including work at
Renmark, advising the village settlements, managing an irrigation
company near Waikerie, and helping closer settlement of the reclaimed
swamps near Murray Bridge.

McIntosh had returned from an overseas tour in 1911 with two important
objectives realised eventually at Cobdogla.  The volume of water
expected to be required there for irrigation warranted the
installation of a powerful and economic pumping plant; the best
available at the time appeared to be the distinctive Humphrey Internal
Combustion Pump.  An order was placed for two units but delivery was
frustrated by the world war and installation was not made before 1920.
His other idea was to follow the example of some Californian
irrigation colonies and provide a pressurised piped supply for
irrigation.  McIntosh learnt of the recent invention in South
Australia by W.\,R.~Hume of concrete pipes manufactured by recourse to
a centrifugal method; he found that these pipes had been installed in
1911 for irrigation on the riverside property known as Murray Heights
near Ramco.  The Hume pipes were used by the government only in the
Loveday Division, but not before 1920.

Besides the development of the Berri and Cobdogla areas, the
government decided to extend settlement at Renmark by the development
of its Block~E and creation under government control of the new
Chaffey Irrigation Area having the single Ral Ral Division.

The settlers at the various irrigation settlements were required to
pay annual rates for the use of irrigation water.  The rate in each
irrigation settlement was fixed by reference to the unimproved value
of the land and included a proportion of interest on the cost of the
works and of their maintenance and management.82 The South Australian
settlements on the Murray involved pumping from the river but had no
requirement for storage of water.

Although other irrigation schemes were considered for action by the
government, the only one not dependent on supplies from the Murray
River was the Pekina scheme in the southern Flinders Ranges, not far
from the township of Orroroo.  In this district there was interest in
establishing a reservoir in the hills to hold water drained by Pekina
Creek so as to irrigate land on a nearby plain where the average
annual rainfall is 13\,in.  The scheme involved an embankment 70 ft
high for a reservoir with a capacity of 220,000,000\,gal
(800\,acre-foot) to be delivered through more than 7 miles of
underground main to the irrigation area of 429\,ac divided into 45
blocks, each with an annual allowance of 12\,in of water and equipped
with a water meter.  The technical development of irrigation was
undertaken by the engineering staff of the Public Works Department,
and an experimental farm undertaken by the Ministry of Agriculture was
managed by Samuel McIntosh.  Construction began in 1908 and was
completed in 1910, when lucerne and fodder crops were sown with the
object of helping the dairying industry.83 The Pekina scheme was
developed under the Water Conservation Act of 1886 by the Public Works
Dept and was later leased to the municipal authority of the
district.84

A feature of irrigation development in South Australia before 1920 was
the responsibility of the Commissioner for Crown Lands.  Until 1910
the Surveyor-General's office provided survey and engineering staff
for the reclamation and irrigation works along the Murray river.  When
the Irrigation and Reclamation Department was set up in 1910, it
required detachment of staff from the Surveyor-General's department,
with both departments responsible to the Commissioner of Crown Lands.
G.\,W.~Goyder, the Surveyor-General for 25 years up to his retirement
in 1894, had been involved with many aspects of land utilisation in
South Australia, including areas adjacent to the Murray River.  Long
after his departure, the Surveyor-General's office continued to have
responsibility for irrigation development to an extent much greater
than in other States.

Irrigation in South Australia relied on government support for
settlements to be provided with water from the River Murray.  Private
diversions from that stream probably began after the collapse of
several village settlements of the 1890s, and by 1912 several farms
were irrigating orchards, vineyards or fodder crops along the river.85
However even when private diversions were regularised by the passage
of the Control of Waters Act in 1919, there were less than 1000\,ac so
irrigated in 1921.86 Private irrigation using groundwater was a minor
activity on the Adelaide Plains before 1920.

\section{Western Australia}

The earliest irrigation undertaken by the government was the
installation in 1908 of an irrigation system to grow lucerne on the
Brunswick State Farm near Bunbury.87 The next development was prompted
by the need for irrigation of citrus groves near the Harvey River
south of Perth. A scheme for this area involving construction of a dam
on the Harvey River was proposed in 1912 when the government first
attempted to gain legislation known as the Rights in Water and
Irrigation Act 1914.88 Work started in 1915 to provide storage of 520
million gallons for irrigation of up to 3,000\,ac at the foot of the
Darling Range.  The Harvey Irrigation District which opened in 1916,
after work by the Public Works Department and the Water Supply
Department, was the sole entry by the Western Australian government
for irrigation development before 1920.  Hugh Oldham, who had worked
earlier at Renmark and Mildura, was an engineer of the Water Supply
Department who referred to the Harvey scheme in 1913 and in 1918 to
the Royal Commission on Agricultural Industries of the South-west.89

\section{Queensland}

Irrigation of sugar cane by private companies continued to be a
feature of agriculture in Queensland following its use on the Burdekin
River delta at Ayr.  Efforts to introduce irrigation to the Bundaberg
district began in 1889 when John Cran of the Millaquin refinery
negotiated with the Chaffey Brothers for irrigation of land at
Woongarra.90 That scheme was abandoned but success attended other
moves to irrigate sugar cane in the district after the arrival in 1900
of Dr~Maxwell, the recently appointed director of the Sugar Experiment
Stations who had experience of irrigation in the Hawaiian sugar
plantations.  The Bingera plantation of Gibson and Howes began to use
irrigation in 1903, with water pumped from the Burnett River, and in
the same year Young Brothers were supplying their Fairymead plantation
with underground water in the same way as on the Burdekin delta.91 A
revival of interest in irrigation of sugar cane in the Woongarra area
occurred in 1903 with an unsuccessful proposal to pump water from the
Elliott River to 3000\,ac.92

Queensland governments gave attention to the underground water
resources both of the inland regions and in coastal areas where they
might be exploited for sugar cane irrigation, but no direct government
involvement in irrigation settlements was undertaken before 1920.
However, after a visit by Elwood Mead to Queensland in 1909, the
government followed the recommendations in his report for legislation
known as the Rights in Water and Water Conservation and Utilisation
Act of 1910, which provided state control of water resources.93

By 1920 the total area reported as irrigated in Queensland rose to
more than 10,000\,ac, mostly for production of sugar cane in the Ayr
and Bundaberg districts.  In the Bowen, Cunnamulla, Hungerford,
Rockhampton and Townsville districts irrigation involved only a few
hundred acres.94 No irrigation works were constructed by the
government before 1920.

\section{Tasmania}

The main concern of Tasmanian governments with irrigation during the
early years of the new century continued to be with the use of waters
held in the adjacent Lakes Sorell and Crescent.  Earlier efforts by
the government to divert sufficient water from these lakes to irrigate
land in the Midlands had failed, leaving the Clyde Water Trust as the
sole beneficiary of water in Lake Crescent --- the source of the Clyde
River.  Then several outbreaks of typhoid fever along that river
during the early 1890s led to criticism of the Trust's control of
water and to renewed consideration of sharing lake water between
landholders in the Clyde valley and those in the Tunbridge district of
the Midlands.95

The government then legislated to recast the Clyde Water Trust, by the
Clyde Water Act of 1898, and to form the Midland Water Trust, under
the Midland Water Act of 1898.  The latter trust would control water
supply, in excess of the Clyde Trust's quota, for irrigation and other
purposes in the Midlands.  When the Midland Trust agreed in 1901 to
supply water to a company for generation of electricity to manufacture
calcium carbide, the Minister for Land and Works intervened and began
negotiations with another company for a hydro-electric scheme based on
use of the lake waters.  One consequence was further legislation ---
the Lakes Sorell and Crescent Conservation Act 1901 --- which
safeguarded the daily entitlement of water for the Clyde valley, made
the delivery of water to the Midlands conditional on an adequate
supply, and gave the Minister authority to carry out conservation
works at the Lakes.96

Subsequent gaugings of rainfall and lake discharge indicated an
inadequate supply of water during dry years, with the result that
proposed conservation works were abandoned, irrigation development by
the Midlands Water Trust was frustrated, and the Clyde Water Trust
continued to be an important irrigation authority.97

Hydro-electric power generation began before 1900 and was undertaken
mainly by mining companies. A major project based on the Great Lake
was begun in 1911 by a company requiring power for processing zinc
ore, and its first operation in 1916 came after sale of the company to
the government and creation of its hydro-electric department directed
by J.\,H.~Butters, formerly chief engineer of the company.
Coincidentally, low rainfall over several of these years led to a
revival of interest in irrigation and proposals for its integration
with hydro-electic power generation.  One example was the scheme
suggested by W.\,E.~Shoobridge for use of waters held in Arthurs Lakes
in the central highlands to generate power and provide irrigation in
the Lake River valley.98 As with earlier proposals concerning Lakes
Sorell and Crescent, Shoobridge's suggestion failed because of
uncertainty over rainfall and lake resources, as indicated in reports
by visiting irrigation engineers.99 In 1919 he called unsuccessfully
for legislation giving full control of Tasmanian water resources to
the State.  By 1920 the hydro-electric department was heavily involved
in collecting data on water resources and their use for power
generation, while public control of irrigation was limited to the
Clyde Water Trust and the Macquarie Water Trust set up in 1892 for
management of Tooms Lake.

\section{Conclusions}

Promotion of irrigation by governments became decisive in Victoria,
New South Wales, and South Australia by 1910, a few years earlier than
similar development in Western Australia.  Progress was made in
Queensland without state control, while in Tasmania state involvement
with water resources concerned only hydro-electric power generation.

Commercial failure of the irrigation settlements launched by the
Chaffeys, the fiasco produced by the Victorian irrigation trusts, and
the experience of widespread drought during 1902 led Victorian
governments to accept greater responsibility for irrigation.  The
first step was the creation of the SRWSC, the second was the
appointment in 1907 of Elwood Mead as its chairman, a position he used
to establish several new irrigation settlements.

Special government authorities to deal with irrigation were
established later in New South Wales and South Australia to launch and
control major irrigation settlements dependent on supplies from the
River Murray or its tributaries.  Engineers and surveyors were the
mainstay of this work which included the provision of irrigation farms
for returned servicemen after World War~I.  Numerous agricultural and
horticultural problems confronting settlers new to irrigation soon
taxed the resources of the government departments of agriculture.
		
\section{References}

1. J.M.Powell, Watering The Garden State,1989, p.121.

2. S.Murray 1889, quoted by L.R.East in Aqua vol.9, 1958, p.248.

3.  J.M.Powell 1989, Fig.35,p.148.

4.  J.M.Powell 1989 p.150 re W.S.Manifold \& VicPD vol.110
      30/8/1905, p.1247-8.

5.  S.Murray, Irrigation In Victoria, A Report for the Hon. Geo. Graham, 1892, 
      L.R.East Vic Hist. Mag. 1967 p.182 and A.Deakin, Yearbook Of Australia, 
      1892, p.9

6.  C.G.McCoy, The Supply Of Water For Irrigation In Victoria From 1881
       to 1981, 1988 p.54 \& Ann.Rept SRWSC 1906-07.

7.  VicPP Annual Rept SRWSC 1914-15, also  E.Mead 1914 , BAAS Hdbk
      Vic, p.266. 

8.  L.R.East, Vic.Hist.Mag, 1967 vol.38, p.205, \& R.F.McNab, 
     The Early Settlement And Water Supply Of The Wimmera And 
     Mallee, 1944.

9.  Vic Water Act 1905,1909, and 1915.

10.  Annual rept. SRWSC 1908, \& J.Rutherford in J.M.Powell(ed.) 
       The Making Of Rural Australia, 1974,p.126.

11. S.Whitehead, BAAS Hdbk Vic1914, p.95.

12. CSIR Bull. No.123, Merbein Rept.

13. E.Mead , BAAS Hdbk Vic 1914 p.265

14. S.Wells, Paddle Steamers To Cornucopia 1986 p.180; E.Mead BAAS 
      Hdbk Vic1914; A.J.McIntyre, Sunraysia, 1948, \& CSIR Bull. 123, 1939, 
      Merbein  Rept .

15. E.Mead 1914.

16. CSIR Bull. No 73, 1933.

17. D.C.S.Sissons in ADB vol.12,p.163,  Lesley Scholes, A History Of
      The Shire Of Swan Hill, 1989, p.117, \& The Australian Encyclopedia,
      1927, article on rice.

18. T.Smith, 1916, J.Agric Vic. vol.14 p.493. 

19. C.G.McCoy 1988 p.16.

20. E.Mead 1914 p.265.

21. Aqua Jan 69, K.N.James, Werribee, 1985 p.86, 90, 97.

22. R.Broome, Arriving, 1984, p.131.

23. R.Broome, 1984,  p.133

24. Lesley Scholes 1989 p.116.

25. Helen Coulson, Echuca-Moama On The Murray, 1995, \& J.Trevaks,
      Shepparton, A Successful Experiment In Settling Jews On The Land
      In Australia, Australian Jewish almanac, 1937, \& E.Mead 1914 p.265.

26. Lesley Scholes, 1989 p.124.

27. Melbourne Argus, 20/3/1913, p.15.

28. VicPP R.C.Closer Settlement 1916, final rept and Argus 20/3/1913.

29. Jenny Keating, The Drought Walked Through, 1992, p.99, \& H.W.Forster,
     Waranga, 1965  p.97.

30. C.G.McCoy,1988, p.14, \& A.J. Tisdall, Irrigation In Victoria, in
       Australian Academy of Science,  Water Resources Use And Management,
       1964.   

31. J.N.Churchyard, Aqua 1960 v.11(12), pp.259-268.

32. J.N.Churchyard Aqua 1960 \& CSIR Bull. 137.

33. E.Mead 1914.

34. VicPP No 32 of 1925, \& Lesley Scholes, 1989.

35. I.Meacham  Aqua , 1961,p.184.

36. C.G.McCoy,1988,  p.54-55.

37. E.Mead 1914 \&Ann.Rept.SRWSC 1918.

38. G.J.Evatt , Public Administration 1938, vol 1(2), pp.16-31.

39. C.J.Lloyd, Either Drought Or Plenty, 1988, p.183.

40. NSW Rept R.C.Water Conserv.1885-87 \& NSW LA V\&P 1897,vol.5 
      Rept.on the prospects of irrigation  in NSW by Col.F.J.Home. 

41. C.J. Lloyd , 1988, pp.178-80, 182-3.

42. C.J.Lloyd, 1988,  pp.124-5.

43. G.J.Evatt, Public Administration, vol 1(2),1938, p.16.

44. VicPP 35 of 1902-03.

45. Patricia McCaughey,  Samuel McCaughey, A Biography, 1955.

46. J.A.Gibson 1962, Proc.Hay Historical Soc.,

47. NSW V\&P 1906 vol.5.Standing Comm.Public Works,  Murrumbidgee 
      Rept \& MoE, MoE p.96. 

48. Patricia McCaughey 1955 \& C.J.Lloyd 1988.

49. NSW LA V\&P 1903 vol.1, \&  C.H.Munro, Australian Water 
      Resources And Their Development 1974 p.135 re McKinney as the real 
      planner.

50. C.J.Lloyd, 1988 p.187.

51. R.T.McKay, The Utilisation Of The Murrumbidgee Waters, 
     Agr.Gaz.NSW., vol. 18, 1907,p.103.      

52. NSW PP V\&P Session 1906, Parl.Standing Comm Public Works,Rept. \&
      MoE. 

53. NSW PP 1906  vol.5. 

54. W.H.Williamson, A Century Of Scientific Progress, Royal Soc.
      New South Wales, 1968 Ch. 2, p.63.

55. H.J.Frith \& G.Sawer(eds), The Murray Waters, 1974 p.127.

56. K.Jeffcoat, More Precious Than Gold, 1988 , p.114.

57. B.M.Kelly, From Wilderness To Eden, 1988.

58. C.J.Lloyd, 1988.

59. T.Langford -Smith \& J.Rutherford, Water and Land, 1966 
      Part 1 \& C.J.Lloyd, 1988, Ch. 14.

60. L.A.B.Wade, 1914 BAAS Hdbk NSW p.147-48.

61. C.J.Lloyd,1988,  p.212.

62. L.A.B.Wade 1914, BAAS Hdbk NSW,  p.147.

63. Information by H.N.England in Appendix A of 1945 CSIR report by
      J.A.Prescott and others on waterlogging in Yanco No.1 Irrigation Area.

64. K.Murley, The Dethridge Meter, SRWSC 1967.

65. C.J.Lloyd,1988,  p215, \& B.R.Davidson,Australia Wet Or Dry?,1969 p.71.

66. CSIR Bull. 107,1937.

67. Ann.Rept to June 1916 of WCIC, NSW. 

68. NSW WCIC annual reports 1918,1919,1920.

69. NSW PP Joint Vol 4 1920, R.Murray Commission , NSW 
      report to 30/6/1920.

70. P.J.Hallows \& D.G.Thompson , The History Of Irrigation In
      Australia, 1996(?), pp.71-81.

71. VicPP No 35 of 1902/03 vol 3.

72. D.J.Gordon, Handbook Of South Australia, 1908. 

73. SAPP 29 of 1910, S.Murray's report.

74. J.K.Taylor \& H.G.Poole, Report On The Soils Of The Bed Of Lake 
      Albert, South Australia, J. CSIR 1932 vol.5,pp129-130.

75. S. McIntosh J.Agric SA vol. 15 Irrigation And Reclamation, March 1912 
      p.812, \& A.J.Perkins J.Agric.SA ,vol.6, 1903.

76. B.J.Menzies \& P.N.Gray, Irrigation And Settlement In The South 
      Australian Riverland, Tech.Paper Dept. Agric. S.A.No.7, 1983, p.191.

77. SAPP 86 of 1913, p.4. 

78. D.Mack, The Village Settlements On The River Murray In South Australia 
      1894-1909, 1994,  p.56 and T.J.Marshall \& N.J.King, CSIR Bull. No.62,
      1932.

79. Mrs L.M.Andison, Berri, Hub Of The Upper Murray,1953 p.4.

80. B.J.Menzies \& P.N.Gray, 1983,p.221.

81. G.Woolmer, The Barmera Story, 1973 p.30.

82. SAIrrigation and Reclamation Act no 1178 of 1914, Section 32. 

83. S.McIntosh J.Agric.SA 1912. 

84. Marianne Hammerton, Water South Australia, 1986, p.56.

85. S.McIntosh J.Agric SA March 1912 , p.810.

86. S.McIntosh in SAPP32 of 1921, cited in Menzies \& Gray , 1983, p.201.

87. WA V \& P 1917-18 Paper 15, 2nd Prog.Rept. R.C. Agric. Industries.

88. WAPD 1912 Vol.43, p.1853, \& WAPD 1913 vol.449.

89. H.Oldham 1913, WA V \& P 1917-18, vol 2, Paper 15, 2nd Prog.Rept 
      R.C.Agric Industries.

90. Brisbane Evening Observer, 6 April 1889. 

91. H.E.A.Eklund Qld Agric.J.1923. 

92. H.E.A.Eklund 1923.

93. Qld J \& P 1910 vol 3, Rept.on . . . water resources of Queensland,
      by E.Mead,\& Qld PD vol. 107, 29/11/1910, p.2356, \& P.J.Hallows \& 
      D.C.Thompson 1996(?).

94. Qld Agr.J. 1924, irrigation statistics, pp.289-308.

95. Margaret Mason-Cox, Lifeblood Of A Colony, 1993,pp.152-54.

96. Margaret Mason-Cox, 1993 p.151. 

97. TasPP No.46 of 1902, Rept by Rahbek.

98. Margaret Mason-Cox, 1993, p.170.

99. TasPP No.11 of 1916 \& No.28 of 1917.
