% $Id$
% CHAPTER FIFTEEN
% 2183 words at 3/5/99

\chapter{Underground Water for Irrigation}

Until the 1880s irrigation generally depended on water available in
streams and lakes or from springs.  Up to that time sinking a well was
generally undertaken with the expectation of obtaining water only for
household use or livestock.  Pastoral use of arid areas remote from
dependable streams depended on wells equipped with horse-driven whims
or a windmill; in these circumstances there was no expectation of
irrigating crops.  However, shallow wells in some riverine areas might
give supplies for small-scale irrigation.

Discoveries by 1850 of artesian water in Europe and other continents
led to interest in the possible occurrence of this type of underground
water in Australia.  In 1857, W.\,B.~Clarke, the geologist, suggested
its occurrence in the inland.1 Borings for artesian water started near
Sale, Victoria, in 1858, without success until 1880.  In 1871, another
geologist, H.\,Y.\,L.~Brown, recommended boring at Kelmscott on the
Perth coastal plain, which gave a flow of artesian water from a depth
of 171\,ft.  Then the notable mound springs near Lake Eyre were
claimed as natural artesian wells.2 A bore put down at Kallara in
western New South Wales, not far from other mud- or mound-springs,
gave a good supply of artesian water in 1879 and was the first in what
became known as the Great Artesian Basin.3

During the 1880s the search for artesian water involved modern
equipment of the type used successfully in drilling for artesian water
and oil in USA.  Many bores in New South Wales, Queensland and South
Australia gave dramatic flows of water.  The Australian discoveries
brought awareness of great variation in the quality of artesian water.
Most flows in the arid inland would satisfy thirsty livestock but not
human palates or digestive systems.  On the other hand some artesian
bores in southern Australia gave water useful for town supply and
irrigation, as at Sale and near Perth.  The fact that artesian water
was used for irrigation in USA was significant to Australian advocates
of irrigation development.4 The search for artesian water added
greatly to knowledge of the underground water resources.  The boundary
of the Great Artesian Basin was investigated, and attention was given
to other basins providing artesian supplies.

\section{Irrigation with artesian water}

While the intricacies of the Great Artesian Basin, including the
quality of water and the limits of the basin, were being investigated,
some attempts were made to use the bore water for irrigation.
Probably the first efforts were made at the South Australian town of
Hergott Springs (Marree), a telegraph station and railway station
about a mile from a small mound-spring.5 The Conservator of Forests
advised in 1884 that experimental plantations of dates should be
established at several places including a locality near Hergott
Springs (Marree).6 A somewhat similar interest was displayed by the
Conservator of Water, whose department had established a bore in the
township; by 1888 he arranged for a garden to be established in the
township by Ah Tan to cultivate maize, peas, and yam tubers with water
from the town bore.7 A few years later the forestry authority started
a date plantation, apparently on the garden site, and by mid 1891 more
than 300 young date palms had been planted out.8 A plantation with
more than 3000 palms was established later near a bore at Lake Harry,
20 miles north-east of Hergott Springs.9 Although good quality dates
were produced and sold for more than 20 years, the palms languished
--- possibly as the artesian flows declined.

More ambitious efforts with irrigated agriculture in this artesian
basin were made in New South Wales.  At the Native Dog Bore, 45 miles
from Bourke, four acres were enclosed and planted with fruit trees and
some crops, all under irrigation in 1892.  A further 16\,ac were then
enclosed and cleared for satisfactory cultivation of vegetables under
irrigation.  In 1894 the Minister for Mines and Agriculture, Sidney
Smith, visited the bore and was impressed with the results obtained
under irrigation.  He also saw the Pera Bore, recently completed
9\,miles west of Bourke, and decided to make a major effort there with
irrigation, involving further experiments and leasing small allotments
to individuals.  Part of the 640\,ac of land around the bore was
divided into 20\,ac blocks which were made available to lessees under
provisions for homestead selection in the Crown Lands Act.10
Demonstration farming was started on 50\,ac in 1895 when 11 of the 26
surveyed blocks were applied for. The bore, 1154\,ft deep, supplied
more than 600,000\,gal/day at a temperature of 98$^\circ$\,F.  The
water was sprayed for cooling and aeration and held in an elevated
tank of 20,000\,gal capacity before it flowed in an open flume through
the selected blocks.  The manager of the demonstration farm,
C.\,H.~Gorman, reported results achieved with wheat, lucerne, sorghum
millet, potatoes and fruit trees, which included deciduous varieties,
citrus, and dates.11

Crops of wheaten hay, lucerne, sorghum and millet were produced over
several years.  Lucerne was less successful but gave many tons of hay
over the years and one stand was still used for grazing after 12
years.12 Deciduous fruit trees proved less vigorous than citrus or
vines and were gradually replaced by oranges which after ten years
were exported to Britain with good results.  Dates established in 1895
began to give fruit in 1900.  The group settlement at Pera Bore
attracted good support at first but gradually declined due to the
alkalinity of the bore water.13 The reports by technical officers
involved with the venture indicate that useful production lasted no
more than 15 years.

Another experimental farm was established in 1900 at Moree, where a
bore to 2900\,ft supplied 1.25\,million gal/day at 114$^\circ$\,F.14
This bore in the township supplied municipal baths, a wool-washing
business, and a farm area of 250\,ac of which 50\,ac was used by the
NSW Department of Agriculture as an irrigation farm and the balance
divided into allotments of 13 to 15\,ac available to lessees and
supplied with bore water.15 The irrigation farm was situated chiefly
on black cracking clay and included 40\,ac of wheat, lucerne, sorghum
and millet and an orchard of 4\,ac.16 Cereal crops were cut for hay,
with profitable results from sales during the drought of 1902.17 The
Moree water was not highly mineralised, being close to intake areas,
and this circumstance was used in criticism of the experimental farm:
some landholders also claimed that the possibility of irrigating small
areas with bore water, as at Moree, had no appeal for pastoralists
dealing with large areas.18 The Moree farm was abandoned in 1910 on
the score that the water supply could be better used for the curative
baths, a wool-washing plant and for the small allotments.19

In Queensland a bore put down by the government 35 miles north of
Cunnamulla was intended to supply several holdings made available for
selection. A daily flow of more than 2 million~gal of extremely pure
water was obtained at 2090\,ft; it was expected that this would be
used on seven adjacent farms with a total area of 126,000\,ac.
Although the report of this project indicated irrigation was intended,
the main use for the supply was for stock and domestic use.20

Barcaldine in central Queensland had experience of artesian water from
the late 1880s when its first bore was put down by the government.
Many bores were made later for individual landholders, with some use
of the water for irrigation.  An unusual development there was the use
of bore water for irrigation at the Alice River settlement, a
cooperative settlement referred to in another chapter.

Other artesian basins were recognised in Gippsland,Victoria, and below
the Perth coastal plain.  Both proved useful for irrigation. Near Sale
a flowing bore providing 42,000\,gal/day from 225\,ft enabled William
Craig to irrigate fodder crops at Craigelee; another irrigator was
Mr~Palmer who since 1883 watered 80\,ac of pasture from his bore at
Charlcote yielding 45,000\,gal/day.21 On the Perth coastal area, an
artesian water supply established in 1896 by boring at Woodbridge,
near Guildford, allowed Mr C.~Harper to irrigate fruit trees.22

\section{Irrigation with sub-artesian and unconfined groundwater}

As boring for water continued it was realised that Australia had a
number of sedimentary basins additional to the widespread Great
Artesian Basin, that particular strata could be important aquifers,
and that groundwaters might be unconfined in one part of a basin yet
so confined elsewhere as to be sub-artesian or even artesian.  One of
the basins which provided groundwater suitable for irrigation and gave
evidence of considerable variation in the occurrence and quality of
groundwater is the Murray Basin.  Like the Great Artesian Basin, it
involves three States.

Settlers in the Victorian Wimmera, within the Murray Basin, had
limited opportunities for water supply from streams or lakes but in
the 1870s and 80s they generally could count on sinking wells to water
at shallow depths.  In the nearby Tatiara district of South Australia,
underground water was exposed in a waterhole and was so readily
available nearby that a Chinese gardener raised vegetables near
Bordertown in 1888 with the aid of the groundwater.23 At about the
same time settlers near Nhill in the West Wimmera took an interest in
developing irrigation based on the use of groundwater from wells or
bores.  They were encouraged by a visit from Alfred Deakin, the
Minister for Water Supply, in 1883 and later tried unsuccessfully for
government support to establish a Lowan Irrigation Trust based on use
of underground water.24 One settler, Bernard Dreher, was regarded as a
successful irrigator; he relied on a bore in the Netherby district
which was described as providing artesian water. In fact Dreher
installed an American deep-well pump giving 2000\,gal/hr used to
irrigate his garden, orchard and four acre vineyard.25 Despite
systematic boring by the Victorian Government in the nearby Mallee,
reserves of good quality water were found only in the Murrayville
district where limited use of bore water for irrigation was made in
1914.26

Where unconfined groundwater with good quality and availability was
found, it was likely to be used for irrigation of crops.  In 1888,
Francis and Robert Gore began irrigation at Yandilla on their pastoral
run of 90,000\,ac on the Condamine River near Pittsworth south-west of
Toowoomba in Queensland.  They pumped groundwater from a depth of
54\,ft to irrigate 80\,ac of lucerne.27 At Emu Plains, near Sydney,
G.H.Dempsey in 1892 took water from a well with a Tangye pump driven
by a 2\,hp Tangye engine and delivered supplies through iron pipes to
his citrus trees.28 Tim Yung at Bourke drew water for his garden on
more than 3\,ac from a well by means of a 6\,hp steam-driven pump.29
By 1892 Mr J.~Shaw was pumping water from a well at Croydon on the
Adelaide plain to water 100\,ac of fodder.30

A major use of groundwater developed on the Burdekin River delta after
sugar-cane plantations were established there under irrigation.  At
first this practice relied on supplies from lagoons.  The Pioneer
plantation, one of several on the delta, had been established by John
Spiller and Henry Brandon, men with long experience of the sugar
industry on the Pioneer River in the Mackay district.  Their new
enterprise on the Burdekin delta was taken over by the Drysdale
brothers in 1884.  John Drysdale, the managing director, began using
groundwater for irrigation in the 90s, after finding the supply of
water from lagoons was sometimes inadequate.  At the Pioneer
plantation, where about 1000\,ac could be irrigated, the main pumping
station raised 4 million gal/day from a depth of 23\,ft.31

\section{References}

1. C.J.Lloyd, Either Drought Or Plenty, 1988 p.103

2. R.Tate, Trans.R.Soc.S.A. vol 2, 1879. 

3. W.H.Williamson, in A Century Of Scientific Progress, 1968, pp.56-57.

4. A.Deakin , Irrigation In Australia, Year-book of Australia for 1892.

5. C.T.Madigan, Central Australia, 1936.

6.  Lois Litchfield , Marree And The Tracks Beyond In Black And White, 1983
      p.94. 

7.  Marianne Hammerton, Water South Australia, 1986 p.78.

8.  Lois Litchfield 1983 p.94.

9.  Lois Litchfield 1983 \& Ursula \& V.J.de Fontenay , J.Aust.Inst.Agric.Sci.
     1960, vol 26, p.246. 

10.  NSW Agr.Gaz. 1898 v.9,p.273. 

11. W.J.Allen, NSW Agr.Gaz. 1910 v.21 p.887.

12. W.J.Allen, NSW Agr.Gaz.1908 V.19,p.17.

13. NSW LA V \& P 1897 vol.5, Rept by F.J.Home,  K.Jeffcoat, More 
      Precious Than Gold, 1988 p.77.

14. W.R.Fry, NSW Agr.Gaz. 1904 v.15 p.1130.

15.  B.L.Thomson, NSW Agr.Gaz. 1903 v.14, p.943.

16.  B.L.Thompson, NSW Agr.Gaz. 1902 v.13,p.577.

17. W.R.Fry, NSW Agr.Gaz.1906 v.17,p.581.

18. W.R.Fry,  NSW Agr.Gaz. 1904 vol.15 p.1130.

19. A.E.Darvall, NSWAgr.Gaz. 1910, vol.21 pp.802-809.

20. Anon., Qld Agr.J. 1899 v.5,p.459.

21. VicPP 53 0f 1885 p.xxxvii.

22.  WA J.Agric. 1904. vol.9, p.145.

23.  SAPP 28 of 1888 MoE p.88.

24.  L.J.Blake, Land Of The Lowan, 1976.

25.  L.J.Blake 1976 p.125.

26.  R.East ,1967 Vict.Hist.Mag. v.38,p.169, \& Jenny Keating, The Drought
       Walked Through, 1992, p. 98, refers to MelbourneAge 10 Oct 1914.

27.  Queenslander, 6/4/1889 \& 27/4/1889.

28.  NSW Agr Gaz.1892, vol.3,p.710.

29.  NSWPP 118a of 1885,p.223.

30.  J.J.Green \& A.Molineux, Irrigation, 1892.

31.  G.C.Bolton, A Thousand Miles Away, 1972,p.237
