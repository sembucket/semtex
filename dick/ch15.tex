% $Id$
% CHAPTER FIFTEEN
% 2183 words at 3/5/99

\chapter{Underground Water for Irrigation}

\index{artesian water|(}
\index{aquifers|(}
\label{ch:underground}\addtoendnotes{\protect\subsection*{Chapter \thechapter}}
\fancyhead[RE]{\sffamily \small Chapter \thechapter.\ %
               Underground Water}

\setcounter{endnote}{0}

Until the 1880s irrigation generally depended on water available in
streams and lakes or from springs.  Up to that time sinking a well was
generally undertaken with the expectation of obtaining water only for
household use or livestock.  Pastoral use of arid areas remote from
dependable streams depended on wells equipped with
\index{horses}horse-driven whims or a wind-mill; in these circumstances
there was no expectation of irrigating crops.  However, shallow wells
in some riverine areas might give supplies for small-scale irrigation.

Discoveries by 1850 of artesian water \index{artesian water} in
\index{Europe}Europe and other continents led to interest in the
possible occurrence of this type of underground water in Australia.
In 1857, W.\,B.~Clarke, \index{Clarke, W.\,B.}  the geologist,
suggested its occurrence in the inland. Borings for artesian water
started near Sale, \index{Sale, Vic.} Victoria, in 1858, without
success until 1880.  In 1871, another geologist, H.\,Y.\,L.~Brown,
\index{Brown, H.\,Y.\,L.} recommended boring at Kelmscott
\index{Kelmscott, WA} on the Perth coastal plain, which gave a flow of
artesian water from a depth of 171\,feet.  Then the notable mound
springs \index{mound spring} near Lake Eyre \index{lake!Eyre} were
claimed as natural artesian wells. A \index{technology!bore|(}bore put
down at Kallara \index{Kallara, NSW} in western New South Wales, not
far from other mud- or mound-springs, gave a good supply of artesian
water in 1879 and was the first in what became known as the Great
Artesian \index{Great Artesian Basin}
Basin.\fn{\cite[p.\,103]{lloyd1988}; R.~Tate, \textit{Trans.\ R.\
Soc.\ SA}, vol.\,2, (1879); \cite[pp.\,56--57]{williamson1968}.}

During the 1880s the search for artesian water employed modern
equipment of the type used successfully in drilling for artesian water
and oil in \index{USA}USA.  Many bores in New South Wales, Queensland
and South Australia gave dramatic flows of water.  The Australian
discoveries brought awareness of great variation in the quality of
artesian water.  Most flows in the arid inland would satisfy thirsty
livestock but not human palates or digestive systems.  On the other
hand some artesian bores in southern Australia gave water useful for
town supply and irrigation, as at Sale and near Perth.  The fact that
artesian water was used for irrigation in USA \index{USA} was
significant to Australian advocates of irrigation development.  The
search for artesian water added greatly to knowledge of the
underground water resources.  The boundary of the Great Artesian Basin
was investigated, and attention was given to other basins providing
artesian supplies.\fn{\cite{deakin1892}}

\section*{Irrigation with artesian water}
\index{irrigation!artesian water}

While the intricacies of the Great Artesian Basin, including the
quality of water and the limits of the basin, were being investigated,
some attempts were made to use the bore water for irrigation.
Probably the first efforts were made at the South Australian town of
Hergott Springs, \index{Hergott Springs, SA} a telegraph station and
\index{technology!railways}railway station about a mile from a small
mound-spring.  \index{mound spring} The Conservator of Forests advised
in 1884 that experimental plantations of dates should be established
at several places including a locality near Hergott Springs.  A
somewhat similar interest was displayed by the Conservator of Water,
whose department had established a bore in the township; by 1888 he
arranged for a \index{gardening}garden to be established in the
township by Ah Tan \index{Ah Tan} to cultivate \index{maize}maize,
\index{vegetables}peas, and yam tubers with water from the town bore.
A few years later the forestry authority started a date plantation,
apparently on the garden site, and by mid 1891 more than 300 young
date palms had been planted out.  A plantation with more than 3000
palms was established later near a bore at Lake Harry,
\index{lake!Harry} 20 miles north-east of Hergott Springs. Although
good quality \index{fruit}dates were produced and sold for more than
20 years, the palms languished---possibly as the artesian flows
declined.\fn{\cite{madigan1936}; \cite[p.\,94]{litchfield1983};
\cite[p.\,78]{hammerton1986}; U.~\& V.\,J.~de Fontenay,
\textit{J.\,Aust.\ Inst.\ Agric.\ Sci.}, vol.\,26, (1960), p.\,246.}

More ambitious efforts with irrigated agriculture in the artesian
basin were made in New South Wales.  At the Native Dog Bore,
\index{bore!Native Dog} 45 miles from Bourke, \index{Bourke, NSW} four
acres were enclosed and planted with \index{fruit}fruit trees and some
crops, all under irrigation in 1892.  A further 16\,acres were then
enclosed and cleared for satisfactory cultivation of vegetables under
irrigation.  In 1894 the Minister for Mines and Agriculture, Sidney
Smith, \index{Smith, S.} visited the bore and was impressed with the
results obtained under irrigation.  He also saw the Pera Bore,
\index{bore!Pera} recently completed nine miles west of Bourke, and
decided to make a major effort there with irrigation, involving
further experiments and leasing small allotments to individuals.  Part
of the 640\,acres of land around the bore was divided into 20-acre
blocks which were made available to lessees under provisions for
homestead selection in the Crown Lands Act. Demonstration farming was
started on 50\,acres in 1895 when 11 of the 26 surveyed blocks were
applied for. The bore, 1154\,feet deep, supplied more than
600\,000\,gallons per day at a temperature of 98$^\circ$F.  The water
was sprayed for cooling and aeration and held in an elevated tank of
20\,000 gallons capacity before it flowed in an open flume through the
selected blocks.  The manager of the demonstration farm,
C.\,H.~Gorman, \index{Gorman, C.\,H.}  reported results achieved with
\index{wheat}wheat, \index{fodder}lucerne, sorghum millet, potatoes
and \index{fruit}fruit trees, which included deciduous varieties,
citrus, and dates.\fn{\textit{NSW Agr.\ Gaz.}, vol.\,9, (1898),
p.\,273; W.\,J.~Allen, \textit{NSW Agr.\ Gaz.}, vol.\,21 (1910),
p.\,887.}

Crops of \index{fodder}wheaten hay, lucerne, sorghum and millet were
produced over several years.  Lucerne was less successful but gave
many tons of hay over the years and one stand was still used for
grazing after 12~years.  Deciduous \index{fruit}fruit trees proved
less vigorous than citrus or \index{vineyards}vines and were gradually
replaced by oranges which after ten years were exported to
\index{Britain}Britain with good results.  Dates established in 1895
began to give fruit in 1900.  The group settlement at Pera Bore
attracted good support at first but gradually declined due to the
alkalinity \index{alkalinity} of the bore water. The reports by
technical officers involved with the venture indicate that useful
production lasted no more than 15 years.\fn{W.\,J.~Allen, \textit{NSW
Agr.\ Gaz.}, vol.\,19, (1908), p.\,17; NSW LA VP 1897 vol.\,5, Rept by
F.\,J.~Home; \cite[p.\,77]{jeffcoat1988}.}

Another experimental farm was established in 1900 at Moree,
\index{Moree, NSW} where a bore to 2900\,feet supplied 1.25\,million
gallons per day at 114$^\circ$F.  This bore in the township supplied
municipal baths, a wool-washing business, and a farm area of
250\,acres of which 50\,acres was used by the NSW Department of
Agriculture as an irrigation farm and the balance divided into
allotments of 13 to 15\,acres available to lessees and supplied with
bore water. The irrigation farm was situated chiefly on black cracking
clay and included 40\,acres of \index{wheat}wheat,
\index{fodder}lucerne, sorghum and millet and an \index{fruit}orchard
of four acres.  \index{cereals}Cereal crops were cut for
\index{fodder}hay, with profitable results from sales during the
\index{drought}drought of 1902. The Moree water was not highly
mineralised, being close to intake areas, and this circumstance was
used in criticism of the experimental farm: some landholders also
claimed that the possibility of irrigating small areas with bore
water, as at Moree, had no appeal for pastoralists dealing with large
areas. The Moree farm was abandoned in 1910 on the score that the
water supply could be better used for the curative baths, a
wool-washing plant and for the small allotments.\fn{B.\,L.~Thompson,
\textit{NSW Agr.\ Gaz.}, vol.\,13, (1902), p.\,577; B.\,L.~Thomson,
\textit{NSW Agr.\ Gaz.}, vol.\,14, (1903), p.\,943; W.\,R.~Fry,
\textit{NSW Agr.\ Gaz.}, vol.\,15 (1904), p.\,1130; W.\,R.~Fry,
\textit{NSW Agr.\ Gaz.}, vol.\,17, (1906), p.\,581; A.\,E.~Darvall,
\textit{NSW Agr.\ Gaz.}, vol.\,21 (1910), pp.\,802--809.}

In Queensland a bore put down by the government 35 miles north of
Cunnamulla \index{Cunnamulla, Qld} was intended to supply several
holdings made available for selection.  A daily flow of more than two
million gallons of extremely pure water was obtained at 2090\,feet; it
was expected that this would be used on seven adjacent farms with a
total area of 126,000 acres.  Although the report of this project
indicated irrigation was intended, the main use for the supply was for
stock and domestic use.\fn{\textit{Qld Agr.\ J.}, vol.\,5, (1899),
p.\,459.}

Barcaldine \index{Barcaldine, Qld} in central Queensland had
experience of artesian water from the late 1880s when its first bore
was put down by the government.  Many bores were made later for
individual landholders, with some use of the water for irrigation.  An
unusual development there was the use of bore water for irrigation at
the Alice River \index{river!Alice} settlement, a cooperative
settlement referred to in chapter\,\ref{ch:groups}.

Other artesian basins were recognised in Gippsland, Victoria,
\index{Gippsland, Vic.} and below the Perth coastal proved useful for
irrigation. Near Sale \index{Sale, Vic.} a flowing bore providing
42\,000\,gallons per day from 225\,feet enabled William Craig
\index{Craig, W.}  to irrigate \index{fodder}fodder crops at
Craigelee; another irrigator was Mr Palmer who since 1883 watered
80\,acres of \index{pasture}pasture from his bore at Charlcote
yielding 45\,000\,gallons per day. On the Perth coastal area, an
artesian water supply established in 1896 by boring at Woodbridge,
\index{Woodbridge, WA} near Guildford, allowed Mr C.~Harper to
irrigate \index{fruit}fruit trees.\fn{Vic PP 53 of 1885 p.\,xxxvii;
\textit{WA J.\,Agric.}, vol.\,9, (1904), p.\,145.}

\section*{Sub-artesian and unconfined\\
groundwater}
\index{sub-artesian|(}

As boring for water continued it was realised that Australia had a
number of sedimentary basins additional to the widespread Great
\index{Great Artesian Basin}
Artesian Basin, that particular strata could be important aquifers,
and that groundwaters might be unconfined in one part of a basin yet
so confined elsewhere as to be sub-artesian or even artesian.  One of
the basins which provided groundwater suitable for irrigation and gave
evidence of considerable variation in the occurrence and quality of
groundwater is the Murray Basin.  Like the Great Artesian Basin, it
involves three States.

Settlers in the Victorian Wimmera, \index{district!Wimmera} within the
Murray Basin, had limited opportunities for water supply from streams
or lakes but in the 1870s and 80s they generally could count on
sinking wells to water at shallow depths.  In the nearby Tatiara
district of South Australia, underground water was exposed in a
waterhole and was so readily available nearby that a
\index{Chinese}Chinese \index{gardening}gardener raised vegetables
near Bordertown \index{Bordertown, SA} in 1888 with the aid of the
groundwater. At about the same time settlers near Nhill \index{Nhill,
Vic.}  in the West Wimmera took an interest in developing irrigation
based on the use of groundwater from wells or
\index{technology!bore}bores.  They were encouraged by a visit from
Alfred Deakin, \index{Deakin, A.} the Minister for Water Supply, in
1883 and later tried unsuccessfully for government support to
establish a Lowan Irrigation Trust \index{trust!irrigation!Lowan}
based on use of underground water.  One settler, Bernard Dreher,
\index{Dreher, B.}  was regarded as a successful irrigator; he relied
on a bore in the Netherby district \index{district!Netherby} which was
described as providing artesian water.  In fact Dreher installed an
\index{American}American \index{technology!pump!deep-well}deep-well
pump giving 2000 gallons per hour used to irrigate his
\index{gardening}garden, \index{fruit}orchard and four acre
\index{vineyards}vineyard.  Despite systematic boring by the Victorian
Government in the nearby Mallee, \index{district!Mallee} reserves of
good quality water were found only in the Murrayville district where
limited use of bore \index{technology!bore|)}water for irrigation was
made in 1914.\fn{SA PP 28 of 1888 MoE p.\,88; \cite{blake1976};
R.~East, \textit{Vic.\ Hist.\ Mag.}, vol.\,38, (1967), p.\,169;
\cite[p.\,98, refers to \textit{Melbourne Age} 10 Oct.\
1914]{keating1992}.}

Where unconfined groundwater with good quality and availability was
found, it was likely to be used for irrigation of crops.  In 1888,
Francis and Robert Gore \index{Gore Bros} began irrigation at Yandilla
on their pastoral run of 90\,000\,acres on the Condamine River
\index{river!Condamine} near Pittsworth \index{Pittsworth, Qld}
south-west of Toowoomba in Queensland.  They pumped groundwater from a
depth of 54\,feet to irrigate 80\,acres of lucerne. At Emu Plains,
\index{Emu Plains, NSW} near Sydney, G.\,H.~Dempsey \index{Dempsey,
G.\,H.} in 1892 took water from a well with a Tangye pump
\index{technology!pump!Tangye} driven by a two-horsepower Tangye
engine and delivered supplies thr\-ough iron
\index{technology!pipe!cast-iron}pipes to his citrus trees.  Tim Yung
at Bourke \index{Bourke, NSW} drew water for his
\index{gardening}garden on more than three acres from a well by means
of a \index{technology!pump!steam-driven}six-horsepower steam-driven
pump. By 1892 Mr J.~Shaw was pumping water from a well at Croydon
\index{Croydon, SA} on the Adelaide plain to water 100\,acres of
\index{fodder}fodder.\fn{\textit{Queenslander}, 6 Apr.\ 1889 \& 27
Apr.\ 1889; \textit{NSW Agr.\ Gaz.}, vol.\,3, (1892), p.\,710; NSW PP
118a of 1885, p.\,223; \cite{green1892}}

A major use of groundwater developed on the Burdekin River
\index{river!Burdekin} delta after \index{sugar cane}sugar cane
plantations were established there under irrigation.  At first this
practice relied on supplies from lagoons.  The Pioneer plantation, one
of several on the delta, had been established by John Spiller and
\index{Spiller, J.}  Henry Brandon, \index{Brandon, H.} men with long
experience of the sugar industry on the Pioneer River
\index{river!Pioneer} in the Mackay district.  Their new enterprise on
he Burdekin delta was taken over by the Drysdale brothers
\index{Drysdale Bros} in 1884.  John Drysdale, the managing director,
began using groundwater for irrigation in the 1890s, after finding the
supply of water from lagoons was sometimes inadequate.  At the Pioneer
plantation, where about 1000\,acres could be irrigated, the main
pumping station raised four million gallons per day from a depth
of\index{artesian water|)} \index{aquifers|)}\index{sub-artesian|)}
23\,feet.\fn{\cite[p.\,237]{bolton1972}.}

%\section*{References}
%1. C.J.Lloyd, Either Drought Or Plenty, 1988 p.103
%2. R.Tate, Trans.R.Soc.S.A. vol 2, 1879. 
%3. W.H.Williamson, in A Century Of Scientific Progress, 1968, pp.56-57.
%4. A.Deakin , Irrigation In Australia, Year-book of Australia for 1892.
%5. C.T.Madigan, Central Australia, 1936.
%6.  Lois Litchfield , Marree And The Tracks Beyond In Black And White, 1983
%      p.94. 
%7.  Marianne Hammerton, Water South Australia, 1986 p.78.
%8.  Lois Litchfield 1983 p.94.
%9.  Lois Litchfield 1983 \& Ursula \& V.J.de Fontenay , J.Aust.Inst.Agric.Sci.
%     1960, vol 26, p.246. 
%10.  NSW Agr.Gaz. 1898 v.9,p.273. 
%11. W.J.Allen, NSW Agr.Gaz. 1910 v.21 p.887.
%12. W.J.Allen, NSW Agr.Gaz.1908 V.19,p.17.
%13. NSW LA V \& P 1897 vol.5, Rept by F.J.Home,  K.Jeffcoat, More 
%      Precious Than Gold, 1988 p.77.
%14. W.R.Fry, NSW Agr.Gaz. 1904 v.15 p.1130.
%15.  B.L.Thomson, NSW Agr.Gaz. 1903 v.14, p.943.
%16.  B.L.Thompson, NSW Agr.Gaz. 1902 v.13,p.577.
%17. W.R.Fry, NSW Agr.Gaz.1906 v.17,p.581.
%18. W.R.Fry,  NSW Agr.Gaz. 1904 vol.15 p.1130.
%19. A.E.Darvall, NSWAgr.Gaz. 1910, vol.21 pp.802-809.
%20. Anon., Qld Agr.J. 1899 v.5,p.459.
%21. VicPP 53 0f 1885 p.xxxvii.
%22.  WA J.Agric. 1904. vol.9, p.145.
%23.  SAPP 28 of 1888 MoE p.88.
%24.  L.J.Blake, Land Of The Lowan, 1976.
%25.  L.J.Blake 1976 p.125.
%26.  R.East ,1967 Vict.Hist.Mag. v.38,p.169, \& Jenny Keating, The Drought
%       Walked Through, 1992, p. 98, refers to MelbourneAge 10 Oct 1914.
%27.  Queenslander, 6/4/1889 \& 27/4/1889.
%28.  NSW Agr Gaz.1892, vol.3,p.710.
%29.  NSWPP 118a of 1885,p.223.
%30.  J.J.Green \& A.Molineux, Irrigation, 1892.
%31.  G.C.Bolton, A Thousand Miles Away, 1972,p.237

