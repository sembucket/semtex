% $Id$
% CHAPTER 16
% 1526 words at 3/5/99; 2211 at 21/1/03 (revised by Sue, including format notes)
\setcounter{endnote}{0}

\chapter{Conclusion}
\label{ch:reflect}
\markboth{Chapter \thechapter. Conclusion}%
{Pioneering Irrigation in Australia}

By the 1920s the pioneering era of Australian irrigation had
ended. For a century individuals in different places had experimented
with different methods, lured by the prospect of greatly increased
production in regions of low and/or unreliable rainfall. Mostly on a
small scale, their projects were implemented with cheap and simple
equipment and the labour of men and horses.

In the twentieth century agriculture changed in Australia, and with it
the technology of irrigation. Just as farms became larger and
mechanised, so the scale of irrigation schemes increased. By the late
nineteenth century projects like those of the Chaffey brothers
\index{Chaffey bros} on the River Murray \index{river!Murray} involved
thousands of acres of land, hundreds of miles of excavated irrigation
channels, and expensive steam-driven pumps to raise water from the
river. Such projects were complex in management and funding and could
easily go wrong. In fact most of the irrigation sch\-emes described in
this book failed, leaving disappointment behind. Too many people, in
the speculative spirit of the colonial capitalism, expected to make
their fortunes too quickly and easily.

Governments increasingly became involved in irrigation schemes,
supplanting the pioneers. Politicians had a stake in the success of
irrigation, sometimes a personal financial stake, but also a political
one if their constituents were participants or if they hoped to build
their electorate by attracting new settlers. Economically it was
always desirable to find ways of increasing production. For many,
irrigation also offered the promise of providing a good living to
people on small farms, a dream that has so often failed in Australian
history. Politicians hoped that the prospects for the deserving poor,
like ex-servicemen, could be improved through blocks of irrigated
land.

Even if governments had been reluctant to support irrigation, they
found themselves drawn in by the problems faced by irrigation schemes,
problems that could only be resolved at higher levels. One of these
was conflict over riparian rights. \index{riparian rights} How was the
use of river water to be regulated? Governments stepped in to assert
control in order to share this scarce resource more equitably, only to
find that inter-state conflicts resulted in the case of Australia's
major water resource, the River Murray. Finally, the funding of
irrigation schemes became so demanding that governments were forced to
provide financial backing of one sort or another, or even to take over
large projects where private capital was insufficient. Their gradual
involvement with irrigation was just another aspect of the way in
which governments came to play a major role where, in the early
nineteenth century, they had stood aloof. Interestingly, public
control of irrigation moved most slowly in Tasmania, \index{Tasmania}
where farmers first introduced irrigation in the early nineteenth
century, illustrating some of the differences between that relatively
well-watered, compact state and the mainland regions.

Looking back, then, at the pioneering period, what were its
characteristics, and how did they differ from the later history of
irrigation in Australia?

%\section*{The irrigation activists or promoters}
\bigskip\noindent
Those concerned with using irrigation are generally landholders not
particularly concerned with inducing others to follow their
example. The acceptance of this kind of land use in Australia owed
much to the work of a relatively small number of promoters or
activists including engineers, pastoralists, legislators,
horticulturalists, community leaders, and entrepreneurs.  Prominent
activists included Arthur and Hugh Cotton, Paul Strzelecki, Benjamin
Hawkins Dods and Hugh McColl, William Lyne, Alfred Deakin, George and
W.\,B.~Chaffey, E.\,C.~De Garis, Samuel McIntosh, Hugh McKinney,
George Gordon, John Derry, Elwood Mead, W.\,E.~Shoobridge, Samuel
McCaughey, John West, Stuart Murray, W.\,W.~Culcheth, and George
Swinburne. How much irrigation would have been undertaken without
these people?

Some of the promoters were British migrants who had gained experience
of irrigation as engineers in India; two gained acquaintance with
irrigation in America; four were prominent men without significant
experience of irrigation; and three had used irrigation in Australia.

One of the most active promoters of irrigation was Alfred Deakin
\index{Deakin, A.}  during his term as a Victorian cabinet minister
with responsibility for water supply.  Deakin's interest developed
during his term in the Service ministry beginning in March 1883.  With
an electorate which included significant irrigation at Bacchus Marsh,
he paid attention to Hugh McColl's \index{McColl, H.} persistent
efforts in parliament to win government support for provision of water
supply and irrigation in northern Victoria.  Deakin later obtained
legislation for provision of irrigation in particular areas by elected
irrigation trusts representing landholders.  His comprehensive
Irrigation Act of 1886 enabled those trusts to obtain government
financial support, a step which led to their proliferation, and
provided for state control of water resources.  He also negotiated the
scheme proposed by the Chaffey Brothers \index{Chaffey bros} for an
irrigation colony at Mildura.

There were also influential journalists and newspaper editors who
helped to advance irrigation. Deakin's enthusiasm for irrigation
development in Victoria was strongly supported by David Syme,
\index{Syme, D.} the influential newspaper proprietor, who encouraged
his visit to USA in 1885 and to Ceylon and India in 1891.

While Deakin's contributions had their main effect in Victoria, his
legislation on state control of water resources had effects elsewhere
in Australia.  The other promoters of irrigation who had widespread
influence were the Chaffey brothers before the collapse of the land
boom and Elwood Mead \index{Mead, E.} who was consulted by governments
in other states than Victoria.

In terms of ethnic background, the pioneers reflected the
multicultural nature of immigration to Australia in the nineteenth
century. Irrigation before the mid 1850s was undertaken generally by
people from the British Isles, the known exceptions being Germans
\index{Germans} in South Australia.  Then with the arrival of Chinese
\index{Chinese} there began widespread demonstration in eastern
Australia of the advantages given by irrigation.  Their operations
were confined to small areas and have been ignored in publications
concerning major irrigation schemes. Nevertheless their achievements
were recognised before 1900 as important examples of irrigation.

Many Europeans and Americans came to Australia in response to
discoveries of gold and a few of them later became involved with
irrigation.  One was Alessandro Martelli, \index{Martelli, A.} the
Italian architect and engineer who in 1860 was involved in Victoria
with a scheme for irrigation at Heidelberg and was later appointed to
make official enquiries on irrigation in Tasmania.  There was also
James Bladier, \index{Bladier, J.} the French vigneron at Bendigo who
was partner in a scheme for irrigation of vines at Adelaide Vale in
1860.  A few Americans involved with production of tobacco and hops in
north-eastern Victoria included Hiram Crawford, \index{Crawfrod, H.}
who irrigated hops on his Everton farm on the Ovens River.
\index{river!Ovens} The American A.~Spawn \index{Spawn, A.} promoted
irrigation colonies near Horsham \index{Horsham, Vic} in the 1880s.

In 1910 Elwood Mead of the Victorian State Rivers and Water Supply
Commission went to Britain, Europe, and America seeking migrants to
take up irrigation in Victoria.  Subsequently some Americans visited
the new irrigation settlements but few remained there.  A major
migration affecting irrigation was the arrival of the Welsh party from
Patagonia, where they had experience of irrigation, as settlers at the
Murrumbidgee Irrigation Authority.

The last significant groups of European migrants actually involved in
irrigation before 1920 were the Spaniards \index{Spanish} from
Catalonia who engaged in market-gardening at Bendigo, Echuca and
Cohuna, and the Russians \index{Russians} who in 1913 began the Jewish
colony engaged in irrigated fruit production near Shepparton.

%\section*{Technology transfer}
\bigskip\noindent
Many of the pioneers were inspired by their observation or experience
of the use of irrigation elsewhere in the world, and some were adept
in developing technologies suited to Australia.

The use of irrigation in Tasmania by the 1840s suggests experience in
the British Isles with water-meadows and the use of water power for
machinery.  Likewise some irrigation by Chinese
gardeners\,---\,whether employed by pastoralists or independent
producers\,---\,is known to have involved traditional Chinese
waterwheels to raise supplies for irrigation.  An association of
irrigation with mining technology is indicated by the use of a pump
widely known as the California pump, \index{pump!Californian} a type
of belt-pump \index{pump!belt} widely used by Australian alluvial
goldminers and by small-scale irrigators.  Specifically English and
Italian systems of irrigation were used in Victoria during the
1860s. Australian irrigation was affected considerably by the transfer
of technology from British India and western USA during and after the
1880s.

Australians have developed innovations that appear to qualify as local
contributions to irrigation technology.  One of the earliest was the
introduction in Tasmania of irrigation for hop-growing.  Delivery of
irrigation water by underground concrete pipes \index{concrete pipes}
was first tried in Australia on a South Australian farm using the
recently invented Hume pipes which were later installed on part of an
irrigation settlement in about 1920.  There is also the significant
invention of the Dethridge wheel\,---\,a \index{Dethridge wheel} meter
for water deliveries to irrigation blocks.

%\section*{Uses and sources of irrigation}
\bigskip\noindent
Whereas the common image of irrigation today is its use in growing
crops like rice, fruit and cotton, in the nineteenth century it is
striking how often it was used to provide fodder for animals, in
particular for horses: irrigated pastures then were the equivalent of
an oil well today as far as transport was concerned. Ultimately the
main benefit of irrigation was to livestock industries dependent on
fodder crops and pastures. One outcome was greater production of dairy
produce in southern Australia.

Access to markets, involving transport, was often the key to the use
of irrigation. Production of fruit under irrigation at Mildura
\index{Mildura} and Renmark \index{Renmark} attracted wide attention
but with limited opportunities for sales it could not make a major
contribution to agriculture. The full potential of irrigated fruit
production along the River Murray \index{river!Murray} had to await
better transportation in the twentieth century.

Drought was a major motivation for the adoption of irrigation by many
agriculturalists. Water drawn from dependable sources near the River
Murray was first used extensively in aid of wheatgrowing in parts of
Victoria subject to drought. Irrigation of sugar cane became a
significant aid to production in parts of Queensland.

The sources of irrigation were increasingly ambitious during the
pioneering period. At first farmers used easily available water on or
immediately adjacent to their properties. By the late nineteenth
century people were experimenting with sewage, artesian bores, and
river water carried over considerable distances for the purposes of
irrigation.

The extension of irrigation on the continent before 1920 depended on
diversion of water from streams and principally from those of the
Murray River system.  Efforts to utilise underground water were highly
successful only for sugar-cane in Queensland.

Supplies from the river system were obtained at first mainly by
pumping together with occasional diversion of flood waters.  The need
to store water for summer irrigation led to structures on major
tributaries of the River Murray but there had been no construction of
reservoirs on the main stream before 1920. The huge dams familiar to
Australians were the product of the post-pioneering phase, when
engineering skills increased and governments mobilised the required
capital.

%\section*{Economics of irrigation settlements}
\bigskip\noindent
The first irrigation settlements were dependent entirely on provision
of capital from private sources and payments by settlers to meet the
cost of water supplied.  The deliberate establishment later of
irrigation settlements by state governments involved their expenditure
on facilities and compulsory annual charges to meet the costs of
supplying irrigation water.  This system was introduced in Victoria in
1909 and remained in force for many years.  In New South Wales the
charges for irrigation water on the Murrumbidgee Irrigation Authority
were calculated to meet the cost of supply plus interest on capital
invested in the storage reservoir, weir and distribution channel.
This arrangement continued beyond 1920.  The South Australian
settlements on the River Murray did not require provision of
reservoirs and their charges for water, based on unimproved land
values, were expected to meet the cost of supplying water and the
maintenance and management as well as interest on the cost of the
works.

The management of irrigation settlements often had to contend with
disputes arising from the disparity between charges levied on settlers
and their returns from sales of produce. Increasingly these became
political issues in which governments had a stake.

%\section*{The outlook for irrigation in 1920}
\bigskip\noindent
Irrigation had in fact a very chequered history in Australia in its
first century. Most of its problems were amply illustrated, and most
ventures failed. Many were under-funded, some used inappropriate
technology, some encountered disputes over water usage, many were
based on inadequate understanding of local conditions (climate, soils,
river flows). Government intervention was intended to help correct
these faults.

It is noticeable that among the problems of this early period, some
current concerns were missing or only beginning to become
visible. When riparian rights were at issue, no one thought about the
rights of indigenous peoples to use of the water or land. Nor was much
attention given to the longer-term impact of irrigation on the
environment.

The warning signs were there. By 1920 irrigation based on the Murray
River system had incurred problems of salinity and waterlogging in
several areas of horticultural production.

But in the 1920s irrigation was still a heroic endeavour, the more so
as the scale of projects increased and hearts lifted at the sight of
water gushing from pumping stations into an orderly and extensive
network of channels. The environment of Australia was always a
challenge. If seepage occurred from irrigation ditches, they should be
lined with concrete. If river levels fell too low to allow pumping in
some years, then weirs should be built. Part of the pioneering spirit
lay in the optimism that technological solutions would be found to all
the problems that nature posed. That confidence, with all its
strengths and weaknesses, continued well beyond the 1920s.
