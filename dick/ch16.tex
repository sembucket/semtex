% $Id$
% CHAPTER 16
% 1526 words at 3/5/99

\chapter{Reflections}

\section{Technology transfer}

Australian irrigation was affected considerably by the transfer of
technology from India/Pakistan and western USA during and after the
1880s.  The earlier irrigation was a response to other
influences. Thus its use in Tasmania by the 1840s suggests experience
in the British Isles with water-meadows and the use of water power for
machinery.  Likewise some irrigation by Chinese gardeners --- whether
employed by pastoralists or independent producers --- is known to have
involved traditional Chinese waterwheels to raise supplies for
irrigation.  An association of irrigation with mining technology is
indicated by the use of a pump widely known as the California pump, a
type of belt-pump widely used by Australian alluvial goldminers and by
small-scale irrigators.  Reference to the use of specific English and
Italian systems of irrigation in Victoria during the 1860s is made in
another chapter.

Australian use of irrigation has involved innovations which appear to
qualify as local contributions to irrigation technology.  One of the
earliest was the introduction in Tasmania of irrigation for
hop-growing.  Delivery of irrigation water by underground concrete
pipes was first tried in Australia on a South Australian farm using
the recently invented Hume pipes which were later installed on part of
an irrigation settlement about 1920.  There is also the significant
invention of the Dethridge wheel --- a meter for water deliveries to
irrigation blocks.

\section{The irrigation activists or promoters}

Those concerned with using irrigation are generally landholders not
particularly concerned with inducing others to follow their example.
The acceptance of this kind of land use in Australia owed much to the
work of a relatively small number of promoters or activists including
engineers, pastoralists, legislators, horticulturalists, community
leaders, and entrpreneurs.  Prominent activists included Arthur and
Hugh Cotton, Paul Strzelecki, Benjamin Hawkins Dods and Hugh McColl,
William Lyne, Alfred Deakin, George and W.\,B.~Chaffey, E.\,C.~De
Garis, Samuel McIntosh, Hugh McKinney, George Gordon, John Derry,
Elwood Mead, W.\,E.~Shoobridge, Samuel McCaughey, John West, Stuart
Murray, W.\,W.~Culcheth, and George Swinburne. How much irrigation
would have been undertaken without these people?

Some of the promoters were British migrants who had gained experience
of irrigation as engineers in India; two gained acquaintenance with
irrigation in America; four were prominent men without significant
experience of irrigation; and three had used irrigation in Australia.

There were also influential journalists and newspaper editors who
helped to advance irrigation.

One of the most active promoters of irrigation was Alfred Deakin
during his term as a Victorian cabinet minister with responsibility
for water supply.  Deakin's interest developed during his term in the
Service ministry beginning in March 1883.  With an electorate which
included significant irrigation at Bacchus Marsh, he paid attention to
Hugh McColl's persistent efforts in parliament to win government
support for provision of water supply and irrigation in northern
Victoria.  Deakin later obtained legislation for provision of
irrigation in particular areas by elected irrigation trusts
representing landholders.  His comprehensive Irrigation Act of 1886
enabled those trusts to obtain government financial support, a step
which led to their proliferation, and provided for state control of
water resources.  He also negotiated the scheme proposed by the
Chaffey Brothers for an irrigation colony at Mildura.

Deakin's enthusiasm for irrigation development in Victoria was
strongly supported by David Syme, the influential newspaper
proprietor, who encouraged his visit to USA in 1885 and to Ceylon and
India in 1891. Before his departure from government, Deakin knew that
little progress had been made under the irrigation trust system, as
when in 1890 in a defensive speech to an irrigators' conference, he
said `not one of the 25 Victorian trusts was last year in a position
to invite its constituents to water their crops.'

Irrigation received little of his attention after 1892, when he
visited Mildura to support the Chaffeys at a time when many settlers
were turning against them.  Legal practice and campaigning for
Federation then became his main interests. Deakin was called as a
witness to the 1896 royal commission investigating the debacle at
Mildura and his involvement with irrigation trusts was probed by
another royal commission.  The latter inquiry criticised him for
allowing the establishment of three `private' irrigation trusts
intended in each case to secure profit by subdivision of a single
property.  By 1903 he regarded his irrigation policy as a failure.

The economic recession in the 1890s probably influenced Deakin to turn
away from interest in irrigation; his work for Federation mainly
concernd tariff protection and defence policy.  He regarded irrigation
as a matter for the States rather than the Commonwealth and during his
period in Federal Parliament gave no obvious support to moves for an
interstate agreement on use of the Murray River.

While Deakin's contributions had their main effect in Victoria, his
legislation on state control of water resources had effects in other
Australian states.  The other promoters of irrigation who had
widespread influence were the Chaffey brothers before the collapse of
the land boom and Elwood Mead who was consulted by governments in
other states than Victoria.

\section{Economics of irrigation settlements}

The first irrigation settlements were dependent entirely on provision
of capital from private sources and payments by settlers to meet the
cost of water supplied.  The deliberate establishment later of
irrigation settlements by state governments involved their expenditure
on facilities and compulsory annual charges to meet the costs of
supplying irrigation water.  This system was introduced in Victoria in
1909 and remained in force for many years.  In New South Wales the
charges for irrigation water on the MIA were calculated to meet the
cost of supply plus interest on capital invested in the storage
reservoir, weir and distribution channel.  This arrangement continued
beyond 1920.  The South Australian settlements on the River Murray did
not require provision of reservoirs and their charges for water, based
on unimproved land values, were expected to meet the cost of supplying
water and the maintenance and management as well as interest on the
cost of the works.

The management of irrigation settlements often had to contend with
disputes arising from the disparity between charges levied on settlers
and their returns from sales of produce.

\section{Multicultural irrigation}

Irrigation before the mid 1850s was undertaken generally by people
from the British Isles, the known exceptions being Germans in South
Australia.  Then with the arrival of Chinese there began widespread
demonstration in eastern Australia of the advantages given by
irrigation.  Their operations were confined to small areas and have
been ignored in publications concerning major irrigation schemes.
Nevertheless their achievements were recognised before 1900 as
important examples of irrigation.

Many Europeans and Americans came to Australia in response to
discoveries of gold and a few of them later became involved with
irrigation.  One was Alessandro Martelli, the Italian architect and
engineer who in 1860 was involved in Victoria with a scheme for
irrigation at Heidelberg and was later appointed to make official
enquiries on irrigation in Tasmania.  There was also James Bladier,
the French vigneron at Bendigo who was partner in a scheme for
irrigation of vines at Adelaide Vale in 1860.  A few Americans
involved with production of tobacco and hops in north-eastern Victoria
included Hiram Crawford, who irrigated hops on his Everton farm on the
Ovens River.  The American A.~Spawn promoted irrigation colonies near
Horsham in the 1880s.

In 1910 Elwood Mead of the SRWSC went to Britain, Europe, and America
seeking migrants to take up irrigation in Victoria.  Subsequently some
Americans visited the new irrigation settlements but few remained
there.  A major migration affecting irrigation was the arrival of the
Welsh party from Patagonia, where they had experience of irrigation,
as settlers at the MIA.

The last significant groups of European migrants actually involved in
irrigation before 1920 were the Spaniards from Catalonia who engaged
in market-gardening at Bendigo, Echuca and Cohuna, and the Russians
who in 1913 began the Jewish colony engaged in irrigated fruit
production near Shepparton.

\section{Irrigation to help agriculture}

Water drawn from dependable sources near the River Murray was first
used extensively in aid of wheatgrowing in parts of Victoria subject
to drought.  Production of fruit under irrigation at Mildura and
Renmark attracted wide attention but with limited opportunities for
sales it could not make a major contribution to agriculture.
Irrigation of sugar cane became a significant aid to production in
parts of Queensland.  Ultimately the main benefit of irrigation was to
livestock industries dependent on fodder crops and pastures; one
outcome was greater production of dairy produce in southern Australia.

\section{The Murray River system}

The extension of irrigation on the continent before 1920 depended on
diversion of water from streams and principally from those of the
Murray River system.  Efforts to utilise underground water were highly
successful only for sugar-cane in Queensland.

Supplies from the river system were obtained at first mainly by
pumping together with occasional diversion of flood waters.  The need
to store water for summer irrigation led to structures on major
tributaries of the River Murray but there had been no construction of
reservoirs on the main stream before 1920.

By 1920 irrigation based on this river system had incurred problems of
salinity and waterlogging in several areas of horticultural
production.



