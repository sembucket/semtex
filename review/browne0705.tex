\documentclass[12pt,twoside]{letter}
\usepackage{pstricks}
\usepackage{graphicx}
\usepackage{a4wide}
%\setlength{\textheight}{235mm}
%\setlength{\textwidth}{165mm}
%\setlength{\topmargin}{-10mm} 
%\setlength{\oddsidemargin}{-0.5cm}
%\setlength{\evensidemargin}{-0.2cm}
%\setlength{\unitlength}{1mm}
%\renewcommand{\baselinestretch}{1.00}
\def\undertext#1{$\underline{\smash{\hbox{#1}}}$}

\input{hmb.mac} 

\pagestyle{empty}
\thispagestyle{empty}
\raggedbottom

\date{}

\begin{document}  

\address{
\mbox{ }\\
\begin{tabular}{ll}
phone:  &(+61 3) 9905 1828\\
fax:    &(+61 3) 9905 3558\\
mailto: &hugh.blackburn@eng.monash.edu.au\\
& 3 May 2007
\end{tabular}
}

\begin{letter}
{
To whom it may concern
}
\signature{\vspace*{-10mm}
\includegraphics[scale=0.8]{hmbsig.ps}\\
Dr Hugh M. Blackburn\\
\undertext{Associate Professor, Dept of Mechanical Engineering}
}

\opening{
\begin{pspicture}(0,0)(0,0)
\rput(35mm,50mm){
\includegraphics%
[width=0.45\textwidth]{MonashEng3.eps}}
\end{pspicture}
\centerline{\undertext{\textbf{Re: Masters Thesis, P Browne}}}
}

I have carried out a review of Patrick Browne's thesis, as
requested. It contains a lot of good work but I believe that the data
analysis needs reconsideration and that the thesis should be revised
and resubmitted.  Please find my detailed comments and assessment
attached.

\closing{Yours sincerely,}
\encl{Assessment}
\end{letter}


%%%%%%%%%%%%%%%%%%%%%%%%%%%%%%%%%%%%%%%%%%%%%%%%%%%%%%%%%%%%%%%%%%%%%%%%%%%%%%
\vspace*{-10mm}
\begin{center}
\textbf{
\undertext{Masters Thesis, P Browne}\\[5pt]
Vortex-Induced Vibrations of a Tethered Circular Cylinder in a Free Stream
}
\end{center}

\textbf{Subject matter}

The thesis describes a program of experimental measurement of the
wakes and flow-induced responses of tethered buoyant circular
cylinders in a water channel. The experimental setup has a single
well-submerged circular cylinder placed parallel to the water free
surface, axis normal to the oncoming flow direction and spanning the
width of the channel; the cylinder's ends are attached to thin stalks
or tethers that are connected to pivot points on the outer wall so
that the cylinder is free to move in a circular arc. Two different
cylinder diameters were used and the other variables were speed of
transverse flow in the channel, cylinder buoyancy, and tether length.
Compared to past experimental work on response of freely moving
cylinders (Govardhan \& Williamson 2002) and tethered cylinders
(Carberry \& Sheridan 2006) the present work mainly examines the
effect of dimensionless tether length on cylinder response and wake
structure.

\textbf{Major points}

The experimental measurements seem to have been well conducted and
described, the introductory chapter is well done, the findings are
sufficiently novel, and I feel that in general the writing and data
analysis is well up to the standard expected for a Masters
thesis. However there is one central issue that I think needs to be
more carefully considered, and this relates to the reduction of
results.

Dimensional analysis suggests that a minimum set of appropriate
independent dimensionless groups for the problem are: mass ratio
$m^\ast=4m/\pi\rho D^2$; tether length ratio $L^\ast=L/D$ (or perhaps
$L/R=2L/D$: the tether length ratio does not seem to be defined in the
thesis); Reynolds number $\Rey=U_\infty D/\nu$; reduced Froude number
$\Fr^\prime=U_\infty/\sqrt{(1-m^\ast)\cg D}$.  Sometimes the author
has used the Froude number as $\Fr=U_\infty/\sqrt{\cg D}$, but it
seems clear that the reduced Froude number --- the ratio of inertia
force to buoyancy force specific to the cylinder --- is more
appropriate.

There are two more dimensionless groups which might perhaps be
relevant; these are the relative submergence depth of the pivot point,
$S^\ast=S/L$ where $S$ is submergence depth and the channel flow
Froude number $\Fr_c=U_\infty/\sqrt{\cg H}$, where $H$ is the depth of
water in the channel, which describes the ratio of flow speed to the
speed of long waves on the free surface. Neither of these has been
mentioned explicitly in the thesis (other than on p.\,44 where it is
stated that the `minimum distance from the free surface to the
cylinder was 25 diameters for the 16\,mm cylinder and 16 diameters for
the 25\,mm cylinder, therefore the effect of the free surface was
assumed to be negligible') but they are potentially important
dimensionless groups. It may be that they were ruled out of
consideration by Carberry \& Sheridan (2006), but the present thesis
does not touch on the issue. I am labouring this point because I feel
that one of the central conclusions of the thesis cannot be correct,
and the resolution of the problem might potentially lie in
consideration of free surface effects.

To come to the point, I have great difficulty with the statement in
\S\,3.3 that
\begin{quote}
A result concluded from the investigation of cylinder diameter in the
previous section was that the cylinder response is essentially
diameter independent. The normalized oscillation amplitude, wake
states, drag coefficient and critical mass ratio appeared to collapse
for a given mass ratio and tether length ratio as a function of
$U_\infty$.
\end{quote}
Now examining the results in \S\,3.2, especially figures~3.12
and~3.14, I just cannot credit this statement is true. Why? Because
first if one looks at the mean layover angle in figure~3.12, the
layover angles for the two different cylinder diameters are plainly
not going to collapse with $U_\infty$, contrary to the statement
above; this is confirmed when we look at figure~3.14, where the curves
of mean layover angles instead collapse with $\Fr^\prime$.  My guess
is that if the oscillation envelope points from figure~3.12 were
included on figure~3.14 they would also collapse quite well with
$\Fr^\prime$.  What does seem to remain invariant in figure~3.12 is
the value of $U_\infty$ at which transition occurs between the two
response curves (`initial' and `upper').  I think the candidate
understands that if anything the dimensionless characterisation of
response must be a described by a dimensionless number, rather than
the free-stream speed, because he introduces the group $\Rey\Fr^2$
(which does not contain the diameter, but does contain the flow speed)
to do this. However, no convincing argument is given as to why this
group might be physically relevant (other than it does not contain the
diameter!).  Further, the candidate is apparently happy to have it
both ways because even after this statement we see (figures~3.27,
3.37) mean layover angles plotted as functions of $\Fr^\prime$.

As an alternative explanation of the observations I propose that the
flow speed at which the transition occurs could easily be a function
of the Froude number $\Fr_c=U_\infty/\sqrt{\cg H}$ of the flow in the
channel, i.e.\ that the \emph{transition} between the two curves is
influenced by free surface effects.  This seems such an obvious
possibility that I am really surprised that it does not seem to have
occurred to the candidate. (Or has it? I could find no discussion on
this point.)  This could have been checked e.g.\ by varying the depth
of water in the channel, or perhaps by eliminating/influencing the
free surface. Granted these things may now be difficult: I think that
(in lieu of an experimental investigation) the candidate needs either
to reconsider, discuss and re-evaluate the results in light of this
possibility, or convincingly rebut it. Perhaps I have missed some
vital piece of explanation that would allay my concerns.

What is to be done where subsequently many of the results are reported
without differentiating between the results for cylinder diameters
16\,mm and 25\,mm, but at a value of $U_\infty$ (typically the maximum
value used, $U_\infty=0.456$\,m/s or equivalently $\Rey\Fr^2=9700$)?
My guess is that what could often make it somewhat reasonable to
compare the results for the two cylinder diameters at this common
value is that the responses tend to become asymptotically independent
of dimensionless flow speed (i.e.\ $\Fr^\prime$).  Obviously one needs
to give this careful consideration before deciding how to deal with
presenting undifferentiated data such as appears in
figures~3.39--3.41. Perhaps one might conclude that such figures may
(however unfortunately) need to be omitted. I leave this to the
candidate to resolve.

\textbf{Minor points}
\begin{enumerate}
\item
On the question addressed in \S\,3.16 of whether the reduced Froude
number is an appropriate parameter for heavy cylinders: it is (as
stated) not surprising that in the initial state, where there is
effectively no dynamic response of the cylinder, one can use
$|1-m^\ast|$ in a reduced Froude number that will collapse the buoyant
and non-buoyant results.  However I would guess that the results could
be quite different in the upper response branch, because the ratio of
cylinder mass to mass of displaced fluid, i.e.\ added mass, will
differ markedly. I think some kind of statement to this effect could
be added.
\item
I suggest that the questions explicitly raised for evaluation in
\S\,1.5 be also given explicit answers in chapter~4, Conclusions.
\item
Dimensionless tether length $L^\ast$ not defined.
\item
I suggest that when the Froude number is used as a response plot
variable that the author standardises on reduced Froude number. There
seems no convincing rationale to use the cylinder Froude number
$U_\infty/\sqrt{\cg D}$.
\item
In figure~3.33, I think the mass ratios reported for (b) and (c) may
be wrong; surely they should match those in figure~3.32\,(b,\,c)?
\item
Has the designation W for a `hyper' \Ka\ vortex street been used
before?  Perhaps it is now conventional?  If this is a novel concept I
suggest giving some more space to talking about it.
\item
The `maximum angular oscillation amplitude' $\theta_{\max}$ does not
seem to be defined anywhere.  Is it in degrees or radians; is it a
peak-to-peak amplitude?
\end{enumerate}

\end{document}

