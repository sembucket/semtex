\documentclass[12pt,twoside]{letter}
\usepackage{pstricks}
\usepackage{graphicx}
\usepackage{a4wide}
%\setlength{\textheight}{235mm}
%\setlength{\textwidth}{165mm}
%\setlength{\topmargin}{-10mm} 
%\setlength{\oddsidemargin}{-0.5cm}
%\setlength{\evensidemargin}{-0.2cm}
%\setlength{\unitlength}{1mm}
%\renewcommand{\baselinestretch}{1.00}
\def\undertext#1{$\underline{\smash{\hbox{#1}}}$}

\pagestyle{empty}
\thispagestyle{empty}
\raggedbottom

\date{}

\begin{document}  

\address{
\mbox{ }\\
\begin{tabular}{ll}
phone:  &(+61 3) 9905 1828\\
fax:    &(+61 3) 9905 3558\\
mailto: &hugh.blackburn@eng.monash.edu.au\\
& 1 May 2007
\end{tabular}
}

\begin{letter}
{
To whom it may concern
}
\signature{\vspace*{-10mm}
\includegraphics[scale=0.8]{hmbsig.ps}\\
Dr Hugh M. Blackburn\\
\undertext{Associate Professor, Dept of Mechanical Engineering}
}

\opening{
\begin{pspicture}(0,0)(0,0)
\rput(35mm,50mm){
\includegraphics%
[width=0.45\textwidth]{MonashEng3.eps}}
\end{pspicture}
\centerline{\undertext{\textbf{Re: PhD Thesis, A Comerford}}}
}

I have carried out a review of Andrew Comerford's thesis, as
requested, and consider it should be passed as it stands.  Please find
my detailed comments and assessment attached.

\closing{Yours sincerely,}
\encl{Assessment}
\end{letter}


%%%%%%%%%%%%%%%%%%%%%%%%%%%%%%%%%%%%%%%%%%%%%%%%%%%%%%%%%%%%%%%%%%%%%%%%%%%%%%
\vspace*{-10mm}
\begin{center}
\textbf{
\undertext{PhD Thesis, A Comerford}\\[5pt]
Computational Models of Endothelial and Nucleotide Function
}
\end{center}

The thesis describes computational modelling of flow and scalar
(nucleotide) transport in arterial geometries, and the coupling of
boundary values of wall shear stress (WSS) and nucleotide (ATP/ADP)
concentration to a cellular dynamics model of endothelial cell
response. The goal of the work is to produce a workable computational
model to allow prediction of regions of arterial wall where
atherosclerotic lesions are likely to occur. 

The computational fluid dynamics and scalar transport has been
implemented with commercial concurrent software (Fluent), linked to
custom-coded numerical solution of ordinary differential equations
that describe the cellular dynamics model of endothelial response. The
cellular dynamics model is based on previously published works by
Wiesner, Berk and Neerem (1996, 1997), and Plank, Wall \& David (2006)
and takes wall nucleotide concentrations and WSS as inputs to model
cellular Ca$^{2+}$ activity and endothelial nitric oxide synthase
(eNOS) production.  Ultimately, regions of depressed eNOS production
(and hence by implication low levels of NO) are taken as sites at
which atherogenesis is likely, although the mechanisms underlying this
linkage are not yet completely clear.  The dynamics of each
discretised cell-dynamics domain is linked to the fluid dynamics wall
cell with which it is associated, but there is no interdomain
communication.  Also, while it is possible for WSS to trigger
endothelial cell ATP release, this release does not appear to be
back-coupled to the flow solver (the coupling of scalar transport from
the CFD model is one-way).

The research work is of a high standard, also novel, and I believe
that overall it well justifies the award of PhD. The characteristics
and locations of depressed eNOS production suggested by the
computational model are observed to be in good qualitative agreement
with locations of atherosclerosis found in laboratory and clinical
studies.  A point of interest was that the locations tended to be more
closely linked to low WSS than to low mural nucleotide concentration,
although it is unclear just how much this outcome is determined by the
parameters adopted in the cellular dynamics model, also in the wall
uptake model. Another interesting finding was that the addition of
pulsatility did not greatly alter the time-average eNOS values from
those predicted by steady flow of equivalent mean Reynolds numbers,
although again it is not clear how general this result is, or how
sensitive it is to the input parameters.  

I suggest that this issue of sensitivity of the outcomes to input
parameters could receive attention before the work is submitted for
journal publication, but the level of treatment here is quite
sufficient to convince me that the work is of PhD standard, and that
the main goals of the research program have been met.

I have to say that I felt that the writing and proofing of the thesis
should have received more care, but finally I do not think that these
considerations would justify a demand for the thesis to be re-written,
and I will accept it as it stands.


%\begin{enumerate}
%\item
%The cellular dynamics model is based on earlier work by Wiesner, Berk
%\& Neerem (1996, 1997) and simplified by Plank, Wall \& David (2006).
%The model is outlined in Chapter~5, but it has a large number of
%coefficients (Table~5.1) that are accepted here without comment, and
%with no apparent attempt to examine the sensitivity of the computed
%results to variation in the coefficients.  I feel that some more
%comment and analysis is required.
%\item
%Similarly, values used for wall mass transfer/reaction rates
%(Table~6.1) are accepted without comment from earlier work. When in
%Chapter~7 the characteristic stress for ATP release $\tau_0$,
%equation~(4.21), is varied, no attempt is made to justify (or to
%provide units) for the values used.  This would seem to be important,
%since the inclusion of wall release of ATP can apparently have a large
%effect on concentrations of ATP at the wall (compare for example
%Figures~7.29\,a and 7.29\,b).
%\item
%In general, the fonts sizes used for labelling of line graphs was far
%too small, to the point of near-unintelligibility in many
%cases. However since the amount of work required to remedy this at the
%present stage is far too large, I do not believe it is reasonable to
%require that the plots be re-made.
%\item
%There were a number of spelling errors that could easily have been
%found and corrected with a spell checker. Some of the errors (such as
%the use of `tract' instead of `tracked' in the caption to
%Figure~8.24), could not readily be identified without careful
%proof-reading, however. There were also quite a number of sentences in
%which words were missing, and again such problems are difficult to
%identify without careful proof reading.
%\item
%The List of Abbreviations (actually a list of acronyms) on page~xxix
%should really have been far more extensive, particularly in this field
%of research, where acronyms are heavily used.
%\end{enumerate}


%\noindent\textbf{Recommended grade: A}\\[5pt]
\end{document}

