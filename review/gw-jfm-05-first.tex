%%%%%%%%%%%%%%%%%%%%%%%%%%%%%%%%%%%%%%%%%%%%%%%%%%%%%%%%%%%%%%%%%%%%%%%%%%%%%%
%
%%%%%%%%%%%%%%%%%%%%%%%%%%%%%%%%%%%%%%%%%%%%%%%%%%%%%%%%%%%%%%%%%%%%%%%%%%%%%%
\documentclass[a4paper,11pt]{article}
\usepackage{graphicx}
%\pagestyle{plain}

\setlength{\textheight}			{245mm}
\setlength{\textwidth}			{160mm}
\setlength{\topmargin}			{-8mm}
\setlength{\headsep}                    {0mm}
\setlength{\oddsidemargin}              {0mm}
\setlength{\parsep}                     {0mm}
\setlength{\parindent}                  {2ex}


\renewcommand{\baselinestretch}	{1.0}
	
\def\undertext#1{$\underline{\smash{\hbox{#1}}}$}

\begin{document}
\begin{center}
\noindent
\rule{\textwidth}{0.25mm}
\end{center}

\noindent
\textbf{Authors:}
Govardhan and Williamson

\noindent
\textbf{Title:}
Defining the `modified Griffin plot' in vortex-induced vibration:
Revealing the effect of Reynolds number using controlled positive
and negative damping

\begin{center}
\noindent
\rule{\textwidth}{0.25mm}
\end{center}

This is an interesting and well-written manuscript which makes a nice
contribution to the study of relatively high-amplitude vortex-induced
vibration of circular cylinders in a cross flow. Chiefly by examining
Reynolds-number effects on vibration amplitudes in an experiment in
which they were independently able to vary Reynolds number, density
ratio and damping, they have been able to resolve a number of key
points and unify many of the past results in this important
applications area. I recommend publication in JFM after some
revisions. I hope that the revisions I have to suggest will aid in the
authors' work becoming established as a central point of reference in
prediction of vortex-induced vibration.

\subsubsection*{Major points}

\begin{enumerate}

\item
I feel that the quadratic relationship fitted between mass-damping
parameter and normalized motion amplitude is not the best choice, for a
number of reasons, and I strongly suggest that the authors adopt the
simpler linear relationship between the two variables as a major
outcome of their work, concentrating on the higher motion
amplitudes. Reasons:
\begin{enumerate}
\item
The quadratic relationship they suggest has a turning point at
moderate values of mass-damping and then predicts that amplitudes
increase boundlessly with increased damping, which cannot be
physically correct.  If the authors revise their major findings to
adopt the linear relationship instead of the quadratic one they have a
unified, simple and robust method for predicting response amplitudes
at comparatively high amplitudes, say for $\alpha<0.75$.
\item
At higher values of mass-damping, or lower values of motion amplitude,
cylinder motion will tend towards that of a lightly damped mode in
random vibration. The authors' current treatment does not deal
correctly with this end of the mass-damping spectrum. Quite a volume
of work has been done in this area by wind engineers, Barry Vickery in
particular.
\item
A linear fit has less parameters.
\end{enumerate}

\item
Although the authors have referenced an early piece of work by
Vickery, I feel they could profitably examine his later work and
models which tried to address the cross-over between low and high
motion amplitude regimes, in which the cylinder-velocity-correlated
forces are characterized as a negative self-limiting aerodynamic
damping. Two relevant papers (out of many) are: Vickery \& Basu
(1983), `Across-wind vibration of structures of circular cross
section. Part 1.' \textit{J Wind Eng Ind Aero} \textbf{12}: 49--73 and
Vickery (1990), `Progress and problems in the prediction of the
response of prototype structures to vortex-induced excitation'
\textit{J Wind Eng Ind Aero} \textbf{33}: 181--196.


While Vickery suggested, based mainly on analysis of experiments by
Wooton, a quadratic relationship between negative aerodynamic damping
and motion amplitude, in fact the present work would seem to provide
extremely good support for a linear relationship at high motion
amplitudes. There is also previous experimental support for this
outcome at low motion amplitudes, as presented in Blackburn \&
Melbourne (1993), `Cross flow response prediction of slender
circular-cylindrical structures: prediction models and recent
experimental results', \textit{J Wind Eng Ind Aero} \textbf{49}:
167--176. If one substitutes this linear relationship into Vickery's
response-amplitude equation, one gets a response-amplitude vs.\
mass-damping curve as shown in figure~5 of that paper. In terms of the
current authors' variables, with a linear relationship one has an
equation like:
\begin{equation}
A^{\ast} = \lambda/[\alpha - \alpha_0(1-A^{\ast}/A^{\ast}_M)]^{1/2}
\label{eq.amp}
\end{equation}
where $\lambda$ incorporates modal stiffness and cross-flow excitation
in the low-amplitude limit, while the other variable $\alpha_0$
expresses the negative aerodynamic damping in the low-amplitude limit,
and can be simply obtained from the linear fit supplied by the
authors. The outcome of fitting this equation to the authors' results
is shown here in figure~\ref{fig.alpha}. This seems to me to be quite
convincing, and has the benefit of predicting the correct form for the
dependence of motion amplitudes on mass-damping in the high
mass-damping limit. Of course, since modal stiffness and
fixed-cylinder lift is subsumed in the parameter $\lambda$, this needs
to be worked out for each individual structure, but suitable values
for the fluid-mechanical parts are widely available, for example in
ESDU data sheets.  This parameter has no effect in the high-amplitude
limit.

\begin{figure}[h]
\begin{center}
\includegraphics[bb=25 180 570 520,width=0.6\textwidth]
{alpha.eps}
\end{center}
\caption{Form of response curve obtained from equation~(\ref{eq.amp}),
  compared to the authors' normalized response amplitude data, with
  fitted value $\lambda=0.065$. Parameter $\alpha_0=1/1.0514$ comes
  from the authors' linear relationship.}
\label{fig.alpha}
\end{figure}

\item
The authors provide evidence for the dependence of response amplitudes
on Reynolds number, but as far as could see, nowhere do they
explicitly point out that at least this part of the parameterization
is for subcritical Reynolds numbers only. One would expect substantial
variations of the parameters once Reynolds numbers reach the drag
crisis and above, and where one might also expect them to become
sensitive to surface roughness, freestream turbulence, and the
like. The authors need to issue caveats on this issue, and supply the
likely range of Reynolds numbers for the validity of their
characterization.

\end{enumerate}

\subsubsection*{Minor points}

\begin{enumerate}

\setcounter{enumi}{3}

\item
The manuscript seems a little over-long, and some of the figures
(e.g.\ 11b, 15) seem redundant. I doubt we need to know the details of
alternative curve fits and coefficients of variation, as I think a
better justification for choosing between close alternatives is the
physics of what they predict, say in asymptotic limits. I suggest the
authors consider re-working the manuscript with a view to shortening
it.

\item
In table~2 the details of mass-damping for the numerical simulations
seem to have been omitted. Also, both 2D (laminar) and 3D (turbulent)
DNS results from Blackburn, Govardhan and Williamson (2001) could have
been included and also have appeared in the associated figure~11. Most
likely the authors intended to do so, as the paper is included in the
list of references but does not seem to be referred to in the
manuscript.

\end{enumerate}

\end{document}


