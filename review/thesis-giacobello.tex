%%%%%%%%%%%%%%%%%%%%%%%%%%%%%%%%%%%%%%%%%%%%%%%%%%%%%%%%%%%%%%%%%%%%%%%%%%%%%%
% Matteo Giacobello PhD
%
% $Id$
%%%%%%%%%%%%%%%%%%%%%%%%%%%%%%%%%%%%%%%%%%%%%%%%%%%%%%%%%%%%%%%%%%%%%%%%%%%%%%
\documentclass[a4paper,12pt]		{article}
\usepackage{amsmath,amssymb,amsxtra,bm,natbib,graphicx,epic,eepic}
%\usepackage[all,import]{xy}

\setlength{\textheight}			{245mm}
\setlength{\textwidth}			{155mm}
\setlength{\topmargin}			{-5mm}
\setlength{\headsep}			{0mm}
\setlength{\oddsidemargin}		{0mm}
\setlength{\parsep}                     {0mm}

\renewcommand{\baselinestretch}		{1.00}

%%%%%%%%%%%%%%%%%%%%%%%%%%%%%%%%%%%%%%%%%%%%%%%%%%%%%%%%%%%%%%%%%%%%%%%%%%%%%%
% General purpose macros, font selections.

\newfont{\cmssLarge}{cmss10 scaled\magstep3}        % large sans-serif font
\def\undertext#1{$\underline{\smash{\hbox{#1}}}$}   % underline running text
\def\refitem{\noindent\hangindent=2em}              % hanging index for refs.

%%%%%%%%%%%%%%%%%%%%%%%%%%%%%%%%%%%%%%%%%%%%%%%%%%%%%%%%%%%%%%%%%%%%%%%%%%%%%%
% Mathematical operators

\def\ol#1{\overline{#1}}
\def\div{{\nabla\!\cdot}}
\def\curl{{\nabla\times}}
\def\divb#1{{\nabla\!\cdot\left( #1 \right)}}
\def\grad{{\nabla}}
\def\laplacian{{\nabla^2}}
\def\gradb#1{{\nabla\left( #1 \right)}}

%%%%%%%%%%%%%%%%%%%%%%%%%%%%%%%%%%%%%%%%%%%%%%%%%%%%%%%%%%%%%%%%%%%%%%%%%%%%%%
% Abbreviations.

\newcommand\Rey{\mbox{\textit{Re}}}
\newcommand\Sto{\mbox{\textit{St}}}
\newcommand\Kc{\mbox{\textit{KC}}}

\newcommand\NS{Neimark--Sacker}
\newcommand\NavSto{Navier--Stokes}
\newcommand\TC{Taylor--Couette}
\newcommand\TG{Taylor--G{\"{o}}rtler}
\newcommand\KH{Kelvin--Helmholtz}
\newcommand{\KC}{Keulegan--Carpenter}
\newcommand\Ka{K{\'{a}}rm{\'{a}}n}
\newcommand\Po{Poincar{\'{e}}}

\newcommand\oned{one-di\-men\-sion\-al}
\newcommand\twod{two-di\-men\-sion\-al}
\newcommand\threed{three-di\-men\-sion\-al}
\newcommand\twoc{two-com\-po\-nent}
\newcommand\threec{three-com\-po\-nent}
\newcommand\qp{qua\-si-perio\-dic}
\newcommand\cc{com\-plex-con\-ju\-gate}
\newcommand\st{spa\-tio-tem\-po\-ral}
\newcommand\hprm{half-pe\-riod-flip map}

\newcommand{\ie}{i.e.\ }
\newcommand{\eg}{e.g.\ }

\newcommand\real{{\mbox{Re}}}
\newcommand\imag{{\mbox{Im}}}
\newcommand\ci{\mathrm{i}}
\newcommand\ce{\mathrm{e}}
\newcommand\cd{\mathrm{d}}

\newcommand\Pm{\mathcal{P}}
\newcommand\Hm{\mathcal{H}}
\newcommand{\Fm}{\mathcal{F}}

\newcommand\TW{{TW}}
\newcommand\TWp{TW$_+$}
\newcommand\TWm{TW$_-$}
\newcommand\SW{{SW}}

\newcommand{\Tone}{\hbox{$\mathbb{T}^1$}}
\newcommand{\Ttwo}{\hbox{$\mathbb{T}^2$}}
\newcommand{\Tthree}{\hbox{$\mathbb{T}^3$}}

\newcommand\R{\mathbb{R}}
\newcommand\C{\mathbb{C}}
\newcommand\I{\mathbb{I}}
\newcommand\Z{\mathbb{Z}}
\newcommand\Poly{\mathbb{P}}

%%%%%%%%%%%%%%%%%%%%%%%%%%%%%%%%%%%%%%%%%%%%%%%%%%%%%%%%%%%%%%%%%%%%%%%%%%%%%%
\begin{document}

%=============================================================================
\subsection*{Examiner's report, M. Giacobello PhD thesis}

This thesis deals with a direct numerical simulation investigation of
low Reynolds number incompressible flow past a sphere that rotates
about an axis normal to the oncoming flow direction. The parameters
varied for steady rotation are Reynolds number and rotation rate,
while oscillatory rotation is investigated for one Reynolds number and
rotation rate, with oscillation frequency as a parameter. The
numerical method solves the flow in a spherical coordinate system,
employs spectral expansions in all three coordinate directions, and is
based on previous works by Mittal and Balachandar. While the
methodology is described in some depth, the focus is mainly on fluid
mechanics.

\textbf{I have no hesitation in recommending that the thesis be
accepted as it stands.} It presents a thorough, rigorous treatment of
the topic, and is very well presented and written. Relevant existing
work in the area is reviewed and where relevant compared to the
results obtained in the investigation. Discrepancies and problems with
some previous investigations are identified. New and interesting
results are provided. There should be no problem in adapting the
material for publication in premier international journals of the
discipline.

I did identify some topics that the candidate might consider when
writing up the work for journal publication. The first is that the
spatial convergence properties of the numerical method are not
formally demonstrated\,---\,a fully spectral method if correctly
conceived and implemented should provide solutions to the \NavSto\
equations that converge exponentially with increasing basis function
order. One approach here would be to take an analytical solution given
in Cartesian coordinates, and demonstrate the convergence of the
numerical solution in spherical coordinates. An example of this
approach (but for cylindrical rather than spherical coordinates) is
given in Blackburn \&~Sherwin, \textsl{J. Comput.\ Phys.}\/
\textbf{197}, 759--778. The second concerns the nature of the novel
wake mode found for the steadily rotating sphere at higher Reynolds
numbers and rotation rates. It seems possible that because the mode
appears to originate in the shear layer, it is a convective rather
than absolute instability. If that is so then it could be sensitive to
numerical noise and for example might reduce in amplitude with
increasing mesh resolution, or change its characteristics when driven
with low-amplitude forcing at a different frequency. I suggest it
would be prudent to give this possibility some careful thought.

%%%%%%%%%%%%%%%%%%%%%%%%%%%%%%%%%%%%%%%%%%%%%%%%%%%%%%%%%%%%%%%%%%%%%%%%%%%%%%
\end{document}
