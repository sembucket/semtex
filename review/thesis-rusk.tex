\documentclass[12pt,a4paper]{letter}
\usepackage{pstricks}
\usepackage{graphicx}
\setlength{\textheight}{235mm}
\setlength{\textwidth}{165mm}
\setlength{\topmargin}{-10mm} 
\setlength{\oddsidemargin}{-0.5cm}
\setlength{\evensidemargin}{-0.2cm}
\setlength{\unitlength}{1mm}
\renewcommand{\baselinestretch}{1.00}
\def\undertext#1{$\underline{\smash{\hbox{#1}}}$}

\pagestyle{empty}
\thispagestyle{empty}
\raggedbottom

\date{}

\begin{document}  

\address{
\mbox{ }\\
\begin{tabular}{ll}
phone:  &(+61 3) 9252 6330\\
fax:    &(+61 3) 9252 6240\\
mailto: &hugh.blackburn@csiro.au\\
& 31 August 2004
\end{tabular}
}

\begin{letter}
{
To whom it may concern
}
\signature{\vspace*{-10mm}
\includegraphics[scale=0.8]{hmbsig.ps}\\
Dr Hugh M. Blackburn\\
\undertext{Principal Research Scientist}
}

\opening{
\begin{pspicture}(0,0)(0,0)
\rput(82mm,50mm){
\includegraphics%
[bb=0 655 594 842,clip,width=1.20\linewidth]{MIT_letter.ps}}
\end{pspicture}
\centerline{\undertext{\textbf{Re: Master's Thesis, R\,D Rusk}}}
}

I have carried out a review of Rueben Rusk's thesis, as requested. I
consider it a fine example of a Master's thesis, and recommend the
thesis be awarded with the grade of A. Please find my detailed
comments and assessment attached.

\closing{Yours sincerely,}
\encl{Assessment}
\end{letter}


%%%%%%%%%%%%%%%%%%%%%%%%%%%%%%%%%%%%%%%%%%%%%%%%%%%%%%%%%%%%%%%%%%%%%%%%%%%%%%
\vspace*{-10mm}
\begin{center}
\textbf{
\undertext{Master's Thesis, R\,D Rusk}\\[5pt]
Large-Scale Oscillatory Stabilisation of the Boundary Layer
}
\end{center}

\noindent\textbf{Subject material}\\[5pt] The work is concerned with
stabilisation of the Blasius boundary layer solution via application
of streamwise and cross-flow forcing. Numerical Floquet stability
analysis of one-dimensional boundary layer solutions was carried out
using special-purpose codes written in Matlab for the cases of
streamwise and cross flow Lorenz-type (localised body force)
forcing. For the case of travelling-wave type cross flow forcing, base
flow fields were investigated via two-dimensional direct numerical
simulation, apparently employing the FIDAP code with arbitrary
Lagrangian--Eulerian formulation to accommodate wall motion, but
stability analysis was not attempted.

\noindent\textbf{Comments}\\[5pt] The whole thesis is very nicely
written and presented, although there were a few minor errors and
omissions that should have been caught during proof reading.

The first six chapters provide a broad review of boundary layer
instabilities, turbulence and stabilisation. This section is easy to
follow and indicates that the author has a good grasp of the subject area.

Chapter 7 describes with linear stability analysis of Orr--Sommerfeld
type as applied to both spatial and temporal stability analysis of
steady 1D boundary layer Navier--Stokes solutions, and extension to
temporal Floquet problems. For the numerical implementation, mapped
Chebyshev expansions are used in the wall-normal direction.

In chapter 8, these methods are used first to verify the codes against
available solutions for steady plane Couette, Poiseuille, and Blasius
boundary layer flows, for both spatial and temporal type
analyses. Following that, they are used to study the effect of
streamwise and cross flow oscillatory forcing of Lorenz type. The
emphasis is on streamwise forcing, though it appears less effective in
stabilisation than cross flow, on the basis that the cost of the cross
flow-forcing analysis is much higher. Given the development and
emphasis on stability analysis that had come previously, I found the
latter part of chapter~8 a little brief. Also, I didn't find (although
it may have appeared somehere in the thesis) an explanation of how the
shape of the Lorenz forcing was established or chosen. Likewise, there
seemed to be no examination of forcing by direct wall oscillation,
something that might have been expected after reading the introductory
chapters. But on the other hand, the investigation could not be
anything like exhaustive within the context of a Master's thesis, and
as the author points out, the size of the possible parameter space is
large.

The final part of the thesis (chapters 9--11) deals with a
computational investigation of flows induced by two-dimensional,
two-component travelling-wave wall motions. Because the motion is
spatially periodic, Floquet assumptions may be employed in deriving
full and approximate (low-motion-amplitude) stream-function
formulations, which are given in chapter~9. Ultimately however, a
primitive-variable finite element formulation was used to examine the
flow generated at moderate travelling wave amplitudes, and was
validated against the approximate stream-function formulation at low
wave amplitudes. The travelling wave wall motion generates a
time-average secondary streaming flow, which is shown at the end of
chapter~11.

Overall, this is a fine example of a Master's thesis.

\noindent\textbf{Recommended grade: A}\\[5pt]



\end{document}

